\Gls{ptt} estimates from \gls{hp} metamorphic rocks and global \gls{shf} rates evidently encode information about \gls{pts} fields deep in \glspl{sz}. Previous work demonstrates the possibility of decoding such geodynamic information by comparing physics-based numerical models with empirical observations of \gls{shf} and the metamorphic rock record. However, antithetical interpretations of (non)uniformity with respect to \gls{pts} fields are emerging from this line of inquiry. For example, while
mechanical coupling depths inverted from \gls{shf} are narrowly distributed among \glspl{sz}, maximum \gls{pt} conditions inverted from exhumed metamorphic rocks are relatively wide-ranging, and yet also uniformly distributed across pressures up to 2.4 GPa. This dissertation scrutinizes (dis)similarities among \glspl{sz} inferred from large numerical and empirical datasets by applying a variety of computational techniques. First, coupling depths for 13 modern \glspl{sz} are predicted after observing coupling in 64 numerical
geodynamic simulations. Second, spatial patterns of \gls{shf} are assessed in two-dimensions by interpolating thousands of \gls{shf} observations near several \gls{sz} segments. Third, \gls{ptt} distributions of over one million markers traced from the previous set of 64 \gls{sz} simulations are compared with hundreds of empirical \gls{ptt} estimates from the rock record to assess the effects of \glspl{tkbc} on deep mechanical processing of rock in \glspl{sz}. These studies conclude the following. Mechanical
coupling between plates is primarily controlled by the upper plate lithospheric thickness, with marginal responses to other \glspl{tkbc}. \Gls{shf} interpolations show high variance within and among \gls{sz} segments, suggesting local, rather than widespread, continuity of \gls{pts} fields deep within \glspl{sz}. Computed marker recovery rates correlate with \glspl{tkbc}, and are therefore expected to vary among \glspl{sz}. Finally, computed \gls{ptt} distributions
of markers show patterns consistent with transient, localized recovery from a cooling, serpentinizing plate interface. Together, this work encourages more antireductionist and diversified views of subduction geodynamics until \gls{shf} and \gls{ptt} datasets can more precisely distinguish (dis)similarities in \gls{pts} fields within and among \glspl{sz}. Strategically scaling \gls{ptt} and \gls{shf} datasets in the future will improve computational precision and confidence, and thus will advance subduction zone research.
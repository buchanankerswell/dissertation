\Gls{pt} estimates from exhumed \gls{hp} metamorphic rocks and global surface heat flow observations evidently encode information about subduction zone thermal structure and the nature of mechanical and chemical processing of subducted materials along the interface between converging plates. Previous work demonstrates the possibility of decoding such geodynamic information by comparing numerical geodynamic models with empirical observations of surface heat flow and the metamorphic rock record. However, ambigous interpretations can arise from this line of inquiry with respect to thermal gradients, plate coupling, and detachment and recovery of subducted materials. This dissertation applies a variety of computational techniques to explore changes in plate interface behavior among subduction zones from large numerical and empirical datasets. First, coupling depths for 13 modern subduction zones are predicted after observing mechanical coupling in 64 numerical geodynamic simulations. Second, upper-plate surface heat flow patterns are assessed by applying two methods of interpolation to thousands of surface heat flow observations near subduction zone segments. Third, \gls{pt} distributions of over one million markers traced from the previous set of 64 subduction simulations are compared with hundreds of empirical \gls{pt} estimates from the rock record to assess the effects of thermo-kinematic boundary conditions on detachment and recovery of rock along the plate interface. These studies conclude the following. Mechanical coupling between plates is primarily controlled by the upper plate lithospheric thickness, with marginal responses to other thermo-kinematic boundary conditions. Upper-plate surface heat flow patterns are highly variable within and among subduction zone segments, suggesting both uniform and nonuniform subsurface thermal structure and/or geodynamics. Finally, \gls{pt} distributions of recovered markers show patterns consistent with trimodal detachment (recovery) of rock from distinct depths coinciding with the continental Moho at ~35-40 km, the onset of plate coupling at ~80 km, and an intermediate recovery mode around ~55 km. Together, this work identifies important biases in geodynamic numerical models (insufficient implementation of recovery mechanisms and/or heat generation/transfer), surface heat flow observations (poor spatial coverage and/or oversampling of specific regions), and petrologic datasets (selective sampling of metamorphic rocks amenable to petrologic modelling techniques) that, if addressed, could significantly improve the current understandings of subduction interface behavior.
\Gls{ptt} estimates from \gls{hp} metamorphic rocks and global surface heat flow rates evidently encode information about pressure-temperature-strain fields deep in subduction zones. Previous work demonstrates the possibility of decoding such geodynamic information by comparing physics-based numerical models with empirical observations of surface heat flow and the metamorphic rock record. However, antithetical interpretations of (non)uniformity with respect to pressure-temperature-strain fields are emerging from this line of inquiry. For example, while
mechanical coupling depths inverted from surface heat flow are narrowly distributed among subduction zones, maximum \gls{pt} conditions inverted from exhumed metamorphic rocks are relatively wide-ranging, and yet also uniformly distributed across pressures up to 2.4 GPa. This dissertation scrutinizes (dis)similarities among subduction zones inferred from large numerical and empirical datasets by applying a variety of computational techniques. First, coupling depths for 13 modern subduction zones are predicted after observing coupling in 64 numerical
geodynamic simulations. Second, spatial patterns of surface heat flow are assessed in two-dimensions by interpolating thousands of surface heat flow observations near several subduction zone segments. Third, \gls{ptt} distributions of over one million markers traced from the previous set of 64 subduction zone simulations are compared with hundreds of empirical \gls{ptt} estimates from the rock record to assess the effects of thermo-kinematic boundary conditions on deep mechanical processing of rock in subduction zones. These studies conclude the following. Mechanical
coupling between plates is primarily controlled by the upper plate lithospheric thickness, with marginal responses to other thermo-kinematic boundary conditions. Surface heat flow interpolations show high variance within and among subduction zone segments, suggesting local, rather than widespread, continuity of pressure-temperature-strain fields deep within subduction zones. Computed marker recovery rates correlate with thermo-kinematic boundary conditions, and are therefore expected to vary among subduction zones. Finally, computed \gls{ptt} distributions
of markers show patterns consistent with transient, localized recovery from a cooling, serpentinizing plate interface. Together, this work encourages more antireductionist and diversified views of subduction geodynamics until surface heat flow and \gls{ptt} datasets can more precisely distinguish (dis)similarities in pressure-temperature-strain fields within and among subduction zones. Strategically scaling \gls{ptt} and surface heat flow datasets in the future will improve computational precision and confidence, and thus will advance subduction zone research.
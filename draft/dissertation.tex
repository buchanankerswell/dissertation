% This is a combination of pandoc's default latex template:
% https://github.com/jgm/pandoc/blob/master/data/templates/default.latex

% and Dylan Mikesell's BSU thesis template:
% https://github.com/dylanmikesell/BSU_LaTeX_Thesis_Template/blob/master/src/BSUmain.tex

% and Dylan Mikesell's BSU style file:
% https://github.com/dylanmikesell/BSU_LaTeX_Thesis_Template/blob/master/src/BSUthesis.sty

% For BSU style requirements see:
% https://github.com/dylanmikesell/BSU_LaTeX_Thesis_Template/blob/master/src/BSU_checklist.pdf

% Pass options to packages loaded elsewhere
  \PassOptionsToPackage{dvipsnames,svgnames,x11names}{xcolor}

% Document class options
% Note: everything defined between \documentclass{} and \begin{document}
% is the "preamble"
\documentclass[
      12pt,
          twoside]{report}

% Loading packages and options

% Page geometry settings
\usepackage[
  left=1.5in,
  right=1in,
  top=1in,
  bottom=1in,
  letterpaper,
  includehead,
  includefoot,
  headheight=14.5pt
]{geometry}

% For colors
\usepackage{xcolor} % Handles colors

% For making corrections to functions
\usepackage{etoolbox}

% For strikeout and underline text
\usepackage[normalem]{ulem}

% Linespacing using setspace package
\usepackage{setspace}
\setstretch{2} % double space

% Math packages
\usepackage{amsmath,amssymb}

% Textcase for handling upper/lower case
\usepackage{textcase}

% Changepage package for changing layout in the middle of a document
\usepackage{changepage}

% For month year format
\usepackage{datetime}

% For chapter (and other) headings required by BSU
\usepackage{fancyhdr}

% Set captions to bold
\usepackage[labelfont=bf,textfont=bf]{caption}

% For making tables and final reading approval page
\usepackage{tabularx}

% For setting (sub)section heading formatting and first paragraph spacing
\usepackage[explicit]{titlesec}

% For TOC style
\usepackage{titletoc}

% Font encoding
% Defaults to 8-bit T1 encoding with 256 glyphs
% https://ctan.org/pkg/encguide
% http://www.micropress-inc.com/fonts/encoding/t1.htm
\usepackage[T1]{fontenc}
\usepackage[utf8]{inputenc}
\usepackage{textcomp} % provide euro and other symbols

% Use upquote if available, for straight quotes in verbatim environments
\IfFileExists{upquote.sty}{\usepackage{upquote}}{}
\IfFileExists{microtype.sty}{% use microtype if available
  \usepackage[]{microtype}
  \UseMicrotypeSet[protrusion]{basicmath} % disable protrusion for tt fonts
}{}

% Font family setting
  % Default to adobe times new roman with math support
  \usepackage{mathptmx}

% Allow pandoc to inject code highlighting environments

% Tables settings
\usepackage{longtable,booktabs,array}
\usepackage{multirow}
\usepackage{calc} % for calculating minipage widths

% Correct order of tables after \paragraph or \subparagraph
\makeatletter
\patchcmd\longtable{\par}{\if@noskipsec\mbox{}\fi\par}{}{}
\makeatother

% Block quote shaded style
\usepackage{framed}
\AtBeginEnvironment{quote}{\par\singlespacing\small}
\let\oldquote=\quote
\let\endoldquote=\endquote
\colorlet{shadecolor}{gray!15}
\renewenvironment{quote}{\begin{shaded*}\begin{oldquote}}{\end{oldquote}\end{shaded*}}

% Allow footnotes in longtable head/foot
\IfFileExists{footnotehyper.sty}{\usepackage{footnotehyper}}{\usepackage{footnote}}
\makesavenoteenv{longtable}

% Graphics settings
\usepackage{graphicx}
\makeatletter
\def\maxwidth{\ifdim\Gin@nat@width>\linewidth\linewidth\else\Gin@nat@width\fi}
\def\maxheight{\ifdim\Gin@nat@height>\textheight\textheight\else\Gin@nat@height\fi}
\makeatother

% Scale images if necessary, so that they will not overflow the page
% margins by default, and it is still possible to overwrite the defaults
% using explicit options in \includegraphics[width, height, ...]{}
\setkeys{Gin}{width=\maxwidth,height=\maxheight,keepaspectratio}
% Set default figure placement to htbp
\makeatletter
\def\fps@figure{htbp}
\makeatother

% Prevent overfull lines
\setlength{\emergencystretch}{3em}
\providecommand{\tightlist}{\setlength{\itemsep}{0pt}\setlength{\parskip}{0pt}}

% Csl environment (required by pandoc)
  \newlength{\cslhangindent}
  \setlength{\cslhangindent}{1.5em}
  \newlength{\csllabelwidth}
  \setlength{\csllabelwidth}{3em}
  \newlength{\cslentryspacingunit} % times entry-spacing
  \setlength{\cslentryspacingunit}{\parskip}
  \newenvironment{CSLReferences}[2] % #1 hanging-ident, #2 entry spacing
   {% don't indent paragraphs
    \setlength{\parindent}{0pt}
    % turn on hanging indent if param 1 is 1
    \ifodd #1
    \let\oldpar\par
    \def\par{\hangindent=\cslhangindent\oldpar}
    \fi
    % set entry spacing
    \setlength{\parskip}{#2\cslentryspacingunit}
   }%
   {}
  \usepackage{calc}
  \newcommand{\CSLBlock}[1]{#1\hfill\break}
  \newcommand{\CSLLeftMargin}[1]{\parbox[t]{\csllabelwidth}{#1}}
  \newcommand{\CSLRightInline}[1]{\parbox[t]{\linewidth - \csllabelwidth}{#1}\break}
  \newcommand{\CSLIndent}[1]{\hspace{\cslhangindent}#1}

% Expand header includes

% Bibliography settings
% Natbib settings
  \usepackage[]{natbib}
  \bibliographystyle{assets/bib/authordate1}

% Nocite

% Some options for hyperlinks
\usepackage[bookmarks=true,pageanchor=false]{hyperref}
\hypersetup{
      colorlinks=true,
    linkcolor={Maroon},
    filecolor={Maroon},
    citecolor={Blue},
    urlcolor={Blue},
  }
\usepackage{xurl} % add URL line breaks if available
\usepackage{bookmark}
\urlstyle{same} % disable monospaced font for URLs

% Abbreviations and acronyms
\usepackage[nonumberlist,acronym,toc]{glossaries-extra}
% http://ctan.mirrors.hoobly.com/macros/latex/contrib/glossaries/glossariesbegin.pdf
\setabbreviationstyle[acronym]{long-short} % glossaries-extra.sty only
% For abbreviations
  \makeglossaries
  \loadglsentries{assets/tex/abbreviations}

% Nomenclature
\usepackage[noprefix,intoc]{nomencl}
% For symbols and nomenclature 
  \makenomenclature
  \nomenclature{$wt.\%$}{weight percent}
\nomenclature{$\vec{q}$}{surface heat flow}
\nomenclature{$\Phi$}{thermal parameter}
\nomenclature{$\vec{v}$}{convergence velocity}
\nomenclature{$Z_{UP}$}{upper plate thickness}
\nomenclature{$Z_{cpl}$}{mechanical coupling depth}
\nomenclature{$\eta$}{viscosity}
\nomenclature{$\rho$}{density}
\nomenclature{$H_2O$}{water (mineral-bound or liquid)}
\nomenclature{$an75$}{plagioclase composition of 75\% anorthite and 25\% albite}
\nomenclature{$A$}{material constant}
\nomenclature{$E$}{activation energy}
\nomenclature{$V$}{activation volume}
\nomenclature{$n$}{power law exponent}
\nomenclature{$\phi$}{internal friction angle}
\nomenclature{$\sigma_{crit}$}{critical stress for brittle/plastic deformation}
\nomenclature{$H$}{volumetric heat production}
\nomenclature{$k$}{thermal conductivity}
\nomenclature{$C_p$}{specific heat capacity}
\nomenclature{$\alpha$}{thermal expansivity}
\nomenclature{$\beta$}{compressibility}
\nomenclature{$R$}{gas constant}
\nomenclature{$G$}{shear modulus}
\nomenclature{$m$}{grain size exponent}
\nomenclature{$b$}{Burgers vector}
\nomenclature{$\dot{\varepsilon}$}{strain rate tensor}
\nomenclature{$\dot{\varepsilon}_{II}$}{second invariant of the strain rate tensor}
\nomenclature{$C$}{cohesive strength}
\nomenclature{$\hat{\gamma}$}{experimental variogram}
\nomenclature{$\gamma$}{variogram model}
\nomenclature{$c$}{variogram lag cutoff constant}
\nomenclature{$n_{lag}$}{number of variogram lags}
\nomenclature{$n_{max}$}{maximum number of nearby observation pairs for local Kriging}
\nomenclature{$h$}{variogram lag distance}
\nomenclature{$l$}{variogram lag shift constant}
\nomenclature{$N(h)$}{number of observation pairs separated by a lag distance $h$}
\nomenclature{$M$}{number of observations in a Kriging domain}
\nomenclature{$\delta$}{lag binwidth}
\nomenclature{$Z(u)$}{observation of a random variable at location $u$}
\nomenclature{$\hat{Z}(u)$}{estimation of a random variable at location $u$}
\nomenclature{$C(\Theta)$}{cost function with parameters $\Theta$}
\nomenclature{$C_{interp}(\Theta)$}{Kriging error with parameters $\Theta$}
\nomenclature{$C_{vgrm}(\Theta)$}{variogram error with parameters $\Theta$}
\nomenclature{$w_{vgrm}$}{variogram weight for cost function}
\nomenclature{$w_{interp}$}{Kriging weight for cost function}
\nomenclature{$n$}{variogram nugget}
\nomenclature{$s$}{variogram sill}
\nomenclature{$a$}{variogram effective range}
\nomenclature{$K_1$}{modified bessel function}
\nomenclature{$Bes$}{Bessel variogram model}
\nomenclature{$Cir$}{circular variogram model}
\nomenclature{$Exp$}{exponential variogram model}
\nomenclature{$Gau$}{Gaussian variogram model}
\nomenclature{$Lin$}{linear variogram model}
\nomenclature{$Sph$}{spherical variogram model}


%% End packages and options

% Frontmatter pages (title, approval, copyright)

% Make title page
  \title{Computational Approaches to Understanding Surface Heat Flow, the Metamorphic Rock Record, and Subduction Geodynamics}

% Change title of contents name
\renewcommand{\contentsname}{Table of contents}

\def\maketitle{
  \cleardoublepage
  \begin{titlepage}
    \pagenumbering{roman}
    \begin{center}
        % Title
        {\huge Computational Approaches to Understanding Surface Heat Flow, the Metamorphic Rock Record, and Subduction Geodynamics \par}
        \vspace*{0.5in}

        % Author
        {by\\}
        {Buchanan C. Kerswell}
        \vspace*{1in}

        % Description
        A dissertation\\
        submitted in partial fulfillment \\
        of the requirements for the degree of\\
        Doctor of Philosophy~in~Geosciences\\
        Boise State University
        \vspace*{0.5in}

        % Date
        November 2021
    \end{center}
  \end{titlepage}
  \let\maketitle\relax
}

% Make final reading approval page
\def\makesubmittalsheet{
  \cleardoublepage 
  \begin{center}
    BOISE STATE UNIVERSITY GRADUATE COLLEGE\\
    \vspace{\baselineskip}
    \textbf{DEFENSE COMMITTEE AND FINAL READING APPROVALS}\\
    \vspace{\baselineskip}
    of the dissertation submitted by\\
    \vspace{\baselineskip}
    {Buchanan C. Kerswell}\\
    \vspace{\baselineskip}
  \end{center}
  \begin{flushleft}
    \begin{singlespace}
      \begin{tabularx}{\textwidth}{@{}lX} 
        Dissertation Title: & {Computational Approaches to Understanding Surface Heat Flow, the Metamorphic Rock Record, and Subduction Geodynamics}
      \end{tabularx}
    \end{singlespace}
    \begin{tabularx}{\textwidth}{@{}lX} 
      Date of Final Oral Examination: & {August 27, 2021}
    \end{tabularx}
  \end{flushleft}
  \begin{singlespace}
    \noindent The following individuals read and discussed the dissertation submitted by student {Buchanan C. Kerswell}, and they evaluated the student’s presentation and response to questions during the final oral examination. They found that the student passed the final oral examination.\\
  \end{singlespace}
  \begin{flushleft}
    \begin{tabular}{@{}ll} 
      {Matthew J. Kohn} {Ph.D.} \hspace{2cm} & {Chair, Supervisory Committee} \\ 
      {C.J. Northrup} {Ph.D.} \hspace{2cm} & {Member, Supervisory Committee} \\ 
      {H.P. Marshall} {Ph.D.} \hspace{2cm} & {Member, Supervisory Committee} \\
      {Philippe Agard} {Ph.D.} \hspace{2cm} & {External Member, Supervisory Committee}
    \end{tabular}
  \end{flushleft}
  \begin{singlespace}
    \noindent The final reading approval of the dissertation was granted by {Matthew J. Kohn} {Ph.D.}, Chair of the Supervisory Committee. The dissertation was approved by the Graduate College.
  \end{singlespace}
  \thispagestyle{empty}
  \par\vfil\null\newpage
  \let\makesubmittalsheet\relax
}

% Make copyright page
\def\makecopyright{
  \null
  \vfill
  \begin{center}
    {$\copyright$ \number\year \par Buchanan C. Kerswell}\\
    {\sc ALL RIGHTS RESERVED}
  \end{center}
  \thispagestyle{empty}
  \let\maketitle\relax\let\makecopyright\relax
}

% End frontmatter pages

% Styling headings and table of contents to meet BSU requirements

% Chapter headings
% \chapter{} headings
\makeatletter
\titlespacing*{\chapter}{0pt}{50pt}{12pt}
\titleformat{\chapter}[block]
  {\normalfont\bfseries\centering}
  {\huge\MakeUppercase\@chapapp\space\thechapter:}
  {0pt}
  {}
  [\LARGE\MakeUppercase{#1}]
\makeatother

% \chapter{}* headings (e.g. acknowledgment, abstract, etc.)
\makeatletter
\titlespacing*{\chapter}{0pt}{50pt}{12pt}
\titleformat{name=\chapter,numberless}[block]
  {\normalfont\bfseries\centering}
  {\huge\MakeUppercase{#1}}
  {0pt}
  {}
  []
\makeatother

% Section headings
\titlespacing*{\section}{0pt}{0pt}{0pt}
\titleformat{\section}[hang]
  {\normalfont\Large\bfseries\centering}
  {\thetitle}
  {1em}
  {#1}

% Subsection headings
\titlespacing*{\subsection}{0pt}{0pt}{0pt}
\titleformat{\subsection}[hang]
  {\normalfont\large\bfseries}
  {\thetitle}
  {1em}
  {\underline{#1}}

% Table of contents style
\dottedcontents{chapter}[0em]{\large}{1em}{1pc}
\dottedcontents{section}[2em]{\large}{2em}{1pc}
\dottedcontents{subsection}[3em]{\large}{3em}{1pc}

% End styling

% Define document layout

% Reset some settings before main body
\def\begintext{
  \cleardoublepage
  \setcounter{page}{1}
  \pagenumbering{arabic}
  \pagestyle{myheadings}
    % For the special first page of a chapter:
    \fancypagestyle{plain}{
    \fancyhf{}
    \fancyhead[RO]{\hfill \thepage}
    \renewcommand\headrulewidth{0pt}
    \renewcommand\footrulewidth{0pt}
    \renewcommand\headsep{0pt}
    \renewcommand\footskip{4.5pt}
    }
}

\begin{document}

% Define a bunch of fields for makeing the title page,
% copyright page, and final approval page
  \author{Buchanan C. Kerswell}

% Title page
  \maketitle

% Copyright page
  \makecopyright

% Final approval page
\makesubmittalsheet

\setcounter{page}{4}

% Other front matter before body
% Dedication
  \chapter*{Dedication}
  \phantomsection
  \addcontentsline{toc}{chapter}{Dedication}
  \markboth{Dedication}{Dedication}
  To my mentors, colleagues, friends, and loved ones who take special interests in my life. This work is yours as much as it is mine.

% Acknowledments
  \chapter*{Acknowledgment}
  \phantomsection
  \addcontentsline{toc}{chapter}{Acknowledgment}
  \markboth{Acknowledgment}{Acknowledgment}
  This work was only possible through the efforts of many individuals. My advisor, Dr. Matthew Kohn, deserves special recognition for his contributions, mentorship, and relentless support during the course of my studies. Special thanks to my committee members, Dr. H.P. Marshall, Dr. C.J. Northrup, Dr. Philippe Agard, and Dr. Steve Utych who served as the Graduate College Representative for Boise State University. Dr. Taras Gerya and the Geophysical Fluid Dynamics group at the Institut für Geophysik, ETH Zürich, generously offered their high-performance computing resources from the Euler cluster, invaluable instruction, discussion, and support on the numerical modelling methods, and many free meals in Zürich. Additional high-performance computing support from the Borah cluster was provided by the Research Computing Department at Boise State University. Thanks to Dr. D. Hasterok for providing references and guidance on citing the large dataset in chapter three. Special thanks to Dr. Philippe Agard, Dr. Laetitia Le Pourhiet, and graduate students at Sorbonne Université for their incredible expertise and showing me the best of summertime Paris. Thanks to many anonymous reviewers, graduate students, and colleagues for helpful comments on technical aspects of each chapter. My deep appreciation of metamorphic rocks and Alpine geology was formed thanks to outstanding field excursions expertly guided by EFIRE and ZiP graduate students, faculty, and affiliates. Funding for this work was provided by the National Science Foundation grant OIA1545903 awarded to Dr. Matthew Kohn, Dr. Sarah Penniston-Dorland, and Dr. Maureen Feineman. Datasets and code for reproducing this research are available at \url{https://github.com/buchanankerswell}.

% Abstract
  \chapter*{Abstract}
  \phantomsection
  \addcontentsline{toc}{chapter}{Abstract}
  \markboth{Abstract}{Abstract}
  \Gls{ptt} estimates from \gls{hp} metamorphic rocks and global surface heat flow rates evidently encode information about pressure-temperature-strain fields deep in subduction zones. Previous work demonstrates the possibility of decoding such geodynamic information by comparing physics-based numerical models with empirical observations of surface heat flow and the metamorphic rock record. However, antithetical interpretations of (non)uniformity with respect to pressure-temperature-strain fields are emerging from this line of inquiry. For example, while
mechanical coupling depths inverted from surface heat flow are narrowly distributed among subduction zones, maximum \gls{pt} conditions inverted from exhumed metamorphic rocks are relatively wide-ranging, and yet also uniformly distributed across pressures up to 2.4 GPa. This dissertation scrutinizes (dis)similarities among subduction zones inferred from large numerical and empirical datasets by applying a variety of computational techniques. First, coupling depths for 13 modern subduction zones are predicted after observing coupling in 64 numerical
geodynamic simulations. Second, spatial patterns of surface heat flow are assessed in two-dimensions by interpolating thousands of surface heat flow observations near several subduction zone segments. Third, \gls{ptt} distributions of over one million markers traced from the previous set of 64 subduction zone simulations are compared with hundreds of empirical \gls{ptt} estimates from the rock record to assess the effects of thermo-kinematic boundary conditions on deep mechanical processing of rock in subduction zones. These studies conclude the following. Mechanical
coupling between plates is primarily controlled by the upper plate lithospheric thickness, with marginal responses to other thermo-kinematic boundary conditions. Surface heat flow interpolations show high variance within and among subduction zone segments, suggesting local, rather than widespread, continuity of pressure-temperature-strain fields deep within subduction zones. Computed marker recovery rates correlate with thermo-kinematic boundary conditions, and are therefore expected to vary among subduction zones. Finally, computed \gls{ptt} distributions
of markers show patterns consistent with transient, localized recovery from a cooling, serpentinizing plate interface. Together, this work encourages more antireductionist and diversified views of subduction geodynamics until surface heat flow and \gls{ptt} datasets can more precisely distinguish (dis)similarities in pressure-temperature-strain fields within and among subduction zones. Strategically scaling \gls{ptt} and surface heat flow datasets in the future will improve computational precision and confidence, and thus will advance subduction zone research.

% Table of contents
\tableofcontents

% List of figures
\phantomsection
\clearpage
\addcontentsline{toc}{chapter}{\listfigurename}
\markboth{\listfigurename}{\listfigurename}
\listoffigures

% List of tables
\phantomsection
\clearpage
\addcontentsline{toc}{chapter}{\listtablename}
\markboth{\listtablename}{\listtablename}
\listoftables

% List of abbreviations
  \clearpage
  \printglossary[title={List of Abbreviations},type=\acronymtype]
  \markboth{List of Abbreviations}{List of Abbreviations}

% List of symbols
  \clearpage
  \renewcommand{\nomname}{List of Symbols}
  \markboth{List of Symbols}{List of Symbols}
  \printnomenclature

% Reset settings before body
\begintext

% Body (everything in .Rmd beneath YAML)
\hypertarget{introduction}{%
\chapter{Introduction}\label{introduction}}

\begin{quote}
\textbf{Keypoints:}:

\begin{itemize}
\item
  Large proxy datasets are key for inferring information about geodynamics deep in \glspl{sz}
\item
  Computational approaches leverage large datasets to infer, build, and test models about \gls{sz} geodynamics
\end{itemize}
\end{quote}

\hypertarget{effects-of-thermo-kinetic-boundary-conditions-on-mechanical-plate-coupling-in-subduction-zones}{%
\chapter{Effects of thermo-kinetic boundary conditions on mechanical plate coupling in subduction zones}\label{effects-of-thermo-kinetic-boundary-conditions-on-mechanical-plate-coupling-in-subduction-zones}}

\begin{quote}
\textbf{Keypoints}:

\begin{itemize}
\item
  Mechanical coupling responds strongly with changes to upper plate thickness
\item
  Inverting backarc surface heat flow allows coupling depth estimation
\item
  Consistent backarc heat flow would support common coupling depths
\end{itemize}
\end{quote}

\hypertarget{a-comparison-of-heat-flow-interpolations-near-subduction-zones}{%
\chapter{A comparison of heat flow interpolations near subduction zones}\label{a-comparison-of-heat-flow-interpolations-near-subduction-zones}}

\begin{quote}
\textbf{Keypoints}:

\begin{itemize}
\item
  Inconsistent spatial patterns characterize heat flow near subduction zones
\item
  Investigations favour 2D interpolations over 1D transects
\item
  Developing composite interpolation schema and scaling datasets will improve \gls{sz} research
\end{itemize}
\end{quote}

\hypertarget{abstract}{%
\section{Abstract}\label{abstract}}

Heat fluxing through the Earth's surface provides indirect observations of \glsfirst{pts} fields deep in \glspl{sz}. Global heat flow databases, therefore, are invaluable for generating and testing belief about \gls{sz} geodynamics. Here we argue that investigating \glsfirst{shf} in two-dimensions by interpolation, rather than in one-dimension by projection, forms better interpretations about spatial continuity of deep processes. We directly compare interpolations based on the First (spatial continuity) and Third (similarity) Laws of Geography applied to the most updated global heat flow database. We observe inconsistent spatial patterns and of \gls{shf} in magnitude and variance near subduction zones, regardless of interpolation method. The implications include discontinuous \gls{pts} fields at depth, countering hypotheses of commonly thin upper plate lithospheres and mechanical coupling depths among subduction zones. Strategic scaling of \gls{shf} datasets will improve interpolation precision and confidence---leading to better tools for distinguishing differences within and among \glspl{sz}. We propose new data acquisition and composite interpolation schema as avenues for future \gls{sz} research.

\hypertarget{refs}{}
\begin{CSLReferences}{0}{0}
\end{CSLReferences}

% Bibliography
\renewcommand\bibname{REFERENCES}
\cleardoublepage
\addcontentsline{toc}{chapter}{References}
\bibliography{assets/bib/example.bib}

\end{document}

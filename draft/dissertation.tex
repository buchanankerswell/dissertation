% This is a combination of pandoc's default latex template:
% https://github.com/jgm/pandoc/blob/master/data/templates/default.latex

% and Dylan Mikesell's BSU thesis template:
% https://github.com/dylanmikesell/BSU_LaTeX_Thesis_Template/blob/master/src/BSUmain.tex

% and Dylan Mikesell's BSU style file:
% https://github.com/dylanmikesell/BSU_LaTeX_Thesis_Template/blob/master/src/BSUthesis.sty

% For BSU style requirements see:
% https://github.com/dylanmikesell/BSU_LaTeX_Thesis_Template/blob/master/src/BSU_checklist.pdf

% Pass options to packages loaded elsewhere
  \PassOptionsToPackage{dvipsnames,svgnames,x11names}{xcolor}

% Document class options
% Note: everything defined between \documentclass{} and \begin{document}
% is the "preamble"
\documentclass[
      12pt,
          twoside]{report}

% Loading packages and options

% Page geometry settings
\usepackage[
  left=1.5in,
  right=1in,
  top=1in,
  bottom=1in,
  letterpaper,
  includehead,
  includefoot,
  headheight=14.5pt
]{geometry}

% For colors
\usepackage{xcolor} % Handles colors

% For making corrections to functions
\usepackage{etoolbox}

% For strikeout and underline text
\usepackage[normalem]{ulem}

% Linespacing using setspace package
\usepackage{setspace}
\setstretch{2} % double space

% Math packages
\usepackage{amsmath,amssymb}

% Textcase for handling upper/lower case
\usepackage{textcase}

% Changepage package for changing layout in the middle of a document
\usepackage{changepage}

% For month year format
\usepackage{datetime}

% For chapter (and other) headings required by BSU
\usepackage{fancyhdr}

% Set captions to bold
\usepackage[labelfont=bf,textfont=bf]{caption}

% For making tables and final reading approval page
\usepackage{tabularx}

% For setting (sub)section heading formatting and first paragraph spacing
\usepackage[explicit]{titlesec}

% For TOC style
\usepackage{titletoc}

% Font encoding
% Defaults to 8-bit T1 encoding with 256 glyphs
% https://ctan.org/pkg/encguide
% http://www.micropress-inc.com/fonts/encoding/t1.htm
\usepackage[T1]{fontenc}
\usepackage[utf8]{inputenc}
\usepackage{textcomp} % provide euro and other symbols

% Use upquote if available, for straight quotes in verbatim environments
\IfFileExists{upquote.sty}{\usepackage{upquote}}{}
\IfFileExists{microtype.sty}{% use microtype if available
  \usepackage[]{microtype}
  \UseMicrotypeSet[protrusion]{basicmath} % disable protrusion for tt fonts
}{}

% Font family setting
  % Default to adobe times new roman with math support
  \usepackage{mathptmx}

% Allow pandoc to inject code highlighting environments

% Tables settings
\usepackage{longtable,booktabs,array}
\usepackage{multirow}
\usepackage{calc} % for calculating minipage widths

% Correct order of tables after \paragraph or \subparagraph
\makeatletter
\patchcmd\longtable{\par}{\if@noskipsec\mbox{}\fi\par}{}{}
\makeatother

% Block quote shaded style
\usepackage{framed}
\AtBeginEnvironment{quote}{\par\singlespacing\small}
\let\oldquote=\quote
\let\endoldquote=\endquote
\colorlet{shadecolor}{gray!15}
\renewenvironment{quote}{\begin{shaded*}\begin{oldquote}}{\end{oldquote}\end{shaded*}}

% Allow footnotes in longtable head/foot
\IfFileExists{footnotehyper.sty}{\usepackage{footnotehyper}}{\usepackage{footnote}}
\makesavenoteenv{longtable}

% Graphics settings
\usepackage{graphicx}
\makeatletter
\def\maxwidth{\ifdim\Gin@nat@width>\linewidth\linewidth\else\Gin@nat@width\fi}
\def\maxheight{\ifdim\Gin@nat@height>\textheight\textheight\else\Gin@nat@height\fi}
\makeatother

% Scale images if necessary, so that they will not overflow the page
% margins by default, and it is still possible to overwrite the defaults
% using explicit options in \includegraphics[width, height, ...]{}
\setkeys{Gin}{width=\maxwidth,height=\maxheight,keepaspectratio}
% Set default figure placement to htbp
\makeatletter
\def\fps@figure{htbp}
\makeatother

% Prevent overfull lines
\setlength{\emergencystretch}{3em}
\providecommand{\tightlist}{\setlength{\itemsep}{0pt}\setlength{\parskip}{0pt}}

% Csl environment (required by pandoc)
  \newlength{\cslhangindent}
  \setlength{\cslhangindent}{1.5em}
  \newlength{\csllabelwidth}
  \setlength{\csllabelwidth}{3em}
  \newlength{\cslentryspacingunit} % times entry-spacing
  \setlength{\cslentryspacingunit}{\parskip}
  \newenvironment{CSLReferences}[2] % #1 hanging-ident, #2 entry spacing
   {% don't indent paragraphs
    \setlength{\parindent}{0pt}
    % turn on hanging indent if param 1 is 1
    \ifodd #1
    \let\oldpar\par
    \def\par{\hangindent=\cslhangindent\oldpar}
    \fi
    % set entry spacing
    \setlength{\parskip}{#2\cslentryspacingunit}
   }%
   {}
  \usepackage{calc}
  \newcommand{\CSLBlock}[1]{#1\hfill\break}
  \newcommand{\CSLLeftMargin}[1]{\parbox[t]{\csllabelwidth}{#1}}
  \newcommand{\CSLRightInline}[1]{\parbox[t]{\linewidth - \csllabelwidth}{#1}\break}
  \newcommand{\CSLIndent}[1]{\hspace{\cslhangindent}#1}

% Expand header includes

% Bibliography settings
% Natbib settings
  \usepackage[]{natbib}
  \bibliographystyle{assets/bib/authordate1}

% Nocite

% Some options for hyperlinks
\usepackage[bookmarks=true,pageanchor=false]{hyperref}
\hypersetup{
      colorlinks=true,
    linkcolor={Brown},
    filecolor={Brown},
    citecolor={CornflowerBlue},
    urlcolor={Blue},
  }
\usepackage{xurl} % add URL line breaks if available
\usepackage{bookmark}
\urlstyle{same} % disable monospaced font for URLs

% Abbreviations and acronyms
\usepackage[nonumberlist,acronym,toc]{glossaries-extra}
% http://ctan.mirrors.hoobly.com/macros/latex/contrib/glossaries/glossariesbegin.pdf
\setabbreviationstyle[acronym]{long-short} % glossaries-extra.sty only
% For abbreviations
  \makeglossaries
  \loadglsentries{assets/tex/abbreviations}

% Nomenclature
\usepackage[noprefix,intoc]{nomencl}
% For symbols and nomenclature 
  \makenomenclature
  \nomenclature{$^{\circ}C$}{Celcius}
\nomenclature{$Ma$}{\textit{Mega annum} or million-years}
\nomenclature{$GPa$}{Gigapascal}
\nomenclature{$K$}{Kelvin}
\nomenclature{$wt.\%$}{weight percent}
\nomenclature{$km$}{kilometer}
\nomenclature{$\vec{q}$}{surface heat flow}
\nomenclature{$\Phi$}{Thermal parameter}
\nomenclature{$\vec{v}_{conv}$}{convergence velocity}
\nomenclature{$t_{OP}$}{oceanic plate age}
\nomenclature{$Z_{UP}$}{Upper plate thickness}
\nomenclature{$Z_{cpl}$}{Mechanical coupling depth}
\nomenclature{$\eta$}{viscosity}


%% End packages and options

% Frontmatter pages (title, approval, copyright)

% Make title page
  \title{Computational Approaches to Understanding Surface Heat Flow, the Metamorphic Rock Record, and Subduction Geodynamics}

% Change title of contents name
\renewcommand{\contentsname}{Table of contents}

\def\maketitle{
  \cleardoublepage
  \begin{titlepage}
    \pagenumbering{roman}
    \begin{center}
        % Title
        {\huge Computational Approaches to Understanding Surface Heat Flow, the Metamorphic Rock Record, and Subduction Geodynamics \par}
        \vspace*{0.5in}

        % Author
        {by\\}
        {Buchanan C. Kerswell}
        \vspace*{1in}

        % Description
        A dissertation\\
        submitted in partial fulfillment \\
        of the requirements for the degree of\\
        Doctor of Philosophy~in~Geosciences\\
        Boise State University
        \vspace*{0.5in}

        % Date
        November 2021
    \end{center}
  \end{titlepage}
  \let\maketitle\relax
}

% Make final reading approval page
\def\makesubmittalsheet{
  \cleardoublepage 
  \begin{center}
    BOISE STATE UNIVERSITY GRADUATE COLLEGE\\
    \vspace{\baselineskip}
    \textbf{DEFENSE COMMITTEE AND FINAL READING APPROVALS}\\
    \vspace{\baselineskip}
    of the dissertation submitted by\\
    \vspace{\baselineskip}
    {Buchanan C. Kerswell}\\
    \vspace{\baselineskip}
  \end{center}
  \begin{flushleft}
    \begin{singlespace}
      \begin{tabularx}{\textwidth}{@{}lX} 
        Dissertation Title: & {Computational Approaches to Understanding Surface Heat Flow, the Metamorphic Rock Record, and Subduction Geodynamics}
      \end{tabularx}
    \end{singlespace}
    \begin{tabularx}{\textwidth}{@{}lX} 
      Date of Final Oral Examination: & {August 27, 2021}
    \end{tabularx}
  \end{flushleft}
  \begin{singlespace}
    \noindent The following individuals read and discussed the dissertation submitted by student {Buchanan C. Kerswell}, and they evaluated the student’s presentation and response to questions during the final oral examination. They found that the student passed the final oral examination.\\
  \end{singlespace}
  \begin{flushleft}
    \begin{tabular}{@{}ll} 
      {Matthew J. Kohn} {Ph.D.} \hspace{2cm} & {Chair, Supervisory Committee} \\ 
      {C.J. Northrup} {Ph.D.} \hspace{2cm} & {Member, Supervisory Committee} \\ 
      {H.P. Marshall} {Ph.D.} \hspace{2cm} & {Member, Supervisory Committee} \\
      {Philippe Agard} {Ph.D.} \hspace{2cm} & {External Member, Supervisory Committee}
    \end{tabular}
  \end{flushleft}
  \begin{singlespace}
    \noindent The final reading approval of the dissertation was granted by {Matthew J. Kohn} {Ph.D.}, Chair of the Supervisory Committee. The dissertation was approved by the Graduate College.
  \end{singlespace}
  \thispagestyle{empty}
  \par\vfil\null\newpage
  \let\makesubmittalsheet\relax
}

% Make copyright page
\def\makecopyright{
  \null
  \vfill
  \begin{center}
    {$\copyright$ \number\year \par Buchanan C. Kerswell}\\
    {\sc ALL RIGHTS RESERVED}
  \end{center}
  \thispagestyle{empty}
  \let\maketitle\relax\let\makecopyright\relax
}

% End frontmatter pages

% Styling headings and table of contents to meet BSU requirements

% Chapter headings
% \chapter{} headings
\makeatletter
\titlespacing*{\chapter}{0pt}{50pt}{12pt}
\titleformat{\chapter}[block]
  {\normalfont\bfseries\centering}
  {\huge\MakeUppercase\@chapapp\space\thechapter:}
  {0pt}
  {}
  [\LARGE\MakeUppercase{#1}]
\makeatother

% \chapter{}* headings (e.g. acknowledgment, abstract, etc.)
\makeatletter
\titlespacing*{\chapter}{0pt}{50pt}{12pt}
\titleformat{name=\chapter,numberless}[block]
  {\normalfont\bfseries\centering}
  {\huge\MakeUppercase{#1}}
  {0pt}
  {}
  []
\makeatother

% Section headings
\titlespacing*{\section}{0pt}{0pt}{0pt}
\titleformat{\section}[hang]
  {\normalfont\Large\bfseries\centering}
  {\thetitle}
  {1em}
  {#1}

% Subsection headings
\titlespacing*{\subsection}{0pt}{0pt}{0pt}
\titleformat{\subsection}[hang]
  {\normalfont\large\bfseries}
  {\thetitle}
  {1em}
  {\underline{#1}}

% Table of contents style
\dottedcontents{chapter}[0em]{\large}{1em}{1pc}
\dottedcontents{section}[2em]{\large}{2em}{1pc}
\dottedcontents{subsection}[3em]{\large}{3em}{1pc}

% End styling

% Define document layout

% Reset some settings before main body
\def\begintext{
  \cleardoublepage
  \setcounter{page}{1}
  \pagenumbering{arabic}
  \pagestyle{myheadings}
    % For the special first page of a chapter:
    \fancypagestyle{plain}{
    \fancyhf{}
    \fancyhead[RO]{\hfill \thepage}
    \renewcommand\headrulewidth{0pt}
    \renewcommand\footrulewidth{0pt}
    \renewcommand\headsep{0pt}
    \renewcommand\footskip{4.5pt}
    }
}

\begin{document}

% Define a bunch of fields for makeing the title page,
% copyright page, and final approval page
  \author{Buchanan C. Kerswell}

% Title page
  \maketitle

% Copyright page
  \makecopyright

% Final approval page
\makesubmittalsheet

\setcounter{page}{4}

% Other front matter before body
% Dedication
  \chapter*{Dedication}
  \phantomsection
  \addcontentsline{toc}{chapter}{Dedication}
  \markboth{Dedication}{Dedication}
  To my mentors, colleagues, friends, and loved ones who take special interests in my life. This work is yours as much as it is mine.

% Acknowledments
  \chapter*{Acknowledgment}
  \phantomsection
  \addcontentsline{toc}{chapter}{Acknowledgment}
  \markboth{Acknowledgment}{Acknowledgment}
  This work was only possible through the efforts of many individuals. My advisor, Dr. Matthew Kohn, deserves special recognition for his contributions, mentorship, and relentless support during the course of my studies. Special thanks to my committee members, Dr. H.P. Marshall, Dr. C.J. Northrup, Dr. Philippe Agard, and Dr. Steve Utych who served as the Graduate College Representative for Boise State University. Dr. Taras Gerya and the Geophysical Fluid Dynamics group at the Institut für Geophysik, ETH Zürich, generously offered their high-performance computing resources from the Euler cluster, invaluable instruction, discussion, and support on the numerical modelling methods, and many free meals in Zürich. Additional high-performance computing support from the Borah cluster was provided by the Research Computing Department at Boise State University. Thanks to Dr. D. Hasterok for providing references and guidance on citing the large dataset in chapter three. Special thanks to Dr. Philippe Agard, Dr. Laetitia Le Pourhiet, and graduate students at Sorbonne Université for their incredible expertise and showing me the best of summertime Paris. Thanks to many anonymous reviewers, graduate students, and colleagues for helpful comments on technical aspects of each chapter. My deep appreciation of metamorphic rocks and Alpine geology was formed thanks to outstanding field excursions expertly guided by EFIRE and ZiP graduate students, faculty, and affiliates. Funding for this work was provided by the National Science Foundation grant OIA1545903 awarded to Dr. Matthew Kohn, Dr. Sarah Penniston-Dorland, and Dr. Maureen Feineman. Datasets and code for reproducing this research are available at \url{https://github.com/buchanankerswell}.

% Abstract
  \chapter*{Abstract}
  \phantomsection
  \addcontentsline{toc}{chapter}{Abstract}
  \markboth{Abstract}{Abstract}
  \Gls{ptt} estimates from \gls{hp} metamorphic rocks and global \gls{shf} rates evidently encode information about \gls{pts} fields deep in \glspl{sz}. Previous work demonstrates the possibility of decoding such geodynamic information by comparing physics-based numerical models with empirical observations of \gls{shf} and the metamorphic rock record. However, antithetical interpretations of (non)uniformity with respect to \gls{pts} fields are emerging from this line of inquiry. For example, while
mechanical coupling depths inverted from \gls{shf} are narrowly distributed among \glspl{sz}, maximum \gls{pt} conditions inverted from exhumed metamorphic rocks are relatively wide-ranging, and yet also uniformly distributed across pressures up to 2.4 GPa. This dissertation scrutinizes (dis)similarities among \glspl{sz} inferred from large numerical and empirical datasets by applying a variety of computational techniques. First, coupling depths for 13 modern \glspl{sz} are predicted after observing coupling in 64 numerical
geodynamic simulations. Second, spatial patterns of \gls{shf} are assessed in two-dimensions by interpolating thousands of \gls{shf} observations near several \gls{sz} segments. Third, \gls{ptt} distributions of over one million markers traced from the previous set of 64 \gls{sz} simulations are compared with hundreds of empirical \gls{ptt} estimates from the rock record to assess the effects of \gls{tkbc} on deep mechanical processing of rock in \glspl{sz}. These studies conclude the following. Mechanical
coupling between plates is primarily controlled by the upper plate lithospheric thickness, with marginal responses to other \gls{tkbc}. \Gls{shf} interpolations show high variance within and among \gls{sz} segments, suggesting local, rather than widespread, continuity of \gls{pts} fields deep within \glspl{sz}. Computed marker recovery rates correlate with \gls{tkbc}, and are therefore expected to vary among \glspl{sz}. Finally, computed \gls{ptt} distributions
of markers show patterns consistent with transient, localized recovery from a cooling, serpentinizing plate interface. Together, this work encourages more antireductionist and diversified views of subduction geodynamics until \gls{shf} and \gls{ptt} datasets can more precisely distinguish (dis)similarities in \gls{pts} fields within and among \glspl{sz}. Strategically scaling \gls{ptt} and \gls{shf} datasets in the future will improve computational precision and confidence, and thus will advance subduction zone research.

% Table of contents
\tableofcontents

% List of figures
\phantomsection
\clearpage
\addcontentsline{toc}{chapter}{\listfigurename}
\markboth{\listfigurename}{\listfigurename}
\listoffigures

% List of tables
\phantomsection
\clearpage
\addcontentsline{toc}{chapter}{\listtablename}
\markboth{\listtablename}{\listtablename}
\listoftables

% List of abbreviations
  \phantomsection
  \clearpage
  \printglossary[title={List of Abbreviations},type=\acronymtype]
  \markboth{List of Abbreviations}{List of Abbreviations}

% List of symbols
  \phantomsection
  \renewcommand{\nomname}{List of Symbols}
  \clearpage
  \markboth{\nomname}{\nomname}
  \printnomenclature

% Reset settings before body
\begintext

% Body (everything in .Rmd beneath YAML)
\hypertarget{introduction}{%
\chapter{Introduction}\label{introduction}}

\markboth{Chapter 1: Introduction}{Chapter 1: Introduction}

\begin{quote}
\textbf{Keypoints:}

\begin{itemize}
\item
  Proxy datasets are key for inference about geodynamics deep in \glspl{sz}
\item
  Computation leverages large data to infer, build, and test geodynamic models
\end{itemize}
\end{quote}

\cleardoublepage

\hypertarget{effects-of-thermo-kinetic-boundary-conditions-on-mechanical-plate-coupling-in-subduction-zones}{%
\chapter{Effects of thermo-kinetic boundary conditions on mechanical plate coupling in subduction zones}\label{effects-of-thermo-kinetic-boundary-conditions-on-mechanical-plate-coupling-in-subduction-zones}}

\markboth{Chapter 2: Coupling Depths}{Chapter 2: Coupling Depths}

\begin{quote}
\textbf{Keypoints:}

\begin{itemize}
\item
  Mechanical coupling responds strongly to \gls{upt}
\item
  Inverting \glsfirst{shf} allows \glsfirst{cd} estimation
\item
  Globally consistent \gls{upt} would support globally uniform \glspl{cd}
\end{itemize}
\end{quote}

\hypertarget{abstract}{%
\section{Abstract}\label{abstract}}

Deep mechanical coupling between converging plates is a key feature of \glsfirst{sz} geodynamics. Onset of coupling likely corresponds with metamorphic dehydration reactions, and therefore, is dependent on \glsfirst{pts} fields within \glspl{sz}. Here we consider the effects of changing \glsfirst{tkbc} on coupling using two-dimensional numerical models of oceanic-continental convergent margins. We focus specifically on responses to thermal parameter (\(\Phi\)) and \glsfirst{upt}. Coupling is implemented numerically by including experimentally-determined (de)hydration reactions of antigorite and olivine. During the experiments, we observe thermal feedbacks regulating (de)hydration self-consistently within the mantle wedge---stabilizing coupling after ca. 5 \(Ma\). \glspl{cd} respond strongly to \gls{upt} and weakly to \(\Phi\). Regression of our results allows \glsfirst{cd} estimation for modern \gls{sz} segments by inverting \gls{upt} from \glsfirst{shf}. We consider the implications for common \glspl{cd} among \glspl{sz}, which would require consistent \gls{upt}, and thus, globally consistent \gls{shf} in the backarc region.

\hypertarget{introduction-1}{%
\section{Introduction}\label{introduction-1}}

Deep subduction geodynamics strongly depend on the depth where the subducting plate and overlying mantle transition from mechanically decoupled (moving differentially with respect to each other) to mechanically coupled \citep[moving with the same local velocity,][]{Furukawa1993, Peacock1994, Wada2008}. Traction forces drive mantle wedge circulation at the (de)coupling transition, defining a rapid increase in temperature along the top of the subducting plate \citep{Peacock1996}. Many observations from numerical experiments and \gls{shf} infer \glspl{cd} occurring globally at 70-80 \(km\) in modern \glspl{sz}, essentially independent of other \gls{tkbc} including oceanic-plate age (\(t_{OP}\)), convergence velocity (\(\vec{v}_{conv}\)), and subduction geometry \citep{Furukawa1993, Wada2008, Wada2009}. It is significant and curious why modern subduction zones appear to achieve similar depths of coupling despite their different physical characteristics.

Notwithstanding, many numerical geodynamic models use \glspl{cd} of 70-80 \(km\) as a boundary condition \citep[e.g.,][]{Abers2017, Currie2004, Syracuse2010, VanKeken2011, VanKeken2018, Wada2012, Gao2014, Wilson2014}, although not exclusively \citep[e.g.~40-56 \(km\),][]{England2010, Peacock1996}. Similar \glspl{cd} among \glspl{sz} is an attractive hypothesis for at least two reasons: 1) it helps explain the relatively narrow range of sub-arc slab depths \citep{England2004, Syracuse2006}, as mechanical coupling is expected to be closely associated with the onset of flux melting, and 2) since mechanical coupling is required to detach and recover rocks from the subducting plate \citep{Agard2016}, a common depth of coupling may also help explain why the maximum pressures recorded by subducted oceanic material worldwide is ca. 2.3-2.5 \(GPa\) \citep[roughly 80 \(km\),][]{Agard2009}.

The location and extent of mechanical coupling along the plate interface is implicated in myriad geodynamic phenomena \citep[seismicity, metamorphism, volatile fluxes into the mantle wedge, volcanism, and plate motions, e.g.,][]{Cizkova2013, Gonzalez2016, Peacock1990, Peacock1991, Peacock1993, Peacock1996, Peacock1999a, Hacker2003, VanKeken2011, Grove2012, Gao2017}. Consequently, the mechanics of coupling have been extensively studied and discussed. Coupling fundamentally depends on the strength (\(\eta\); viscosity) of materials above, within, and below the plate interface. In general, high water fluxes due to compaction and dehydration of clays and other hydrous minerals in the shallow forearc mantle wedge, coupled with increases in \gls{pt}, form layers of low viscosity sheet silicates---especially talc and serpentine---that inhibit transmission of shear stress from the slab to the mantle wedge \citep{Peacock1999a}. The lack of traction along the interface combined with cooling from the subducting plate surface ensures the shallow mantle wedge remains cold and rigid. Experimentally determined flow laws \citep[e.g.,][]{Agard2016}, petrologic observations \citep[e.g.,][]{Agard2016}, and geophysical observations \citep[e.g.,][]{Gao2014, Peacock1999a} all support the plausibility of this conceptual model of subduction interface behaviour.

This chapter focuses on two fundamental questions: 1) how does mechanical \gls{cd} respond to \gls{tkbc}, and 2) how stable is mechanical \gls{cd} through time? We use two-dimensional numerical geodynamic models of subduction to investigate potential correlations between \gls{cd}, \gls{upt} (inverted from backarc heat flow), and the \(\Phi\). \citet{Wada2009} previously investigated steady-state slab-mantle \glspl{cd} by modelling 17 active subduction zones. Among other parameters, their models specify convergence rate, subduction geometry, thermal structure of incoming and overriding plate, and degree of coupling along the subduction interface. Their experiments control for interface rheology and discriminate the best-fit depth based on observed fore-arc heat flow. In our models, we similarly specify \gls{tkbc} to simulate the range of modern \gls{sz} systems. However, subduction dip angle and, most importantly, interface rheology are regulated self-consistently by evolving \gls{pts} fields in a deforming mantle wedge. \gls{cd} in each of our models is not a fully determined feature, therefore, but rather a spontaneous model outcome within the range of boundary conditions. As in other previous studies \citep[e.g.,][]{Ruh2015}, we include the rheological effect of the dehydration reaction \(antigorite \allowbreak \Leftrightarrow olivine + orthopyroxene + H_{2}O\), which drives mechanical coupling by an abrupt viscosity increase with antigorite loss. The position of this reaction along the subduction interface determines the \gls{cd}.

In this study we simulate subduction and observe mechanical plate coupling for 64 convergent margins with variable \gls{upt} and \(\Phi\). We quantify \gls{cd} responses to \(\Phi\) and \gls{upt} using multi-variate linear regression. We then visualize thermal feedbacks within the system in terms of mantle temperature, viscosity, and velocity fields. Lastly, we discuss how feedbacks stabilize \glspl{cd} through millions of years of active subduction.

\cleardoublepage

\hypertarget{a-comparison-of-heat-flow-interpolations-near-subduction-zones}{%
\chapter{A comparison of heat flow interpolations near subduction zones}\label{a-comparison-of-heat-flow-interpolations-near-subduction-zones}}

\markboth{Chapter 3: Heat Flow Interpolations}{Chapter 2: Heat Flow Interpolations}

\begin{quote}
\textbf{Keypoints:}

\begin{itemize}
\item
  Inconsistent spatial patterns characterize heat flow near subduction zones
\item
  Heat flow investigations favour 2D interpolations over 1D transects
\item
  Scaling datasets and new interpolation schema will advance \gls{sz} research
\end{itemize}
\end{quote}

\hypertarget{abstract-1}{%
\section{Abstract}\label{abstract-1}}

Heat fluxing through the Earth's surface provides indirect observations of \glsfirst{pts} fields deep in \glspl{sz}. Global heat flow databases, therefore, are invaluable for generating and testing belief about \gls{sz} geodynamics. Here we argue that investigating \glsfirst{shf} in two-dimensions by interpolation, rather than in one-dimension by projection, forms better interpretations about spatial continuity of deep processes. We directly compare interpolations based on the First (spatial continuity) and Third (similarity) Laws of Geography applied to the most updated global heat flow database. We observe inconsistent spatial patterns and of \gls{shf} in magnitude and variance near subduction zones, regardless of interpolation method. The implications include discontinuous \gls{pts} fields at depth, countering hypotheses of commonly thin upper plate lithospheres and mechanical \glspl{cd} among subduction zones. Strategic scaling of \gls{shf} datasets will improve interpolation precision and confidence---leading to better tools for distinguishing differences within and among \glspl{sz}. We propose new data acquisition and composite interpolation schema as avenues for future \gls{sz} research.

\cleardoublepage

\hypertarget{refs}{}
\begin{CSLReferences}{0}{0}
\end{CSLReferences}

\markboth{References}{References}

% Bibliography
\renewcommand\bibname{REFERENCES}
\phantomsection
\cleardoublepage
\addcontentsline{toc}{chapter}{References}
\bibliography{assets/bib/chpt2ref.bib}

\end{document}

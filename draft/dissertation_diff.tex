% This is a combination of pandoc's default latex template:
%DIF LATEXDIFF DIFFERENCE FILE
%DIF DEL dissertation_v1-2-2.tex   Fri Mar 11 16:22:41 2022
%DIF ADD dissertation_v1-2-3.tex   Mon Mar 14 14:03:53 2022
% https://github.com/jgm/pandoc/blob/master/data/templates/default.latex

% and Dylan Mikesell's BSU thesis template:
% https://github.com/dylanmikesell/BSU_LaTeX_Thesis_Template/blob/master/src/BSUmain.tex

% and Dylan Mikesell's BSU style file:
% https://github.com/dylanmikesell/BSU_LaTeX_Thesis_Template/blob/master/src/BSUthesis.sty

% For BSU style requirements see:
% https://github.com/dylanmikesell/BSU_LaTeX_Thesis_Template/blob/master/src/BSU_checklist.pdf

% Pass options to packages loaded elsewhere
  \PassOptionsToPackage{dvipsnames,svgnames,x11names}{xcolor}

% Document class options
% Note: everything defined between \documentclass{} and \begin{document}
% is the "preamble"
\documentclass[
      12pt,
          twoside]{report}

% Loading packages and options

% Page geometry settings
\usepackage[
  left=1.5in,
  right=1in,
  top=1in,
  bottom=1in,
  letterpaper,
  includehead,
  includefoot,
  headheight=14.5pt
]{geometry}

% For colors
\usepackage[table]{xcolor} % Handles colors

% For better hyphenations
\usepackage{soulutf8}

% For landscape pages
\usepackage{pdflscape}
\newcommand{\blandscape}{\begin{landscape}}
\newcommand{\elandscape}{\end{landscape}}

% For making corrections to functions
\usepackage{etoolbox}

% For strikeout and underline text
\usepackage[normalem]{ulem}

% Linespacing using setspace package
\usepackage{setspace}
\setstretch{2} % double space

% Math packages
\usepackage{amsmath,amssymb}

% Textcase for handling upper/lower case
\usepackage{textcase}

% Changepage package for changing layout in the middle of a document
\usepackage{changepage}

% For month year format
\usepackage{datetime}

% For chapter (and other) headings required by BSU
\usepackage{fancyhdr}

% Set captions styling
\usepackage[labelfont=bf,textfont=bf]{caption}
\captionsetup[figure]{
  font={
    stretch=0.6,
          small
      }
}

% For better float environments
\usepackage{float}

% For making tables and final reading approval page
\usepackage{tabularx}

% For setting (sub)section heading formatting and first paragraph spacing
\usepackage[explicit]{titlesec}

% For TOC style
\usepackage{titletoc}

% Font encoding
% Defaults to 8-bit T1 encoding with 256 glyphs
% https://ctan.org/pkg/encguide
% http://www.micropress-inc.com/fonts/encoding/t1.htm
\usepackage[T1]{fontenc}
\usepackage[utf8]{inputenc}
% Font family setting
  \usepackage{kpfonts}

% Line numbers
  \usepackage{lineno}

% Use upquote if available, for straight quotes in verbatim environments
\IfFileExists{upquote.sty}{\usepackage{upquote}}{}
\IfFileExists{microtype.sty}{% use microtype if available
  \usepackage[]{microtype}
  \UseMicrotypeSet[protrusion]{basicmath} % disable protrusion for tt fonts
}{}

% Allow pandoc to inject code highlighting environments

% Tables settings
\usepackage{booktabs,array,threeparttable}
\usepackage{multirow}
\usepackage{calc} % for calculating minipage widths

% Correct order of tables after \paragraph or \subparagraph
%\makeatletter
%\patchcmd\longtable{\par}{\if@noskipsec\mbox{}\fi\par}{}{}
%\makeatother

% Block quote shaded style
\usepackage{framed}
\AtBeginEnvironment{quote}{\par\singlespacing\small}
\let\oldquote=\quote
\let\endoldquote=\endquote
\colorlet{shadecolor}{gray!15}
\renewenvironment{quote}{\begin{shaded*}\begin{oldquote}}{\end{oldquote}\end{shaded*}}

% Allow footnotes in longtable head/foot
%\IfFileExists{footnotehyper.sty}{\usepackage{footnotehyper}}{\usepackage{footnote}}
%\makesavenoteenv{longtable}

% Graphics settings
\usepackage{graphicx}
\makeatletter
\def\maxwidth{\ifdim\Gin@nat@width>\linewidth\linewidth\else\Gin@nat@width\fi}
\def\maxheight{\ifdim\Gin@nat@height>\textheight\textheight\else\Gin@nat@height\fi}
\makeatother

% Scale images if necessary, so that they will not overflow the page
% margins by default, and it is still possible to overwrite the defaults
% using explicit options in \includegraphics[width, height, ...]{}
\setkeys{Gin}{width=\maxwidth,height=\maxheight,keepaspectratio}
% Set default figure placement to htbp
\makeatletter
\def\fps@figure{htbp}
\makeatother

% Prevent overfull lines
\setlength{\emergencystretch}{3em}
\providecommand{\tightlist}{\setlength{\itemsep}{0pt}\setlength{\parskip}{0pt}}

% Csl environment (required by pandoc)
  \newlength{\cslhangindent}
  \setlength{\cslhangindent}{1.5em}
  \newlength{\csllabelwidth}
  \setlength{\csllabelwidth}{3em}
  \newlength{\cslentryspacingunit} % times entry-spacing
  \setlength{\cslentryspacingunit}{\parskip}
  \newenvironment{CSLReferences}[2] % #1 hanging-ident, #2 entry spacing
   {% don't indent paragraphs
    \setlength{\parindent}{0pt}
    % turn on hanging indent if param 1 is 1
    \ifodd #1
    \let\oldpar\par
    \def\par{\hangindent=\cslhangindent\oldpar}
    \fi
    % set entry spacing
    \setlength{\parskip}{#2\cslentryspacingunit}
   }%
   {}
  \usepackage{calc}
  \newcommand{\CSLBlock}[1]{#1\hfill\break}
  \newcommand{\CSLLeftMargin}[1]{\parbox[t]{\csllabelwidth}{#1}}
  \newcommand{\CSLRightInline}[1]{\parbox[t]{\linewidth - \csllabelwidth}{#1}\break}
  \newcommand{\CSLIndent}[1]{\hspace{\cslhangindent}#1}

% Expand header includes
  \usepackage{booktabs}
  \usepackage{longtable}
  \usepackage{array}
  \usepackage{multirow}
  \usepackage{wrapfig}
  \usepackage{float}
  \usepackage{colortbl}
  \usepackage{pdflscape}
  \usepackage{tabu}
  \usepackage{threeparttable}
  \usepackage{threeparttablex}
  \usepackage[normalem]{ulem}
  \usepackage{makecell}
  \usepackage{xcolor}

% Bibliography settings
% Natbib settings

% Nocite

% Some options for hyperlinks
\usepackage[bookmarks=true,pageanchor=false]{hyperref}
\hypersetup{
      colorlinks=true,
    linkcolor={Brown},
    filecolor={Brown},
    citecolor={CornflowerBlue},
    urlcolor={Blue}
  }
\usepackage{xurl} % add URL line breaks if available
\usepackage{bookmark}
\urlstyle{same} % disable monospaced font for URLs

% Abbreviations and acronyms
\usepackage[nonumberlist,acronym,toc]{glossaries-extra}
% http://ctan.mirrors.hoobly.com/macros/latex/contrib/glossaries/glossariesbegin.pdf
\setabbreviationstyle[acronym]{long-short} % glossaries-extra.sty only
% For abbreviations
  \makeglossaries
  \loadglsentries{assets/tex/abbreviations}

% Nomenclature
\usepackage[noprefix,intoc]{nomencl}
% For symbols and nomenclature 
  \makenomenclature
  \nomenclature{$wt.\%$}{weight percent}
\nomenclature{$\vec{q}$}{surface heat flow}
\nomenclature{$\Phi$}{thermal parameter}
\nomenclature{$\vec{v}$}{convergence velocity}
\nomenclature{$Z_{UP}$}{upper plate thickness}
\nomenclature{$Z_{cpl}$}{mechanical coupling depth}
\nomenclature{$\eta$}{viscosity}
\nomenclature{$\rho$}{density}
\nomenclature{$H_2O$}{water (mineral-bound or liquid)}
\nomenclature{$an75$}{plagioclase composition of 75\% anorthite and 25\% albite}
\nomenclature{$A$}{material constant}
\nomenclature{$E$}{activation energy}
\nomenclature{$V$}{activation volume}
\nomenclature{$n$}{power law exponent}
\nomenclature{$\phi$}{internal friction angle}
\nomenclature{$\sigma_{crit}$}{critical stress for brittle/plastic deformation}
\nomenclature{$H$}{volumetric heat production}
\nomenclature{$k$}{thermal conductivity}
\nomenclature{$C_p$}{specific heat capacity}
\nomenclature{$\alpha$}{thermal expansivity}
\nomenclature{$\beta$}{compressibility}
\nomenclature{$R$}{gas constant}
\nomenclature{$G$}{shear modulus}
\nomenclature{$m$}{grain size exponent}
\nomenclature{$b$}{Burgers vector}
\nomenclature{$\dot{\varepsilon}$}{strain rate tensor}
\nomenclature{$\dot{\varepsilon}_{II}$}{second invariant of the strain rate tensor}
\nomenclature{$C$}{cohesive strength}
\nomenclature{$\hat{\gamma}$}{experimental variogram}
\nomenclature{$\gamma$}{variogram model}
\nomenclature{$c$}{variogram lag cutoff constant}
\nomenclature{$n_{lag}$}{number of variogram lags}
\nomenclature{$n_{max}$}{maximum number of nearby observation pairs for local Kriging}
\nomenclature{$h$}{variogram lag distance}
\nomenclature{$l$}{variogram lag shift constant}
\nomenclature{$N(h)$}{number of observation pairs separated by a lag distance $h$}
\nomenclature{$M$}{number of observations in a Kriging domain}
\nomenclature{$\delta$}{lag binwidth}
\nomenclature{$Z(u)$}{observation of a random variable at location $u$}
\nomenclature{$\hat{Z}(u)$}{estimation of a random variable at location $u$}
\nomenclature{$C(\Theta)$}{cost function with parameters $\Theta$}
\nomenclature{$C_{interp}(\Theta)$}{Kriging error with parameters $\Theta$}
\nomenclature{$C_{vgrm}(\Theta)$}{variogram error with parameters $\Theta$}
\nomenclature{$w_{vgrm}$}{variogram weight for cost function}
\nomenclature{$w_{interp}$}{Kriging weight for cost function}
\nomenclature{$n$}{variogram nugget}
\nomenclature{$s$}{variogram sill}
\nomenclature{$a$}{variogram effective range}
\nomenclature{$K_1$}{modified bessel function}
\nomenclature{$Bes$}{Bessel variogram model}
\nomenclature{$Cir$}{circular variogram model}
\nomenclature{$Exp$}{exponential variogram model}
\nomenclature{$Gau$}{Gaussian variogram model}
\nomenclature{$Lin$}{linear variogram model}
\nomenclature{$Sph$}{spherical variogram model}


%% End packages and options

% Frontmatter pages (title, approval, copyright)

% Make title page
  \title{Computational Approaches to Understanding Subduction Zone Geodynamics, Surface Heat Flow, and the Metamorphic Rock Record}

% Change title of contents name
\renewcommand{\contentsname}{Table of contents}

\def\maketitle{
  \cleardoublepage
  \begin{titlepage}
    \pagenumbering{roman}
    \begin{center}
        % Title
        {\huge Computational Approaches to Understanding Subduction Zone Geodynamics, Surface Heat Flow, and the Metamorphic Rock Record \par}
        \vspace*{0.5in}

        % Author
        {by\\}
        {Buchanan C. Kerswell}
        \vspace*{1in}

        % Description
        A dissertation\\
        submitted in partial fulfillment \\
        of the requirements for the degree of\\
        Doctor of Philosophy~in~Geosciences\\
        Boise State University
        \vspace*{0.5in}

        % Date
        May 2022
    \end{center}
  \end{titlepage}
  \let\maketitle\relax
}

% Make final reading approval page
\def\makesubmittalsheet{
  \cleardoublepage 
  \begin{center}
    BOISE STATE UNIVERSITY GRADUATE COLLEGE\\
    \vspace{\baselineskip}
    \textbf{DEFENSE COMMITTEE AND FINAL READING APPROVALS}\\
    \vspace{\baselineskip}
    of the dissertation submitted by\\
    \vspace{\baselineskip}
    {Buchanan C. Kerswell}\\
    \vspace{\baselineskip}
  \end{center}
  \begin{flushleft}
    \begin{singlespace}
      \begin{tabularx}{\textwidth}{@{}lX} 
        Dissertation Title: & {Computational Approaches to Understanding Subduction Zone Geodynamics, Surface Heat Flow, and the Metamorphic Rock Record}
      \end{tabularx}
    \end{singlespace}
    \begin{tabularx}{\textwidth}{@{}lX} 
      Date of Final Oral Examination: & {27 August 2021}
    \end{tabularx}
  \end{flushleft}
  \begin{singlespace}
    \noindent The following individuals read and discussed the dissertation submitted by student {Buchanan C. Kerswell}, and they evaluated the student’s presentation and response to questions during the final oral examination. They found that the student passed the final oral examination.\\
  \end{singlespace}
  \begin{flushleft}
    \begin{tabular}{@{}ll} 
      {Matthew J. Kohn} {Ph.D.} \hspace{2cm} & {Chair, Supervisory Committee} \\ 
      {C.J. Northrup} {Ph.D.} \hspace{2cm} & {Member, Supervisory Committee} \\ 
      {H.P. Marshall} {Ph.D.} \hspace{2cm} & {Member, Supervisory Committee} \\
      {Philippe Agard} {Ph.D.} \hspace{2cm} & {External Member, Supervisory Committee}
    \end{tabular}
  \end{flushleft}
  \begin{singlespace}
    \noindent The final reading approval of the dissertation was granted by {Matthew J. Kohn} {Ph.D.}, Chair of the Supervisory Committee. The dissertation was approved by the Graduate College.
  \end{singlespace}
  \thispagestyle{empty}
  \par\vfil\null\newpage
  \let\makesubmittalsheet\relax
}

% Make copyright page
\def\makecopyright{
  \null
  \vfill
  \begin{center}
    {$\copyright$ \number\year \par Buchanan C. Kerswell}\\
    {\sc ALL RIGHTS RESERVED}
  \end{center}
  \thispagestyle{empty}
  \let\maketitle\relax\let\makecopyright\relax
}

% End frontmatter pages

% Styling headings and table of contents to meet BSU requirements

% Chapter headings
% \chapter{} headings
\makeatletter
\titlespacing*{\chapter}{0pt}{50pt}{12pt}
\titleformat{\chapter}[block]
  {\normalfont\bfseries\centering}
  {\huge\MakeUppercase\@chapapp\space\thechapter:}
  {0pt}
  {}
  [\LARGE\MakeUppercase{#1}]
\makeatother

% \chapter{}* headings (e.g. acknowledgment, abstract, etc.)
\makeatletter
\titlespacing*{\chapter}{0pt}{50pt}{12pt}
\titleformat{name=\chapter,numberless}[block]
  {\normalfont\bfseries\centering}
  {\huge\MakeUppercase{#1}}
  {0pt}
  {}
  []
\makeatother

% Section headings
\titlespacing*{\section}{0pt}{0pt}{0pt}
\titleformat{\section}[hang]
  {\normalfont\Large\bfseries\centering}
  {\thetitle}
  {1em}
  {#1}

% Subsection headings
\titlespacing*{\subsection}{0pt}{0pt}{0pt}
\titleformat{\subsection}[hang]
  {\normalfont\large\bfseries}
  {\thetitle}
  {1em}
  {\underline{#1}}

% Table of contents style
\dottedcontents{chapter}[0em]{}{1em}{1pc}
\dottedcontents{section}[2em]{}{2em}{1pc}
\dottedcontents{subsection}[5em]{}{3em}{1pc}

% End styling

% Define document layout

% Reset some settings before main body
\def\begintext{
  \cleardoublepage
  \setcounter{page}{1}
  \pagenumbering{arabic}
  \pagestyle{myheadings}
    % For the special first page of a chapter:
    \fancypagestyle{plain}{
    \fancyhf{}
    \fancyhead[RO]{\hfill \thepage}
    \renewcommand\headrulewidth{0pt}
    \renewcommand\footrulewidth{0pt}
    \renewcommand\headsep{0pt}
    \renewcommand\footskip{4.5pt}
    }
}

% Allow slightly different vertial spacing on pages
% useful for two-sided docs with figs, etc.
\raggedbottom
%DIF PREAMBLE EXTENSION ADDED BY LATEXDIFF
%DIF CFONT PREAMBLE %DIF PREAMBLE
\RequirePackage{color}\definecolor{RED}{rgb}{1,0,0}\definecolor{BLUE}{rgb}{0,0,1} %DIF PREAMBLE
\DeclareOldFontCommand{\sf}{\normalfont\sffamily}{\mathsf} %DIF PREAMBLE
\providecommand{\DIFaddtex}[1]{{\protect\color{blue} \sf #1}} %DIF PREAMBLE
\providecommand{\DIFdeltex}[1]{{\protect\color{red} \scriptsize #1}} %DIF PREAMBLE
%DIF SAFE PREAMBLE %DIF PREAMBLE
\providecommand{\DIFaddbegin}{} %DIF PREAMBLE
\providecommand{\DIFaddend}{} %DIF PREAMBLE
\providecommand{\DIFdelbegin}{} %DIF PREAMBLE
\providecommand{\DIFdelend}{} %DIF PREAMBLE
\providecommand{\DIFmodbegin}{} %DIF PREAMBLE
\providecommand{\DIFmodend}{} %DIF PREAMBLE
%DIF FLOATSAFE PREAMBLE %DIF PREAMBLE
\providecommand{\DIFaddFL}[1]{\DIFadd{#1}} %DIF PREAMBLE
\providecommand{\DIFdelFL}[1]{\DIFdel{#1}} %DIF PREAMBLE
\providecommand{\DIFaddbeginFL}{} %DIF PREAMBLE
\providecommand{\DIFaddendFL}{} %DIF PREAMBLE
\providecommand{\DIFdelbeginFL}{} %DIF PREAMBLE
\providecommand{\DIFdelendFL}{} %DIF PREAMBLE
%DIF HYPERREF PREAMBLE %DIF PREAMBLE
\providecommand{\DIFadd}[1]{\texorpdfstring{\DIFaddtex{#1}}{#1}} %DIF PREAMBLE
\providecommand{\DIFdel}[1]{\texorpdfstring{\DIFdeltex{#1}}{}} %DIF PREAMBLE
\newcommand{\DIFscaledelfig}{0.5}
%DIF HIGHLIGHTGRAPHICS PREAMBLE %DIF PREAMBLE
\RequirePackage{settobox} %DIF PREAMBLE
\RequirePackage{letltxmacro} %DIF PREAMBLE
\newsavebox{\DIFdelgraphicsbox} %DIF PREAMBLE
\newlength{\DIFdelgraphicswidth} %DIF PREAMBLE
\newlength{\DIFdelgraphicsheight} %DIF PREAMBLE
% store original definition of \includegraphics %DIF PREAMBLE
\LetLtxMacro{\DIFOincludegraphics}{\includegraphics} %DIF PREAMBLE
\newcommand{\DIFaddincludegraphics}[2][]{{\color{blue}\fbox{\DIFOincludegraphics[#1]{#2}}}} %DIF PREAMBLE
\newcommand{\DIFdelincludegraphics}[2][]{% %DIF PREAMBLE
\sbox{\DIFdelgraphicsbox}{\DIFOincludegraphics[#1]{#2}}% %DIF PREAMBLE
\settoboxwidth{\DIFdelgraphicswidth}{\DIFdelgraphicsbox} %DIF PREAMBLE
\settoboxtotalheight{\DIFdelgraphicsheight}{\DIFdelgraphicsbox} %DIF PREAMBLE
\scalebox{\DIFscaledelfig}{% %DIF PREAMBLE
\parbox[b]{\DIFdelgraphicswidth}{\usebox{\DIFdelgraphicsbox}\\[-\baselineskip] \rule{\DIFdelgraphicswidth}{0em}}\llap{\resizebox{\DIFdelgraphicswidth}{\DIFdelgraphicsheight}{% %DIF PREAMBLE
\setlength{\unitlength}{\DIFdelgraphicswidth}% %DIF PREAMBLE
\begin{picture}(1,1)% %DIF PREAMBLE
\thicklines\linethickness{2pt} %DIF PREAMBLE
{\color[rgb]{1,0,0}\put(0,0){\framebox(1,1){}}}% %DIF PREAMBLE
{\color[rgb]{1,0,0}\put(0,0){\line( 1,1){1}}}% %DIF PREAMBLE
{\color[rgb]{1,0,0}\put(0,1){\line(1,-1){1}}}% %DIF PREAMBLE
\end{picture}% %DIF PREAMBLE
}\hspace*{3pt}}} %DIF PREAMBLE
} %DIF PREAMBLE
\LetLtxMacro{\DIFOaddbegin}{\DIFaddbegin} %DIF PREAMBLE
\LetLtxMacro{\DIFOaddend}{\DIFaddend} %DIF PREAMBLE
\LetLtxMacro{\DIFOdelbegin}{\DIFdelbegin} %DIF PREAMBLE
\LetLtxMacro{\DIFOdelend}{\DIFdelend} %DIF PREAMBLE
\DeclareRobustCommand{\DIFaddbegin}{\DIFOaddbegin \let\includegraphics\DIFaddincludegraphics} %DIF PREAMBLE
\DeclareRobustCommand{\DIFaddend}{\DIFOaddend \let\includegraphics\DIFOincludegraphics} %DIF PREAMBLE
\DeclareRobustCommand{\DIFdelbegin}{\DIFOdelbegin \let\includegraphics\DIFdelincludegraphics} %DIF PREAMBLE
\DeclareRobustCommand{\DIFdelend}{\DIFOaddend \let\includegraphics\DIFOincludegraphics} %DIF PREAMBLE
\LetLtxMacro{\DIFOaddbeginFL}{\DIFaddbeginFL} %DIF PREAMBLE
\LetLtxMacro{\DIFOaddendFL}{\DIFaddendFL} %DIF PREAMBLE
\LetLtxMacro{\DIFOdelbeginFL}{\DIFdelbeginFL} %DIF PREAMBLE
\LetLtxMacro{\DIFOdelendFL}{\DIFdelendFL} %DIF PREAMBLE
\DeclareRobustCommand{\DIFaddbeginFL}{\DIFOaddbeginFL \let\includegraphics\DIFaddincludegraphics} %DIF PREAMBLE
\DeclareRobustCommand{\DIFaddendFL}{\DIFOaddendFL \let\includegraphics\DIFOincludegraphics} %DIF PREAMBLE
\DeclareRobustCommand{\DIFdelbeginFL}{\DIFOdelbeginFL \let\includegraphics\DIFdelincludegraphics} %DIF PREAMBLE
\DeclareRobustCommand{\DIFdelendFL}{\DIFOaddendFL \let\includegraphics\DIFOincludegraphics} %DIF PREAMBLE
%DIF COLORLISTINGS PREAMBLE %DIF PREAMBLE
\RequirePackage{listings} %DIF PREAMBLE
\RequirePackage{color} %DIF PREAMBLE
\lstdefinelanguage{DIFcode}{ %DIF PREAMBLE
%DIF DIFCODE_CFONT %DIF PREAMBLE
  moredelim=[il][\color{red}\scriptsize]{\%DIF\ <\ }, %DIF PREAMBLE
  moredelim=[il][\color{blue}\sffamily]{\%DIF\ >\ } %DIF PREAMBLE
} %DIF PREAMBLE
\lstdefinestyle{DIFverbatimstyle}{ %DIF PREAMBLE
	language=DIFcode, %DIF PREAMBLE
	basicstyle=\ttfamily, %DIF PREAMBLE
	columns=fullflexible, %DIF PREAMBLE
	keepspaces=true %DIF PREAMBLE
} %DIF PREAMBLE
\lstnewenvironment{DIFverbatim}{\lstset{style=DIFverbatimstyle}}{} %DIF PREAMBLE
\lstnewenvironment{DIFverbatim*}{\lstset{style=DIFverbatimstyle,showspaces=true}}{} %DIF PREAMBLE
%DIF END PREAMBLE EXTENSION ADDED BY LATEXDIFF

\begin{document}

% Define a bunch of fields for makeing the title page,
% copyright page, and final approval page
  \author{Buchanan C. Kerswell}

% Title page
  \maketitle

% Copyright page
  \makecopyright

% Final approval page
\makesubmittalsheet

\setcounter{page}{4}

% Other front matter before body
% Dedication
  \chapter*{Dedication}
  \phantomsection
  \addcontentsline{toc}{chapter}{Dedication}
  \markboth{Dedication}{Dedication}
  To my mentors, colleagues, friends, and loved ones who take special interests in my life. This work is yours as much as it is mine.
  \clearpage

% Acknowledments
  \chapter*{Acknowledgment}
  \phantomsection
  \addcontentsline{toc}{chapter}{Acknowledgment}
  \markboth{Acknowledgment}{Acknowledgment}
  This work was only possible through the efforts of many individuals. My advisor, Dr. Matthew Kohn, deserves special recognition for his contributions, mentorship, and relentless support during the course of my studies. Special thanks to my committee members, Dr. H.P. Marshall, Dr. C.J. Northrup, Dr. Philippe Agard, and Dr. Steve Utych who served as the Graduate College Representative for Boise State University. Dr. Taras Gerya and the Geophysical Fluid Dynamics group at the Institut für Geophysik, ETH Zürich, generously offered their high-performance computing resources from the Euler cluster, invaluable instruction, discussion, and support on the numerical modelling methods, and many free meals in Zürich. Additional high-performance computing support from the Borah cluster was provided by the Research Computing Department at Boise State University. Thanks to Dr. D. Hasterok for providing references and guidance on citing the large dataset in chapter three. Special thanks to Dr. Philippe Agard, Dr. Laetitia Le Pourhiet, and graduate students at Sorbonne Université for their incredible expertise and showing me the best of summertime Paris. Thanks to many anonymous reviewers, graduate students, and colleagues for helpful comments on technical aspects of each chapter. My deep appreciation of metamorphic rocks and Alpine geology was formed thanks to outstanding field excursions expertly guided by EFIRE and ZiP graduate students, faculty, and affiliates. Funding for this work was provided by the National Science Foundation grant OIA1545903 awarded to Dr. Matthew Kohn, Dr. Sarah Penniston-Dorland, and Dr. Maureen Feineman. Datasets and code for reproducing this research are available at \url{https://github.com/buchanankerswell}.
  \clearpage

% Abstract
  \chapter*{Abstract}
  \phantomsection
  \addcontentsline{toc}{chapter}{Abstract}
  \markboth{Abstract}{Abstract}
  \Gls{ptt} estimates from \gls{hp} metamorphic rocks and global surface heat flow rates evidently encode information about pressure-temperature-strain fields deep in subduction zones. Previous work demonstrates the possibility of decoding such geodynamic information by comparing physics-based numerical models with empirical observations of surface heat flow and the metamorphic rock record. However, antithetical interpretations of (non)uniformity with respect to pressure-temperature-strain fields are emerging from this line of inquiry. For example, while
mechanical coupling depths inverted from surface heat flow are narrowly distributed among subduction zones, maximum \gls{pt} conditions inverted from exhumed metamorphic rocks are relatively wide-ranging, and yet also uniformly distributed across pressures up to 2.4 GPa. This dissertation scrutinizes (dis)similarities among subduction zones inferred from large numerical and empirical datasets by applying a variety of computational techniques. First, coupling depths for 13 modern subduction zones are predicted after observing coupling in 64 numerical
geodynamic simulations. Second, spatial patterns of surface heat flow are assessed in two-dimensions by interpolating thousands of surface heat flow observations near several subduction zone segments. Third, \gls{ptt} distributions of over one million markers traced from the previous set of 64 subduction zone simulations are compared with hundreds of empirical \gls{ptt} estimates from the rock record to assess the effects of thermo-kinematic boundary conditions on deep mechanical processing of rock in subduction zones. These studies conclude the following. Mechanical
coupling between plates is primarily controlled by the upper plate lithospheric thickness, with marginal responses to other thermo-kinematic boundary conditions. Surface heat flow interpolations show high variance within and among subduction zone segments, suggesting local, rather than widespread, continuity of pressure-temperature-strain fields deep within subduction zones. Computed marker recovery rates correlate with thermo-kinematic boundary conditions, and are therefore expected to vary among subduction zones. Finally, computed \gls{ptt} distributions
of markers show patterns consistent with transient, localized recovery from a cooling, serpentinizing plate interface. Together, this work encourages more antireductionist and diversified views of subduction geodynamics until surface heat flow and \gls{ptt} datasets can more precisely distinguish (dis)similarities in pressure-temperature-strain fields within and among subduction zones. Strategically scaling \gls{ptt} and surface heat flow datasets in the future will improve computational precision and confidence, and thus will advance subduction zone research.
  \clearpage

% Table of contents
\tableofcontents
\clearpage

% List of figures
\phantomsection
\addcontentsline{toc}{chapter}{\listfigurename}
\markboth{\listfigurename}{\listfigurename}
\listoffigures
\clearpage

% List of tables
\phantomsection
\addcontentsline{toc}{chapter}{\listtablename}
\markboth{\listtablename}{\listtablename}
\listoftables
\clearpage

% List of abbreviations
  \phantomsection
  \printglossary[title={List of Abbreviations},type=\acronymtype]
  \markboth{List of Abbreviations}{List of Abbreviations}
  \clearpage

% List of symbols
  \phantomsection
  \renewcommand{\nomname}{List of Symbols}
  \printnomenclature
  \markboth{\nomname}{\nomname}
  \clearpage

% Reset settings before body
\begintext

% Line numbers
  \linenumbers

% Body (everything in .Rmd beneath YAML)
\hypertarget{chpt1}{%
\chapter{Introduction}\label{chpt1}}

\markboth{Chapter 1: Introduction}{Chapter 1: Introduction}

As noted by Gerya (\protect\hyperlink{ref-gerya2014}{2014}), a scarcity of observational constraints through time and space makes the study of geodynamics on Earth extraordinarily challenging (Figure \ref{fig:sparse}). Fortunately, application of various computational approaches---simulation, interpolation, and applied statistics (machine learning)---enable geodynamic inquiry despite sparse datasets. This dissertation leverages the above computational methods to investigate a fundamental component of Plate Tectonic theory, \emph{subduction}.



\begin{figure}[htbp]

{\centering \includegraphics[width=1\linewidth,]{assets/figs/chpt1/gerya2014} 

}

\caption[The Geodynamicist's dilemma]{The Geodynamicist's dilemma. Time-depth diagram representing the data availability on Earth. The rock record (red circles) encodes information about geodynamic processes throughout Earth's history, but only within approximately 100 km of Earth's surface. Geophysical data (blue circles) provide images of Earth's deep interior, but only since the 20th century CE (or 10\(^{-7}\) Ga). Direct observations (green circle) are limited to the present-day surface. Size of the circles represents the abundance of available data. Reprinted from Gerya (\protect\hyperlink{ref-gerya2014}{2014}) with permission.}\label{fig:sparse}
\end{figure}

Subduction occurs when two lithospheric plates converge and the denser plate \emph{subducts} beneath the other at a \emph{subduction zone}. Subduction zones drive many geodynamic phenomena, including plate motions, seismicity, metamorphism, volatile flux, volcanism, and crustal deformation (\protect\hyperlink{ref-cizkova2013}{Čížková \& Bina, 2013}; \protect\hyperlink{ref-gao2017}{Gao \& Wang, 2017}; \protect\hyperlink{ref-gonzalez2016}{Gonzalez et al., 2016}; \protect\hyperlink{ref-grove2012}{Grove et al., 2012}; \protect\hyperlink{ref-hacker2003}{Hacker et al., 2003}; \protect\hyperlink{ref-hirauchi2010}{Hirauchi et al., 2010}; \protect\hyperlink{ref-peacock1990}{Peacock, 1990}, \protect\hyperlink{ref-peacock1991}{1991}, \protect\hyperlink{ref-peacock1993}{1993}, \protect\hyperlink{ref-peacock1996}{1996}; \protect\hyperlink{ref-peacock1999a}{Peacock \& Hyndman, 1999}; \protect\hyperlink{ref-vankeken2011}{van Keken et al., 2011}). These phenomena are largely defined by plate motions and mechanical behavior along the interface between the subducting plate and overriding (upper) plate (\protect\hyperlink{ref-furukawa1993}{Furukawa, 1993}; \protect\hyperlink{ref-peacock1994}{Peacock et al., 1994}; \protect\hyperlink{ref-peacock1996}{Peacock, 1996}). Important thermo-kinematic boundary conditions exerting first-order control on subduction zone geodynamics (plate velocity, subduction angle, plate thickness, sediment thickness, crustal structure, subduction duration, and others) vary considerably among presently active subduction zones worldwide (e.g. \protect\hyperlink{ref-syracuse2010}{Syracuse et al., 2010}; \protect\hyperlink{ref-syracuse2006}{Syracuse \& Abers, 2006}).

Intuition suggests diverse thermo-kinematic boundary conditions for various subduction zone systems should influence mechanical behavior differently along the plate interface. Yet previous work comparing surface heat flow with numerical simulations of subduction argues for rather uniform depths of plate coupling among subduction zones (\protect\hyperlink{ref-furukawa1993}{Furukawa, 1993}; \protect\hyperlink{ref-wada2008}{Wada et al., 2008}; \protect\hyperlink{ref-wada2009}{Wada \& Wang, 2009}) and implies some aspects of subduction zone mechanics are minimally affected by thermo-kinematic boundary conditions. Compounding the ambiguity are global compilations of \gls{pt} estimates from exhumed \gls{hp} metamorphic rocks that imply detachment of subducting material is either rather continuous along the plate interface (\protect\hyperlink{ref-agard2018}{Agard et al., 2018}; \protect\hyperlink{ref-penniston2015}{Penniston-Dorland et al., 2015}) or discontinuous (\protect\hyperlink{ref-agard2009}{Agard et al., 2009}, \protect\hyperlink{ref-agard2016}{2016}; \protect\hyperlink{ref-groppo2016}{Groppo et al., 2016}; \protect\hyperlink{ref-monie2009}{Monie \& Agard, 2009}; \protect\hyperlink{ref-plunder2015}{Plunder et al., 2015}). Thus, the spatial variability (with depth \emph{and} along strike) of plate interface mechanics remains largely unconstrained and difficult to quantify.

This dissertation is motivated by the following question. How can spatial variations in plate interface mechanics be evaluated across a range of subduction zones with currently available petrologic and geophysical datasets? Each chapter focuses on quantifying an aspect of subduction zone mechanics using different computational approaches and datasets.

Chapter \ref{chpt2} numerically simulates oceanic-continental subduction across a range of thermo-kinematic boundary conditions. Plate coupling is observed after 10 Ma and multivariate linear regression is then used to formulate an expression for estimating coupling depth. The expression requires estimates for upper-plate thickness, which can be inverted from surface heat flow. Average upper-plate surface heat flow for 13 presently active subduction zones yield a narrow distribution of coupling depths.

Chapter \ref{chpt3} takes a closer look surface heat flow by quantifying its spatial variability across large adjacent regions (sectors) in the upper-plate. Two interpolations methods, Kriging and Similarity, are compared to assess differences in their surface heat flow predictions near \DIFdelbegin \DIFdel{17 }\DIFdelend \DIFaddbegin \DIFadd{13 }\DIFaddend subduction zone segments. Kriging and Similarity accuracies are comparable on average and both approaches show lateral (along strike) surface heat flow variability in the upper-plate. Discontinuous upper-plate surface heat flow implies nonuniform thermal structure and/or discontinuous geodynamics.

Finally, Chapter \ref{chpt4} applies machine learning techniques to recognize detachment of subducting markers (representing rock fragments) from the numerical simulations in Chapter \ref{chpt2}. A large (122,236) \gls{pt} dataset of recovered markers is compared across numerical experiments and with global compilation of \gls{pt} estimates for rocks exhumed from subduction zones (the rock record, \protect\hyperlink{ref-agard2018}{Agard et al., 2018}; \protect\hyperlink{ref-penniston2015}{Penniston-Dorland et al., 2015}). Marker \gls{pt} distributions are distinct from the rock record for most numerical simulations, except for slowly-converging systems (40 km/Ma) with young oceanic plates (\(\leq\) 55 Ma) and thin upper-plate lithospheres. A sizeable gap in marker recovery around 2 GPa and 550 \(^\circ\)C, closely coinciding with the highest density region of natural samples, implies certain biases may be affecting numerical geodynamic models, the rock record, or both.

\cleardoublepage

\hypertarget{chpt2}{%
\chapter{Effects of Thermo-Kinematic Boundary Conditions on Plate Coupling in Subduction Zones}\label{chpt2}}

\markboth{Chapter 2: Coupling Depths}{Chapter 2: Coupling Depths}

\hypertarget{chpt2Abstract}{%
\section{Abstract}\label{chpt2Abstract}}

Deep mechanical coupling between converging plates is implicated in dynamic plate motions, crustal deformation, seismic cycles, arc magmatism, detachment (recovery) of subducting material, and is considered a key feature of subduction zone geodynamics. This study uses two-dimensional numerical simulations of oceanic-continental convergent margins to investigate effects of thermo-kinematic boundary conditions on coupling---specifically focusing on thermal parameter (\(\Phi\)) and upper-plate thickness. Numerical simulations implement coupling by including the metamorphic (de)hydration reaction \(antigorite \allowbreak \Leftrightarrow olivine + orthopyroxene + H_{2}O\) in the upper-plate mantle. Visualizing \gls{pt}-strain fields show thermal feedbacks regulating coupling depth dynamically with strong responses to upper-plate thickness and weak responses to \(\Phi\). The results imply estimation of coupling depth is possible by inverting upper-plate thickness from surface heat flow. Moreover, surface heat flow sampled from the backarc region near 17 presently active subduction zones imply uniform upper-plate thickness, and thus uniform coupling depths among subduction zones.

\hypertarget{chpt2Intro}{%
\section{Introduction}\label{chpt2Intro}}

Subduction geodynamics are largely defined by plate motions and mechanical behavior along the plate interface. For example, a transition from mechanically decoupled (plates moving differentially with respect to each other) to coupled plates (plates moving with the same local velocity) dramatically increases temperature by inducing mantle circulation in the upper-plate asthenospheric mantle (\protect\hyperlink{ref-peacock1994}{Peacock et al., 1994}; \protect\hyperlink{ref-peacock1996}{Peacock, 1996}). Observations from numerical simulations and forearc surface heat flow imply coupling transitions occur globally within a narrow range of depths in modern subduction zones (70-80 km). Further, coupling appears essentially unresponsive to diverse thermo-kinematic boundary conditions, including oceanic plate age, convergence velocity, and subduction geometry (\protect\hyperlink{ref-furukawa1993}{Furukawa, 1993}; \protect\hyperlink{ref-wada2008}{Wada et al., 2008}; \protect\hyperlink{ref-wada2009}{Wada \& Wang, 2009}). While uniform coupling depths among subduction zones are inferred from numerical simulations and surface heat flow, this phenomenon remains curious and unconfirmed to a large extent. To understand subduction zone geodynamics, it is essential to understand why modern subduction zones appear to achieve similar coupling depths despite differences in their physical characteristics.

Notwithstanding, many numerical geodynamic models use coupling depths of 70-80 km as a boundary condition (\protect\hyperlink{ref-abers2017}{Abers et al., 2017}; \protect\hyperlink{ref-currie2004}{Currie et al., 2004}; \protect\hyperlink{ref-gao2014}{Gao \& Wang, 2014}; \protect\hyperlink{ref-syracuse2010}{Syracuse et al., 2010}; \protect\hyperlink{ref-vankeken2011}{van Keken et al., 2011}, \protect\hyperlink{ref-vankeken2018}{2018}; \protect\hyperlink{ref-wada2012}{Wada et al., 2012}; \protect\hyperlink{ref-wilson2014}{Wilson et al., 2014}), although not exclusively (e.g.~40-56 km, \protect\hyperlink{ref-england2010}{England \& Katz, 2010}; \protect\hyperlink{ref-peacock1996}{Peacock, 1996}). Similar coupling depths among subduction zones is an attractive hypothesis for at least two reasons. First, it helps explain a relatively narrow range of depths to subducting oceanic plates beneath volcanic arcs (\protect\hyperlink{ref-england2004}{England et al., 2004}; \protect\hyperlink{ref-syracuse2006}{Syracuse \& Abers, 2006}) as mechanical coupling is expected to be closely associated with the onset of flux melting. Second, mechanical coupling is required to detach crustal fragments from the subducting plate (\protect\hyperlink{ref-agard2016}{Agard et al., 2016}), so uniform coupling depths may also help explain why maximum pressures recorded by subducted oceanic material worldwide is \(\leq\) 2.3-2.5 GPa (roughly 80 km, \protect\hyperlink{ref-agard2009}{Agard et al., 2009}, \protect\hyperlink{ref-agard2018}{2018}).

The location and extent of mechanical coupling along the plate interface is implicated in myriad geodynamic phenomena, including seismicity, metamorphism, volatile flux, volcanism, plate motions, and crustal deformation (\protect\hyperlink{ref-cizkova2013}{Čížková \& Bina, 2013}; \protect\hyperlink{ref-gao2017}{Gao \& Wang, 2017}; \protect\hyperlink{ref-gonzalez2016}{Gonzalez et al., 2016}; \protect\hyperlink{ref-grove2012}{Grove et al., 2012}; \protect\hyperlink{ref-hacker2003}{Hacker et al., 2003}; \protect\hyperlink{ref-hirauchi2010}{Hirauchi et al., 2010}; \protect\hyperlink{ref-peacock1990}{Peacock, 1990}, \protect\hyperlink{ref-peacock1991}{1991}, \protect\hyperlink{ref-peacock1993}{1993}, \protect\hyperlink{ref-peacock1996}{1996}; \protect\hyperlink{ref-peacock1999a}{Peacock \& Hyndman, 1999}; \protect\hyperlink{ref-vankeken2011}{van Keken et al., 2011}). Consequently, the mechanics of coupling have been extensively studied and discussed. Coupling fundamentally depends on the strength (viscosity) of materials above, within, and below the plate interface. Water flux from compaction and dehydration of hydrous minerals with increasing \gls{pt} forms layers of low viscosity sheet silicates near the plate interface. Transmission of shear stress between plates is inhibited by formation of talc and serpentine in the shallow upper-plate mantle (\protect\hyperlink{ref-peacock1999a}{Peacock \& Hyndman, 1999}). Lack of traction along the interface, combined with cooling from the subducting plate surface, ensures a positive feedback between hydrous mineral formation and mechanical decoupling. Experimentally determined flow laws, petrologic observations, and geophysical observations all support the plausibility of this conceptual model of subduction interface behavior (e.g. \protect\hyperlink{ref-agard2016}{Agard et al., 2016}, \protect\hyperlink{ref-agard2018}{2018}; \protect\hyperlink{ref-gao2014}{Gao \& Wang, 2014}; \protect\hyperlink{ref-peacock1999a}{Peacock \& Hyndman, 1999}).

Experimental control over important thermo-kinematic boundary conditions make geodynamic numerical simulations essential for investigating such complex geodynamic environments. Wada \& Wang (\protect\hyperlink{ref-wada2009}{2009}) previously investigated the effects of \(\Phi\) on coupling depths by numerically simulating 17 presently active subduction zones. Among other thermo-kinematic boundary conditions, their models specify convergence rate, subduction geometry, thermal structure of oceanic- and overriding-plates, and degree of coupling along the subduction interface. Notably, their experiments control for interface rheology and discriminate best-fit coupling depths based on observed forearc surface heat flow.

This study similarly specifies thermo-kinematic boundary conditions to numerically simulate the range of modern subduction zone systems, but regulates interface rheology dynamically by implementing metamorphic reactions that respond to evolving \gls{pt}-strain fields. Subduction geometry and coupling depth are not fully determined features, in other words, but spontaneous model outcomes within the range of specified boundary conditions discussed in Section \ref{numMethods}. As in previous studies (e.g. \protect\hyperlink{ref-ruh2015}{Ruh et al., 2015}), rheological effects of the dehydration reaction \(antigorite \allowbreak \Leftrightarrow olivine + orthopyroxene + H_{2}O\) are implemented to drive mechanical coupling. An abrupt viscosity increase accompanies antigorite (serpentine) destabilization, thereby inducing mechanical coupling, as defined by empirically-determined flow laws used in the numerical experiments.

This chapter focuses on two fundamental questions. How does coupling depth respond to \(\Phi\) \emph{and} upper-plate thickness? And how stable is coupling depth through time? First, 64 convergent margins with variable upper-plate thickness and \(\Phi\) are numerically simulated and mechanical plate coupling is observed. Thermal feedbacks within the system are visualized in terms of mantle temperature, viscosity, and velocity fields and coupling depth responses to a range of \(\Phi\) and upper-plate thickness are quantified using multivariate linear regression. Three different regression models are then used to estimate coupling depths for 17 presently active subduction zones. Coupling depth estimates are narrowly distributed, regardless of regression model form. Finally, implications and questions regarding uniformity among subduction zones in terms of surface heat flow, upper-plate thickness, and coupling depth are discussed.

\hypertarget{numMethods}{%
\section{Numerical Modelling Methods}\label{numMethods}}

This study simulates converging oceanic-continental plates, where an ocean basin is being consumed by subduction at a continental margin (Figure \ref{fig:init}). Initial conditions are modified from previous numerical experiments of active margins (\protect\hyperlink{ref-gorczyk2007}{Gorczyk et al., 2007}; \protect\hyperlink{ref-sizova2010}{Sizova et al., 2010}) using the code \texttt{I2VIS} (\protect\hyperlink{ref-gerya2003}{Gerya \& Yuen, 2003}), although plate coupling was not the focus of their studies. An identical rheologic model with identical material properties (Table \ref{tab:materials}), and a identical hydration/melt model (Table \ref{tab:melts} \& Appendix \ref{deHydration}) to Sizova et al. (\protect\hyperlink{ref-sizova2010}{2010}) are implemented. However, the version of \texttt{I2VIS} in this study differs from Sizova et al. (\protect\hyperlink{ref-sizova2010}{2010}) in its initial setup, overall dimension, resolution, continental geotherm, dehydration model, and left boundary condition (origin of new oceanic lithosphere). Differences are outlined in this section and in Appendix \ref{deHydration}. Sixty-four \texttt{I2VIS} models constructed with varying convergence rates (\(\vec{v}\)), oceanic plate ages, and upper-plate thickness (Figure \ref{fig:params}) were ran for at least 100 timesteps.

\begin{landscape}



\begin{figure}[htbp]

{\centering \includegraphics[width=1\linewidth,]{assets/figs/chpt2/fig1} 

}

\caption[Initial model configuration and boundary conditions]{Initial model configuration and boundary conditions. (a) A free sedimentation/erosion boundary at the surface is maintained by implementing a layer of "sticky" air and water, and an infinite-like open boundary at the bottom allows for spontaneous oceanic plate descent and subduction angle. Left and right boundaries are free slip and thermally insulating. Initial material distribution includes 7 km of oceanic crust (2 km basalt, 5 km gabbro), 1 km of oceanic sediments, and 35 km of continental crust, thinning ocean-ward. (b) Oceanic lithosphere is continually created at the left boundary. The oceanic geotherm is calculated using a half-space cooling model and the continental geotherm is calculated using a one-dimensional steady-state conductive cooling model to 1300 $^{\circ}$C. The base of the upper-plate lithosphere ($Z_{UP}$) is defined by visualizing viscosity and generally coincides with the 1100 $^{\circ}$C isotherm. (c) Oceanic crust is bent under loading from passive margin sediments, and a weak zone extends through the lithosphere to help induce subduction. Convergence velocities are imposed at stationary, high-viscosity regions far from the trench. Rock type colors are: [1] air, [2] water, [4,5] sediments, [6,7] felsic crust, [8] basalt, [9] gabbro, [10,11] dry mantle, [12] hydrated mantle, [14] serpentinized mantle.}\label{fig:init}
\end{figure}


\end{landscape}

\begin{landscape}\begin{table}

\caption{\label{tab:materials}Material properties used in numerical experiments}
\centering
\resizebox{\linewidth}{!}{
\begin{threeparttable}
\begin{tabular}[t]{lrrlrrrrrrrrlr}
\toprule
\multicolumn{1}{c}{Material} & \multicolumn{1}{c}{$\rho$} & \multicolumn{1}{c}{$H_2O$} & \multicolumn{1}{c}{Flow Law} & \multicolumn{1}{c}{$log_{10}A$} & \multicolumn{1}{c}{$E$} & \multicolumn{1}{c}{$V$} & \multicolumn{1}{c}{$n$} & \multicolumn{1}{c}{$\phi$} & \multicolumn{1}{c}{$\sigma_{crit}$} & \multicolumn{1}{c}{$k_1$} & \multicolumn{1}{c}{$k_2$} & \multicolumn{1}{c}{$k_3$} & \multicolumn{1}{c}{$H$} \\
\cmidrule(l{0pt}r{0pt}){1-1} \cmidrule(l{0pt}r{0pt}){2-2} \cmidrule(l{0pt}r{0pt}){3-3} \cmidrule(l{0pt}r{0pt}){4-4} \cmidrule(l{0pt}r{0pt}){5-5} \cmidrule(l{0pt}r{0pt}){6-6} \cmidrule(l{0pt}r{0pt}){7-7} \cmidrule(l{0pt}r{0pt}){8-8} \cmidrule(l{0pt}r{0pt}){9-9} \cmidrule(l{0pt}r{0pt}){10-10} \cmidrule(l{0pt}r{0pt}){11-11} \cmidrule(l{0pt}r{0pt}){12-12} \cmidrule(l{0pt}r{0pt}){13-13} \cmidrule(l{0pt}r{0pt}){14-14}
 & kg/m$^3$ & $wt.\%$ &  &  & kJ/mol & J/MPa$\cdot$mol &  &  & MPa &  &  &  & $\mu$W/m$^3$\\
\midrule
\cellcolor{gray!6}{sediments} & \cellcolor{gray!6}{2600} & \cellcolor{gray!6}{5.0} & \cellcolor{gray!6}{wet quartzite} & \cellcolor{gray!6}{-3.5} & \cellcolor{gray!6}{154.0} & \cellcolor{gray!6}{3.0} & \cellcolor{gray!6}{2.3} & \cellcolor{gray!6}{0.15} & \cellcolor{gray!6}{0.03} & \cellcolor{gray!6}{0.64} & \cellcolor{gray!6}{807} & \cellcolor{gray!6}{4e-06} & \cellcolor{gray!6}{2.000}\\
felsic crust & 2700 &  & wet quartzite & -3.5 & 154.0 & 3.0 & 2.3 & 0.45 & 0.03 & 0.64 & 807 & 4e-06 & 1.000\\
\cellcolor{gray!6}{basalt} & \cellcolor{gray!6}{3000} & \cellcolor{gray!6}{5.0} & \cellcolor{gray!6}{plag an75} & \cellcolor{gray!6}{-3.5} & \cellcolor{gray!6}{238.0} & \cellcolor{gray!6}{8.0} & \cellcolor{gray!6}{3.2} & \cellcolor{gray!6}{0.45} & \cellcolor{gray!6}{0.03} & \cellcolor{gray!6}{1.18} & \cellcolor{gray!6}{474} & \cellcolor{gray!6}{4e-06} & \cellcolor{gray!6}{0.250}\\
gabbro & 3000 &  & plag an75 & -3.5 & 238.0 & 8.0 & 3.2 & 0.45 & 0.03 & 1.18 & 474 & 4e-06 & 0.250\\
\cellcolor{gray!6}{mantle dry} & \cellcolor{gray!6}{3300} & \cellcolor{gray!6}{} & \cellcolor{gray!6}{dry olivine} & \cellcolor{gray!6}{4.4} & \cellcolor{gray!6}{540.0} & \cellcolor{gray!6}{20.0} & \cellcolor{gray!6}{3.5} & \cellcolor{gray!6}{0.45} & \cellcolor{gray!6}{0.30} & \cellcolor{gray!6}{0.73} & \cellcolor{gray!6}{1293} & \cellcolor{gray!6}{4e-06} & \cellcolor{gray!6}{0.022}\\
mantle hydrated & 3300 & 0.5 & wet olivine & 3.3 & 430.0 & 10.0 & 3.0 & 0.45 & 0.24 & 0.73 & 1293 & 4e-06 & 0.022\\
\cellcolor{gray!6}{serpentine} & \cellcolor{gray!6}{3200} & \cellcolor{gray!6}{2.0} & \cellcolor{gray!6}{serpentine} & \cellcolor{gray!6}{3.3} & \cellcolor{gray!6}{8.9} & \cellcolor{gray!6}{3.2} & \cellcolor{gray!6}{3.8} & \cellcolor{gray!6}{0.15} & \cellcolor{gray!6}{3.00} & \cellcolor{gray!6}{0.73} & \cellcolor{gray!6}{1293} & \cellcolor{gray!6}{4e-06} & \cellcolor{gray!6}{0.022}\\
\bottomrule
\end{tabular}
\begin{tablenotes}
\item \uline{\textit{key}}: $A$: material constant, $E$, $V$: activation energy and volume, $n$: power law exponent, $\phi$: internal friction angle, $\sigma_{crit}$: critical stress, $k_1$-$k_3$: thermal conductivity constants, $H$: heat production
\item \uline{\textit{constants}}: $C_p$: heat capacity = 1 [kJ/kg], $\alpha$: expansivity = 2$\times 10^{-5}$ [1/K], $\beta$: compressibility = 0.045 [1/MPa]
\item \uline{\textit{thermal conductivity}}: $k$ [W/mK] = $(k_1+\frac{k_2}{T+77})\times exp(k_3 \cdot P)$ with $P$ in [MPa] and $T$ in [K]
\item \uline{\textit{references}}: Turcotte \& Schubert (2002), Ranalli (1995), Hilairet et al. (2007), Karato \& Wu (1993)
\end{tablenotes}
\end{threeparttable}}
\end{table}
\end{landscape}

\hypertarget{numBCs}{%
\subsection{Initial Setup and Boundary Conditions}\label{numBCs}}

Simulations are 2000 km wide and 300 km deep (Figure \ref{fig:init}). In the model domain, three governing equations of heat transport, momentum, and continuity are discretized and solved with a conservative finite-difference marker-in-cell approach on a fully staggered grid as outlined in Gerya \& Yuen (\protect\hyperlink{ref-gerya2003}{2003}). Numerical resolution is nonuniform with higher resolution (1 \(\times\) 1 km) in a 600 km wide area surrounding the contact between the oceanic plate and continental margin, then gradually changing to lower resolution towards the model boundaries (5 \(\times\) 1 km, x- and z-directions, respectively). The left and right boundaries are free-slip and thermally insulating (Figure \ref{fig:init}a, b). Implementation of ``sticky'' air and water allows for a free topographical surface with a simple linear sedimentation and erosion model. The lower boundary is open to allow for oceanic plate descent with a spontaneous subduction angle (\protect\hyperlink{ref-burg2005}{Burg \& Gerya, 2005}).



\begin{figure}[htbp]

{\centering \includegraphics[width=1\linewidth,]{assets/figs/chpt2/fig2} 

}

\caption[Range of boundary conditions used in numerical experiments]{Range of thermo-kinematic boundary conditions used in numerical experiments. (a) Thermal parameters (grayscale) range from 13 to 110 km/100 and broadly reflect the distribution of oceanic plate ages and convergence velocities in modern subduction zones. Model names include the prefix ``cd'' for ``coupling depth'' with increasing alphabetic suffixes. Note that neither axes are continuous. (b) Upper-plate thickness (\(Z_{UP}\)) is defined by a range of continental geotherms. Geotherms were constructed using a one-dimensional steady-state conductive cooling model with T(z=0) = 0 \(^{\circ}\)C, \(\vec{q}\)(z=0) = 59, 63, 69, 79 mW/m\(^2\), and constant radiogenic heating of 1.0 \(\mu\)W/m\(^3\) for a 35 km-thick crust and 0.022 \(\mu\)W/m\(^3\) for the mantle. Continental geotherms are calculated up to 1300 \(^{\circ}\)C with a constant 0.5 \(^{\circ}\)C/km gradient (the mantle adiabat) extending to the base of the model domain.}\label{fig:params}
\end{figure}

A horizontal convergence force is applied to both plates in a rectangular region far from the continental margin (Figure \ref{fig:init}c). An initial weak layer cutting the lithosphere permits subduction to initiate. The high-viscosity (\(\eta = 10^{25}\) Pa \(\cdot\) s) rectangular convergence regions apply constant horizontal velocities without deforming the lithosphere. Subduction angle is governed by free-motion of the subducting plate. Similarly, subduction velocity varies with time in response to extension or shortening of the overriding plate. \(\Phi\) is thus calculated as the product of the horizontal convergence velocity and the oceanic plate age (cf. \protect\hyperlink{ref-kirby1991}{Kirby et al., 1991}; \protect\hyperlink{ref-mckenzie1969}{McKenzie, 1969}). For convenience and consistency with the literature, this study presents \(\Phi\) in units of km/100 (Figure \ref{fig:params}a).

\hypertarget{rheologicModel}{%
\subsection{Rheologic Model}\label{rheologicModel}}

Contributions from dislocation and diffusion creep are accounted for by computing a composite rheology for ductile rocks, \(\eta_{eff}\):
\begin{equation}
  \begin{aligned}
    \frac{1}{\eta_{eff}} = \frac{1}{\eta_{diff}} + \frac{1}{\eta_{disl}}
  \end{aligned} 
  \label{eq:ductile}
\end{equation}
where \(\eta_{diff}\) and \(\eta_{disl}\) are effective viscosities for diffusion and dislocation creep.

\noindent For the crust and serpentinized mantle, \(\eta_{diff}\) and \(\eta_{disl}\) are computed as:
\begin{equation}
  \begin{aligned}
    \eta_{diff} &= \frac{1}{2} \ A \ \sigma_{crit}^{1-n} \ \exp\left[\frac{E+PV}{RT}\right] \\
    \eta_{disl} &= \frac{1}{2} \ A^{1/n} \ \dot{\varepsilon}_{II}^{(1-n)/n} \ \exp\left[\frac{E+PV}{nRT}\right]
  \label{eq:crust}
  \end{aligned}
\end{equation}
where \(R\) is the gas constant, \(P\) is pressure, \(T\) is temperature in \(K\), \({\dot{\varepsilon}}_{II} = \sqrt{\frac{1}{2}{{\dot{\varepsilon}}_{ij}}^{2}}\) is the square root of the second invariant of the strain rate tensor, \(\sigma_{crit}\) is an assumed diffusion-dislocation transition stress, and \(A\), \(E\), \(V\) and \(n\) are the material constant, activation energy, activation volume, and stress exponent, respectively (Table \ref{tab:materials}, \protect\hyperlink{ref-hilairet2007}{Hilairet et al., 2007}; \protect\hyperlink{ref-ranalli1995}{Ranalli, 1995}).

\noindent For the mantle, \(\eta_{diff}\) and \(\eta_{disl}\) are computed as (\protect\hyperlink{ref-karato1993}{Karato \& Wu, 1993}):
\begin{equation}
  \begin{aligned}
    \eta_{diff} &= \frac{1}{2} \ A^{-1} \ G \ \left[\frac{h}{b}\right]^{m/n} \ \exp\left[\frac{E+PV}{RT}\right] \\
    \eta_{disl} &= \frac{1}{2} \ A^{-1/n} \ G \ \dot{\varepsilon}_{II}^{(1-n)/n} \ \exp\left[\frac{E+PV}{nRT}\right]
  \end{aligned}
  \label{eq:mantle}
\end{equation}
where \(b\) = 5 \(\times\) 10\(^{-10}\) m is the Burgers vector, \(G\) = 8 \(\times\) 10\(^{10}\) Pa is shear modulus, \(h\) = 1 \(\times\) 10\(^{-3}\) m is the assumed grain size, \(m\) = 2.5 is the grain size exponent, and the other flow law parameters are given in Table \ref{tab:materials}. Viscosity is limited in all numerical experiments from \(\eta_{min}\) = 10\(^{17}\) Pa \(\cdot\) s to \(\eta_{max}\) = 10\(^{25}\) Pa \(\cdot\) s.

\noindent An effective visco-plastic rheology is achieved by limiting \(\eta_{eff}\) with a brittle (plastic) yield criterion:
\begin{equation}
  \eta_{eff} \leq \frac{C + \phi \ P}{2 \ \dot{\varepsilon}_{II}}
  \label{eq:plastic}
\end{equation}
where \(\phi\) is the internal friction coefficient, \(C\) cohesive strength at \(P\) = 0, and \({\dot{\varepsilon}}_{ij}\) is the strain rate tensor (Table \ref{tab:materials}).

\hypertarget{numGeotherms}{%
\subsection{Defining Geotherms and Lithospheric Thickness}\label{numGeotherms}}

Oceanic crust is modelled as 1 km of sediment cover overlying 2 km of basalt and 5 km of gabbro (Figure \ref{fig:init}a). Oceanic lithosphere is continually made at a pseudo-mid-ocean ridge at the left boundary of the model (Figure \ref{fig:init}b). An enhanced vertical cooling condition applied at 200 km from left boundary adjusts for the proper oceanic plate age, and therefore its lithospheric thickness as it enters the trench (\protect\hyperlink{ref-agrusta2013}{Agrusta et al., 2013}). Oceanic plate ages range from 32.6 to 110 Ma and convergence velocities from 40 to 100 km/Ma (Figure \ref{fig:params}a). This range of parameters broadly reflects the middle-range of modern global subduction systems (\protect\hyperlink{ref-syracuse2006}{Syracuse \& Abers, 2006}).

Initial continental geotherms are determined by solving the heat flow equation in one-dimension to 1300 \(^{\circ}\)C (Figure \ref{fig:params}b). This study assumes a fixed temperature of 0 \(^{\circ}\)C at the surface, constant radiogenic heating of 1 \(\mu\)W/m\(^3\) in the 35 km-thick continental crust, 0.022 \(\mu\)W/m\(^3\) in the mantle, with thermal conductivities of 2.3 W/mK and 3.0 W/mK for the continental crust and mantle, respectively. Above, 1300 \(^{\circ}\)C, temperature is assumed to constantly increase by 0.5 \(^{\circ}\)C/km (the mantle adiabat) to the base of the model domain.

Many studies define the base of the continental lithosphere at the 1300 \(^{\circ}\)C isotherm, but it can be determined directly by visualizing viscosity and strain rate as the model progresses. The mechanical base of the lithosphere (\(Z_{UP}\)) in the models generally occurs near the 1100 \(^{\circ}\)C isotherm---characterized by a rapid decrease in viscosity and increase in strain rate (Figures \ref{fig:cdfStep1}, \ref{fig:cdfStep2}, \ref{fig:cdfStep3}). As such, this study considers oceanic and continental lithospheres as mechanical layers defined by viscosity, rather than defined merely by temperature. \(Z_{UP}\) corresponding to backarc surface heat flow of 59, 63, 69, and 79 mW/m\(^2\) are used in this study (Figure \ref{fig:params}b).

\hypertarget{numHydration}{%
\subsection{Metamorphic (De)hydration Reactions}\label{numHydration}}

Using Lagrangian markers (\protect\hyperlink{ref-harlow1962}{Harlow, 1962}, \protect\hyperlink{ref-harlow1964}{1964}) to store and update material properties and \gls{pt}-strain fields allows for straight-forward numerical implementation of metamorphic reactions. This approach is key to regulating mechanical coupling dynamically in subduction zone simulations. For example, dehydration (eclogitization) of the oceanic plate and (de)stabilization of serpentine in the upper-plate mantle may be effectively modelled by tracing marker \gls{pt}-time paths while changing marker properties according to thermodynamically-stable mineral assemblages (e.g. \protect\hyperlink{ref-connolly2005}{Connolly, 2005}). For computational efficiency, however, water contents in this study are not computed by iteratively solving thermodynamic systems of equations.

Instead, gradual eclogitization of oceanic crust is computed as a linear function of lithostatic pressure to a maximum depth of 150 km, or temperature of 1427 \(^{\circ}\)C, while including garnet-in and plagioclase-out reactions defined by Ito \& Kennedy (\protect\hyperlink{ref-ito1971}{1971}). Mantle (de)hydration is computed according reactions boundaries defined by Schmidt \& Poli (\protect\hyperlink{ref-schmidt1998}{1998}) with a maximum water content of 2 \(wt.\%\) (explained below). This approach effectively simulates continuous influx of water to the upper-plate mantle with relatively low computational cost, beginning with compaction and release of connate water at shallow depths, followed by a sequence of reactions consuming major hydrous phases (chlorite, lawsonite, zoisite, chloritoid, talc, amphibole, and phengite) in different parts of the hydrated basaltic crust (\protect\hyperlink{ref-schmidt1998}{Schmidt \& Poli, 1998}; \protect\hyperlink{ref-vankeken2011}{van Keken et al., 2011}).

The extent of metamorphic reaction effects on mechanical coupling, and the exact (de)hydration reaction(s) likely responsible, are unknown. However, formation of brucite and serpentine from dry olivine near the plate interface are inferred to strongly regulate mechanical behavior (\protect\hyperlink{ref-agard2016}{Agard et al., 2016}; \protect\hyperlink{ref-hyndman2003}{Hyndman \& Peacock, 2003}; \protect\hyperlink{ref-peacock1999a}{Peacock \& Hyndman, 1999}). Brucite notably breaks down at much lower temperatures than serpentine (\protect\hyperlink{ref-schmidt1998}{Schmidt \& Poli, 1998}), so serpentine (de)stabilization arguably represents the key transition from a weak-to-strong upper-plate mantle deep in subduction zones. This study elects an implementation of serpentine (de)hydration for this reason. The reaction is assumed to be abrupt and discontinuous, which is a fine approximation for near-endmember compositions like (Mg-rich) peridotites. The \gls{pt} conditions of the reaction \(antigorite \Leftrightarrow olivine + orthopyroxene + H_{2}O\) were numerically implemented by the following equation (after \protect\hyperlink{ref-schmidt1998}{Schmidt \& Poli, 1998}):
\begin{equation}
  T_{atg-out}(z)=
  \begin{cases}
    751.50+6.008\times10^{-3}z-3.469\times10^{-8}z^2,& \text{for } z < 63000m \\
    1013.2-6.039\times10^{-5}z-4.289\times10{-9}z^2,& \text{for } z>63000m
  \end{cases}
  \label{eq:antstab}
\end{equation}
where \(z\) is the depth of a marker from the surface in meters and \(T\) is temperature in Kelvins. This reaction boundary is consistent to within 25 \(^{\circ}\)C of more recent experiments by Shen et al. (\protect\hyperlink{ref-shen2015}{2015}). Markers with internal temperature exceeding \(T_{atg-out}(z)\) spontaneously form \(olivine + orthopyroxene + H_{2}O\) and release their crystal-bound water. This implementation tacitly assumes thermodynamic equilibrium and is common to many versions of \texttt{I2VIS}.

Oceanic plates of different ages are also tacitly assumed to dehydrate similarly with the above implementation. However, older (colder) oceanic plates are expected to carry water to greater depths than younger (warmer) plates because of relatively delayed water-releasing reactions (\protect\hyperlink{ref-peacock1996}{Peacock, 1996}). Abrupt water release with serpentine dehydration (Equation \eqref{eq:antstab}) was tested to model deep water retention in cold oceanic plates. Mechanical coupling behavior was indistinguishable from gradual water release models. This implies rates of water release are less important than the depth of serpentine stability. Explicitly modelling other major dehydration reactions are thus unlikely to significantly affect mechanical coupling behavior, yet likely to introduce numerical artifacts at great computational cost. A simplified gradual water release model for all oceanic plates is therefore preferred.

Water released by rock forms discrete fluid particles that migrate with relative velocities defined by local deviatoric (non-lithostatic) pressure gradients (see Appendix \ref{deHydration}, \protect\hyperlink{ref-faccenda2009}{Faccenda et al., 2009}). Fluid velocities are scaled by a 10 cm/yr vertical percolation velocity to account for purely lithostatic pressure gradients in the mantle (\protect\hyperlink{ref-gorczyk2007}{Gorczyk et al., 2007}). Fluid particles migrate until encountering rock that can consume additional water by equilibrium hydration or melting reactions, (Equation \ref{tab:melts}).

The shallow upper-plate mantle can theoretically store large amounts of water as serpentine may contain up to 13 \(wt.\%\) water (\protect\hyperlink{ref-reynard2013}{Reynard, 2013}) and is generally stable at shallow mantle conditions. Thermodynamic models predict 8 \(wt.\%\) water in the shallow upper-plate mantle (\protect\hyperlink{ref-connolly2005}{Connolly, 2005}). However, seismic studies suggest most shallow upper-plate mantles are only partially serpentinized (\textless{} 20-40\%), equating to water contents of approximately 3-6 \(wt.\%\) (\protect\hyperlink{ref-abers2017}{Abers et al., 2017}; \protect\hyperlink{ref-carlson2003}{Carlson \& Miller, 2003}). Many modes of mantle hydration are documented or inferred, including evidence for channelized fluid flow within ophiolites exhumed from subduction zones (\protect\hyperlink{ref-angiboust2012a}{Angiboust et al., 2012a}, \protect\hyperlink{ref-angiboust2014a}{2014a}; \protect\hyperlink{ref-plumper2017}{Plümper et al., 2017}; \protect\hyperlink{ref-zack2007}{Zack \& John, 2007}). This study limits mantle wedge hydration to \(\leq\) 2 \(wt.\%~H_{2}O\) and assumes any excess \(H_{2}O\) exits the system through channelized fluid flow during plastic or brittle deformation (\protect\hyperlink{ref-davies1999a}{Davies, 1999b}).

\hypertarget{visualization-and-determination-of-coupling-depth}{%
\subsection{Visualization and Determination of Coupling Depth}\label{visualization-and-determination-of-coupling-depth}}

The rheologic model and thermo-kinematic boundary conditions described in the previous sections always results in plate motions towards the left boundary (slab-rollback). Relatively high dip angles and extreme subduction velocities in some high-\(\Phi\) experiments cause chaotic behavior by 10 Ma as the upper-plate is stretched thin and mechanical interference occurs between trench sediments and the high-viscosity convergence region 200 km from the left boundary. Numerical solutions are stable for most experiments, however, reaching quasi-steady state by 5 Ma. An additional 5 Ma is allowed to ensure stable geodynamics before observing coupling depth. Surface heat flow, rock type, temperature, viscosity, strain rate, shear heating, and velocity fields are visualized at approximately 10 Ma (e.g.~Figure \ref{fig:comp}) for all 64 experiments to assess geodynamics and solution stability (Figure \ref{fig:antDepth}).



\begin{figure}[htbp]

{\centering \includegraphics[width=1\linewidth,]{assets/figs/chpt2/fig3} 

}

\caption[Numerical experiment visualization]{Visualizing model cdf with a 78 km upper-plate lithosphere at approximately 10 Ma. (a) Rock type shows a thin serpentine layer (pink) lubricating the plate interface. Note that low melt volumes are inconspicuous and quickly extracted. (b) Viscosity shows high contrast between the oceanic plate and serpentinized upper-plate mantle at shallow levels. Viscosity contrast disappears where serpentine becomes unstable. (c) Streamlines show focused mantle flow towards the interface, coinciding with the lower limit of serpentine stability. Note the converging isotherms that imply a feedback between heat transfer, serpentine destabilization, and mechanical coupling. (d) Strain rate shows localized deformation in the serpentine layer along the plate interface. Note that deformation in the upper-plate mantle is restricted to viscous flow beneath the lithosphere and along narrow, subvertical melt conduits. Rock type colors are the same as Figure \ref{fig:init}.}\label{fig:comp}
\end{figure}

After approximately 10 Ma of subduction coupling depth is determined directly from viscosity by finding the approximate area where strength contrasts between serpentinized- and non-serpentinized upper-plate mantle diminishes to \textless{} 10\(^2\) Pa \(\cdot\) s. The node nearest to this region is assigned as the coupling depth. This study assumes mechanical coupling occurs instantaneously and at a single node. Mechanical coupling in reality must be dispersed across a finite length along the plate interface, however. At the numerical resolution the experiments, coupling-like viscosity contrasts are similar within a small area (approximately 5 \(\times\) 5 km or 5 \(\times\) 5 nodes), giving a qualitative uncertainty coupling depth on the order of 2.5 km.

\hypertarget{chpt2Results}{%
\section{Results}\label{chpt2Results}}

\hypertarget{cdEstimators}{%
\subsection{Coupling Depth Estimators}\label{cdEstimators}}

Coupling depth (\(Z_{cpl}\)) correlates strongly with upper-plate thickness (\(Z_{UP}\)) and weakly with \(\Phi\) across all 64 numerical models (Table \ref{tab:zcResults}, Figures \ref{fig:results} \& \ref{fig:biv}). The responsiveness of coupling depth to \(Z_{UP}\) but not to \(\Phi\) is a key result of this study. The following equation minimizes standard least squares while optimizing the number of parameters, \emph{p value}, and \(R^2\) for all possible permutations of the variables \(Z_{UP}\) and \(\Phi\) in linear and quadratic forms:
\begin{equation}
  Z_{cpl} = 4.95\times 10^{-3}\ Z_{UP}^{2}\ -\ 9.27\times 10^{-2}\ \Phi\ +\ 63.6
  \label{eq:zCpl}
\end{equation}
where \(Z_{cpl}\) is coupling depth in km and \(\Phi\) is the thermal parameter in km/100. Regression summaries show both linear and quadratic models of \(Z_{cpl}\) vs.~\(Z_{UP}\) and \(\Phi\) fit experimental results well (Tables \ref{tab:anova} \& \ref{tab:regSummary}). Equation \eqref{eq:zCpl} represents a statistical model formulated with observations from physics-based simulations of subduction. Equation \eqref{eq:zCpl} is useful for estimating coupling depths in active subduction zones where \(\Phi\) is known and \(Z_{UP}\) can be inverted from surface heat flow.

Sensitivity of coupling depth to upper-plate thickness and \(\Phi\) is apparent when visualizing Equation \eqref{eq:zCpl} and other regression models across \(Z_{UP}\) and \(\Phi\) space \ref{fig:multiv}. Applying Equation \eqref{eq:zCpl} to 17 active subduction zone segments (Table \ref{tab:segs}) gives a wide range of estimated coupling depths, similar to the numerical simulations in this study. The distribution of estimated coupling depths, however, is relatively narrow and quasi-normal, with a mean of \(\sim\) 82 km and standard deviation of 7 km (Figure \ref{fig:multiv}d).



\begin{figure}[htbp]

{\centering \includegraphics[width=1\linewidth,]{assets/figs/chpt2/fig4} 

}

\caption[Estimated coupling depths for 17 active subduction zone segments]{Multivariate regressions and estimated coupling depth (\(Z_{cpl}\)) for 17 active subduction zone segments. Contoured plots show estimated \(Z_{cpl}\) (contours) as a function of thermal parameter (\(\Phi\)) and upper-plate thickness (\(Z_{UP}\)) for linear (a) and quadratic (b, c) regressions. The best fit regression is panel b (Equation \eqref{eq:zCpl}, see Tables \ref{tab:anova} and \ref{tab:regSummary}). Black squares are numerical experiments used to fit the contours. White points represent active subduction zones (Table \ref{tab:segs}). Contours imply \(Z_{cpl}\) depends strongly on \(Z_{UP}\). While some estimated \(Z_{cpl}\) for subduction zones with similar \(\Phi\) are quite different (e.g.~Alaska vs.~N. Sumatra), some estimated \(Z_{cpl}\) are quite similar for subduction zones with very different \(\Phi\) (e.g.~Kamchatka vs.~N. Cascadia). (d) Distributions of estimated \(Z_{cpl}\) for 17 active subduction zones shown in (a), (b), and (c). These 17 segments span a large range of \(\Phi\) but are expected to have a relatively narrow distribution of \(Z_{cpl}\) (82 \(\pm\) 14 km) according to the regressions in (a), (b), and (c).}\label{fig:multiv}
\end{figure}

\begin{table}

\caption{\label{tab:segs}Estimated coupling depths for active subduction zones}
\centering
\begin{threeparttable}
\begin{tabular}[t]{lrrrrrr}
\toprule
\multicolumn{1}{c}{Segment} & \multicolumn{1}{c}{$\vec{q}$} & \multicolumn{1}{c}{$Z_{UP}$} & \multicolumn{1}{c}{$\Phi$} & \multicolumn{1}{c}{$Z_{cpl}^a$} & \multicolumn{1}{c}{$Z_{cpl}^b$} & \multicolumn{1}{c}{$Z_{cpl}^c$} \\
\cmidrule(l{0pt}r{0pt}){1-1} \cmidrule(l{0pt}r{0pt}){2-2} \cmidrule(l{0pt}r{0pt}){3-3} \cmidrule(l{0pt}r{0pt}){4-4} \cmidrule(l{0pt}r{0pt}){5-5} \cmidrule(l{0pt}r{0pt}){6-6} \cmidrule(l{0pt}r{0pt}){7-7}
 & mW/m$^2$ & km & km/100 & km & km & km\\
\midrule
\cellcolor{gray!6}{N. Cascadia} & \cellcolor{gray!6}{75} & \cellcolor{gray!6}{74.2} & \cellcolor{gray!6}{3.4} & \cellcolor{gray!6}{92} & \cellcolor{gray!6}{91} & \cellcolor{gray!6}{90}\\
Nankai & 69 & 96.3 & 6.9 & 107 & 109 & 110\\
\cellcolor{gray!6}{Mexico} & \cellcolor{gray!6}{72} & \cellcolor{gray!6}{98.1} & \cellcolor{gray!6}{7.2} & \cellcolor{gray!6}{108} & \cellcolor{gray!6}{111} & \cellcolor{gray!6}{112}\\
Columbia-Ecuador & 80 & 66.4 & 10.4 & 86 & 84 & 84\\
\cellcolor{gray!6}{S.C. Chile} & \cellcolor{gray!6}{80} & \cellcolor{gray!6}{66.4} & \cellcolor{gray!6}{20.0} & \cellcolor{gray!6}{85} & \cellcolor{gray!6}{84} & \cellcolor{gray!6}{83}\\
Kyushu & 69 & 83.2 & 13.5 & 97 & 97 & 96\\
\cellcolor{gray!6}{N. Sumatra} & \cellcolor{gray!6}{120} & \cellcolor{gray!6}{26.8} & \cellcolor{gray!6}{25.0} & \cellcolor{gray!6}{57} & \cellcolor{gray!6}{65} & \cellcolor{gray!6}{68}\\
Alaska & 80 & 66.4 & 25.3 & 85 & 83 & 82\\
\cellcolor{gray!6}{N. Chile} & \cellcolor{gray!6}{85} & \cellcolor{gray!6}{58.7} & \cellcolor{gray!6}{38.4} & \cellcolor{gray!6}{78} & \cellcolor{gray!6}{77} & \cellcolor{gray!6}{77}\\
N. Costa Rica & 80 & 58.5 & 20.4 & 80 & 79 & 78\\
\cellcolor{gray!6}{Aleutians} & \cellcolor{gray!6}{75} & \cellcolor{gray!6}{51.6} & \cellcolor{gray!6}{39.6} & \cellcolor{gray!6}{73} & \cellcolor{gray!6}{73} & \cellcolor{gray!6}{73}\\
N. Hikurangi & 80 & 58.5 & 41.0 & 78 & 77 & 76\\
\cellcolor{gray!6}{Mariana} & \cellcolor{gray!6}{80} & \cellcolor{gray!6}{47.5} & \cellcolor{gray!6}{54.6} & \cellcolor{gray!6}{69} & \cellcolor{gray!6}{70} & \cellcolor{gray!6}{70}\\
Kermadec & 80 & 47.5 & 60.0 & 68 & 69 & 70\\
\cellcolor{gray!6}{Kamchatka} & \cellcolor{gray!6}{70} & \cellcolor{gray!6}{80.2} & \cellcolor{gray!6}{77.0} & \cellcolor{gray!6}{89} & \cellcolor{gray!6}{88} & \cellcolor{gray!6}{88}\\
Izu & 80 & 47.5 & 77.0 & 67 & 68 & 68\\
\cellcolor{gray!6}{NE Japan} & \cellcolor{gray!6}{88} & \cellcolor{gray!6}{47.7} & \cellcolor{gray!6}{107.9} & \cellcolor{gray!6}{64} & \cellcolor{gray!6}{65} & \cellcolor{gray!6}{65}\\
\bottomrule
\end{tabular}
\begin{tablenotes}
\item a: $Z_{cpl}=Z_{UP}+\Phi$, b: $Z_{cpl}=Z_{UP}^2+\Phi$, c: $Z_{cpl}=Z_{UP}+Z_{UP}^2+\Phi$
\item \uline{\textit{references}}: Currie \& Hyndman (2006), Wada \& Wang (2009)
\end{tablenotes}
\end{threeparttable}
\end{table}

\hypertarget{surface-heat-flow}{%
\subsection{Surface Heat Flow}\label{surface-heat-flow}}

Upper-plate surface heat flow remains relatively stable and reflects initial upper-plate geotherms in the backarc region for experiments with low to moderate \(\Phi\) (Figure \ref{fig:hf}). However, high-amplitude and high-frequency positive surface heat flow deviations in the upper-plate are common in all experiments, especially for high-\(\Phi\) experiments. These deviations correspond to extensional deformation and heat transport via lithospheric thinning and melt migration. These features are apparent as subvertical low viscosity, high strain rate columns originating from the plate interface (Figure \ref{fig:comp}b, d) and point to potential sources of error when inverting surface heat flow in active subduction zones. Notably, the backarc is relatively unaffected by fluid and melt migration compared to the forearc. Estimating upper-plate thickness by inverting surface heat flow in the backarc is therefore preferable to forearc surface heat flow.

Surface heat flow across all numerical experiments is similar in the forearc region (normalized distance \(\leq\) 0.75, Figure \ref{fig:hf78}). In contrast, surface heat flow extending behind the arc region (normalized distance \textgreater{} 0.75, Figure \ref{fig:hf78}) increases systematically, then levels off at values reflecting initial continental geotherms (i.e.~reflecting initial upper-plate thickness). In reality, surface heat flow depend on fault slip rates and rates of volcanic outputs. However, heat flow in the behind the arc may remain in steady-state if rates of volcanism and crustal thinning by extension are low (\protect\hyperlink{ref-currie2004}{Currie et al., 2004}; \protect\hyperlink{ref-currie2006}{Currie \& Hyndman, 2006}).



\begin{figure}[htbp]

{\centering \includegraphics[width=1\linewidth,]{assets/figs/chpt2/fig5} 

}

\caption[Surface heat flow determined from numerical experiments]{Surface heat flow (\(\vec{q}\)) vs.~normalized distance for model cdf with upper-plate thickness (\(Z_{UP}\)) ranging from 46 to 94 km. The distribution of \(\vec{q}\) in the forearc (normalized distance between 0.0 and 1.0) is narrow and shows little variance until near the arc (normalized distance between 0.75 and 1.0). The broad distribution of \(\vec{q}\) behind the arc (normalized distance \textgreater{} 1.0) reflects the broad distribution of initial continental geotherms (\(Z_{UP}\)). Any simple relationship between \(\vec{q}\) and \(Z_{UP}\) may be obscured by heating from extension or vertical migration of fluids, especially within the arc-region (high-amplitude fluctuations).}\label{fig:hf78}
\end{figure}

\hypertarget{chpt2Discussion}{%
\section{Discussion}\label{chpt2Discussion}}

\hypertarget{dynamic-feedbacks-regulating-plate-coupling}{%
\subsection{Dynamic Feedbacks Regulating Plate Coupling}\label{dynamic-feedbacks-regulating-plate-coupling}}

A clear association between plate coupling and the reaction \(antigorite \allowbreak \Leftrightarrow olivine + orthopyroxene + H_{2}O\) is observed in all experiments. A relatively narrow serpentine channel quickly forms above the dehydrating oceanic plate, localizing strain, lubricating the plate interface, and inhibiting transfer of shear stress between plates (e.g. \protect\hyperlink{ref-agard2016}{Agard et al., 2016}; \protect\hyperlink{ref-ruh2015}{Ruh et al., 2015}). This mechanical behavior is a direct consequence of a sharp rheologic change dependent on the location of serpentine dehydration reaction described in Section \ref{numHydration} and its effect on the rheologic model described in Section \ref{rheologicModel}. Interactions among viscosity changes, serpentine dehydration, and heat transfer are regulated by competing dynamic feedbacks acting in the upper-plate. In summary, cooling and hydration of the shallow upper-plate mantle (serpentine stabilization) and heating from circulating asthenospheric mantle beneath the upper-plate lithosphere (driven by mechanical coupling) compete to stabilize coupling depth (Figure \ref{fig:flow}).



\begin{figure}[htbp]

{\centering \includegraphics[width=1\linewidth,]{assets/figs/chpt2/fig6} 

}

\caption[Visualizing viscosity and mantle flow]{Visualizing viscosity and mantle flow near the coupling region at approximately 10 Ma for model cdf with upper-plate thickness of 78 km. Strong mantle flow beneath the lithospheric base (1100\(^{\circ}\)C) transfers heat towards the coupling region. Viscosity indicates coupling at the point where the viscosity contrast between the slab and mantle approaches zero (between points b \& d). Reference points a-e are used for discussing coupling dynamics and thermal feedbacks (see Section \ref{cplResponses}).}\label{fig:flow}
\end{figure}

The entire process can be conceptualized with Figure \ref{fig:flow} as follows. The upper-plate mantle cools via diffusive heat loss to the oceanic plate along the entire length of the plate interface (Figure \ref{fig:flow}a). At shallow depths, water released from the oceanic plate stabilizes serpentine in the overriding upper-plate mantle, effectively decoupling the two plates (Figure \ref{fig:flow}b, point a). A positive feedback stabilizes serpentine to greater depths as decoupled plates stagnate the upper-plate mantle, promoting further cooling and formation of serpentine. Numerical experiments imply only a thin layer of serpentine is sufficient to trigger this feedback.

Deeper along the plate interface, beyond the stability of serpentine, diffusive heat loss from the upper-plate mantle to the slab forms a thickening layer of high-viscosity mantle atop the oceanic plate (Figure \ref{fig:flow}b, point b). Downward motion of the oceanic plate, plus accreted high-viscosity mantle (Figure \ref{fig:flow}b, point b) relative to the deepest extent of the stiff upper-plate mantle (Figure \ref{fig:flow}b, point c) creates a pressure gradient that attracts flow of the weakest materials---serpentine from the up-dip direction (Figure \ref{fig:flow}b, point d)---and hot mantle from below (Figure \ref{fig:flow}b, point e). Flow of hot mantle into the necking region between points b and c in Figure \ref{fig:flow} is analogous to passive asthenospheric upwelling toward a mid-ocean ridge where two strong cooling lithospheric plates diverge. Highly efficient heat advection from the warm upper-plate asthenospheric mantle (Figure \ref{fig:flow}a) prevents formation of sperentine---thus regulating and stabilizing the coupling depth.

Coupling mechanics apparent from numerical experiments can be described in terms of competing positive and negative feedbacks. The positive feedback involves addition of water into a diffusively cooling, shallow mantle to produce serpentine. The negative feedback involves serpentine destabilization by advection of heat from the deeper upper-plate asthenospheric mantle. Such thermal-petrologic-mechanical feedbacks drive coupling depth towards steady-state. The numerical experiments in this study imply a finely-tuned balance of serpentine stability can maintain coupling depths in subduction zones for potentially 10's of Ma.

\hypertarget{cplResponses}{%
\subsection{\texorpdfstring{Coupling Responses to \(Z_{UP}\) and \(\Phi\)}{Coupling Responses to Z\_\{UP\} and \textbackslash Phi}}\label{cplResponses}}

How does upper-plate thickness influence coupling depth? Numerical experiments point to the upper-plate lithosphere-asthenosphere boundary as an important feature constraining coupling mechanics as it defines the permissible flow field in the upper-plate (Figure \ref{fig:streams}a-d). Thin upper-plate lithospheres (Figure \ref{fig:streams}a, b) permit shallow mantle flow and advection of heat farther up the plate interface. Thin upper-plate lithospheres thereby raise coupling depths by raising serpentine stability up the plate interface. Thick upper-plate lithospheres (Figure \ref{fig:streams}c, d) restrict mantle wedge flow to deeper levels, deepening serpentine stability and mechanical coupling.

The thermal state of the slab, as represented by \(\Phi\), has almost no effect on coupling depth by comparison. Relative insensitivity of coupling depth to \(\Phi\) is consistent with previous studies of active subduction zones (\protect\hyperlink{ref-furukawa1993}{Furukawa, 1993}; \protect\hyperlink{ref-wada2009}{Wada \& Wang, 2009}). The irresponsiveness of coupling depth to changes in \(\Phi\) is perhaps due to competing cooling and heating effects driven by the subducting oceanic plate. For example, high-\(\Phi\) oceanic plates (older plates with higher velocities) cool the upper-plate mantle more effectively, but also effectively heat the interface by driving stronger mantle circulation. In contrast, low-\(\Phi\) oceanic plates (younger plates with lower velocities) are less effective in cooling the upper-plate mantle, but also ineffectively heat the interface by ineffectively driving mantle circulation. That is, the shallow vs.~deep dynamic effects of \(\Phi\) tend to cancel each other, explaining the lack of correlation between coupling depth and \(\Phi\).

\hypertarget{estimating-coupling-depths-in-subduction-zones}{%
\subsection{Estimating Coupling Depths in Subduction Zones}\label{estimating-coupling-depths-in-subduction-zones}}

Theoretically, coupling depth can be estimated directly by fitting forearc surface heat flow data using forward modelling approaches (e.g. \protect\hyperlink{ref-wada2009}{Wada \& Wang, 2009}). However, forward approaches typically adjust coupling depth independently from upper-plate thickness, which is inconsistent with an inherent link between coupling depth and upper-plate thickness discussed in Section \ref{chpt2Discussion} (e.g.~Figures \ref{fig:hf78} \& \ref{fig:streams}). Moreover, many additional heat sources (e.g.~shear heating and crustal plutonism, \protect\hyperlink{ref-gao2014}{Gao \& Wang, 2014}; \protect\hyperlink{ref-reesjones2018}{Rees Jones et al., 2018}) may contribute to forearc surface heat flow---increasing uncertainty when inverting upper-plate thickness from surface heat flow.

Assuming low degrees of backarc extension, estimating coupling depth in active subduction zones using Equation \eqref{eq:zCpl} with \(Z_{UP}\) inverted from backarc surface heat flow is preferable to avoid additional uncertainties stemming from seismic and volcanic activity in the forearc. However, while \(\Phi\) is inventoried for most active subduction zones (\protect\hyperlink{ref-syracuse2006}{Syracuse \& Abers, 2006}), a corresponding dataset of \(Z_{UP}\) does not exist. Several geophysical and petrologic methods might be considered for independent estimates of \(Z_{UP}\) (e.g.~seismic velocities, flexure, heat flow, mantle xenoliths). Backarc surface heat flow is still a good choice, however, because of its direct correspondence with \(Z_{UP}\). For example, \(Z_{UP}\) may be estimated using simple one-dimensional heat transport models assuming values for radiogenic heat production in the crust (\protect\hyperlink{ref-rudnick1998}{Rudnick et al., 1998}). Special attention must be paid to crustal processes, including extension and magmatism, because additional heating will underestimate \(Z_{UP}\) and, consequently, underestimate coupling depth.



\begin{figure}[htbp]

{\centering \includegraphics[width=1\linewidth,]{assets/figs/chpt2/fig7} 

}

\caption[Visualizing mantle flow and coupling]{Visualizing mantle flow at approximately 10 Ma for model cdf with upper-plate thickness of (a) 46, (b) 62, (c) 78, and (d) 94 km. All experiments are plotted on the same scale and location within the model domain. The flow of warm mantle is restricted to below the 1100\(^{\circ}\)C isotherm, which corresponds to the base of the upper-plate lithosphere (\(Z_{UP}\)). A minimum coupling depth (\(Z_{cpl}\)) appears to exist as models with extremely thin lithospheres (a) exhibit coupling at \(\sim\) 70-80 km depth. \(Z_{cpl}\) generally increases with increasing \(Z_{UP}\) as mantle flow and advective heat transport are restricted to greater depths.}\label{fig:streams}
\end{figure}

\hypertarget{globally-similar-coupling-depths}{%
\subsection{Globally Similar Coupling Depths?}\label{globally-similar-coupling-depths}}

A \(Z_{cpl}\) distribution of 82 \(\pm\) 14 km (2\(\sigma\)) estimated for active subduction zones in this study (Figure \ref{fig:multiv}d) roughly match the preferred \(Z_{cpl}\) inferred from forearc surface heat flow for Cascadia and NE Japan (75-80 km, \protect\hyperlink{ref-syracuse2010}{Syracuse et al., 2010}; \protect\hyperlink{ref-wada2009}{Wada \& Wang, 2009}) km. The range of \(Z_{cpl}\) estimated for active subduction zones in this study (Figure \ref{fig:multiv}d) is relatively broad, however. For example, omitting Mexico and Nankai because their \(\Phi\) values fall outside the range of \(\Phi\) used for numerical experiments, estimated coupling depths range from almost 100 km (Kyushu) to approximately 65 km (Sumatra and NE Japan, Table \ref{tab:segs}).

Coupling depth in active subduction zones are commonly assumed to be narrowly distributed around 70-80 km (\protect\hyperlink{ref-syracuse2010}{Syracuse et al., 2010}; \protect\hyperlink{ref-wada2009}{Wada \& Wang, 2009}). The strong correlation between \(Z_{UP}\) and \(Z_{cpl}\) found from numerical experiments imply uniform coupling depths are possible if upper-plate thickness are globally uniform. The surface heat flow dataset compiled by Wada \& Wang (\protect\hyperlink{ref-wada2009}{2009}) (Table \ref{tab:segs}) shows average backarc surface heat flow are indeed similar among active subduction zones---implying a narrow distribution of coupling depths (Figure \ref{fig:multiv}d). Much of their dataset is based on Currie \& Hyndman (\protect\hyperlink{ref-currie2006}{2006}), who estimate upper-plate thickness for 10 circum-Pacific subduction zones of 50-60 km (defined by the 1200 \(^{\circ}\)C isotherm). Uniformly thin upper-plate thickness are corroborated by uniformly high heat flow (\textgreater{} 70 mW/m\(^2\)), thermobarometric constraints on mantle xenoliths, and P-wave velocities (\protect\hyperlink{ref-currie2006}{Currie \& Hyndman, 2006}). An attempt is made to further corroborate the uniformity of upper-plate thickness in Chapter \ref{chpt3} by interpolating surface heat flow near active subduction zones.

Although it still curious why upper-plates among subduction zones may have similar thicknesses, one can assume it is likely related to some processes of lithospheric erosion proposed for subarc lithosphere. These include: lithospheric delamination induced by lower crust eclogitization (\protect\hyperlink{ref-sobolev2005}{Sobolev \& Babeyko, 2005}), small-scale convection caused by hydration-induced mantle wedge weakening (\protect\hyperlink{ref-arcay2006}{Arcay et al., 2006}), thermal erosion (\protect\hyperlink{ref-england2010}{England \& Katz, 2010}), mechanical weakening by percolating melts (\protect\hyperlink{ref-gerya2011}{Gerya \& Meilick, 2011}), and subarc foundering of magmatic cumulates (\protect\hyperlink{ref-jull2001}{Jull \& Kelemen, 2001}). Most of these mechanisms are thus strongly related to mantle wedge hydration, melting, and melt transport toward volcanic arcs.

The metamorphic rock record may also imply consistency among coupling depths in subduction zones. For example, the demise of a serpentine channel and onset of coupling may provide a natural barrier such that rocks are more likely to be exhumed from within the channel than from below it. The relative abundance of blueschists and eclogites should then be greater for pressures below estimated coupling depths (approximately 2.4 GPa or 70-80 km) than above them.

\hypertarget{conclusions}{%
\section{Conclusions}\label{conclusions}}

Three important results are highlighted in this study:

\begin{enumerate}
\def\labelenumi{\arabic{enumi}.}
\tightlist
\item
  Coupling depth is stabilized near the base of the upper-plate lithosphere by competing dynamic feedbacks regulating heat transport, serpentine dehydration, and mechanical coupling in the upper-plate mantle.
\item
  A simple expression fitted to coupling depths observed in numerical experiments allows the coupling depths to be estimated for active subduction zones by inverting upper-plate thickness from surface heat flow.
\item
  Uniform surface heat flow in circum-Pacific subduction zones (\protect\hyperlink{ref-currie2006}{Currie \& Hyndman, 2006}; \protect\hyperlink{ref-wada2009}{Wada \& Wang, 2009}) may indicate uniform coupling depths at approximately 80 km.
\end{enumerate}

Questions remain, however, including: how do warm (thin) upper-plates persist over 100's of kilometers behind arcs and throughout the lifespan of subduction zones? How abruptly are dehydration reaction occurring along the subduction interface? How can expressions like Equation \eqref{eq:zCpl} be improved using natural datasets? Each of these questions may be considered for future research.

\cleardoublepage

\hypertarget{chpt3}{%
\chapter{A Comparison of Surface Heat Flow Interpolations Near Subduction Zones}\label{chpt3}}

\markboth{Chapter 3: Heat Flow Interpolations}{Chapter 3: Heat Flow Interpolations}

\hypertarget{chpt3Abstract}{%
\section{Abstract}\label{chpt3Abstract}}

The magnitude and spatial extent of heat fluxing through the Earth's surface depend on the integrated thermal state of Earth's lithosphere (conductive heat loss) plus heat \DIFdelbegin \DIFdel{generating }\DIFdelend \DIFaddbegin \DIFadd{generation }\DIFaddend (e.g.~from seismic cycles and radioactive decay) and heat \DIFdelbegin \DIFdel{transferring subsurface processes }\DIFdelend \DIFaddbegin \DIFadd{transfer via advection }\DIFaddend (e.g.~by fluids, melts, and plate motions). Surface heat flow observations are thus critically important for understanding the thermo-mechanical evolution of subduction zones. Yet evaluating regional surface heat flow patterns across tectonic features remains difficult due to sparse observations irregularly-spaced at distances from 10\(^{-1}\) to 10\(^3\) km. Simple sampling methods (e.g.~1D trench-perpendicular transects across subduction zones) can provide excellent location-specific information but are insufficient for evaluating lateral (along-strike) variability. Robust interpolation methods are therefore required. This study compares two interpolation methods based on fundamentally different principles, \emph{Similarity} and \emph{Kriging}, to (1) investigate the spatial variability of surface heat flow near \DIFdelbegin \DIFdel{17 }\DIFdelend \DIFaddbegin \DIFadd{13 }\DIFaddend presently active subduction zone segments and (2) provide insights into the reliability of such methods for subduction zone research. Similarity and Kriging predictions show diverse surface heat flow distributions and profiles among subduction zone segments and broad systematic changes along strike. \DIFdelbegin \DIFdel{Such a Kaleidoscope of }\DIFdelend \DIFaddbegin \DIFadd{Median upper-plate surface heat flow varies 25.4 mW/m\(^2\) for Similarity and 42.4 mW/m\(^2\) for Kriging within segments, on average, and up to 40.7 mW/m\(^2\) for Similarity and up to 90.5 mW/m\(^2\) for Kriging among segments. Diverse }\DIFaddend distributions and profiles within \DIFdelbegin \DIFdel{, and among , }\DIFdelend \DIFaddbegin \DIFadd{and among }\DIFaddend subduction zone segments \DIFdelbegin \DIFdel{implies various degrees of (dis)continuity exist in terms of lithospheric thermal structure}\DIFdelend \DIFaddbegin \DIFadd{imply spatial heterogeneities in lithospheric thickness}\DIFaddend , subsurface geodynamics, \DIFaddbegin \DIFadd{or near-surface perturbations, }\DIFaddend and/or \DIFdelbegin \DIFdel{observational density }\DIFdelend \DIFaddbegin \DIFadd{undersampling }\DIFaddend relative to the \DIFdelbegin \DIFdel{local (unknown) spatial variabilityof surface heat flow. Accuracy }\DIFdelend \DIFaddbegin \DIFadd{scale and magnitude of spatial variability. Average accuracy }\DIFaddend rates of Similarity \DIFdelbegin \DIFdel{and Kriging }\DIFdelend \DIFaddbegin \DIFadd{(28.8 mW/m\(^2\)) and Kriging (32.2 mW/m\(^2\)) }\DIFaddend predictions are comparable among subduction zone segments, \DIFdelbegin \DIFdel{on average, }\DIFdelend implying either method is viable for subduction zone research. Importantly, anomalies and methodological idiosyncrasies identified by comparing Similarity and Kriging can aid in developing more accurate regional surface heat flow interpolations and identifying future survey targets.

\hypertarget{chpt3Intro}{%
\section{Introduction}\label{chpt3Intro}}

The amount of heat escaping Earth's surface depends on the integrated thermal state of Earth's lithosphere, plus heat-transferring and heat-generating subsurface processes like hydrothermal circulation, radioactive decay, fault motion, and mantle convection (\protect\hyperlink{ref-currie2004}{Currie et al., 2004}; \protect\hyperlink{ref-currie2006}{Currie \& Hyndman, 2006}; \protect\hyperlink{ref-fourier1827}{Fourier, 1827}; \protect\hyperlink{ref-furlong2013}{Furlong \& Chapman, 2013}; \protect\hyperlink{ref-furukawa1993}{Furukawa, 1993}; \protect\hyperlink{ref-gao2014}{Gao \& Wang, 2014}; \protect\hyperlink{ref-hasterok2013}{Hasterok, 2013}; \protect\hyperlink{ref-hutnak2008}{Hutnak et al., 2008}; \protect\hyperlink{ref-kelvin1863}{Kelvin, 1863}; \protect\hyperlink{ref-kerswell2021}{Kerswell et al., 2021}; \protect\hyperlink{ref-parsons1977}{Parsons \& Sclater, 1977}; \protect\hyperlink{ref-pollack1977}{Pollack \& Chapman, 1977}; \protect\hyperlink{ref-rudnick1998}{Rudnick et al., 1998}; \protect\hyperlink{ref-stein1992}{Stein \& Stein, 1992}, \protect\hyperlink{ref-stein1994}{1994}; \protect\hyperlink{ref-wada2009}{Wada \& Wang, 2009}). Surface heat flow observations are thus critically important for understanding lithospheric evolution, crustal deformation and seismic hazards, groundwater hydrology and environmental impacts, and exploration of economic resources (e.g.~hydrocarbon, mineral, and geothermal energy). Monumental efforts to take tens of thousands of continental and oceanic surface heat flow measurements (\DIFdelbegin \DIFdel{see }%DIFDELCMD < \protect\hyperlink{tglobe}{ThermoGlobe References} %%%
\DIFdel{with 1,393 citations}\DIFdelend \DIFaddbegin \DIFadd{from more than 1000 individual studies}\DIFaddend ) and compile them into databases (\protect\hyperlink{ref-hasterok2008}{Hasterok \& Chapman, 2008}; \protect\hyperlink{ref-jennings2021}{Jennings \& Hasterok, 2021}; \protect\hyperlink{ref-lucazeau2019}{Lucazeau, 2019}; \protect\hyperlink{ref-pollack1993}{Pollack et al., 1993}) enable multi-disciplinary investigations of lithospheric and crustal processes.

The most recent global surface heat flow database, \emph{ThermoGlobe} (\protect\hyperlink{ref-jennings2021}{Jennings \& Hasterok, 2021}; \protect\hyperlink{ref-lucazeau2019}{Lucazeau, 2019}), currently contains 69,729 observations. Yet the spatial coverage near subduction zones is relatively sparse (n = 13,360 for this study) and \DIFdelbegin \DIFdel{highly-irregular }\DIFdelend \DIFaddbegin \DIFadd{highly irregular }\DIFaddend at the regional scale (10\(^2\) to 10\(^3\) km, see Figure \ref{fig:globalhfComp} \& Table \ref{tab:hfSummaryTable}). Note that ThermoGlobe includes many datasets of high-resolution surface heat flow arrays, often collocated with seismic arrays, that span \(\leq\) 10\(^2\) km in total length. While high-resolution surveys can resolve fine spatial variations in surface heat flow at the study site scale, probing surface heat flow variations along a subduction zone segment requires evaluation of ThermoGlobe data across larger-scales. Thus, the primary challenge in quantifying segment-scale surface heat flow variations is evaluating sparse, irregularly-spaced observations separated by distances from 10\(^{-1}\) to 10\(^3\) km. This study solves the problem of irregularly-spaced data by (1) independently applying two interpolation methods to ThermoGlobe data near subduction zone segments, and then (2) regularly sampling the interpolated surface heat flow across large adjacent regions in the upper-plate (upper-plate sectors).



\begin{figure}[htbp]

{\centering \includegraphics[width=1\linewidth,]{assets/figs/chpt3/ThermoGlobeBufferComp} 

}

\caption[Regional surface heat flow near subduction zone segments]{Regional surface heat flow near subduction zone segments. (a) ThermoGlobe data from Lucazeau (\protect\hyperlink{ref-lucazeau2019}{2019}) cropped within 1000 km-radius buffers around \DIFdelbeginFL \DIFdelFL{17 }\DIFdelendFL \DIFaddbeginFL \DIFaddFL{13 }\DIFaddendFL active subduction zone segments show uneven regional coverage. For example, note the relatively high observational density in the NW Pacific compared to other regions. (b) In contrast, a Similarity interpolation cropped within the same buffers presents an evenly-distributed approximation of regional surface heat flow. Similarity interpolation from Lucazeau (\protect\hyperlink{ref-lucazeau2019}{2019}). Subduction zone boundaries (bold white lines) defined by Syracuse \& Abers (\protect\hyperlink{ref-syracuse2006}{2006}). Plate boundaries (bold black lines) defined by Lawver et al. (\protect\hyperlink{ref-lawver2018}{2018}). AA: Alaska Aleutians, AN: Andes, CA: Central America, KM: Kamchatka Marianas, KR: Kyushu Ryukyu, LA: Lesser Antilles, NBS: New Britain Solomon, NP: N Philippines, SBS: Sumatra Banda Sea, SC: Scotia, SP: S Philippines, TNZ: Tonga New Zealand, VN: Vanuatu.}\label{fig:globalhfComp}
\end{figure}

The two interpolation methods compared in this study, \emph{Kriging} and \emph{Similarity}, are chosen because they represent end-member approaches based on fundamentally different principles and mathematical frameworks. Their comparative differences, therefore, may be important for understanding lithospheric thermal structure, identifying surface heat flow anomalies, evaluating practical limitations of each approach, and developing new methods combining the strengths of Kriging and Similarity techniques.

\clearpage

The rationale for applying Kriging and Similarity methods \DIFdelbegin \DIFdel{are }\DIFdelend \DIFaddbegin \DIFadd{is }\DIFaddend embodied in the First and Third Laws of Geography, respectively:

\begin{quote}
\uline{Three Laws of Geography}:\\
\strut \\
\hspace*{0.333em}1. Everything is related, but nearer things are more related\\
\hspace*{0.333em}\hspace*{0.333em}\hspace*{0.333em}\hspace*{0.333em}(\protect\hyperlink{ref-krige1951}{Krige, 1951}; \protect\hyperlink{ref-matheron1963}{Matheron, 1963})\\
\strut \\
\hspace*{0.333em}2. Geographic phenomena are inherently heterogeneous\\
\hspace*{0.333em}\hspace*{0.333em}\hspace*{0.333em}\hspace*{0.333em}(\protect\hyperlink{ref-goodchild2004}{Goodchild, 2004})\\
\strut \\
\hspace*{0.333em}3. Localities with similar geographic configurations share other attributes\\
\hspace*{0.333em}\hspace*{0.333em}\hspace*{0.333em}\hspace*{0.333em}(\protect\hyperlink{ref-zhu2018}{Zhu et al., 2018})
\end{quote}

Generally speaking, the spatial continuity of surface heat flow reflects variations in lithospheric thermal structure and heat-transferring processes (neglecting variations in radiogenic heat production). For example, broad regions of low surface heat flow on continents outline cratons (\protect\hyperlink{ref-nyblade1993}{Nyblade \& Pollack, 1993}), anomalously low surface heat flow in oceanic crust \DIFdelbegin \DIFdel{imply }\DIFdelend \DIFaddbegin \DIFadd{implies }\DIFaddend significant heat extraction by seawater (\protect\hyperlink{ref-fisher2000}{Fisher \& Becker, 2000}; \protect\hyperlink{ref-hasterok2011}{Hasterok et al., 2011}; \protect\hyperlink{ref-hutnak2008}{Hutnak et al., 2008}; \protect\hyperlink{ref-stein1994}{Stein \& Stein, 1994}), and trench-orthogonal surface heat flow profiles imply uniform upper-plate lithospheric thickness (\protect\hyperlink{ref-currie2004}{Currie et al., 2004}; \protect\hyperlink{ref-currie2006}{Currie \& Hyndman, 2006}; \protect\hyperlink{ref-hyndman2005}{Hyndman et al., 2005}) and mechanical coupling depths (\protect\hyperlink{ref-furukawa1993}{Furukawa, 1993}; \protect\hyperlink{ref-kerswell2021}{Kerswell et al., 2021}; \protect\hyperlink{ref-wada2009}{Wada \& Wang, 2009}) among subduction zones. For Kriging, such patterns and anomalies may be resolved (assuming adequate observational coverage) because Kriging estimation is inherently dependent on the spatial continuity of observed surface heat flow.

In contrast, Similarity may \DIFdelbegin \DIFdel{(or may not) }\DIFdelend impose different patterns than Kriging because the method only depends on the similarity between two localities in terms of their \emph{geographic configuration} (the makeup and structure of geographic variables over some spatial \DIFdelbegin \DIFdel{neighbourhood }\DIFdelend \DIFaddbegin \DIFadd{neighborhood }\DIFaddend around a point, \protect\hyperlink{ref-zhu2018}{Zhu et al., 2018}). Rather than interpolating (\emph{sensu stricto}) like Kriging, Similarity predicts surface heat flow by comparing geographic, geologic, geochronologic, and geophysical information between a target point and the entire ThermoGlobe dataset (see \protect\hyperlink{ref-goutorbe2011}{Goutorbe et al., 2011} for method details). In other words, Similarity predictions are fundamentally geologically-reasoned estimates of surface heat flow. For example, two localities have similar surface heat flow if they have similar bathymetry, lithology, proximity to active or ancient orogens, seafloor age, upper mantle shear wave velocity, etc. (\protect\hyperlink{ref-chapman1975}{Chapman \& Pollack, 1975}; \protect\hyperlink{ref-davies2013}{Davies, 2013}; \protect\hyperlink{ref-lee1965}{Lee \& Uyeda, 1965}; \protect\hyperlink{ref-lucazeau2019}{Lucazeau, 2019}; \protect\DIFdelbegin %DIFDELCMD < \hyperlink{ref-sclater1970}{Sclater et al., 1970a}%%%
\DIFdelend \DIFaddbegin \hyperlink{ref-sclater1970}{Sclater \& Francheteau, 1970}\DIFaddend ; \protect\hyperlink{ref-shapiro2004}{Shapiro \& Ritzwoller, 2004}).

This study compares regional Similarity and Kriging interpolations near 13 presently active subduction zones while considering the following questions: (1) how does surface heat flow vary near subduction zones, especially within the upper-plate? (2) How do Kriging and Similarity predictions compare? (3) What do the differences (if any) imply about geodynamic variability among active subduction zones? First, ordinary Kriging is applied to ThermoGlobe data near 13 presently active subduction zone segments (defined by \protect\hyperlink{ref-syracuse2006}{Syracuse \& Abers, 2006}). Kriging predictions are then directly compared (point-by-point) to Similarity predictions from a previous global-scale study by Lucazeau (\protect\hyperlink{ref-lucazeau2019}{2019}). Interpolation comparisons yield a \DIFdelbegin \DIFdel{Kaleidoscope of predicted }\DIFdelend \DIFaddbegin \DIFadd{variety of }\DIFaddend upper-plate surface heat flow distributions and profiles. Potential implications of mixed upper-plate profiles are discussed, especially with respect to uniform lithospheric thickness (e.g. \protect\hyperlink{ref-currie2004}{Currie et al., 2004}; \protect\hyperlink{ref-currie2006}{Currie \& Hyndman, 2006}; \protect\hyperlink{ref-hyndman2005}{Hyndman et al., 2005}).

\hypertarget{chpt3Methods}{%
\section{Methods}\label{chpt3Methods}}

\hypertarget{the-thermoglobe-database}{%
\subsection{The ThermoGlobe Database}\label{the-thermoglobe-database}}

The ThermoGlobe database is available from the supplementary material of Lucazeau (\protect\hyperlink{ref-lucazeau2019}{2019}) and is accessible online at \url{http://heatflow.org} (\protect\hyperlink{ref-jennings2021}{Jennings \& Hasterok, 2021}). It currently contains 69,729 data points, their locations in latitude/longitude, and important metadata---including a data quality rank (\texttt{Code\ 6}) from A (high-quality) to D (low-quality). \DIFdelbegin \DIFdel{The reader is referred to }\DIFdelend Lucazeau (\protect\hyperlink{ref-lucazeau2019}{2019}) and \url{http://heatflow.org} \DIFdelbegin \DIFdel{for }\DIFdelend \DIFaddbegin \DIFadd{provide }\DIFaddend details on compilation, references, historical perspective on ThermoGlobe, and previous compilations. ThermoGlobe is the most recent database available, has been carefully compiled, and is open-access.

Like Lucazeau (\protect\hyperlink{ref-lucazeau2019}{2019}), 4,661 poor quality observations (\texttt{Code\ 6} = D), 350 data points without heat flow observations, and 2 without geographic information \DIFdelbegin \DIFdel{are }\DIFdelend \DIFaddbegin \DIFadd{were }\DIFaddend excluded from the analysis. Note that quality control of such a large dataset is an ongoing endeavor and 11,712 observations currently have an undetermined quality (\texttt{Code\ 6} = Z). Duplicate observations at the same location \DIFdelbegin \DIFdel{are }\DIFdelend \DIFaddbegin \DIFadd{were }\DIFaddend parsed (to avoid singular covariance matrices during Kriging) by selecting only the best quality measurement. If duplicate measurements \DIFdelbegin \DIFdel{are }\DIFdelend \DIFaddbegin \DIFadd{were }\DIFaddend of equal quality, one \DIFdelbegin \DIFdel{is }\DIFdelend \DIFaddbegin \DIFadd{was }\DIFaddend randomly chosen. Finally, surface heat flow observations for Kriging and Similarity predictions \DIFdelbegin \DIFdel{are }\DIFdelend \DIFaddbegin \DIFadd{were }\DIFaddend both limited to the range (0 - 250{]} mW/m\(^2\). Observations outside of the range (0 - 250{]} mW/m\(^2\) are considered anomalous \DIFdelbegin \DIFdel{high }\DIFdelend (e.g.~collected near geothermal systems, \protect\hyperlink{ref-lucazeau2019}{Lucazeau, 2019}) and unrepresentative of lithospheric-scale thermal structure. \DIFdelbegin \DIFdel{Moreover, anomalously high }\DIFdelend \DIFaddbegin \DIFadd{Anomalous }\DIFaddend observations constitute a small fraction of measurements (4,883 out of 69,729) forming long tails on either side of the global surface heat flow distribution. The final dataset used for Kriging contains 13,360 observations after filtering for quality, missing values, and heat flow range, parsing duplicate pairs, and cropping within subduction zone buffers (Figure \ref{fig:hfSummaryPlot} \& Table \ref{tab:hfSummaryTable}).

\hypertarget{map-projection-and-interpolation-grid}{%
\subsection{Map Projection and Interpolation Grid}\label{map-projection-and-interpolation-grid}}

All geographic operations, including transformation, cropping, Kriging, and comparing interpolations, \DIFdelbegin \DIFdel{are }\DIFdelend \DIFaddbegin \DIFadd{were }\DIFaddend performed using general-purpose functions in the R package \texttt{sf} (\protect\hyperlink{ref-pebesma2018}{Pebesma, 2018}). ThermoGlobe data and Similarity interpolations from Lucazeau (\protect\hyperlink{ref-lucazeau2019}{2019}) \DIFdelbegin \DIFdel{are }\DIFdelend \DIFaddbegin \DIFadd{were }\DIFaddend transformed into a Pacific-centered Robinson coordinate reference system using the open source geographic transformation software \texttt{PROJ} (\protect\hyperlink{ref-proj2021}{PROJ contributors, 2021}). The transformation is defined by the \texttt{proj4} string \texttt{"+proj=robin\ +lon\_0=-155\ +lon\_wrap=-155\ +x\_0=0\ +y\_0=0\ +ellps=WGS84\ +datum=WGS84\ +units=m\ +no\_defs"}. The Kriging domains \DIFdelbegin \DIFdel{are }\DIFdelend \DIFaddbegin \DIFadd{were }\DIFaddend defined by drawing 1000 km-radius buffers around each subduction zone segment defined by Syracuse \& Abers (\protect\hyperlink{ref-syracuse2006}{2006}). Target locations for Kriging (the interpolation grid) \DIFdelbegin \DIFdel{are }\DIFdelend \DIFaddbegin \DIFadd{were }\DIFaddend defined across the same grid used by Lucazeau (\protect\hyperlink{ref-lucazeau2019}{2019}) to compute point-by-point differences with their Similarity interpolation (Figure \ref{fig:domainConstruct}). In this case, grid point locations represent the centroids of 0.5\(^{\circ}\) \(\times\) 0.5\(^{\circ}\) unequal-area grid cells encompassing the entire globe.



\begin{figure}[htbp]

{\centering \includegraphics[width=1\linewidth,]{assets/figs/chpt3/SumatraBandaSeaComp} 

}

\caption[Example of an interpolation domain]{Example of an interpolation domain constructed around the Sumatra Banda Sea segment. ThermoGlobe data (colored squares; from \protect\hyperlink{ref-lucazeau2019}{Lucazeau, 2019}) are cropped within a 1000 km-radius buffer (thin black line) surrounding the segment boundary (bold white line). Target locations for interpolation are defined by the intersections of a 0.5\(^{\circ}\) \(\times\) 0.5\(^{\circ}\) grid (fine black mesh; defined by \protect\hyperlink{ref-lucazeau2019}{Lucazeau, 2019}) cropped to the same buffer. Note that Sumatra Banda Sea is one of the more densely sampled regions, yet still has considerable observational gaps. Segment boundary and volcanoes (gold diamonds) defined by Syracuse \& Abers (\protect\hyperlink{ref-syracuse2006}{2006}). Plate boundaries (bold black lines) defined by Lawver et al. (\protect\hyperlink{ref-lawver2018}{2018}). AUP: Australian Plate, PSP: Philippine Sea Plate, SNP: Sunda Plate.}\label{fig:domainConstruct}
\end{figure}

\hypertarget{kriging}{%
\subsection{Kriging}\label{kriging}}

Kriging is derived from the theory of \emph{regionalized variables} (\protect\hyperlink{ref-matheron1963}{Matheron, 1963}, \protect\hyperlink{ref-matheron2019}{2019}) and estimates an unknown quantity as a linear combination of all nearby known quantities. Kriging is a three-step process that involves: 1) estimating an experimental variogram \(\hat{\gamma}(h)\) that characterizes the spatial \DIFdelbegin \DIFdel{(dis)}\DIFdelend continuity of some quantity within the Kriging domain, 2) fitting one of many variogram models \(\gamma(h)\) to the experimental variogram, and 3) directly solving a linear system of Kriging equations to predict unknown quantities at arbitrary target locations (\protect\hyperlink{ref-cressie2015}{Cressie, 2015}; \protect\hyperlink{ref-krige1951}{Krige, 1951}). The general-purpose functions defined in the R package \texttt{gstat} (\protect\hyperlink{ref-graler2016}{Gräler et al., 2016}; \protect\hyperlink{ref-pebesma2004}{Pebesma, 2004}) \DIFdelbegin \DIFdel{are }\DIFdelend \DIFaddbegin \DIFadd{were }\DIFaddend used to perform all three Kriging steps. The first step \DIFdelbegin \DIFdel{is computing }\DIFdelend \DIFaddbegin \DIFadd{computed }\DIFaddend an experimental variogram (after \protect\hyperlink{ref-bardossy1997}{Bárdossy, 1997}):
\begin{equation}
  \begin{aligned}
    \hat{\gamma}(h) &= \frac{1}{2N(h)}\sum_{N(h)}^{}[Z(u_i) - Z(u_j)]^2 \\
    h &= |u_i - u_j| \\
  \end{aligned}
  \label{eq:variogram}
\end{equation}
where \(Z(u_i)\) and \(Z(u_j)\) are observations located at \(u_i\) and \(u_j\) separated by a lag of \(h\), and \(N(h)\) is the number of observations separated by a given lag distance. The experimental variogram \(\hat{\gamma}(h)\) evaluates the spatial continuity of the set of observations \(Z(u)\) by computing the average variance among pairs of observations separated by increasingly greater lag distances. By convention the average variance is halved and called ``semivariance''.

For regularly-spaced data, lag distances are simply multiples of the grid-step distance, but irregularly-spaced data must be treated differently. In the case of irregularly-spaced surface heat flow in this study, a binwidth \(\delta\) \DIFdelbegin \DIFdel{is }\DIFdelend \DIFaddbegin \DIFadd{was }\DIFaddend defined as:
\begin{equation}
  \begin{aligned}
    &\delta = \frac{\max(h)\ (n_{lag}+shift)}{n_{lag}\ cut} \\
    &N(h) = \#\{h \  \in \  [h - \delta,\  h + \delta)\}
  \end{aligned}
  \label{eq:binwidth}
\end{equation}
where \(\max(h)\) is the maximum separation distance within the Kriging domain, \(n_{lag}\) is the number of lags used to evaluate the variogram, \(shift\) is a lag shift constant that shifts the variogram by an integer number of binwidths, \(cut\) is a lag cutoff constant (by convention \(cut\) = 3). \DIFdelbegin \DIFdel{Then }\DIFdelend \(N(h)\) is the number of observations that fall within \([h-\delta,\ h+\delta)\).

This study \DIFdelbegin \DIFdel{applies }\DIFdelend \DIFaddbegin \DIFadd{applied }\DIFaddend ordinary Kriging with isotropic variogram models (assumes semivariance is spatially invariant) to surface heat flow data projected onto a smooth sphere (neglects elevation). Kriging \DIFdelbegin \DIFdel{is }\DIFdelend \DIFaddbegin \DIFadd{was }\DIFaddend applied locally (to avoid violating stationarity assumptions) by evaluating only the nearest \(n_{max}\) observations at each target location, where ``nearest'' is defined by the distances between the target location and observations. Therefore, the domain of local Kriging expands or shrinks depending on the local observational density at each target location.

Several variogram parameters influence the Kriging result, including the choice of variogram model, the scope of local Kriging \(n_{max}\), and choice of experimental variogram parameters in Equation \eqref{eq:variogram}. Instead of choosing Kriging parameters by eye (a common practice for fitting variograms) this study \DIFdelbegin \DIFdel{uses }\DIFdelend \DIFaddbegin \DIFadd{used }\DIFaddend a constrained non-linear optimization approach to find optimum values for the variogram parameters \(\{model,\ n_{lag},\ cut,\ n_{max},\ shift\}\). A weighted sum of the \gls{rmse} evaluated during variogram fitting and the \gls{rmse} evaluated between Kriging estimates and surface heat flow observations \DIFdelbegin \DIFdel{is }\DIFdelend \DIFaddbegin \DIFadd{was }\DIFaddend used as a cost function to simultaneously optimize variogram and Kriging accuracy (after \protect\hyperlink{ref-li2018}{Li et al., 2018}). The R package \texttt{nloptr} \DIFdelbegin \DIFdel{is }\DIFdelend \DIFaddbegin \DIFadd{was }\DIFaddend used to optimize Kriging parameters by finding a combination of the parameters \(\{model,\ n_{lag},\ cut,\ n_{max},\ shift\}\) that minimizes the cost function. A full description of the Kriging system of equations, underlying assumptions, and optimization methods is presented in Appendix \ref{krigeOpt} with optimization results for all segments and variogram models. All experimental and fitted variograms are in Appendix \ref{interpDiffAppendix} with interpolations for each case not presented in the main text.

\hypertarget{upSectors}{%
\subsection{Upper-Plate Sector Profiles}\label{upSectors}}

Surface heat flow profiles and distributions \DIFdelbegin \DIFdel{are }\DIFdelend \DIFaddbegin \DIFadd{were }\DIFaddend computed for several adjacent upper-plate regions to assess lateral (along-strike) surface heat flow variability. Profiles \DIFdelbegin \DIFdel{are }\DIFdelend \DIFaddbegin \DIFadd{were }\DIFaddend defined by (1) splitting a subduction zone segment (defined by \protect\hyperlink{ref-syracuse2006}{Syracuse \& Abers, 2006}) into 2-14 equidistant parts, (2) defining 500 km-wide single-sided buffers (sectors) around the segment parts, and (3) calculating the orthogonal great circle distance between each surface heat flow prediction (Similarity and Kriging), or observation (ThermoGlobe data), contained within a sector and the segment boundary (trench). Steps (1-3) above closely approximate the projection of surface heat flow onto a 1D trench-orthogonal line at the center of each sector (e.g. \protect\hyperlink{ref-currie2004}{Currie et al., 2004}; \protect\hyperlink{ref-currie2006}{Currie \& Hyndman, 2006}; \protect\hyperlink{ref-hyndman2005}{Hyndman et al., 2005}; \protect\hyperlink{ref-morishige2020}{Morishige \& Kuwatani, 2020}; \protect\hyperlink{ref-wada2009}{Wada \& Wang, 2009}). Profiles \DIFdelbegin \DIFdel{are }\DIFdelend \DIFaddbegin \DIFadd{were }\DIFaddend smoothed by a three-point running average and fit with a local non-parametric regression curve (LOESS, \protect\hyperlink{ref-cleveland1988}{Cleveland \& Devlin, 1988}).

\hypertarget{interpolation-accuracy}{%
\subsection{Interpolation Accuracy}\label{interpolation-accuracy}}

Previous studies evaluate global Similarity accuracy by either applying cross-validation during the interpolation process (e.g. \protect\hyperlink{ref-goutorbe2011}{Goutorbe et al., 2011}) or directly computing residuals between predictions and surface heat flow observations after interpolation (e.g. \protect\hyperlink{ref-lucazeau2019}{Lucazeau, 2019}). Generally speaking, ranking models by comparing cross-validation results is typically preferred over directly comparing residuals for two reasons: (1) cross-validation gives a sense of how a model behaves when \DIFdelbegin \DIFdel{given }\DIFdelend \DIFaddbegin \DIFadd{presented with }\DIFaddend \emph{new} data (not part of the training data set used to fit the model), and (2) cross-validation can distinguish models that are overfit (high-accuracy due to ``memorizing'' the training data set). However, because Similarity is a non-parametric approach that does not involve ``fitting'' models to sets of training data (i.e.~no residuals or cost function to minimize), cross-validating Similarity predictions does not effectively distinguish overfitting, nor does it give a sense of how well Similarity will behave when \DIFdelbegin \DIFdel{given }\DIFdelend \DIFaddbegin \DIFadd{presented with }\DIFaddend new data. Similarity, as typically implemented (e.g.~by \protect\hyperlink{ref-goutorbe2011}{Goutorbe et al., 2011}; \protect\hyperlink{ref-lucazeau2019}{Lucazeau, 2019}), always considers the entire global dataset of surface heat flow observations to make predictions at unknown target locations. Therefore leaving out a few observations has little effect. For example, even removing an entire continent's worth of surface heat flow data does not significantly affect the outcome of Similarity predictions compared to Similarity interpolations including the full ThermoGlobe dataset (see Figure 9 in \protect\hyperlink{ref-lucazeau2019}{Lucazeau, 2019}).

\DIFdelbegin \DIFdel{In order to }\DIFdelend \DIFaddbegin \DIFadd{To }\DIFaddend better compare Kriging (a parametric model fit to training data) and Similarity (a non-parametric model with prescribed weights), this study \DIFdelbegin \DIFdel{computes }\DIFdelend \DIFaddbegin \DIFadd{computed }\DIFaddend interpolation accuracies using a direct approach (similar to \protect\hyperlink{ref-lucazeau2019}{Lucazeau, 2019}) for both methods. More specifically, the \gls{rmse} \DIFdelbegin \DIFdel{is }\DIFdelend \DIFaddbegin \DIFadd{was }\DIFaddend computed for each surface heat flow observation by comparing the observed value to the nearest predicted value made across the 0.5\(^{\circ}\) \(\times\) 0.5\(^{\circ}\) interpolation grid. Compared to cross-validation, this direct method provides a more robust and effective comparison between Similarity and Kriging accuracies\DIFdelbegin \DIFdel{in this case}\DIFdelend . However, the direct approach is particularly susceptible to ignoring overfitting during Kriging estimation. Therefore caution must be taken to avoid misinterpreting unusually low Kriging error rates as indication of a more accurate model\DIFdelbegin \DIFdel{in all cases}\DIFdelend .

\hypertarget{chpt3Results}{%
\section{Results}\label{chpt3Results}}

\hypertarget{interpDiff}{%
\subsection{Similarity and Kriging Interpolations}\label{interpDiff}}

\hypertarget{global-differences}{%
\subsubsection{Global Differences}\label{global-differences}}

Global differences between Similarity and Kriging interpolations across all subduction zone segments are centered near zero with median differences ranging from -1 to 14 mW/m\(^2\), but broadly distributed with \glspl{iqr} from 15 to 50 mW/m\(^2\) and long tails extending from -1000 to 205 mW/m\(^2\) (Table \ref{tab:diffSummaryTable}). Distributions of interpolation differences are either approximately symmetrical, or slightly right-skewed (Figure \ref{fig:diffSummaryPlot}). Slight right skew and positive median differences indicate a general tendency to predict higher surface heat flow by Similarity compared to Kriging. However, much of the right skew can be explained by spreading centers, transform faults, and volcanic regions predicted by Similarity that are unresolved by Kriging due to lack of observations in those regions (e.g.~Scotia), and/or regions of anomalously-low surface heat flow within oceanic crust resolved by Kriging that are effectively overlooked by Similarity (e.g.~Central America).

\hypertarget{regional-differences}{%
\subsubsection{Regional Differences}\label{regional-differences}}

Examples given in this section highlight the range of differences observed between Similarity and Kriging interpolations across subduction zone segments with anomalously-low surface heat flow within oceanic crust (Central America), with complex tectonic boundaries (Vanuatu), with excellent observational coverage (Kyushyu Ryukyu), and with very few observations (Scotia). Refer to Appendix \ref{interpDiffAppendix} for the remaining set of visualized interpolations.

\hypertarget{central-america}{%
\paragraph{Central America}\label{central-america}}

Distance to plate boundaries and the age of oceanic lithosphere are key geologic proxies exerting strong influence on Similarity predictions (\protect\hyperlink{ref-goutorbe2011}{Goutorbe et al., 2011}; \protect\hyperlink{ref-shapiro2004}{Shapiro \& Ritzwoller, 2004}; \protect\hyperlink{ref-stein1992}{Stein \& Stein, 1992}). Consequently, Similarity predicts high surface heat flow along the arms of the Galápagos triple junction and within the (young) converging Cocos Plate near Central America (Figure \ref{fig:centralAmericaDiff}). Kriging, on the other hand, predicts relatively low surface heat flow within the Cocos Plate despite its young age and close proximity to the nearby spreading centers. This is explained by anomalously-low surface heat flow observed within the Cocos Plate that is interpreted as regional modification of the expected surface heat flow by hydrothermal circulation of seawater (\protect\hyperlink{ref-hutnak2008}{Hutnak et al., 2008}). These widespread observations of low surface heat flow constrain Kriging predictions to similarly low values within the Cocos Plate. \DIFdelbegin \DIFdel{Moreover, disagreement }\DIFdelend \DIFaddbegin \DIFadd{Disagreement }\DIFaddend between Similarity and Kriging appears more subdued within the upper-plate, yet Similarity still predicts slightly higher surface heat flow on average.



\begin{figure}[htbp]

{\centering \includegraphics[width=1\linewidth,]{assets/figs/chpt3/CentralAmericaDiffComp} 

}

\caption[Similarity and Kriging interpolations for Central America]{Similarity and Kriging interpolations for Central America. (a) Relatively high surface heat flow is predicted by Similarity within the young Cocos Plate (CP) and along the arms of the Galápagos triple junction (GTJ): the East Pacific Rise (EPR) and Cocos Ridge (CR). In contrast, (b) many anomalously-low surface heat flow observations within the CP (\protect\hyperlink{ref-hutnak2008}{Hutnak et al., 2008}) constrain Kriging predictions to low values. Segment boundary (bold white line) and volcanoes (gold diamonds) defined by Syracuse \& Abers (\protect\hyperlink{ref-syracuse2006}{2006}). Similarity interpolation from Lucazeau (\protect\hyperlink{ref-lucazeau2019}{2019}). Plate boundaries (bold black lines) defined by Lawver et al. (\protect\hyperlink{ref-lawver2018}{2018}).}\label{fig:centralAmericaDiff}
\end{figure}

\hypertarget{vanuatu}{%
\paragraph{Vanuatu}\label{vanuatu}}

The interpolation domain near Vanuatu is characterized by complex tectonic boundaries defining \DIFdelbegin \DIFdel{a number of }\DIFdelend \DIFaddbegin \DIFadd{several }\DIFaddend microplates to the \DIFdelbegin \DIFdel{E }\DIFdelend \DIFaddbegin \DIFadd{east }\DIFaddend of the volcanic arc (Figure \ref{fig:vanuatuDiff}). The resolution of the geologic proxy datasets used to construct Similarity predictions (namely oceanic plate age, upper mantle density anomaly, sediment thickness, and distance to tectonic boundaries) is apparently too coarse to distinguish a small microplate near the northern tip of the Vanuatu segment from the New Hebrides, Balmoral Reef, and Conway Reef microplates. According to Similarity, the entire region is comprised of young oceanic plate with thin sediment cover, and thus is predicted to have uniformly-high surface heat flow. In contrast, excellent observational coverage enables Kriging to clearly distinguish the northern microplate as an anomalously-low surface heat flow region compared to the other microplates. Outside \DIFdelbegin \DIFdel{of }\DIFdelend the cluster of microplates, Kriging predicts lower surface heat flow on average---similar to many other segments.



\begin{figure}[htbp]

{\centering \includegraphics[width=1\linewidth,]{assets/figs/chpt3/VanuatuDiffComp} 

}

\caption[Similarity and Kriging interpolations for Central America]{Similarity and Kriging interpolations for Vanuatu. While (a) Similarity predicts more-or-less uniformly-high surface heat flow within the region defined by many microplates, (b) excellent observational coverage allows Kriging to distinguish the most northern microplate from the New Hebrides Plate (NHP), Balmoral Reef (BR), and Conway Reef (CWR) microplates to the S. The geologic proxy datasets used to construct Similarity interpolations are apparently too coarse to resolve microplate-size features in this case. Segment boundary (bold white line) and volcanoes (gold diamonds) defined by Syracuse \& Abers (\protect\hyperlink{ref-syracuse2006}{2006}). Similarity interpolation from Lucazeau (\protect\hyperlink{ref-lucazeau2019}{2019}). Plate boundaries (bold black lines) defined by Lawver et al. (\protect\hyperlink{ref-lawver2018}{2018}).}\label{fig:vanuatuDiff}
\end{figure}

\hypertarget{kyushu-ryukyu}{%
\paragraph{Kyushu Ryukyu}\label{kyushu-ryukyu}}

The interpolation domain near the Kyushu Ryukyu segment is characterized by a complex juxtaposition of active subduction and volcanism on the margins of the Philippine Sea Plate, and active rifting between the Ryukyu arc and the Eurasian continent (the Okinawa trough, \protect\hyperlink{ref-minami2022}{Minami et al., 2022}). Contrasting oceanic plate ages, topography/bathymetry, sediment thickness, volcanic activity, \DIFaddbegin \DIFadd{and }\DIFaddend active tectonic settings (subduction vs.~rifting) consequently produce a very textured distribution of Similarity predictions throughout the Kyushu Ryukyu domain (Figure \ref{fig:kyushuRyukyuDiff}). For example, Similarity predictions clearly show the influence of multiple volcanic arc chains, plate boundaries, and the age of the subducting oceanic lithosphere. Geologic complexity notwithstanding, excellent coverage of surface heat flow observations throughout the domain enable Kriging predictions to resolve much of the texture predicted by Similarity. Regional Similarity and Kriging differences are small and narrowly distributed near Kyushu Ryukyu (median difference: 4, IQR: 21 mW/m\(^2\)) \DIFdelbegin \DIFdel{compared }\DIFdelend \DIFaddbegin \DIFadd{as compared, for example, }\DIFaddend to Central America (median difference: 12, IQR: 50 mW/m\(^2\); Table \ref{tab:diffSummaryTable}) despite having a comparable number of observations (n = 1,895) as Central America (n = 1,441). While Kriging predictions are smoother overall, both interpolations appear to corroborate each other\DIFdelbegin \DIFdel{. This is especially true }\DIFdelend \DIFaddbegin \DIFadd{, especially }\DIFaddend to the NE of the main Kyushu Ryukyu segment boundary.



\begin{figure}[htbp]

{\centering \includegraphics[width=1\linewidth,]{assets/figs/chpt3/KyushuRyukyuDiffComp} 

}

\caption[Similarity and Kriging interpolations for Kyushyu Ryukyu]{Similarity and Kriging interpolations for Kyushyu Ryukyu. (a) Similarity predicts a textured interpolation that is strongly influenced by multiple volcanic chains along the margins of the Philippine Sea Plate (PSP), contrasting oceanic plate ages, and active rifting in the Okinawa trough (OKT). (b) The Kriging interpolation is generally smoother, but corroborates much of the same texture predicted by Similarity due to relatively high observational density and regularity of observational coverage throughout the domain. Segment boundary (bold white line) and volcanoes (gold diamonds) defined by Syracuse \& Abers (\protect\hyperlink{ref-syracuse2006}{2006}). Similarity interpolation from Lucazeau (\protect\hyperlink{ref-lucazeau2019}{2019}). Plate boundaries (bold black lines) defined by Lawver et al. (\protect\hyperlink{ref-lawver2018}{2018}).}\label{fig:kyushuRyukyuDiff}
\end{figure}

\hypertarget{scotia}{%
\paragraph{Scotia}\label{scotia}}

The Scotia segment illustrates a case where surface heat flow observations are \DIFdelbegin \DIFdel{incredibly }\DIFdelend \DIFaddbegin \DIFadd{extremely }\DIFaddend sparse. Yet Similarity predicts multiple tectonic features including the East Scotia Ridge and the WSW-ENE trending transform boundary separating the Scotia and Sandwich Plates from the Antarctic Plate (Figure \ref{fig:scotiaDiff}). Combinations of geologic proxy datasets enable Similarity to resolve these features despite having very few observations within the interpolation domain. Kriging, on the other hand, shows a high heat flow anomaly more or less in the region of the East Scotia Ridge, and a few low heat flow anomalies on the Antarctic Plate, but does not resolve any structure in a way that is geologically useful. Few surface heat flow observations (n = 25) result in smooth Kriging predictions that approximate the expected \DIFdelbegin \DIFdel{value (mean: }\DIFdelend \DIFaddbegin \DIFadd{mean value (}\DIFaddend 79 mW/m\(^2\)) for most of the domain according to Equation \eqref{eq:linEstimate}.



\begin{figure}[htbp]

{\centering \includegraphics[width=1\linewidth,]{assets/figs/chpt3/ScotiaDiffComp} 

}

\caption[Similarity and Kriging interpolations for Scotia]{Similarity and Kriging interpolations for Scotia. Despite \DIFdelbeginFL \DIFdelFL{incredibly }\DIFdelendFL \DIFaddbeginFL \DIFaddFL{extremely }\DIFaddendFL sparse data (n = 25), (a) Similarity identifies two tectonic features, the East Scotia Ridge (ESR) and a transform fault (TF) separating the Scotia and Sandwich Plates (SP, SAN) from the Antartic Plate (AP). (b) Kriging predicts a high heat flow anomaly in the region of the ESR, and a few low heat flow anomalies in the AP, but otherwise appears featureless due to sparse data. Segment boundary (bold white line) and volcanoes (gold diamonds) defined by Syracuse \& Abers (\protect\hyperlink{ref-syracuse2006}{2006}). Similarity interpolation from Lucazeau (\protect\hyperlink{ref-lucazeau2019}{2019}). Plate boundaries (bold black lines) defined by Lawver et al. (\protect\hyperlink{ref-lawver2018}{2018}).}\label{fig:scotiaDiff}
\end{figure}

\hypertarget{upSectorsResults}{%
\subsubsection{Upper-Plate Sector Samples}\label{upSectorsResults}}

Sampling the interpolation grid and ThermoGlobe data from adjacent upper-plate sectors allows for first-order quantitative evaluation of the along-strike variability in upper-plate surface heat flow. However, ThermoGlobe data within sectors are often too few (n \textless{} 20 observations for 59/100 sectors; Table \ref{tab:sectorSummaryTable}) to compare distributions confidently with other sectors. Therefore, this study compares trench-orthogonal profiles of the dense, regularly-spaced Similarity and Kriging predictions. Generally speaking, distributions of Similarity and Kriging predictions in the upper-plates show a range of overlap and appear to fluctuate systematically across adjacent upper-plate sectors for some subduction zone segments. Moreover, Similarity and Kriging predictions reveal a variety of upper-plate surface heat flow profiles within \DIFdelbegin \DIFdel{, and among , }\DIFdelend \DIFaddbegin \DIFadd{and among }\DIFaddend subduction zone segments (Table \ref{tab:sectorSummaryTable}, Figures \ref{fig:kyushuRyukyuUpper}, \ref{fig:sumatraBandaSeaUpper}, \ref{fig:newBritainSolomonUpper} \& Appendix \ref{lateralDiffAppendix}).

Below are three examples of subduction zone segments that illustrate part of the range of observed upper-plate surface heat flow patterns.

\hypertarget{kyushu-ryukyu-1}{%
\paragraph{Kyushu Ryukyu}\label{kyushu-ryukyu-1}}

Kyushu Ryukyu characterizes a subduction zone segment with relatively consistent upper-plate surface heat flow for thousands of km along-strike. In this case, \emph{consistent} refers to comparable Similarity and Kriging predictions \emph{and} consistent surface heat flow distributions across sectors. That is, medians and \DIFdelbegin \DIFdel{interquartile ranges }\DIFdelend \DIFaddbegin \glspl{iqr} \DIFaddend of Similarity and Kriging predictions overlap relatively well across most sectors---differing by only 6.4 \(\pm\) 10.2 mW/m\(^2\) \DIFdelbegin \DIFdel{on average (Table \ref{tab:sectorSummaryTable} \& Figure \ref{fig:kyushuRyukyuUpper}). However, interquartile ranges of Kriging predictions are consistently larger than Similarity (differing by }\DIFdelend \DIFaddbegin \DIFadd{for medians and }\DIFaddend 19.9 \(\pm\) 34 mW/m\(^2\) \DIFdelbegin \DIFdel{on average), implying a higher degree of variance than Similarity predictions. Notably, upper-plate }\DIFdelend \DIFaddbegin \DIFadd{for }\glspl{iqr}\DIFadd{, on average (Table \ref{tab:sectorSummaryTable} \& Figure \ref{fig:kyushuRyukyuUpper}). Upper-plate }\DIFaddend surface heat flow\DIFaddbegin \DIFadd{, as estimated by Kriging, }\DIFaddend appears to increase systematically from the NE to SW across sectors 8-6 before leveling out through sectors 5-1.

Meanwhile, ThermoGlobe data within Kyushu Ryukyu upper-plate sectors (n = 339) vary considerably. Wide distributions of ThermoGlobe data appear near the trench and at approximately 200 km from the trench, coinciding with the young active rifting in the Okinawa trough (Figure \ref{fig:kyushuRyukyuUpper}). Yet, smoothed trench-orthogonal Similarity and Kriging profiles gently arc through the approximate midrange of ThermoGlobe data. Profile shapes are consistent across sectors and show relatively little spread (\(\leq\) 25 mW/m\(^2\)). All profiles gradually rise from approximately 50 mW/m\(^2\) at the trench to maximums of approximately 75-100 mW/m\(^2\) before gradually decreasing to approximately 75 mW/m\(^2\) at 500 km into the upper-plate.



\begin{figure}[htbp]

{\centering \includegraphics[width=1\linewidth,]{assets/figs/chpt3/kyushuRyukyuUpperPlate} 

}

\caption[Kyushu Ryukyu upper-plate sectors]{Surface heat flow profiles for Kyushu Ryukyu upper-plate sectors. (a) Similarity and Kriging predictions across sectors are largely indistinguishable with overlapping medians and \DIFdelbeginFL \DIFdelFL{interquartile ranges }\DIFdelendFL \DIFaddbeginFL \glspl{iqr} \DIFaddendFL (boxes). (b) Profiles are computed by finding orthogonal distances between the segment boundary (i.e.~the trench, bold black line) and 342 surface heat flow predictions within eight 500 km-wide sectors (colored polygons). Profiles (colored curves with 95\% confidence intervals) are remarkably consistent across sectors for (c) Kriging and (d) Similarity predictions. Colored squares are ThermoGlobe data from Lucazeau (\protect\hyperlink{ref-lucazeau2019}{2019}). Segment boundary and volcanoes (gold diamonds) defined by Syracuse \& Abers (\protect\hyperlink{ref-syracuse2006}{2006}). Plate boundaries (bold black lines) defined by Lawver et al. (\protect\hyperlink{ref-lawver2018}{2018}). Profile curves in (c) are LOESS regressions through three-point running averages (small colored data points).}\label{fig:kyushuRyukyuUpper}
\end{figure}

\hypertarget{sumatra-banda-sea}{%
\paragraph{Sumatra Banda Sea}\label{sumatra-banda-sea}}

Sumatra Banda Sea characterizes a subduction zone segment with moderately consistent upper-plate surface heat flow for thousands of km along-strike. In this case, \emph{moderately consistent} refers to mostly comparable (\DIFdelbegin \DIFdel{overlaping}\DIFdelend \DIFaddbegin \DIFadd{overlapping}\DIFaddend ) Similarity and Kriging predictions that distinctively fluctuate in a similar manner across sectors. That is, medians and \DIFdelbegin \DIFdel{interquartile ranges }\DIFdelend \DIFaddbegin \glspl{iqr} \DIFaddend of Similarity and Kriging predictions overlap well for some sectors, but not others (e.g.~sectors 1, 10, \& 11, Figure \ref{fig:sumatraBandaSeaUpper}). Median Similarity and Kriging predictions differ by 10.7 \(\pm\) 14.2 mW/m\(^2\) on average, and \DIFdelbegin \DIFdel{interquartile ranges }\DIFdelend \DIFaddbegin \glspl{iqr} \DIFaddend differ by 17.3 \(\pm\) 61.2 mW/m\(^2\) on average across all sectors (Table \ref{tab:sectorSummaryTable}). \DIFdelbegin \DIFdel{Notably, }\DIFdelend Similarity and Kriging predictions appear to broadly oscillate between higher and lower surface heat flow across adjacent sectors with a wavelength on the order of several sectors (10\(^3\) km).

Meanwhile, Similarity and Kriging profiles show obvious differences. For example, Similarity predictions are distributed narrowly and increase monotonically from the trench to 500 km into the upper-plate, whereas Kriging profiles generally ramp up more steeply and begin to disperse at approximately 200 km from the trench. Similarity profiles remain narrowly distributed through at least 300 km from the trench, whereas Kriging profiles show up to 25-30 mW/m\(^2\) spread among sectors at 300-500 km from the trench.



\begin{figure}[htbp]

{\centering \includegraphics[width=1\linewidth,]{assets/figs/chpt3/SumatraBandaSeaUpperPlate} 

}

\caption[Sumatra Banda Sea upper-plate sectors]{Surface heat flow profiles for Sumatra Banda Sea upper-plate sectors. (a) Similarity and Kriging predictions across sectors are moderately distinguishable with mostly overlapping \DIFdelbeginFL \DIFdelFL{interquartile ranges}\DIFdelendFL \DIFaddbeginFL \glspl{iqr}\DIFaddendFL , except for sectors 1, 10, \& 11 (boxes). (b) Profiles are computed by finding orthogonal distances between the segment boundary (trench; bold black line) and 870 surface heat flow predictions within ten 500 km-wide sectors (colored polygons). Profiles (colored curves with 95\% confidence intervals) of (c) Kriging predictions show greater overall spread than (d) Similarity profiles (e.g.~\(\geq\) 200 km from the trench), implying nonuniform upper-plate surface heat flow across the segment. Colored squares are ThermoGlobe data from Lucazeau (\protect\hyperlink{ref-lucazeau2019}{2019}). Segment boundary and volcanoes (gold diamonds) defined by Syracuse \& Abers (\protect\hyperlink{ref-syracuse2006}{2006}). Plate boundaries (bold black lines) defined by Lawver et al. (\protect\hyperlink{ref-lawver2018}{2018}). Profile curves in (c) are LOESS regressions through three-point running averages (small colored data points).}\label{fig:sumatraBandaSeaUpper}
\end{figure}

\hypertarget{new-britain-solomon}{%
\paragraph{New Britain Solomon}\label{new-britain-solomon}}

New Britain Solomon characterizes a subduction zone segment with inconsistent upper-plate surface heat flow and poor overlap between Similarity and Kriging predictions. Only one sector (sector 8) shows overlapping \DIFdelbegin \DIFdel{interquartile ranges }\DIFdelend \DIFaddbegin \glspl{iqr} \DIFaddend of Similarity and Kriging predictions, whereas all other sectors strongly diverge (Figure \ref{fig:newBritainSolomonUpper}). For example, median Kriging predictions range by 21.4 mW/m\(^2\) across all sectors, whereas median Similarity predictions range by 42.7 mW/m\(^2\). Moreover, Similarity and Kriging medians across all sectors differ by 32.4 \(\pm\) 50.2 mW/m\(^2\) on average. Notably, opposing wave-like oscillations between higher and lower surface heat flow across adjacent sectors are observed in Similarity and Kriging predictions.

Meanwhile, Similarity and Kriging profiles are obviously distinguishable. For example, Kriging profiles are smooth and closely parallel ThermoGlobe data, whereas Similarity profiles show higher average surface heat flow (Figure \ref{fig:newBritainSolomonUpper}). In contrast to flat Kriging profiles, high surface heat flow regions along Similarity profiles clearly show the influence of certain tectonic features (e.g.~in sector 4, which intersects a volcanic center and ridge segment). Moreover, small confidence intervals around Kriging profiles suggest small uncertainties compared to Similarity. However, Kriging is determined to find the smallest variance solution by definition and can easily overfit the small number (n = 9) of ThermoGlobe data. Divergence between Similarity and Kriging predictions near New Britain Solomon thus appear to be driven by methodological differences and a tendency for Kriging to overfit small sample sets.



\begin{figure}[htbp]

{\centering \includegraphics[width=1\linewidth,]{assets/figs/chpt3/NewBritainSolomonUpperPlate} 

}

\caption[New Britain Solomon upper-plate sectors]{Surface heat flow profiles for New Britain Solomon upper-plate sectors. (a) Similarity and Kriging predictions across sectors are very distinguishable with non-overlapping \DIFdelbeginFL \DIFdelFL{interquartile ranges }\DIFdelendFL \DIFaddbeginFL \glspl{iqr} \DIFaddendFL (boxes). (b) Profiles are computed by finding orthogonal distances between the segment boundary (trench; bold black line) and 163 surface heat flow predictions within five 500 km-wide sectors (colored polygons). Profiles (colored curves with 95\% confidence intervals) of (c) Kriging predictions are lower and show a narrow distribution compared to (d) Similarity profiles. Colored squares are ThermoGlobe data from Lucazeau (\protect\hyperlink{ref-lucazeau2019}{2019}). Segment boundary and volcanoes (gold diamonds) defined by Syracuse \& Abers (\protect\hyperlink{ref-syracuse2006}{2006}). Plate boundaries (bold black lines) defined by Lawver et al. (\protect\hyperlink{ref-lawver2018}{2018}). Profile curves in (c) are LOESS regressions through three-point running averages (small colored data points).}\label{fig:newBritainSolomonUpper}
\end{figure}

\hypertarget{optimum-kriging-parameters}{%
\subsection{Optimum Kriging Parameters}\label{optimum-kriging-parameters}}

Optimized Kriging parameters vary substantially from segment to segment (Table \ref{tab:vgrmSummaryTable}). However, despite a range of domain sizes, observational densities, and diverse plate configurations, Kriging parameters converge on solutions for all Kriging domains (Figure \ref{fig:optTrace}) and show no systematic correlation with cost, with the exception of a negative correlation with the logarithm of the variogram model sill (Figure \ref{fig:vgrmSummaryPlot}). Differences in cost are apparently explained by systematic regional differences in surface heat flow distributions (i.e.~differences in the constant terms \(\sigma_{vgrm}\) and \(\sigma_{interp}\) in Equation \eqref{eq:costExp}) rather than sensitivity to any particular Kriging parameter.

\begin{table}

\caption{\label{tab:vgrmSummaryTable}Optimum variogram models and interpolation accuracy}
\centering
\resizebox{\linewidth}{!}{
\begin{threeparttable}
\begin{tabular}[t]{llrrrrrrrr}
\toprule
\multicolumn{1}{c}{Segment} & \multicolumn{1}{c}{Model} & \multicolumn{1}{c}{Cut} & \multicolumn{1}{c}{Lags} & \multicolumn{1}{c}{Shift} & \multicolumn{1}{c}{$n_{max}$} & \multicolumn{1}{c}{Sill} & \multicolumn{1}{c}{Range} & \multicolumn{1}{c}{$RMSE_S$} & \multicolumn{1}{c}{$RMSE_K$} \\
\cmidrule(l{0pt}r{0pt}){1-1} \cmidrule(l{0pt}r{0pt}){2-2} \cmidrule(l{0pt}r{0pt}){3-3} \cmidrule(l{0pt}r{0pt}){4-4} \cmidrule(l{0pt}r{0pt}){5-5} \cmidrule(l{0pt}r{0pt}){6-6} \cmidrule(l{0pt}r{0pt}){7-7} \cmidrule(l{0pt}r{0pt}){8-8} \cmidrule(l{0pt}r{0pt}){9-9} \cmidrule(l{0pt}r{0pt}){10-10}
 &  &  &  &  &  & $(mW/m^2)^2$ & km & mW/m$^2$ & mW/m$^2$\\
\midrule
\cellcolor{gray!6}{Alaska Aleutians} & \cellcolor{gray!6}{Bes} & \cellcolor{gray!6}{1.0} & \cellcolor{gray!6}{16.3} & \cellcolor{gray!6}{1.0} & \cellcolor{gray!6}{8} & \cellcolor{gray!6}{841} & \cellcolor{gray!6}{77} & \cellcolor{gray!6}{17.6} & \cellcolor{gray!6}{74.6}\\
Andes & Exp & 1.6 & 20.8 & 8.5 & 12 & 4631 & 165 & 52.6 & 34.9\\
\cellcolor{gray!6}{Central America} & \cellcolor{gray!6}{Exp} & \cellcolor{gray!6}{4.9} & \cellcolor{gray!6}{21.2} & \cellcolor{gray!6}{3.9} & \cellcolor{gray!6}{12} & \cellcolor{gray!6}{4683} & \cellcolor{gray!6}{265} & \cellcolor{gray!6}{52.5} & \cellcolor{gray!6}{33.4}\\
Kamchatka Marianas & Sph & 1.7 & 18.5 & 7.5 & 7 & 1787 & 1355 & 33.1 & 31.2\\
\cellcolor{gray!6}{Kyushu Ryukyu} & \cellcolor{gray!6}{Lin} & \cellcolor{gray!6}{3.2} & \cellcolor{gray!6}{19.8} & \cellcolor{gray!6}{3.3} & \cellcolor{gray!6}{8} & \cellcolor{gray!6}{1898} & \cellcolor{gray!6}{183} & \cellcolor{gray!6}{34.5} & \cellcolor{gray!6}{37.8}\\
Lesser Antilles & Lin & 1.5 & 24.2 & 1.1 & 11 & 653 & 77 & 11.5 & 13.3\\
\cellcolor{gray!6}{N Philippines} & \cellcolor{gray!6}{Bes} & \cellcolor{gray!6}{1.4} & \cellcolor{gray!6}{18.3} & \cellcolor{gray!6}{1.0} & \cellcolor{gray!6}{8} & \cellcolor{gray!6}{1258} & \cellcolor{gray!6}{19} & \cellcolor{gray!6}{27.1} & \cellcolor{gray!6}{32.0}\\
New Britain Solomon & Lin & 2.0 & 20.2 & 5.1 & 10 & 693 & 228 & 13.6 & 28.2\\
\cellcolor{gray!6}{S Philippines} & \cellcolor{gray!6}{Lin} & \cellcolor{gray!6}{3.2} & \cellcolor{gray!6}{29.0} & \cellcolor{gray!6}{1.0} & \cellcolor{gray!6}{5} & \cellcolor{gray!6}{1014} & \cellcolor{gray!6}{40} & \cellcolor{gray!6}{25.6} & \cellcolor{gray!6}{22.9}\\
Scotia & Sph & 2.7 & 20.8 & 4.8 & 8 & 3655 & 1766 & 26.5 & 10.9\\
\cellcolor{gray!6}{Sumatra Banda Sea} & \cellcolor{gray!6}{Sph} & \cellcolor{gray!6}{6.6} & \cellcolor{gray!6}{21.0} & \cellcolor{gray!6}{5.1} & \cellcolor{gray!6}{13} & \cellcolor{gray!6}{10598} & \cellcolor{gray!6}{5850} & \cellcolor{gray!6}{18.0} & \cellcolor{gray!6}{20.4}\\
Tonga New Zealand & Lin & 3.7 & 24.9 & 3.6 & 10 & 1293 & 321 & 24.1 & 23.8\\
\cellcolor{gray!6}{Vanuatu} & \cellcolor{gray!6}{Lin} & \cellcolor{gray!6}{1.2} & \cellcolor{gray!6}{20.4} & \cellcolor{gray!6}{2.6} & \cellcolor{gray!6}{11} & \cellcolor{gray!6}{2918} & \cellcolor{gray!6}{286} & \cellcolor{gray!6}{37.1} & \cellcolor{gray!6}{54.6}\\
\bottomrule
\end{tabular}
\begin{tablenotes}
\item \uline{\textit{note}}: showing lowest-cost models from Table \ref{tab:vgrmSummaryTableLong}
\item \uline{\textit{key}}: $n_{max}$: max point-pairs, $RMSE_S$: Similarity accuracy, $RMSE_K$: Kriging accuracy
\end{tablenotes}
\end{threeparttable}}
\end{table}

\hypertarget{similarity-and-kriging-error-rates}{%
\subsection{Similarity and Kriging Error Rates}\label{similarity-and-kriging-error-rates}}

Regional Kriging error rates (ranging from 10.9 to 74.6 mW/m\(^2\)) are very similar to Similarity error rates from the same regions (ranging from 11.5 to 52.6 mW/m\(^2\), Table \ref{tab:vgrmSummaryTable}). Kriging errors can be relatively small compared to Similarity for domains with high observational density (e.g.~Lesser Antilles; n = 3,008, \(\Delta\)RMSE\(_{K-S}\) = \DIFdelbegin \DIFdel{-15.6}\DIFdelend \DIFaddbegin \DIFadd{1.9}\DIFaddend ) but relatively large where observational density is comparatively low (Alaska Aleutians; n = 290, \(\Delta\)RMSE\(_{K-S}\) = 57). \DIFdelbegin \DIFdel{Note the }\DIFdelend \DIFaddbegin \DIFadd{The }\DIFaddend small Kriging error rate computed for Scotia (10.9 mW/m\(^2\)) \DIFdelbegin \DIFdel{that }\DIFdelend likely reflects overfitting of few (n = 25) observations. On average, Kriging error rates are 1.3 times Similarity error rates across all segments. In comparison to previous work, regional Similarity error rates for most subduction zone segments in Table \ref{tab:vgrmSummaryTable} are much higher than the 7 mW/m\(^2\) Similarity error rate reported by Lucazeau (\protect\hyperlink{ref-lucazeau2019}{2019}). However, Similarity error rates in Table \ref{tab:vgrmSummaryTable} are consistent with global Similarity error rates computed by cross-validation on a 1\(^{\circ}\) \(\times\) 1\(^{\circ}\) \DIFaddbegin \DIFadd{grid }\DIFaddend (from 11.6 to 29.0 \(mW/m^{-2}\)) reported previously by Goutorbe et al. (\protect\hyperlink{ref-goutorbe2011}{2011}).

\hypertarget{chpt3Discussion}{%
\section{Discussion}\label{chpt3Discussion}}

\hypertarget{interpDiffDiscussion}{%
\subsection{Comparing Similarity and Kriging Interpolations}\label{interpDiffDiscussion}}

\DIFdelbegin \DIFdel{If the goal is understanding subduction zone thermal structure and geodynamics, comparing }\DIFdelend \DIFaddbegin \DIFadd{Comparing }\DIFaddend two independent interpolation methods has distinct advantages \DIFaddbegin \DIFadd{for understanding subduction zone thermal structure and geodynamics}\DIFaddend . For example, \DIFdelbegin \DIFdel{this study finds }\DIFdelend many cases of Similarity and Kriging predictions \DIFdelbegin \DIFdel{corroborating }\DIFdelend \DIFaddbegin \DIFadd{corroborate }\DIFaddend known, expected, or predicted tectonic features. These include: (1) broad regions of low surface heat flow defining the oceanic plate and forearc along the Kamchatka Marianas segment (Figure \ref{fig:kamchatkaMarianasDiff}), (2) high surface heat flow anomalies defining the volcanic center and transform fault separating the South American Plate and Caribbean Plates near the Lesser Antilles Segment (Figure \ref{fig:lesserAntillesDiff}), (3) the general seafloor thermal structure near the N Philippines segment (Figure \ref{fig:nPhilippinesDiff}), (4) a broad region of high surface heat flow within the NW part of the Sumatra Banda Sea segment upper-plate (Figure \ref{fig:sumatraBandaSeaDiff}), and (5) high surface heat flow defining volcanic arc chains near the Kyushu Ryukyu segment (Figure \ref{fig:kyushuRyukyuDiff}).

While corroboration of known or expected features is \DIFdelbegin \DIFdel{an advantageous outcome }\DIFdelend \DIFaddbegin \DIFadd{advantageous }\DIFaddend when comparing independent interpolation methods, \DIFdelbegin \DIFdel{finding }\DIFdelend inconsistencies between Similarity and Kriging predictions \DIFdelbegin \DIFdel{is at least as }\DIFdelend \DIFaddbegin \DIFadd{are equally }\DIFaddend valuable. For example, \DIFdelbegin \DIFdel{this study finds }\DIFdelend many cases of Similarity and Kriging predictions \DIFdelbegin \DIFdel{identifying }\DIFdelend \DIFaddbegin \DIFadd{identify }\DIFaddend unexpected or poorly resolved tectonic features. These include: (1) much of the thermal structure along the Andes segment (Figure \ref{fig:andesDiff}), (2) the location and extent of two spreading centers, the tip of a transform fault, and the regional thermal structure of the Cocos Plate near the Central America segment (Figure \ref{fig:centralAmericaDiff}), (3) locations of plate boundaries near the New Britain Solomon (Figure \ref{fig:newBritainSolomonDiff}) and Scotia segments (Figure \ref{fig:scotiaDiff}), (4) a large low surface heat flow anomaly near the Sumatra Banda Sea segment (east of Borneo at approximately 120\(^{\circ}\)E and 5\(^{\circ}\)S, Figure \ref{fig:sumatraBandaSeaDiff}), (5) a high heat flow anomaly defining a transform fault near the N tip of the Tonga New Zealand segment (Figure \ref{fig:tongaNewZealandDiff}), and (6) the location of microplate boundaries near the Vanuatu segment (Figure \ref{fig:vanuatuDiff}).

\DIFdelbegin \DIFdel{By definition, such }\DIFdelend \DIFaddbegin \DIFadd{Such }\DIFaddend inconsistencies between Similarity and Kriging interpolations identify tectonic features that either violate geologic proxy datasets, violate local surface heat flow observations, lack sufficient observational coverage to be resolved by Kriging, or are too fine-scale to be resolved by geologic proxy datasets on a 0.5\(^{\circ}\) \(\times\) 0.5\(^{\circ}\) grid. In any case, the above examples demonstrate the utility of comparing independent interpolation methods in identifying relevant targets for future investigation and data acquisition (discussed further below). Maps of regional interpolated surface heat flow prepared in this study (Section \ref{chpt3Results} and Appendices \ref{interpDiffAppendix} \& \ref{lateralDiffAppendix}, or similar) therefore provide important context for subduction zone research.

\hypertarget{comparing-upper-plate-sectors}{%
\subsection{Comparing Upper-Plate Sectors}\label{comparing-upper-plate-sectors}}

\hypertarget{issues-with-irregularly-spaced-data}{%
\subsubsection{Issues with Irregularly-Spaced Data}\label{issues-with-irregularly-spaced-data}}

Surface heat flow profiles in previous studies \DIFdelbegin \DIFdel{are }\DIFdelend \DIFaddbegin \DIFadd{were }\DIFaddend computed with observations sampled from within a single sector (\protect\hyperlink{ref-currie2004}{Currie et al., 2004}; \protect\hyperlink{ref-currie2006}{Currie \& Hyndman, 2006}; \protect\hyperlink{ref-furukawa1993}{Furukawa, 1993}; \protect\hyperlink{ref-hyndman2005}{Hyndman et al., 2005}; \protect\hyperlink{ref-kerswell2021}{Kerswell et al., 2021}; \protect\hyperlink{ref-wada2009}{Wada \& Wang, 2009}). While extending a single-sector sampling approach to many adjacent sectors is simple to implement, inherent pitfalls are immediately obvious when comparing ThermoGlobe data among sectors. For example, the spatial density and regularity of ThermoGlobe data within adjacent sectors can often be drastically different (e.g.~compare ThermoGlobe data counts across sectors from Central America, Sumatra Banda Sea, and Tonga New Zealand in Table \ref{tab:sectorSummaryTable}). Fluctuating sample sizes among upper-plate sectors can make statistical comparisons of ThermoGlobe data equivocal. \DIFdelbegin %DIFDELCMD < 

%DIFDELCMD < %%%
\DIFdel{Indeed, this study finds comparing ThermoGlobe data among upper-plate sectors problematic. }\DIFdelend For instance, ThermoGlobe data are often too few (n \textless{} 20 observations for 59/100 sectors, Table \ref{tab:sectorSummaryTable}) to compare with statistical confidence. Many sectors (n = 10) have a single observation with a singular distribution (\gls{iqr} = 0) or few observations spanning a large range (very large \gls{iqr}). Many sectors encompass zero ThermoGlobe data and therefore cannot be compared at all. In other words, summary statistics necessary for gauging the \DIFdelbegin \DIFdel{(dis)}\DIFdelend continuity of surface heat flow among sectors (e.g.~median, \gls{iqr}, Table \ref{tab:sectorSummaryTable}) can be generally considered unreliable for \DIFdelbegin \DIFdel{up to 59/100 }\DIFdelend \DIFaddbegin \DIFadd{a majority of }\DIFaddend sectors.

The above limitation arising from sampling irregularly-spaced data can be easily overcome by interpolation. That is because sampling a regular interpolation grid allows for more consistent sample sizes and spatial coverage across sectors. For example, many sectors defined in this study have few ThermoGlobe data (n \textless{} 5 observations for 37/100 sectors, Table \ref{tab:sectorSummaryTable}), yet the average number of \DIFdelbegin \DIFdel{interpolation targets }\DIFdelend \DIFaddbegin \DIFadd{Similarity and Kriging predictions }\DIFaddend within those same sectors is 51---about 10 times the sample size on average. Surface heat flow variability among sectors is thus more confidently and consistently evaluated with interpolations \emph{derived from} ThermoGlobe data, rather than from ThermoGlobe data directly.

\DIFdelbegin %DIFDELCMD < \hypertarget{discontinuity-of-upper-plate-surface-heat-flow}{%
%DIFDELCMD < \subsubsection{(Dis)continuity of Upper-Plate Surface Heat Flow}\label{discontinuity-of-upper-plate-surface-heat-flow}}
%DIFDELCMD < %%%
\DIFdelend \DIFaddbegin \hypertarget{continuity-of-upper-plate-surface-heat-flow}{%
\subsubsection{Continuity of Upper-Plate Surface Heat Flow}\label{continuity-of-upper-plate-surface-heat-flow}}
\DIFaddend 

\DIFdelbegin \DIFdel{Despite being derived from nearly-identical sets of processed ThermoGlobe data, comparisons of Similarity and Kriging predictions across thousands of square kilometers of upper-plate sectors very clearly show mixed results. In summary, Similarity and Kriging predictions from 100 upper-plate sectors neither show broadly continuous nor broadly discontinuous upper-plate surface heat flow within, or among subduction zone segments. Rather than observing consistency and uniformity in upper-plate surface heat flow, a Kaleidoscope of surface heat flow profiles and distributions are observed (compare Figures \ref{fig:kyushuRyukyuUpper}, \ref{fig:sumatraBandaSeaUpper}, \ref{fig:newBritainSolomonUpper}, and see Appendix \ref{lateralDiffAppendix} \& Table \ref{tab:sectorSummaryTable}).
}%DIFDELCMD < 

%DIFDELCMD < %%%
\DIFdelend How consistent and continuous is upper-plate surface heat flow within \DIFdelbegin \DIFdel{, }\DIFdelend and among subduction zone segments? \DIFdelbegin \DIFdel{The Kaleidoscope of surface heat flow profiles and distributions observed across all 13 subduction zone segments is somewhat ambiguous in this regard. }\DIFdelend While Similarity and Kriging predictions show discontinuous upper-plate surface heat flow patterns for some segments (e.g.~Andes, Lesser Antilles and Vanuatu, Figures \ref{fig:andesUpper}, \ref{fig:lesserAntillesUpper} \& \ref{fig:vanuatuUpper}), other segments show rather continuous patterns (e.g.~Central America, Kamchatka Marianas, Kyushu Ryukyu, N Philippines, Figures \ref{fig:centralAmericaUpper}, \ref{fig:kamchatkaMarianasUpper}, \ref{fig:kyushuRyukyuUpper}, \ref{fig:nPhilippinesUpper}), and still other segments show mixed patterns depending on the interpolation method (e.g.~Alaska Aleutians, New Britain Solomon, S Philippines, Sumatra Banda Sea, Tonga New Zealand, Figures \ref{fig:alaskaAleutiansUpper}, \ref{fig:newBritainSolomonUpper}, \ref{fig:sPhilippinesUpper}, \ref{fig:sumatraBandaSeaUpper}, \ref{fig:tongaNewZealandUpper}). \DIFdelbegin %DIFDELCMD < 

%DIFDELCMD < %%%
\DIFdel{On }\DIFdelend \DIFaddbegin \DIFadd{On the }\DIFaddend one hand, Similarity and Kriging interpolations can show \DIFdelbegin \DIFdel{nearly-identical profiles continuing }\DIFdelend \DIFaddbegin \DIFadd{nearly identical profiles }\DIFaddend along-strike for 1000's of km (e.g.~Kamchatka Marianas, Kyushu Ryukyu, Sumatra Banda Sea, Figures \ref{fig:kamchatkaMarianasUpper}, \ref{fig:kyushuRyukyuUpper}, \ref{fig:sumatraBandaSeaUpper}). These segments \DIFdelbegin \DIFdel{demonstrates the possibility of }\DIFdelend \DIFaddbegin \DIFadd{demonstrate }\DIFaddend large-scale continuity in upper-plate surface heat flow and may imply spatially homogeneous lithospheric thermal structure and/or spatially homogeneous heat-transferring dynamics \DIFdelbegin \DIFdel{inferred from previous work }\DIFdelend (e.g. \protect\hyperlink{ref-currie2004}{Currie et al., 2004}; \protect\hyperlink{ref-currie2006}{Currie \& Hyndman, 2006}; \protect\hyperlink{ref-furukawa1993}{Furukawa, 1993}; \protect\hyperlink{ref-kerswell2021}{Kerswell et al., 2021}; \protect\hyperlink{ref-wada2009}{Wada \& Wang, 2009}). \DIFdelbegin \DIFdel{An alternative explanation is that }\DIFdelend \DIFaddbegin \DIFadd{Alternatively, }\DIFaddend continuous surface heat flow \DIFdelbegin \DIFdel{is a consequence of }\DIFdelend \DIFaddbegin \DIFadd{may reflect }\DIFaddend undersampling relative to \DIFdelbegin \DIFdel{the }\DIFdelend local spatial variability of surface heat flow. Moreover, most segments show neither completely continuous nor discontinuous upper-plate surface heat flow patterns (Table \ref{tab:sectorSummaryTable}).

\DIFdelbegin \DIFdel{Another type of continuous pattern observed across some segments is }\DIFdelend \DIFaddbegin \DIFadd{Some segments show }\DIFaddend an apparent wave-like oscillation between higher and lower surface heat flow across multiple adjacent upper-plate sectors. \DIFdelbegin \DIFdel{The most notable case is }\DIFdelend \DIFaddbegin \DIFadd{In }\DIFaddend the Sumatra Banda Sea segment (Figure \ref{fig:sumatraBandaSeaUpper}), \DIFdelbegin \DIFdel{where }\DIFdelend median Similarity and Kriging predictions oscillate with a wavelength on the order of 10\(^3\) km (approximately 5-7 sectors). Such large-wavelength oscillations may imply gradual along-strike variation in upper-plate thickness, coupling depths, and/or lithosphere-asthenosphere geodynamics. \DIFdelbegin \DIFdel{Note that near surface }\DIFdelend \DIFaddbegin \DIFadd{Near-surface }\DIFaddend perturbations probably do not significantly \DIFdelbegin \DIFdel{factor into }\DIFdelend \DIFaddbegin \DIFadd{affect }\DIFaddend large-scale oscillations because hydrothermal effects are expected to be locally distributed in accordance with thin (\textless{} 400 m) sediment cover \DIFdelbegin \DIFdel{and/}\DIFdelend or close proximity to seamounts (\textless{} 60 km, \protect\hyperlink{ref-hasterok2011}{Hasterok et al., 2011}).

\hypertarget{identifying-survey-targets}{%
\subsubsection{Identifying Survey Targets}\label{identifying-survey-targets}}

Ideal survey targets for future surface heat flow observations \DIFdelbegin \DIFdel{will }\DIFdelend \DIFaddbegin \DIFadd{should strive to }\DIFaddend simultaneously improve the spatial resolution and \DIFdelbegin \DIFdel{/or }\DIFdelend accuracy of Similarity and Kriging methods. For Similarity \DIFdelbegin \DIFdel{, this means ensuring that the }\DIFdelend geographic configurations of new survey targets (the geologic context) \DIFaddbegin \DIFadd{should }\DIFaddend have the greatest diversity possible and \DIFdelbegin \DIFdel{do }\DIFdelend \DIFaddbegin \DIFadd{should }\DIFaddend not overlap significantly with already oversampled regions in the geologic proxy parameter space. For example, \DIFdelbegin \DIFdel{enormous amounts of }\DIFdelend \DIFaddbegin \DIFadd{numerous }\DIFaddend surface heat flow observations are located close to oceanic ridge systems because of historically productive study sites like Cascadia (western North America, e.g. \protect\hyperlink{ref-currie2004}{Currie et al., 2004}; \protect\hyperlink{ref-davis1990}{Davis et al., 1990}; \protect\hyperlink{ref-hyndman1993}{Hyndman \& Wang, 1993}; \protect\hyperlink{ref-jennings2021}{Jennings \& Hasterok, 2021}; \protect\hyperlink{ref-korgen1971}{Korgen et al., 1971}; \protect\hyperlink{ref-wang1995}{Wang et al., 1995}). This \DIFdelbegin \DIFdel{has the effect of biasing }\DIFdelend \DIFaddbegin \DIFadd{biases }\DIFaddend Similarity predictions to look like Cascadia---as all interpolation targets located near oceanic ridge systems will adopt the same distribution of surface heat flow values measured near Cascadia (and \DIFaddbegin \DIFadd{a }\DIFaddend few other densely sampled regions, Figure \ref{fig:cascadiaBias}). The same principle applies to any other geologic proxy variable sampled heavily from selectively few regions. Oversampling within the geologic proxy parameter space is dually undesirable when applying Similarity because it adds elements of bias and spatial-dependence to a method that is otherwise advantageous because of its spatial-independence.



\begin{landscape}


\begin{figure}[htbp]

{\centering \includegraphics[width=0.9\linewidth,]{assets/figs/chpt3/cascadiaBias} 

}

\caption[Surface heat flow and distances to ridges]{Global distribution of surface heat flow observations and distances to ridges. (a\DIFaddbeginFL \DIFaddFL{, b}\DIFaddendFL ) Maps showing \DIFaddbeginFL \DIFaddFL{the }\DIFaddendFL localities of surface heat flow observations and their distances from ridges, and \DIFdelbeginFL \DIFdelFL{(b) }\DIFdelendFL the complete \DIFaddbeginFL \DIFaddFL{global }\DIFaddendFL distribution of distances to ridges. (c) Normalized density estimates comparing the relative coverage of surface heat flow observations with the global distribution of distances from ridges. Differences in density reveal regions of over- and undersampling within the geologic proxy parameter space. Subduction zone boundaries (bold white lines) defined by Syracuse \& Abers (\protect\hyperlink{ref-syracuse2006}{2006}). Plate boundaries defined by Lawver et al. (\protect\hyperlink{ref-lawver2018}{2018}). Global proxy data from Goutorbe et al. (\protect\hyperlink{ref-goutorbe2011}{2011}).}\label{fig:cascadiaBias}
\end{figure}


\end{landscape}

For Kriging, ideal survey target sites \DIFdelbegin \DIFdel{will }\DIFdelend \DIFaddbegin \DIFadd{should }\DIFaddend provide the most regular coverage over a region of \DIFdelbegin \DIFdel{spatial }\DIFdelend interest (e.g.~a particular subduction zone segment). Evaluating surface heat flow distributions across upper-plate sectors offers opportunities for discovering future survey targets by identifying the least-constrained sectors. For example, segments with the greatest Similarity-Kriging discrepancies among sectors tend to have: (1) very few ThermoGlobe data (e.g.~Alaska Aleutians, N Philippines, New Britain Solomon, S Philippines), (2) highly-irregular spatial coverage of ThermoGlobe data (e.g.~Andes, Central America, Lesser Antilles), \DIFdelbegin \DIFdel{and/}\DIFdelend or (3) complex upper-plate tectonics (Vanuatu). A simple query of the ThermoGlobe dataset by sector can identify individual sectors with low \DIFdelbegin \DIFdel{and/}\DIFdelend or highly-irregular observational density \DIFdelbegin \DIFdel{and/}\DIFdelend or large Similarity-Kriging discrepancies. Thus, current observational gaps in regional surface heat flow can be efficiently identified by comparing independent interpolation methods within multiple-sectors.

\hypertarget{comparing-similarity-and-kriging-accuracies}{%
\subsection{Comparing Similarity and Kriging Accuracies}\label{comparing-similarity-and-kriging-accuracies}}

Neither \DIFdelbegin \DIFdel{Similarity nor Kriging methods are evidently more favorable for every interpolation domain on first principles . Nor do the error rates in Table \ref{tab:vgrmSummaryTable} indicate a clear accuracy advantage of one method over the other with respect to predicting }\DIFdelend \DIFaddbegin \DIFadd{error rates nor first principles favor Similarity vs.~Kriging on }\DIFaddend regional (10\(^2\) to 10\(^3\) km) \DIFdelbegin \DIFdel{distributions of surface heat flow near subduction zone segments. It appears that either method is }\DIFdelend \DIFaddbegin \DIFadd{scales. Rather, both methods are }\DIFaddend successfully generalizable and appropriate for subduction zone research. \DIFdelbegin %DIFDELCMD < 

%DIFDELCMD < %%%
\DIFdelend While some segments do show \DIFdelbegin \DIFdel{a large discrepancy }\DIFdelend \DIFaddbegin \DIFadd{large discrepancies }\DIFaddend between Similarity and Kriging error rates (e.g.~Scotia), \DIFdelbegin \DIFdel{it is important to note that }\DIFdelend low error rates do not necessarily imply more accurate predictions. For Scotia, few observations naturally lead to overfitting and low error rates\DIFdelbegin \DIFdel{during Kriging. But it is also practically possible to reduce Kriging error rates by }\DIFdelend \DIFaddbegin \DIFadd{, but }\DIFaddend choosing different Kriging parameters and/or highly localizing \DIFdelbegin \DIFdel{Kriging---thereby (un)intentionally overfitting }\DIFdelend \DIFaddbegin \DIFadd{Kriging can also unintentionally overfit }\DIFaddend ThermoGlobe data and \DIFdelbegin \DIFdel{artificially increasing accuracy. In such cases, overfitting compromises }\DIFdelend \DIFaddbegin \DIFadd{compromise }\DIFaddend regional interpolation accuracy. At 1.3 times greater error rates than Similarity on average, however, Kriging error rates do not suggest overfitting is \DIFdelbegin \DIFdel{a prevalent issue in this study }\DIFdelend \DIFaddbegin \DIFadd{prevalent }\DIFaddend (Tables \ref{tab:vgrmSummaryTable} and \ref{tab:vgrmSummaryTableLong}).

Differences in error rates notwithstanding, Similarity has a distinct advantage compared to Kriging when applied to regions with relatively low observational density and/or highly-irregular spatial coverage. For example, Similarity predictions appear to be remarkably consistent with known tectonic features even in cases with few observations (e.g.~Scotia and New Britain Solomon, Figures \ref{fig:scotiaDiff} \& \ref{fig:newBritainSolomonDiff}). Integrating geologic proxies is therefore preferred when limited observations preclude practically useful Kriging interpolations.

\hypertarget{layered-interpolation-approach}{%
\subsection{Layered Interpolation Approach}\label{layered-interpolation-approach}}

Similarity and Kriging interpolations are distinguishable by eye at the regional scale \DIFdelbegin \DIFdel{when compared side-by-side }\DIFdelend (e.g.~compare Figures \ref{fig:centralAmericaDiff}, \ref{fig:kyushuRyukyuDiff}, and \ref{fig:scotiaDiff} with the remaining segments in Appendices \ref{interpDiffAppendix} \& \ref{lateralDiffAppendix}). The same unique properties of Similarity and Kriging methods that make them quickly discernible by eye can be independently leveraged. For example, because Similarity is inherently agnostic to the spatial configuration of observations (\protect\hyperlink{ref-goutorbe2011}{Goutorbe et al., 2011}), accurate interpolations with well-defined plate boundaries are still possible for regions with relatively few observations (e.g.~Scotia and New Britain Solomon, Figures \ref{fig:scotiaDiff} \& \ref{fig:newBritainSolomonDiff}). Since surface heat flow observations near subduction zone segments are commonly sparse and irregularly spaced, spatial-independence from observations is a desirable property to maintain during the interpolation process.

On the other hand, conserving the ``ground-truth'' is \DIFaddbegin \DIFadd{an }\DIFaddend equally desirable property\DIFdelbegin \DIFdel{to maintain during the interpolation process}\DIFdelend . Local ordinary Kriging conserves ground-truth by remaining agnostic to all other factors but the spatial configuration of surface heat flow observations (see Appendix \ref{krigeOpt}). For example, Kriging resolves tectonic features near Tonga New Zealand and Vanuatu that are discordant with Similarity predictions, yet compatible with ThermoGlobe data (Figures \ref{fig:tongaNewZealandDiff} \& \ref{fig:vanuatuDiff}). Another example is the young Cocos Plate near Central America where Similarity predicts relatively high heat flow by proximity to two spreading centers and young oceanic plate age, yet observations of anomalously low surface heat flow (e.g. \protect\hyperlink{ref-hutnak2008}{Hutnak et al., 2008}) constrain Kriging predictions to low values. Such contrasting predictions imply ThermoGlobe data violate one or more geologic proxy data sets used by Similarity. In other words, Kriging will tend to highlight anomalies (compared to Similarity) if they exist and have been observed.

In principle, carefully layering Similarity and Kriging methods may combine their properties to produce more accurate regional interpolations \DIFaddbegin \DIFadd{in the future}\DIFaddend . A layered approach simultaneously respects the First (\protect\hyperlink{ref-krige1951}{Krige, 1951}) and Third Laws of Geography (\protect\hyperlink{ref-zhu2018}{Zhu et al., 2018}) by integrating geologic and spatial information. \DIFdelbegin \DIFdel{Any number of }\DIFdelend \DIFaddbegin \DIFadd{Many }\DIFaddend methods may be applied to combine Similarity and Kriging predictions. As a basic example: (1) compare Similarity and Kriging layers to detect anomalies, (2) compute weights proportional to the squared difference between Similarity and Kriging predictions to emphasized \DIFdelbegin \DIFdel{(or subdue ) }\DIFdelend \DIFaddbegin \DIFadd{or subdue }\DIFaddend anomalies, (3) combine Similarity and Kriging layers using a weighted average scheme.
\DIFdelbegin \DIFdel{Similar methods may be the topic of future research.
}\DIFdelend 

\hypertarget{conclusions-1}{%
\section{Conclusions}\label{conclusions-1}}

This study evaluates regional patterns of surface heat flow near subduction zones by comparing Similarity and Kriging interpolations across adjacent upper-plate sectors. Methodological differences between Similarity and Kriging yield \DIFdelbegin \DIFdel{a Kaleidoscope of upper-plate surface }\DIFdelend \DIFaddbegin \DIFadd{both similar and disparate predicted }\DIFaddend heat flow distributions and profiles among subduction \DIFdelbegin \DIFdel{zone segments. Such mixed results highlight at least three important points of consideration regarding }\DIFdelend \DIFaddbegin \DIFadd{zones. Four key conclusions arise from }\DIFaddend regional surface heat flow near active subduction zones:

\begin{enumerate}
\def\labelenumi{\arabic{enumi}.}
\tightlist
\item
  Accurate regional interpolations of irregularly-spaced ThermoGlobe data are key to understanding broad (segment-scale) variations in lithospheric thermal structure near subduction zones\DIFaddbegin \DIFadd{.
}\DIFaddend \item
  Mixed upper-plate surface heat flow distributions and profiles imply various degrees of regional \DIFdelbegin \DIFdel{(dis)continuity exist }\DIFdelend \DIFaddbegin \DIFadd{continuity }\DIFaddend among subduction zones in terms of their lithospheric thermal structure (contrary to expectations from \protect\hyperlink{ref-kerswell2021}{Kerswell et al., 2021}), heat-transferring subsurface dynamics, and/or observational density relative to the local spatial variability of \DIFdelbegin \DIFdel{(unknown) }\DIFdelend surface heat flow\DIFaddbegin \DIFadd{.
}\DIFaddend \item
  Future surface heat flow surveys \DIFdelbegin \DIFdel{should maximize the potential for increasing }\DIFdelend \DIFaddbegin \DIFadd{can maximize }\DIFaddend Similarity and Kriging accuracies by carefully considering the existing spatial distribution of surface heat flow observations and their distribution within geologic proxy parameter space\DIFaddbegin \DIFadd{.
}\DIFaddend \item
  Layered interpolation approaches may produce more accurate surface heat flow predictions by combining the independently-advantageous properties of Similarity and Kriging methods\DIFaddbegin \DIFadd{.
}\DIFaddend \end{enumerate}

\cleardoublepage

\hypertarget{chpt4}{%
\chapter{Computing Rates and Distributions of Rock Recovery in Subduction Zones}\label{chpt4}}

\markboth{Chapter 4: Marker Recovery}{Chapter 4: Marker Recovery}

\hypertarget{chpt4Abstract}{%
\section{Abstract}\label{chpt4Abstract}}

Bodies of rock that are detached (recovered) from subducting oceanic plates and exhumed to Earth's surface become invaluable records of the mechanical and chemical processing of subducted material. Well-studied bodies of exhumed \gls{hp} rocks provide insights into the nature of rock recovery, yet various interpretations concerning thermal gradients, recovery rates, and recovery depths arise when directly comparing the rock record with numerical simulations of subduction. \DIFdelbegin \DIFdel{The crux to constraining }\DIFdelend \DIFaddbegin \DIFadd{Constraining }\DIFaddend recovery rates and depths directly from the rock record \DIFdelbegin \DIFdel{, or by comparison with numerical experiments, stems from }\DIFdelend \DIFaddbegin \DIFadd{relies on }\DIFaddend small sample sizes of \gls{hp} rocks---making statistical inference weak. As an alternative approach, this study implements a ``soft'' clustering classification algorithm to identify rock recovery in numerical simulations of oceanic-continental convergence. Over \DIFdelbegin \DIFdel{one-million }\DIFdelend \DIFaddbegin \DIFadd{one million }\DIFaddend markers are traced and classified from 64 simulations representing a large range of presently active subduction zones on Earth. Marker \gls{pt} distributions are compared across models and with the rock record to address the following three questions: how do recovery rates vary among subduction zones? How is recovery distributed along the plate interface? How does recovery vary among subduction zones and through time? Recovery pressures (depths) correlate strongly with convergence velocity and moderately with oceanic plate age, while \gls{pt} gradients correlate strongly with oceanic plate age and upper-plate thickness. Recovery rates strongly correlate with upper-plate thickness, yet show no correlation with other initial conditions. Likewise, \gls{pt} distributions of recovered markers show \DIFdelbegin \DIFdel{a range of (in)}\DIFdelend \DIFaddbegin \DIFadd{variable }\DIFaddend compatibility with the rock record depending on the collection of natural samples and suite(s) of numerical experiments. A significant gap in \DIFaddbegin \DIFadd{predicted }\DIFaddend marker recovery is found near 2 GPa and 550 \(^{\circ}\)C, coinciding with the \DIFdelbegin \DIFdel{highest-density }\DIFdelend \DIFaddbegin \DIFadd{highest density }\DIFaddend of exhumed \gls{hp} rocks. Implications for such a gap in marker recovery include a simplified numerical model that fails to capture the full range of recovery mechanisms, inconsistent thermal gradients compared to natural samples, and a relative overabundance of rocks studied from \DIFdelbegin \DIFdel{(scientific bias), }\DIFdelend around 2 GPa and 550 \(^{\circ}\)C \DIFaddbegin \DIFadd{(scientific bias)}\DIFaddend .

\hypertarget{chpt4Intro}{%
\section{Introduction}\label{chpt4Intro}}

Maximum \gls{pt} conditions have been estimated for hundreds of \gls{hp} metamorphic rocks exhumed from subduction zones (Figure \ref{fig:rockPTComp}, \protect\hyperlink{ref-agard2018}{Agard et al., 2018}; \protect\hyperlink{ref-hacker1996}{Hacker, 1996}; \protect\hyperlink{ref-penniston2015}{Penniston-Dorland et al., 2015}). Metamorphic rocks (the \emph{rock record}) are the only tangible evidence of \gls{pt}-strain fields experienced by Earth's lithosphere during deformation and chemical processing in subduction zones. Together with geophysical imaging (e.g. \protect\hyperlink{ref-bostock2013}{Bostock, 2013}; \protect\hyperlink{ref-ferris2003}{Ferris et al., 2003}; \protect\hyperlink{ref-hyndman2003}{Hyndman \& Peacock, 2003}; \protect\hyperlink{ref-mann2022}{Mann et al., 2022}; \protect\hyperlink{ref-naif2015}{Naif et al., 2015}; \protect\hyperlink{ref-rondenay2008}{Rondenay et al., 2008}; \protect\hyperlink{ref-syracuse2006}{Syracuse \& Abers, 2006}), surface heat flow (e.g. \protect\hyperlink{ref-currie2006}{Currie \& Hyndman, 2006}; \protect\hyperlink{ref-gao2014}{Gao \& Wang, 2014}; \protect\hyperlink{ref-hyndman2005}{Hyndman et al., 2005}; \protect\hyperlink{ref-kohn2018}{Kohn et al., 2018}; \protect\hyperlink{ref-morishige2020}{Morishige \& Kuwatani, 2020}; \protect\hyperlink{ref-wada2009}{Wada \& Wang, 2009}), and forward numerical modelling (e.g. \protect\hyperlink{ref-gerya2002}{Gerya et al., 2002}, \protect\hyperlink{ref-gerya2008}{2008}; \protect\hyperlink{ref-gerya2006}{Gerya \& Stöckhert, 2006}; \protect\hyperlink{ref-hacker2003}{Hacker et al., 2003}; \protect\hyperlink{ref-kerswell2021}{Kerswell et al., 2021}; \protect\hyperlink{ref-mckenzie1969}{McKenzie, 1969}; \protect\hyperlink{ref-peacock1990}{Peacock, 1990}, \protect\hyperlink{ref-peacock1996}{1996}; \protect\hyperlink{ref-sizova2010}{Sizova et al., 2010}; \protect\hyperlink{ref-syracuse2010}{Syracuse et al., 2010}; \protect\hyperlink{ref-yamato2007}{Yamato et al., 2007}, \protect\hyperlink{ref-yamato2008}{2008}), the rock record underpins contemporary understandings of subduction geodynamics (e.g. \protect\hyperlink{ref-agard2009}{Agard et al., 2009}; \protect\hyperlink{ref-agard2021}{Agard, 2021}; \protect\hyperlink{ref-bebout2007}{Bebout, 2007}).



\begin{figure}[htbp]

{\centering \includegraphics[width=1\linewidth,]{assets/figs/chpt4/rox_comp} 

}

\caption[Compiled \gls{pt} estimates from the rock record]{Distributions of \gls{pt} estimates from the rock record. (a) Pressure vs.~temperature diagram showing the densities of pd15 (solid contours, \protect\hyperlink{ref-penniston2015}{Penniston-Dorland et al., 2015}) and ag18 (filled contours, \protect\hyperlink{ref-agard2018}{Agard et al., 2018}) datasets. Thin lines are thermal gradients labeled in \(^{\circ}\)C/km. Reaction boundaries for eclogitization of oceanic crust and antigorite dehydration are from Ito \& Kennedy (\protect\hyperlink{ref-ito1971}{1971}) and Schmidt \& Poli (\protect\hyperlink{ref-schmidt1998}{1998}), respectively. (b-d) Cumulative probability diagrams of pd15 and ag18 imply relatively uniform recovery of rocks across \gls{pt} space with distinct abundances of different rock types. Metamorphic facies: (1) forbidden zone (\(\leq\) 5 \(^{\circ}\)C/km), (2) lws/ep blueschist to eclogite (5-15 \(^{\circ}\)C/km), (3) greenschist to amphibolite (15-30 \(^{\circ}\)C/km), (4) amphibolite to granulite (\(\geq\) 30 \(^{\circ}\)C/km). Note that ag18 {[}no HT{]} excludes samples above 20 \(^{\circ}\)C/km.}\label{fig:rockPTComp}
\end{figure}

However, it remains difficult to directly interpret the rock record in terms of recovery rates and distributions along the plate interface. For example, large compilations of \gls{pt} estimates representing the global distribution of \gls{hp} rocks exhumed during the Phanerozoic (the pd15 and ag18 datasets, \protect\hyperlink{ref-agard2018}{Agard et al., 2018}; \protect\hyperlink{ref-penniston2015}{Penniston-Dorland et al., 2015}) reveal a kinked \gls{cdf} with respect to pressure (Figure \ref{fig:rockPTComp}b). A smooth linear \gls{cdf} curve (not step-like) interrupted by a sharp change in slope around 2.3-2.4 GPa implies relatively uniform recovery of subducting material up to 2.3-2.4 GPa, but increasingly rare recovery above 2.3-2.4 GPa (\protect\hyperlink{ref-agard2018}{Agard et al., 2018}; \protect\hyperlink{ref-kerswell2021}{Kerswell et al., 2021}; \protect\hyperlink{ref-monie2009}{Monie \& Agard, 2009}; \protect\hyperlink{ref-plunder2015}{Plunder et al., 2015}). While evidence for common mechanical coupling depths among presently active subduction zones near 2.3 GPa (\protect\hyperlink{ref-furukawa1993}{Furukawa, 1993}; \protect\hyperlink{ref-kerswell2021}{Kerswell et al., 2021}; \protect\hyperlink{ref-wada2009}{Wada \& Wang, 2009}) is consistent with the distribution of recovery depths implied by pd15 and ag18 (Figure \ref{fig:rockPTComp}b), upper-plate surface heat flow patterns presented in Chapter \ref{chpt3} are generally inconsistent with uniform coupling depths, and geophysical constraints on mechanical transitions likely to induce rock recovery are still being revised (e.g. \protect\hyperlink{ref-abers2020}{Abers et al., 2020}; \protect\hyperlink{ref-audet2016}{Audet \& Kim, 2016}; \protect\hyperlink{ref-audet2018}{Audet \& Schaeffer, 2018}; \protect\hyperlink{ref-morishige2020}{Morishige \& Kuwatani, 2020}).

Moreover, different compilations of \gls{pt} estimates show relative density variations across \gls{pt} space. For example, Agard et al. (\protect\hyperlink{ref-agard2018}{2018}) \DIFdelbegin \DIFdel{notes }\DIFdelend \DIFaddbegin \DIFadd{note }\DIFaddend that compilations from Plunder et al. (\protect\hyperlink{ref-plunder2015}{2015}) and Groppo et al. (\protect\hyperlink{ref-groppo2016}{2016}) show less dispersion (i.e.~a more step-like \gls{cdf}) than ag18 with tighter bi- or trimodal distributions clustering around important transitions along the plate interface. These peaks (modes) in the distribution of exhumed \gls{hp} rocks are inferred to coincide with the continental Moho at approximately 25-35 km and \DIFaddbegin \DIFadd{the }\DIFaddend transition to mechanical plate coupling at approximately 80 km (\protect\hyperlink{ref-agard2018}{Agard et al., 2018}; \protect\hyperlink{ref-monie2009}{Monie \& Agard, 2009}; \protect\hyperlink{ref-plunder2015}{Plunder et al., 2015}). \DIFdelbegin \DIFdel{However, there seems to be more consensus on the mechanisms of recovery from 25-35 km and from 80 km than from }\DIFdelend \DIFaddbegin \DIFadd{Less consensus explains }\DIFaddend a smaller, yet significant, intermediate mode at 55-60 km (\protect\hyperlink{ref-agard2009}{Agard et al., 2009}, \protect\hyperlink{ref-agard2018}{2018}; \protect\hyperlink{ref-plunder2015}{Plunder et al., 2015}). \DIFdelbegin \DIFdel{On the other hand, the }\DIFdelend \DIFaddbegin \DIFadd{This }\DIFaddend intermediate mode (at 55-60 km) coincides closely with a high density region of \gls{pt} estimates in the pd15 dataset.

Differences in compiled \gls{pt} datasets notwithstanding, key observations regarding rock recovery in subduction zones emerge from pd15 and ag18:

\begin{quote}
\uline{Key observations from the rock record:}\\
\strut \\
\hspace*{0.333em}1. Rocks are recovered relatively uniformly up to 2.5 GPa\\
\hspace*{0.333em}2. 64-66\% of recovered rocks \DIFdelbegin \DIFdel{equilibrate }\DIFdelend \DIFaddbegin \DIFadd{equilibrated }\DIFaddend between 1-2.5 GPa\\
\hspace*{0.333em}3. 5-19\% of recovered rocks \DIFdelbegin \DIFdel{equilibrate }\DIFdelend \DIFaddbegin \DIFadd{equilibrated }\DIFaddend above 2.5 GPa\\
\hspace*{0.333em}4. 26-28\% of recovered rocks \DIFdelbegin \DIFdel{equilibrate }\DIFdelend \DIFaddbegin \DIFadd{equilibrated }\DIFaddend between 350-500 \(^{\circ}\)C\\
\hspace*{0.333em}5. 27-31\% of recovered rocks \DIFdelbegin \DIFdel{equilibrate }\DIFdelend \DIFaddbegin \DIFadd{equilibrated }\DIFaddend above 625 \(^{\circ}\)C\\
\hspace*{0.333em}6. 52-62\% of recovered rocks record gradients between 5-10 \(^{\circ}\)C/km\\
\hspace*{0.333em}7. 18-31\% of recovered rocks record gradients between 10-15 \(^{\circ}\)C/km\\
\hspace*{0.333em}8. 6-30\% of recovered rocks record gradients above 15 \(^{\circ}\)C/km
\end{quote}

\noindent These ranges in the relative abundances of exhumed \gls{hp} rocks compiled in different datasets raise important questions in subduction zone research: are rocks recovered broadly and uniformly along the plate interface or discretely from certain depths? How do recovery rates and \gls{pt} distributions change across diverse subduction zone systems and through time?

Previous work comparing the rock record directly with numerical models has generally produced ambiguous interpretations concerning recovery rates and distributions along the plate interface. For example, comparisons of different geodynamic codes with diverse subsets of the rock record show various extents of agreement in terms of overlapping \gls{pt} paths and thermal gradients (e.g. \protect\hyperlink{ref-angiboust2012b}{Angiboust et al., 2012b}; \protect\hyperlink{ref-burov2014}{Burov et al., 2014}; \protect\hyperlink{ref-holt2021}{Holt \& Condit, 2021}; \protect\hyperlink{ref-penniston2015}{Penniston-Dorland et al., 2015}; \protect\hyperlink{ref-plunder2018}{Plunder et al., 2018}; \protect\hyperlink{ref-roda2010}{Roda et al., 2010}, \protect\hyperlink{ref-roda2012}{2012}, \protect\hyperlink{ref-roda2020}{2020}; \protect\hyperlink{ref-ruh2015}{Ruh et al., 2015}; \protect\hyperlink{ref-yamato2007}{Yamato et al., 2007}, \protect\hyperlink{ref-yamato2008}{2008}). Initial setups for numerical experiments (oceanic plate age, convergence velocity, subduction dip angle, upper-plate thickness, and heating sources\DIFdelbegin \DIFdel{, }\DIFdelend \DIFaddbegin \DIFadd{; }\DIFaddend \protect\hyperlink{ref-kohn2018}{Kohn et al., 2018}; \protect\hyperlink{ref-penniston2015}{Penniston-Dorland et al., 2015}; \protect\hyperlink{ref-ruh2015}{Ruh et al., 2015}; \protect\hyperlink{ref-vankeken2019}{van Keken et al., 2019}), differential recovery rates from subduction zones with favorable thermo-kinematic boundary conditions (\protect\hyperlink{ref-abers2017}{Abers et al., 2017}; \protect\hyperlink{ref-vankeken2018}{van Keken et al., 2018}), and comparisons among suites of undifferentiated \gls{hp} rocks (e.g.~grouping rocks recovered during subduction initiation with rocks recovered during ``steady-state'' subduction, see \protect\hyperlink{ref-agard2018}{Agard et al., 2018}, \protect\hyperlink{ref-agard2020}{2020}) all potentially contribute to nonoverlapping \gls{pt} distributions and thermal gradients. Compounding the ambiguity are arguments that material is sporadically recovered during short-lived mechanical transitions (\protect\hyperlink{ref-agard2016}{Agard et al., 2016}) and/or geodynamic changes (\protect\hyperlink{ref-monie2009}{Monie \& Agard, 2009})---implying compilations of exhumed \gls{hp} rocks may not randomly sample of \gls{pt} conditions along the plate interface. Such ambiguities warrant further investigation into the general response of recovery rates and distributions along the plate interface to broad ranges of thermo-kinematic boundary conditions.

Fortunately, clues about the nature and \gls{pt} limits of rock recovery are provided by many extensively studied examples of exhumed plate interfaces (e.g. \protect\hyperlink{ref-agard2018}{Agard et al., 2018}; \protect\hyperlink{ref-angiboust2011}{Angiboust et al., 2011}; \protect\hyperlink{ref-angiboust2015}{2015}; \protect\hyperlink{ref-cloos1988}{Cloos \& Shreve, 1988}; \protect\hyperlink{ref-fisher2021}{Fisher et al., 2021}; \protect\hyperlink{ref-ioannidi2020}{Ioannidi et al., 2020}; \protect\hyperlink{ref-kitamura2012}{Kitamura \& Kimura, 2012}; \protect\hyperlink{ref-kotowski2019}{Kotowski \& Behr, 2019}; \protect\hyperlink{ref-locatelli2019}{Locatelli et al., 2019}; \protect\hyperlink{ref-monie2009}{Monie \& Agard, 2009}; \protect\hyperlink{ref-okay1989}{Okay, 1989}; \protect\hyperlink{ref-platt1986}{Platt, 1986}; \protect\hyperlink{ref-plunder2013}{Plunder et al., 2013}, \protect\hyperlink{ref-plunder2015}{2015}; \protect\hyperlink{ref-tewksbury2021a}{Tewksbury-Christle et al., 2021}; \protect\hyperlink{ref-wakabayashi2015}{Wakabayashi, 2015}). However, these type localities represent an unknown fraction of subducted material and differ significantly in terms of their geometry, composition (rock types), and interpreted deformation histories (recovery \& exhumation). It is also unclear to what extent ag18 and pd15 (and other smaller compilations or subsets) represent the full range of conditions suitable for rock recovery and/or represent scientific sampling bias (e.g.~rocks are more frequently resampled from the same pristine exposures than from other localities, \protect\hyperlink{ref-agard2018}{Agard et al., 2018}). Thus, a primary challenge to inferring recovery rates and distributions accurately from the rock record fundamentally stems from small nonrandom samples (typically less than a few dozen \gls{pt} estimates from any given exhumed terrane) compared to the \DIFdelbegin \DIFdel{largely unconstrained }\DIFdelend \DIFaddbegin \DIFadd{possibly large }\DIFaddend range of conditions suitable for rock recovery.

This study aims to address the problem of small nonrandom samples by tracing over \DIFdelbegin \DIFdel{one-million }\DIFdelend \DIFaddbegin \DIFadd{one million }\DIFaddend (1,343,369) Lagrangian markers from 64 numerical geodynamic models (\protect\hyperlink{ref-kerswell2021}{Kerswell et al., 2021}). This large \gls{pt} dataset is insensitive to outliers and represents a statistically robust picture of recovery rates and \gls{pt} distributions in subduction zones for this particular type of model. Clear correlations are found among recovery rates, \gls{pt} distributions, and thermo-kinematic boundary conditions that define a range of plausible conditions for reproducing the rock record. Moreover, an unexpected gap in marker recovery is found coinciding with the \DIFdelbegin \DIFdel{highest-densities }\DIFdelend \DIFaddbegin \DIFadd{highest densities }\DIFaddend of natural samples around 2 GPa and 550 \(^{\circ}\)C. Implications for a gap in marker recovery are discussed, including insufficient implementation of recovery mechanisms in numerical models and a potential overabundance of natural samples studied from around 2 GPa and 550 \(^{\circ}\)C (scientific bias).

\hypertarget{chpt4Methods}{%
\section{Methods}\label{chpt4Methods}}

This study presents a dataset of Lagrangian markers (described below) from numerical experiments that simulate 64 oceanic-continental convergent margins with thermo-kinematic boundary conditions (oceanic plate age, convergence velocity, and upper-plate lithospheric thickness). These experiments closely represent the range of presently active subduction zones (\protect\hyperlink{ref-syracuse2006}{Syracuse \& Abers, 2006}; \protect\hyperlink{ref-wada2009}{Wada \& Wang, 2009}). Initial conditions were modified from previous studies of active margins (\protect\hyperlink{ref-gorczyk2007}{Gorczyk et al., 2007}; \protect\hyperlink{ref-sizova2010}{Sizova et al., 2010}) using the geodynamic code \texttt{I2VIS} (\protect\hyperlink{ref-gerya2003}{Gerya \& Yuen, 2003}). \texttt{I2VIS} models visco-plastic flow of geologic materials by solving conservative equations of mass, energy, and momentum on a fully-staggered finite difference grid with a \emph{marker-in-cell} technique (e.g. \protect\hyperlink{ref-harlow1965}{Harlow \& Welch, 1965}). Further details about the initial setup, boundary conditions, and rheologic model are presented in Kerswell et al. (\protect\hyperlink{ref-kerswell2021}{2021}). Details about \texttt{I2VIS} and example code are presented in Gerya \& Yuen (\protect\hyperlink{ref-gerya2003}{2003}) and Gerya (\protect\hyperlink{ref-gerya2019}{2019}).

\DIFdelbegin \DIFdel{This }\DIFdelend \DIFaddbegin \DIFadd{The following }\DIFaddend section defines Lagrangian markers (now referred to as \emph{markers}) and briefly elaborates on their usefulness in understanding \DIFdelbegin \DIFdel{fluid flow }\DIFdelend \DIFaddbegin \DIFadd{flow of geologic materials}\DIFaddend , followed by a description of the marker classification algorithm. A full mathematical description of the classification algorithm is presented in Appendix \ref{gmm}.

\hypertarget{lagrangian-markers}{%
\subsection{Lagrangian Markers}\label{lagrangian-markers}}

Markers are mathematical objects representing discrete parcels of material flowing in a continuum (\protect\hyperlink{ref-harlow1962}{Harlow, 1962}, \protect\hyperlink{ref-harlow1964}{1964}). Tracing markers (saving marker information at each timestep) is distinctly advantageous for investigating subduction dynamics in the following two ways.

First, modelling subduction requires solving equations of mass, motion, and heat transport in a partly layered, partly heterogeneous, high-strain region known as the \emph{plate interface}, \emph{subduction interface}, or \emph{subduction channel} (\protect\hyperlink{ref-gerya2002}{Gerya et al., 2002}). Current conceptual models regard the plate interface as a visco-plastic continuum with complex geometry and structure, sharp thermal, chemical, and strain gradients, strong advection, and abundant fluid flow (\protect\hyperlink{ref-agard2016}{Agard et al., 2016}, \protect\hyperlink{ref-agard2018}{2018}; \protect\hyperlink{ref-bebout2007}{Bebout, 2007}; \protect\hyperlink{ref-bebout2002}{Bebout \& Barton, 2002}; \protect\hyperlink{ref-cloos1988}{Cloos \& Shreve, 1988}; \protect\hyperlink{ref-gerya2003}{Gerya \& Yuen, 2003}; \protect\hyperlink{ref-penniston2015}{Penniston-Dorland et al., 2015}; \protect\hyperlink{ref-shreve1986}{Shreve \& Cloos, 1986}; \protect\hyperlink{ref-stockhert2002}{Stöckhert, 2002}; \protect\hyperlink{ref-tewksbury2021a}{Tewksbury-Christle et al., 2021}). Finite-difference numerical approaches do not perform well with strong local gradients, and interpolating and updating temperature, strain, and chemical fields with markers greatly improves accuracy and stability of numerical solutions (\protect\hyperlink{ref-gerya2019}{Gerya, 2019}; \protect\hyperlink{ref-gerya2003}{Gerya \& Yuen, 2003}; \protect\hyperlink{ref-moresi2003}{Moresi et al., 2003}).

Second, tracing a marker closely proxies for tracing a rock's \gls{pt}-time history. Strictly speaking, some deviations in calculated \gls{pt}-time histories are possible because our models make assumptions, including: (1) markers move in an incompressible continuum (\protect\hyperlink{ref-batchelor1953}{Batchelor, 1953}; \protect\hyperlink{ref-boussinesq1897}{Boussinesq, 1897}), (2) marker material properties are governed by a simple petrologic model describing eclogitization of oceanic crust (\protect\hyperlink{ref-ito1971}{Ito \& Kennedy, 1971}) and (de)hydration of upper mantle (\(antigorite \allowbreak \Leftrightarrow olivine + orthopyroxene + H_{2}O\), \protect\hyperlink{ref-schmidt1998}{Schmidt \& Poli, 1998}), and (3) marker stress and strain are related by a highly non-linear rheologic model derived from empirical flow laws (\protect\hyperlink{ref-hilairet2007}{Hilairet et al., 2007}; \protect\hyperlink{ref-karato1993}{Karato \& Wu, 1993}; \protect\hyperlink{ref-ranalli1995}{Ranalli, 1995}; \protect\hyperlink{ref-turcotte2002}{Turcotte \& Schubert, 2002}). For example, if rocks were highly compressible or could sustain large deviatoric stresses, pressures and temperatures might be different. However, insofar as Earth's lithosphere closely behaves like an incompressible visco-plastic fluid (under the assumptions above, \protect\hyperlink{ref-gerya2019}{Gerya, 2019}; \protect\hyperlink{ref-gerya2003}{Gerya \& Yuen, 2003}; \protect\hyperlink{ref-kerswell2021}{Kerswell et al., 2021}), comparisons between marker \gls{pt} distributions and the rock record may be made.

\hypertarget{marker-classification}{%
\subsection{Marker Classification}\label{marker-classification}}

For each numerical experiment, 20,990 markers were initially selected from within a 760 km-long and 8 km-deep section of oceanic crust and seafloor sediments at \(t\) = 0 Ma. Tracing proceeded for 115 timesteps (between 9.3-54.7 Ma depending on convergence velocity), which was sufficient for markers to be potentially subducted very deeply (up to 300 km) from their initial positions. However, only markers that detached from the subducting oceanic plate were relevant for comparison with \gls{pt} estimates of natural rocks (because such markers and rocks were not subducted). The main challenge, therefore, was to first develop a method for determining which markers among 20,990 detached and moved away from the subducting plate without knowing their fate \emph{a priori}. Moreover, the method needed to be generalizable to a large range of numerical experiments. Note that markers that detach \DIFdelbegin \DIFdel{are }\DIFdelend \DIFaddbegin \DIFadd{were }\DIFaddend classified as ``recovered'' even if they do not exhume to the surface in the modelling domain. The rationale is that diverse processes can cause exhumation of subduction zone rocks, including later tectonic events. Although not all detached markers might ever exhume, they all could be exhumed, in contrast to markers that clearly do not detach.

Classifying markers as either ``recovered'' or ``not recovered'' based solely on their undifferentiated traced histories defines an unsupervised classification problem (\protect\hyperlink{ref-barlow1989}{Barlow, 1989}). To solve the unsupervised classification problem, this study implemented a Gaussian mixture model (\protect\hyperlink{ref-reynolds2009}{Reynolds, 2009})---a type of ``soft'' clustering algorithm used extensively for pattern recognition, anomaly detection, and estimating complex probability distribution functions (e.g. \protect\hyperlink{ref-banfield1993}{Banfield \& Raftery, 1993}; \protect\hyperlink{ref-celeux1995}{Celeux \& Govaert, 1995}; \protect\hyperlink{ref-figueiredo2002}{Figueiredo \& Jain, 2002}; \protect\hyperlink{ref-fraley2002}{Fraley \& Raftery, 2002}; \protect\hyperlink{ref-vermeesch2018}{Vermeesch, 2018}). In this case, a Gaussian mixture model organizes markers into groups (clusters) by fitting \(k\) = 14 bivariate Gaussian ellipsoids to the distribution of markers in \gls{pt} space. ``Fitting'' refers to adjusting parameters (centroids and covariance matrices) of all \(k\) Gaussian ellipsoids until the ellipsoids and data achieve maximum likelihood (see Appendix \ref{gmm} for a full mathematical description). Finally, marker clusters with centroids located within certain bounds are classified as ``recovered''. The entire classification algorithm can be summarised as follows:

\begin{quote}
\uline{Classifier algorithm:}\\
\strut \\
\hspace*{0.333em}0. Select markers within a 760 \(km\ \times\) 8 km section of oceanic crust\\
\hspace*{0.333em}1. Trace markers for 115 timesteps\\
\hspace*{0.333em}2. Identify maximum marker \glspl{pt}\\
\hspace*{0.333em}3. Apply Gaussian mixture modelling to maximum marker \glspl{pt}\\
\hspace*{0.333em}5. Check for cluster centroids within the bounds:\\
\hspace*{0.333em}\hspace*{0.333em}\hspace*{0.333em}\hspace*{0.333em}\(\geq\) 3 \(^{\circ}\)C/km AND \(\leq\) 1300 \(^{\circ}\)C AND \(\leq\) 120 km~(3.4 GPa)\\
\hspace*{0.333em}6. Classify marker clusters found in step 5 as recovered\\
\hspace*{0.333em}7. Classify all other markers as subducted
\end{quote}

\noindent Note that maximum marker \glspl{pt} used for clustering are assessed before markers transform (dehydrate or melt) and before the accretionary wedge toe collides with the high-viscosity convergence region positioned at 500 km from the left boundary (to avoid spurious maximum \glspl{pt} from sudden isothermal burial). \DIFdelbegin \DIFdel{Also note that ``hard}\DIFdelend \DIFaddbegin \DIFadd{``Hard}\DIFaddend '' classification is possible by directly applying simple rules to markers without clustering (e.g. \protect\hyperlink{ref-roda2012}{Roda et al., 2012}). However, ``hard'' methods are less generalizable than ``soft'' approaches like Gaussian mixture models, which can be implemented to study many possible features in numerical simulations with Lagrangian reference frames---not just recovery of subducted material. Figures \ref{fig:class} \& \ref{fig:bigComp} illustrate marker classification for 1 of 64 numerical experiments. All other experiments are presented in Appendix \ref{vis}.

\hypertarget{recovery-modes}{%
\subsection{Recovery Modes}\label{recovery-modes}}

To better understand how rock recovery can vary among subduction zones with different boundary conditions, important modes of marker recovery \DIFdelbegin \DIFdel{are }\DIFdelend \DIFaddbegin \DIFadd{were }\DIFaddend determined by finding peaks in marker density with respect to \gls{pt} and thermal gradients. The tallest density peak (mode1) gives a sense of where the greatest abundance of markers are being recovered from. The deepest, or warmest, density peak (mode2) gives a sense of how the most deeply subducted markers (or markers with the highest thermal gradients) vary across different numerical experiments.

Note that correlations are not presented with respect to the thermal parameter \(\Phi\) (\(\Phi\) = \(age \cdot \vec{v}\)), unlike many studies. The rationale is three-fold: (1) the aim is to understand how oceanic plate age and convergence velocity affect marker recovery independently, (2) sample sizes of recovered markers are larger when grouped by oceanic plate age and convergence velocity (n = 335,840) compared to grouping by \(\Phi\) (n = 83,960), and (3) \DIFdelbegin \DIFdel{because }\DIFdelend \(\Phi\) itself was formulated from a general relationship among thermo-kinematic boundary conditions to explain differences in earthquake distributions (\protect\hyperlink{ref-gorbatov1997}{Gorbatov \& Kostoglodov, 1997}; \protect\hyperlink{ref-kirby1991}{Kirby et al., 1991}; \protect\hyperlink{ref-mckenzie1969}{McKenzie, 1969}; \protect\hyperlink{ref-molnar1979}{Molnar et al., 1979}), and likely never intended as a precise tool for investigating subduction geodynamics. Indeed, correlations drawn among \(\Phi\) and thermo-kinematic boundary conditions in this study (not presented) are ambiguous and their inclusion detracts from the correlations drawn independently with respect to oceanic plate age and convergence velocity.



\begin{figure}[htbp]

{\centering \includegraphics[width=1\linewidth,]{assets/figs/chpt4/cde62_k14_class} 

}

\caption[Example of marker classification for model cde62]{Example of marker classification for model cde62. (a) Pressure vs.~temperature diagram showing distributions of 14 marker clusters (20,990 markers) as assigned by Gaussian mixture modelling. Markers belonging to clusters with centroids positioned at \(\geq\) 3 \(^{\circ}\)C/km AND \(\leq\) 1300 \(^{\circ}\)C AND \(\leq\) 120 km (3.4 GPa) are classified as recovered. All others are classified as not recovered. Thin lines are thermal gradients labeled in \(^{\circ}\)C/km. (b) Pressure vs.~temperature diagram showing marker classification in comparison with the pd15 and ag18 datasets. In this experiment, markers are generally recovered from thermal gradients between 5-10 \(^{\circ}\)C/km, but very few are recovered from the high-density region shared by pd15 and ag18 around 2 GPa and 550 \(^{\circ}\)C (light-pink density contours, solid and filled).}\label{fig:class}
\end{figure}



\begin{landscape}


\begin{figure}[htbp]

{\centering \includegraphics[width=1\linewidth,]{assets/figs/chpt4/cde62_bigComp} 

}

\caption[Summary of classification results for model cde62]{Summary of classification results for model cde62. (a) Pressure vs.~temperature diagram showing the density of recovered markers (point cloud and blue Tanaka contours) in comparison with the pd15 (solid density contours) and ag18 (filled density contours) datasets. Thin lines are thermal gradients labeled in \(^{\circ}\)C/km. Reaction boundaries for eclogitization of oceanic crust and antigorite dehydration are from Ito \& Kennedy (\protect\hyperlink{ref-ito1971}{1971}) and Schmidt \& Poli (\protect\hyperlink{ref-schmidt1998}{1998}), respectively. (b-d) Cumulative probability diagrams showing step-like curves that indicate trimodal recovery pressures (depths), bimodal recovery temperatures, and a smooth curve indicating almost all markers (97\%) are recovered from thermal gradients between 5-10 \(^{\circ}\)C/km compared to pd15 (94\%) and ag18 (70\%). (e) Visualization of log viscosity showing the major modes of marker recovery along the plate interface. Marker density is concentrated along relatively cool thermal gradients and within three distinct regions coinciding with the continental Moho (about 1 GPa), the onset of plate coupling (about 2.3 GPa and 625 \(^{\circ}\)C, and an intermediate cluster around 1.6 GPa and 350 \(^{\circ}\)C.}\label{fig:bigComp}
\end{figure}


\end{landscape}

\hypertarget{chpt4Results}{%
\section{Results}\label{chpt4Results}}

\hypertarget{pt-distributions-of-recovered-markers}{%
\subsection{PT Distributions of Recovered Markers}\label{pt-distributions-of-recovered-markers}}

Recovered markers generally overlap with pd15 and ag18 in terms of absolute \gls{pt} range, but show different density distributions across \gls{pt} space (Figure \ref{fig:marxComp}). \DIFdelbegin \DIFdel{Note that there are significant differences }\DIFdelend \DIFaddbegin \DIFadd{Significant differences become evident }\DIFaddend when comparing markers recovered from individual numerical experiments with the rock record vs.~combining results of experiments. For example, many numerical experiments show nonuniform marker recovery, evident by segregated high-density peaks and step-like \glspl{cdf} (e.g.~Figure \ref{fig:bigComp}). Combining recovered markers from multiple numerical experiments (analogous to randomly sampling exhumed \gls{hp} rocks from different subduction zones) effectively fills in density gaps and presents a smoother picture of marker recovery across \gls{pt} space.

Whether comparing the rock record with individual numerical experiments, suites of experiments, or \DIFdelbegin \DIFdel{with }\DIFdelend all numerical experiments, several key observations emerge from recovered markers:

\pagebreak

\begin{quote}
\uline{Key observations from recovered markers:}\\
\strut \\
\hspace*{0.333em}1. Recovered markers from individual numerical experiments show bi-\\
\hspace*{0.333em}\hspace*{0.333em}\hspace*{0.333em}\hspace*{0.333em}or trimodal recovery from discrete \gls{pt} regions with step-like \glspl{cdf}\\
\hspace*{0.333em}\hspace*{0.333em}\hspace*{0.333em}\hspace*{0.333em}(Figure \ref{fig:bigComp} \& Appendix \ref{vis})\\
\strut \\
\hspace*{0.333em}2. A significant gap in marker recovery exists around 2 GPa and 550 \(^{\circ}\)C\\
\hspace*{0.333em}\hspace*{0.333em}\hspace*{0.333em}\hspace*{0.333em}that coincides with the \DIFdelbegin \DIFdel{highest-density }\DIFdelend \DIFaddbegin \DIFadd{highest density }\DIFaddend of natural samples (Figure \ref{fig:marxComp}a)\\
\strut \\
\hspace*{0.333em}3. Markers are recovered from a single major mode near 1 GPa and \DIFdelbegin \DIFdel{trivial}\DIFdelend \DIFaddbegin \DIFadd{a}\DIFaddend \\
\hspace*{0.333em}\hspace*{0.333em}\hspace*{0.333em}\hspace*{0.333em}minor mode near 2.5 GPa with a high rate of recovery from lower\\
\hspace*{0.333em}\hspace*{0.333em}\hspace*{0.333em}\hspace*{0.333em}pressures (80\% from \(\leq\) 1.5 GPa) compared to natural samples (36-59\%\\
\hspace*{0.333em}\hspace*{0.333em}\hspace*{0.333em}\hspace*{0.333em}from \(\leq\) 1.5 GPa, Figure \ref{fig:marxComp}b)\\
\strut \\
\hspace*{0.333em}4. Markers are recovered from a single major mode near 350 \(^{\circ}\)C and a\\
\hspace*{0.333em}\hspace*{0.333em}\hspace*{0.333em}\hspace*{0.333em}\DIFdelbegin \DIFdel{trivial }\DIFdelend minor mode near 625 \(^{\circ}\)C with a high rate of recovery from lower\\
\hspace*{0.333em}\hspace*{0.333em}\hspace*{0.333em}\hspace*{0.333em}temperatures (92\% from \(\leq\) 500 \(^{\circ}\)C) compared to natural samples (37-\\
\hspace*{0.333em}\hspace*{0.333em}\hspace*{0.333em}\hspace*{0.333em}44\% from \(\leq\) 500 \(^{\circ}\)C, Figure \ref{fig:marxComp}c)\\
\strut \\
\hspace*{0.333em}5. The relative abundance of markers recovered along ``typical'' thermal\\
\hspace*{0.333em}\hspace*{0.333em}\hspace*{0.333em}\hspace*{0.333em}gradients for subduction zones (97\% from 5-15 \(^{\circ}\)C/km) is greater than\\
\hspace*{0.333em}\hspace*{0.333em}\hspace*{0.333em}\hspace*{0.333em}natural samples (70-93\% from 5-15 \(^{\circ}\)C/km, Figure \ref{fig:marxComp}d)\\
\strut \\
\hspace*{0.333em}6. Few markers are recovered from the forbidden zone (3\% from \(\leq\) 5\\
\hspace*{0.333em}\hspace*{0.333em}\hspace*{0.333em}\hspace*{0.333em}\(^{\circ}\)C/km, Figure \ref{fig:marxComp}d)\\
\strut \\
\hspace*{0.333em}7. Virtually no markers (0.009\%) are recovered from \(\geq\) 15 \(^{\circ}\)C/km\\
\hspace*{0.333em}\hspace*{0.333em}\hspace*{0.333em}\hspace*{0.333em}compared to natural samples (6-30\% from \(\geq\) 15 \(^{\circ}\)C/km, Figure \ref{fig:marxComp}d)
\end{quote}



\begin{figure}[htbp]

{\centering \includegraphics[width=1\linewidth,]{assets/figs/chpt4/marx_comp} 

}

\caption[\gls{pt} distribution of recovered markers from all simulations]{Recovered markers from all 64 simulations. (a) Pressure vs.~temperature diagram showing the densities of recovered markers (point cloud and blue Tanaka contours) in comparison with the pd15 (solid contours, \protect\hyperlink{ref-penniston2015}{Penniston-Dorland et al., 2015}) and ag18 (filled contours, \protect\hyperlink{ref-agard2018}{Agard et al., 2018}) datasets. Thin lines are thermal gradients labeled in \(^{\circ}\)C/km. Reaction boundaries for eclogitization of oceanic crust and antigorite dehydration are from Ito \& Kennedy (\protect\hyperlink{ref-ito1971}{1971}) and Schmidt \& Poli (\protect\hyperlink{ref-schmidt1998}{1998}), respectively. (b-d) Cumulative probability diagrams showing single-step \glspl{cdf} that indicate high recovery rates from 0.8-1.5 GPa, \(\leq\) 500 \(^{\circ}\)C, and \(\leq\) 15 \(^{\circ}\)C/km compared to natural samples (72\% vs.~28-43\%, 92\% vs.~37-44\%, \textgreater{} 99.99\% vs.~70-94\%, respectively). Metamorphic facies: (1) forbidden zone (\(\leq\) 5 \(^{\circ}\)C/km), (2) lws/ep blueschist to eclogite (5-15 \(^{\circ}\)C/km), (3) greenschist to amphibolite (15-30 \(^{\circ}\)C/km), (4) amphibolite to granulite (\(\geq\) 30 \(^{\circ}\)C/km).}\label{fig:marxComp}
\end{figure}

\hypertarget{correlations-with-boundary-conditions}{%
\subsection{Correlations with Boundary Conditions}\label{correlations-with-boundary-conditions}}

\hypertarget{oceanic-plate-age-effect}{%
\subsubsection{Oceanic Plate Age Effect}\label{oceanic-plate-age-effect}}

The average thermal gradient of recovered markers correlates strongly with oceanic plate age. Both major \gls{pt} gradient modes (PT gradient mode1 \& 2) correlate negatively with oceanic plate age, indicating an increase from about 6.5 \(^{\circ}\)C/km for old plates (\(\geq\) 85 Ma) to about 8.5 \(^{\circ}\)C/km for younger plates (\(\leq\) 55 Ma, Figure \ref{fig:corrComp}). The dominant temperature mode (temperature mode1) does not correlate with oceanic plate age, yet markers recovered from temperatures well in excess of 350 \(^{\circ}\)C (temperature mode2) show a small, but significant, inverse correlation. The dominant pressure mode (pressure mode1) weakly correlates with oceanic plate age, indicating a slightly higher chance of recovery from beyond the continental Moho for the oldest oceanic plates (\(\geq\) 85 Ma). In summary, oceanic plate age primarily affects the average \gls{pt} trajectory of recovered material, but does not significantly shift marker recovery along the plate interface. Finally, recovery rate does not correlate with oceanic plate age.

\hypertarget{convergence-velocity-effect}{%
\subsubsection{Convergence Velocity Effect}\label{convergence-velocity-effect}}

The average pressure of recovered markers correlates strongly with convergence velocity. The dominant pressure mode of recovered markers (pressure mode1) \DIFdelbegin \DIFdel{is strongly correlated }\DIFdelend \DIFaddbegin \DIFadd{strongly correlates }\DIFaddend with convergence velocity, but does not change significantly until convergence velocity drops below 66 km/Ma (Figure \ref{fig:corrComp}). More deeply subducted markers (pressure mode2) are inversely correlated with convergence velocity, showing average pressures increasing from about 1.7 GPa for fast moving plates (100 km/Ma) to about 2.6 GPa for slow moving plates (40 km/Ma). On the other hand, the major \gls{pt} gradient modes do not correlate with convergence velocity. In summary, decreasing convergence velocity shifts marker recovery deeper down the plate interface without significantly changing the average thermal gradient of subducted material. Finally, recovery rate does not correlate with convergence velocity.

\hypertarget{upper-plate-thickness-effect}{%
\subsubsection{Upper-plate Thickness Effect}\label{upper-plate-thickness-effect}}

Kerswell et al. (\protect\hyperlink{ref-kerswell2021}{2021}) \DIFdelbegin \DIFdel{demonstrates }\DIFdelend \DIFaddbegin \DIFadd{demonstrated }\DIFaddend an association between upper-plate thickness and coupling depth in the same numerical experiments used to trace markers. \gls{pt} distributions of markers might therefore respond strongly to changes in upper-plate thickness, especially with respect to pressure. However, a surprisingly negligible effect is observed. For example, neither of the important pressure modes (pressure mode1 \& 2) correlate with upper-plate thickness, nor does temperature mode 2 (usually the most deeply subducted markers) correlate with upper-plate thickness. In contrast, the average thermal gradient of recovered markers (\gls{pt} gradient mode1 \& 2) correlate with upper-plate thickness. The dominant temperature mode (temperature mode1) \DIFdelbegin \DIFdel{is correlated }\DIFdelend \DIFaddbegin \DIFadd{does correlate }\DIFaddend with upper-plate thickness, but not with any other boundary condition. Recovery rate is also correlated with upper-plate thickness, but not with any other boundary condition, indicating higher recovery rates are more likely underneath thick upper-plates (above 10 \% for 94 km-thick plates) than from underneath thin upper-plates (around 7.5 \% for \(\leq\) 62 km-thick plates, Figure \ref{fig:corrComp}). In summary, thin upper-plates are more likely to produce warmer thermal gradients, higher temperatures, and \DIFdelbegin \DIFdel{higher }\DIFdelend \DIFaddbegin \DIFadd{lower }\DIFaddend recovery rates.

\begin{landscape}




\begin{figure}[htbp]

{\centering \includegraphics[width=1\linewidth,]{assets/figs/chpt4/correlations_comp} 

}

\caption[Correlations among recovery modes and boundary conditions]{Correlations among important modes of marker recovery and thermo-kinematic boundary conditions. The dominant recovery mode (mode1) indicates the position of the tallest density peak (i.e. where the greatest number of markers detach), while mode2 indicates the position of the warmest, deepest, or highest gradient density peak (i.e. where the greatest number of deeply subducted markers detach). Oceanic plate age and upper-plate thickness affect the average \gls{pt} gradient of recovered markers (strong correlation with gradient modes), while convergence velocity shifts marker recovery up and down the plate interface (strong correlation with pressure and temperature modes). The dominant temperature mode (temperature mode1) and recovery rate are correlated with upper-plate thickness, but not with any other boundary condition. Symbols indicate the Spearman's rank correlation coeffecient that measures the significance of monotonic correlations. *** $\rho \leq$ 0.001, ** $\rho \leq$ 0.01, * $\rho \leq$ 0.05, - $\rho \geq$ 0.05. Boxplots with darker fills are highly significant (***), whereas faded boxplots are less significant (**, *) or insignificant (-).}\label{fig:corrComp}
\end{figure}


\end{landscape}

\begingroup
\renewcommand{\arraystretch}{0.5}

\begin{landscape}\begingroup\fontsize{10}{12}\selectfont

\begin{longtable}[t]{lrrrrrrrrrrrrr}
\caption{\label{tab:recSummary}Subduction zone parameters and marker classification summary}\\
\toprule
\multicolumn{5}{c}{Initial Boundary Conditions} & \multicolumn{9}{c}{Marker Classification Summary} \\
\cmidrule(l{3pt}r{3pt}){1-5} \cmidrule(l{3pt}r{3pt}){6-14}
\multicolumn{1}{c}{model} & \multicolumn{1}{c}{$\Phi$} & \multicolumn{1}{c}{$Z_{UP}$} & \multicolumn{1}{c}{age} & \multicolumn{1}{c}{$\vec{v}$} & \multicolumn{1}{c}{rec.} & \multicolumn{1}{c}{$\sigma$} & \multicolumn{1}{c}{rate} & \multicolumn{1}{c}{P md1} & \multicolumn{1}{c}{P md2} & \multicolumn{1}{c}{T md1} & \multicolumn{1}{c}{T md2} & \multicolumn{1}{c}{$\nabla$ md1} & \multicolumn{1}{c}{$\nabla$ md2} \\
\cmidrule(l{0pt}r{0pt}){1-1} \cmidrule(l{0pt}r{0pt}){2-2} \cmidrule(l{0pt}r{0pt}){3-3} \cmidrule(l{0pt}r{0pt}){4-4} \cmidrule(l{0pt}r{0pt}){5-5} \cmidrule(l{0pt}r{0pt}){6-6} \cmidrule(l{0pt}r{0pt}){7-7} \cmidrule(l{0pt}r{0pt}){8-8} \cmidrule(l{0pt}r{0pt}){9-9} \cmidrule(l{0pt}r{0pt}){10-10} \cmidrule(l{0pt}r{0pt}){11-11} \cmidrule(l{0pt}r{0pt}){12-12} \cmidrule(l{0pt}r{0pt}){13-13} \cmidrule(l{0pt}r{0pt}){14-14}
 & km & km & Ma & km/Ma &  &  & \% & GPa & GPa & $^{\circ}$C & $^{\circ}$C & $^{\circ}$C/km & $^{\circ}$C/km\\
\midrule
\endfirsthead
\caption[]{\label{tab:recSummary}Subduction zone parameters and marker classification summary \textit{(continued)}}\\
\toprule
\multicolumn{5}{c}{Initial Boundary Conditions} & \multicolumn{9}{c}{Marker Classification Summary} \\
\cmidrule(l{3pt}r{3pt}){1-5} \cmidrule(l{3pt}r{3pt}){6-14}
\multicolumn{1}{c}{model} & \multicolumn{1}{c}{$\Phi$} & \multicolumn{1}{c}{$Z_{UP}$} & \multicolumn{1}{c}{age} & \multicolumn{1}{c}{$\vec{v}$} & \multicolumn{1}{c}{rec.} & \multicolumn{1}{c}{$\sigma$} & \multicolumn{1}{c}{rate} & \multicolumn{1}{c}{P md1} & \multicolumn{1}{c}{P md2} & \multicolumn{1}{c}{T md1} & \multicolumn{1}{c}{T md2} & \multicolumn{1}{c}{$\nabla$ md1} & \multicolumn{1}{c}{$\nabla$ md2} \\
\cmidrule(l{0pt}r{0pt}){1-1} \cmidrule(l{0pt}r{0pt}){2-2} \cmidrule(l{0pt}r{0pt}){3-3} \cmidrule(l{0pt}r{0pt}){4-4} \cmidrule(l{0pt}r{0pt}){5-5} \cmidrule(l{0pt}r{0pt}){6-6} \cmidrule(l{0pt}r{0pt}){7-7} \cmidrule(l{0pt}r{0pt}){8-8} \cmidrule(l{0pt}r{0pt}){9-9} \cmidrule(l{0pt}r{0pt}){10-10} \cmidrule(l{0pt}r{0pt}){11-11} \cmidrule(l{0pt}r{0pt}){12-12} \cmidrule(l{0pt}r{0pt}){13-13} \cmidrule(l{0pt}r{0pt}){14-14}
 & km & km & Ma & km/Ma &  &  & \% & GPa & GPa & $^{\circ}$C & $^{\circ}$C & $^{\circ}$C/km & $^{\circ}$C/km\\
\midrule
\endhead

\endfoot
\bottomrule
\endlastfoot
\cellcolor{gray!6}{cda46} & \cellcolor{gray!6}{13.0} & \cellcolor{gray!6}{46} & \cellcolor{gray!6}{32.6} & \cellcolor{gray!6}{40} & \cellcolor{gray!6}{1486} & \cellcolor{gray!6}{16} & \cellcolor{gray!6}{7.9} & \cellcolor{gray!6}{1.12} & \cellcolor{gray!6}{2.46} & \cellcolor{gray!6}{382.5} & \cellcolor{gray!6}{596.3} & \cellcolor{gray!6}{8.4} & \cellcolor{gray!6}{11.3}\\
cda62 & 13.0 & 62 & 32.6 & 40 & 1347 & 14 & 7.1 & 1.12 & 2.21 & 352.5 & 645.7 & 8.5 & 10.8\\
\cellcolor{gray!6}{cda78} & \cellcolor{gray!6}{13.0} & \cellcolor{gray!6}{78} & \cellcolor{gray!6}{32.6} & \cellcolor{gray!6}{40} & \cellcolor{gray!6}{1859} & \cellcolor{gray!6}{13} & \cellcolor{gray!6}{9.8} & \cellcolor{gray!6}{1.39} & \cellcolor{gray!6}{2.38} & \cellcolor{gray!6}{626.6} & \cellcolor{gray!6}{626.6} & \cellcolor{gray!6}{7.8} & \cellcolor{gray!6}{12.2}\\
cda94 & 13.0 & 94 & 32.6 & 40 & 1938 & 16 & 10.3 & 1.25 & 2.64 & 346.6 & 577.6 & 8.0 & 9.4\\
\cellcolor{gray!6}{cdb46} & \cellcolor{gray!6}{21.5} & \cellcolor{gray!6}{46} & \cellcolor{gray!6}{32.6} & \cellcolor{gray!6}{66} & \cellcolor{gray!6}{1822} & \cellcolor{gray!6}{21} & \cellcolor{gray!6}{9.6} & \cellcolor{gray!6}{1.04} & \cellcolor{gray!6}{1.99} & \cellcolor{gray!6}{377.1} & \cellcolor{gray!6}{679.4} & \cellcolor{gray!6}{8.6} & \cellcolor{gray!6}{12.2}\\
cdb62 & 21.5 & 62 & 32.6 & 66 & 1408 & 11 & 7.5 & 1.00 & 2.16 & 310.1 & 624.1 & 8.6 & 10.2\\
\cellcolor{gray!6}{cdb78} & \cellcolor{gray!6}{21.5} & \cellcolor{gray!6}{78} & \cellcolor{gray!6}{32.6} & \cellcolor{gray!6}{66} & \cellcolor{gray!6}{1887} & \cellcolor{gray!6}{15} & \cellcolor{gray!6}{10.0} & \cellcolor{gray!6}{0.92} & \cellcolor{gray!6}{2.55} & \cellcolor{gray!6}{286.9} & \cellcolor{gray!6}{595.3} & \cellcolor{gray!6}{8.4} & \cellcolor{gray!6}{8.4}\\
cdb94 & 21.5 & 94 & 32.6 & 66 & 2351 & 45 & 12.4 & 1.15 & 2.63 & 320.2 & 584.7 & 7.9 & 8.6\\
\cellcolor{gray!6}{cdc46} & \cellcolor{gray!6}{26.1} & \cellcolor{gray!6}{46} & \cellcolor{gray!6}{32.6} & \cellcolor{gray!6}{80} & \cellcolor{gray!6}{1739} & \cellcolor{gray!6}{23} & \cellcolor{gray!6}{9.2} & \cellcolor{gray!6}{1.02} & \cellcolor{gray!6}{1.37} & \cellcolor{gray!6}{358.6} & \cellcolor{gray!6}{680.2} & \cellcolor{gray!6}{9.5} & \cellcolor{gray!6}{11.4}\\
cdc62 & 26.1 & 62 & 32.6 & 80 & 1294 & 9 & 6.9 & 0.99 & 2.01 & 310.5 & 594.3 & 8.7 & 10.4\\
\cellcolor{gray!6}{cdc78} & \cellcolor{gray!6}{26.1} & \cellcolor{gray!6}{78} & \cellcolor{gray!6}{32.6} & \cellcolor{gray!6}{80} & \cellcolor{gray!6}{1810} & \cellcolor{gray!6}{15} & \cellcolor{gray!6}{9.6} & \cellcolor{gray!6}{0.92} & \cellcolor{gray!6}{2.88} & \cellcolor{gray!6}{316.7} & \cellcolor{gray!6}{592.1} & \cellcolor{gray!6}{8.3} & \cellcolor{gray!6}{9.6}\\
cdc94 & 26.1 & 94 & 32.6 & 80 & 2170 & 13 & 11.5 & 1.15 & 3.01 & 311.0 & 591.2 & 7.7 & 9.9\\
\cellcolor{gray!6}{cdd46} & \cellcolor{gray!6}{32.6} & \cellcolor{gray!6}{46} & \cellcolor{gray!6}{32.6} & \cellcolor{gray!6}{100} & \cellcolor{gray!6}{1060} & \cellcolor{gray!6}{31} & \cellcolor{gray!6}{5.6} & \cellcolor{gray!6}{1.00} & \cellcolor{gray!6}{1.78} & \cellcolor{gray!6}{319.0} & \cellcolor{gray!6}{319.0} & \cellcolor{gray!6}{9.1} & \cellcolor{gray!6}{11.8}\\
cdd62 & 32.6 & 62 & 32.6 & 100 & 1364 & 14 & 7.2 & 0.99 & 1.63 & 334.1 & 596.0 & 8.8 & 11.1\\
\cellcolor{gray!6}{cdd78} & \cellcolor{gray!6}{32.6} & \cellcolor{gray!6}{78} & \cellcolor{gray!6}{32.6} & \cellcolor{gray!6}{100} & \cellcolor{gray!6}{1899} & \cellcolor{gray!6}{11} & \cellcolor{gray!6}{10.1} & \cellcolor{gray!6}{1.00} & \cellcolor{gray!6}{1.93} & \cellcolor{gray!6}{314.1} & \cellcolor{gray!6}{620.2} & \cellcolor{gray!6}{8.7} & \cellcolor{gray!6}{14.1}\\
cdd94 & 32.6 & 94 & 32.6 & 100 & 2717 & 15 & 14.4 & 1.23 & 2.90 & 313.5 & 720.6 & 7.4 & 10.2\\
\cellcolor{gray!6}{cde46} & \cellcolor{gray!6}{22.0} & \cellcolor{gray!6}{46} & \cellcolor{gray!6}{55.0} & \cellcolor{gray!6}{40} & \cellcolor{gray!6}{1627} & \cellcolor{gray!6}{37} & \cellcolor{gray!6}{8.6} & \cellcolor{gray!6}{1.11} & \cellcolor{gray!6}{3.20} & \cellcolor{gray!6}{376.2} & \cellcolor{gray!6}{666.7} & \cellcolor{gray!6}{7.7} & \cellcolor{gray!6}{10.7}\\
cde62 & 22.0 & 62 & 55.0 & 40 & 1794 & 21 & 9.5 & 1.08 & 2.24 & 303.6 & 621.6 & 7.8 & 10.4\\
\cellcolor{gray!6}{cde78} & \cellcolor{gray!6}{22.0} & \cellcolor{gray!6}{78} & \cellcolor{gray!6}{55.0} & \cellcolor{gray!6}{40} & \cellcolor{gray!6}{1867} & \cellcolor{gray!6}{18} & \cellcolor{gray!6}{9.9} & \cellcolor{gray!6}{1.39} & \cellcolor{gray!6}{2.52} & \cellcolor{gray!6}{372.6} & \cellcolor{gray!6}{630.8} & \cellcolor{gray!6}{6.8} & \cellcolor{gray!6}{9.8}\\
cde94 & 22.0 & 94 & 55.0 & 40 & 1804 & 13 & 9.5 & 2.39 & 2.54 & 315.2 & 681.4 & 7.6 & 9.4\\
\cellcolor{gray!6}{cdf46} & \cellcolor{gray!6}{36.3} & \cellcolor{gray!6}{46} & \cellcolor{gray!6}{55.0} & \cellcolor{gray!6}{66} & \cellcolor{gray!6}{2279} & \cellcolor{gray!6}{36} & \cellcolor{gray!6}{12.1} & \cellcolor{gray!6}{1.11} & \cellcolor{gray!6}{2.69} & \cellcolor{gray!6}{351.1} & \cellcolor{gray!6}{695.4} & \cellcolor{gray!6}{8.0} & \cellcolor{gray!6}{10.2}\\
cdf62 & 36.3 & 62 & 55.0 & 66 & 1579 & 25 & 8.4 & 1.14 & 2.20 & 279.4 & 647.1 & 7.5 & 9.5\\
\cellcolor{gray!6}{cdf78} & \cellcolor{gray!6}{36.3} & \cellcolor{gray!6}{78} & \cellcolor{gray!6}{55.0} & \cellcolor{gray!6}{66} & \cellcolor{gray!6}{1629} & \cellcolor{gray!6}{9} & \cellcolor{gray!6}{8.6} & \cellcolor{gray!6}{0.99} & \cellcolor{gray!6}{2.77} & \cellcolor{gray!6}{313.3} & \cellcolor{gray!6}{623.9} & \cellcolor{gray!6}{7.1} & \cellcolor{gray!6}{8.7}\\
cdf94 & 36.3 & 94 & 55.0 & 66 & 1972 & 18 & 10.4 & 0.93 & 2.80 & 224.8 & 671.5 & 6.9 & 9.8\\
\cellcolor{gray!6}{cdg46} & \cellcolor{gray!6}{44.0} & \cellcolor{gray!6}{46} & \cellcolor{gray!6}{55.0} & \cellcolor{gray!6}{80} & \cellcolor{gray!6}{2133} & \cellcolor{gray!6}{18} & \cellcolor{gray!6}{11.3} & \cellcolor{gray!6}{1.20} & \cellcolor{gray!6}{1.97} & \cellcolor{gray!6}{373.4} & \cellcolor{gray!6}{524.4} & \cellcolor{gray!6}{8.7} & \cellcolor{gray!6}{12.4}\\
cdg62 & 44.0 & 62 & 55.0 & 80 & 1363 & 19 & 7.2 & 1.00 & 1.73 & 274.3 & 547.7 & 7.9 & 7.9\\
\cellcolor{gray!6}{cdg78} & \cellcolor{gray!6}{44.0} & \cellcolor{gray!6}{78} & \cellcolor{gray!6}{55.0} & \cellcolor{gray!6}{80} & \cellcolor{gray!6}{1604} & \cellcolor{gray!6}{16} & \cellcolor{gray!6}{8.5} & \cellcolor{gray!6}{1.01} & \cellcolor{gray!6}{2.15} & \cellcolor{gray!6}{298.6} & \cellcolor{gray!6}{633.5} & \cellcolor{gray!6}{7.5} & \cellcolor{gray!6}{10.1}\\
cdg94 & 44.0 & 94 & 55.0 & 80 & 2154 & 13 & 11.4 & 0.98 & 2.69 & 288.9 & 629.0 & 6.9 & 9.2\\
\cellcolor{gray!6}{cdh46} & \cellcolor{gray!6}{55.0} & \cellcolor{gray!6}{46} & \cellcolor{gray!6}{55.0} & \cellcolor{gray!6}{100} & \cellcolor{gray!6}{980} & \cellcolor{gray!6}{40} & \cellcolor{gray!6}{5.2} & \cellcolor{gray!6}{0.98} & \cellcolor{gray!6}{1.65} & \cellcolor{gray!6}{375.0} & \cellcolor{gray!6}{391.8} & \cellcolor{gray!6}{8.9} & \cellcolor{gray!6}{11.0}\\
cdh62 & 55.0 & 62 & 55.0 & 100 & 1469 & 12 & 7.8 & 0.99 & 1.73 & 258.0 & 289.1 & 7.4 & 7.4\\
\cellcolor{gray!6}{cdh78} & \cellcolor{gray!6}{55.0} & \cellcolor{gray!6}{78} & \cellcolor{gray!6}{55.0} & \cellcolor{gray!6}{100} & \cellcolor{gray!6}{1635} & \cellcolor{gray!6}{13} & \cellcolor{gray!6}{8.7} & \cellcolor{gray!6}{0.99} & \cellcolor{gray!6}{1.71} & \cellcolor{gray!6}{296.4} & \cellcolor{gray!6}{473.8} & \cellcolor{gray!6}{7.0} & \cellcolor{gray!6}{8.9}\\
cdh94 & 55.0 & 94 & 55.0 & 100 & 2291 & 13 & 12.1 & 0.88 & 1.20 & 307.0 & 484.7 & 7.0 & 12.3\\
\cellcolor{gray!6}{cdi46} & \cellcolor{gray!6}{34.0} & \cellcolor{gray!6}{46} & \cellcolor{gray!6}{85.0} & \cellcolor{gray!6}{40} & \cellcolor{gray!6}{1249} & \cellcolor{gray!6}{55} & \cellcolor{gray!6}{6.6} & \cellcolor{gray!6}{1.17} & \cellcolor{gray!6}{2.85} & \cellcolor{gray!6}{323.3} & \cellcolor{gray!6}{641.7} & \cellcolor{gray!6}{6.9} & \cellcolor{gray!6}{12.7}\\
cdi62 & 34.0 & 62 & 85.0 & 40 & 1919 & 17 & 10.2 & 1.44 & 2.28 & 274.3 & 622.6 & 6.7 & 8.0\\
\cellcolor{gray!6}{cdi78} & \cellcolor{gray!6}{34.0} & \cellcolor{gray!6}{78} & \cellcolor{gray!6}{85.0} & \cellcolor{gray!6}{40} & \cellcolor{gray!6}{2043} & \cellcolor{gray!6}{11} & \cellcolor{gray!6}{10.8} & \cellcolor{gray!6}{1.65} & \cellcolor{gray!6}{2.56} & \cellcolor{gray!6}{347.2} & \cellcolor{gray!6}{560.2} & \cellcolor{gray!6}{5.7} & \cellcolor{gray!6}{8.4}\\
cdi94 & 34.0 & 94 & 85.0 & 40 & 2014 & 14 & 10.7 & 1.66 & 3.07 & 369.8 & 575.7 & 5.3 & 8.6\\
\cellcolor{gray!6}{cdj46} & \cellcolor{gray!6}{56.1} & \cellcolor{gray!6}{46} & \cellcolor{gray!6}{85.0} & \cellcolor{gray!6}{66} & \cellcolor{gray!6}{1668} & \cellcolor{gray!6}{67} & \cellcolor{gray!6}{8.8} & \cellcolor{gray!6}{1.07} & \cellcolor{gray!6}{2.41} & \cellcolor{gray!6}{344.2} & \cellcolor{gray!6}{553.9} & \cellcolor{gray!6}{9.0} & \cellcolor{gray!6}{9.1}\\
cdj62 & 56.1 & 62 & 85.0 & 66 & 1395 & 20 & 7.4 & 1.09 & 2.13 & 260.5 & 593.2 & 7.0 & 10.1\\
\cellcolor{gray!6}{cdj78} & \cellcolor{gray!6}{56.1} & \cellcolor{gray!6}{78} & \cellcolor{gray!6}{85.0} & \cellcolor{gray!6}{66} & \cellcolor{gray!6}{1335} & \cellcolor{gray!6}{14} & \cellcolor{gray!6}{7.1} & \cellcolor{gray!6}{1.22} & \cellcolor{gray!6}{1.96} & \cellcolor{gray!6}{286.8} & \cellcolor{gray!6}{286.8} & \cellcolor{gray!6}{6.6} & \cellcolor{gray!6}{7.3}\\
cdj94 & 56.1 & 94 & 85.0 & 66 & 1883 & 22 & 10.0 & 1.03 & 1.52 & 217.5 & 346.5 & 6.2 & 8.3\\
\cellcolor{gray!6}{cdk46} & \cellcolor{gray!6}{68.0} & \cellcolor{gray!6}{46} & \cellcolor{gray!6}{85.0} & \cellcolor{gray!6}{80} & \cellcolor{gray!6}{1524} & \cellcolor{gray!6}{13} & \cellcolor{gray!6}{8.1} & \cellcolor{gray!6}{1.06} & \cellcolor{gray!6}{1.11} & \cellcolor{gray!6}{330.3} & \cellcolor{gray!6}{488.5} & \cellcolor{gray!6}{8.5} & \cellcolor{gray!6}{8.6}\\
cdk62 & 68.0 & 62 & 85.0 & 80 & 1262 & 13 & 6.7 & 1.07 & 1.82 & 294.4 & 469.6 & 6.7 & 7.8\\
\cellcolor{gray!6}{cdk78} & \cellcolor{gray!6}{68.0} & \cellcolor{gray!6}{78} & \cellcolor{gray!6}{85.0} & \cellcolor{gray!6}{80} & \cellcolor{gray!6}{1566} & \cellcolor{gray!6}{18} & \cellcolor{gray!6}{8.3} & \cellcolor{gray!6}{1.01} & \cellcolor{gray!6}{1.83} & \cellcolor{gray!6}{280.6} & \cellcolor{gray!6}{593.5} & \cellcolor{gray!6}{6.5} & \cellcolor{gray!6}{10.4}\\
cdk94 & 68.0 & 94 & 85.0 & 80 & 2076 & 17 & 11.0 & 1.04 & 3.18 & 303.1 & 701.9 & 6.4 & 8.5\\
\cellcolor{gray!6}{cdl46} & \cellcolor{gray!6}{85.0} & \cellcolor{gray!6}{46} & \cellcolor{gray!6}{85.0} & \cellcolor{gray!6}{100} & \cellcolor{gray!6}{803} & \cellcolor{gray!6}{11} & \cellcolor{gray!6}{4.2} & \cellcolor{gray!6}{1.09} & \cellcolor{gray!6}{1.56} & \cellcolor{gray!6}{344.6} & \cellcolor{gray!6}{362.4} & \cellcolor{gray!6}{9.0} & \cellcolor{gray!6}{9.0}\\
cdl62 & 85.0 & 62 & 85.0 & 100 & 1132 & 9 & 6.0 & 1.02 & 2.23 & 336.1 & 521.0 & 8.3 & 8.4\\
\cellcolor{gray!6}{cdl78} & \cellcolor{gray!6}{85.0} & \cellcolor{gray!6}{78} & \cellcolor{gray!6}{85.0} & \cellcolor{gray!6}{100} & \cellcolor{gray!6}{1734} & \cellcolor{gray!6}{15} & \cellcolor{gray!6}{9.2} & \cellcolor{gray!6}{1.04} & \cellcolor{gray!6}{1.94} & \cellcolor{gray!6}{257.1} & \cellcolor{gray!6}{317.5} & \cellcolor{gray!6}{6.6} & \cellcolor{gray!6}{10.5}\\
cdl94 & 85.0 & 94 & 85.0 & 100 & 2309 & 21 & 12.2 & 1.24 & 1.75 & 308.5 & 437.4 & 7.0 & 9.6\\
\cellcolor{gray!6}{cdm46} & \cellcolor{gray!6}{44.0} & \cellcolor{gray!6}{46} & \cellcolor{gray!6}{110.0} & \cellcolor{gray!6}{40} & \cellcolor{gray!6}{1412} & \cellcolor{gray!6}{17} & \cellcolor{gray!6}{7.5} & \cellcolor{gray!6}{1.39} & \cellcolor{gray!6}{3.14} & \cellcolor{gray!6}{363.0} & \cellcolor{gray!6}{612.5} & \cellcolor{gray!6}{7.1} & \cellcolor{gray!6}{13.8}\\
cdm62 & 44.0 & 62 & 110.0 & 40 & 2332 & 17 & 12.3 & 1.21 & 2.45 & 321.4 & 602.8 & 6.1 & 8.2\\
\cellcolor{gray!6}{cdm78} & \cellcolor{gray!6}{44.0} & \cellcolor{gray!6}{78} & \cellcolor{gray!6}{110.0} & \cellcolor{gray!6}{40} & \cellcolor{gray!6}{1826} & \cellcolor{gray!6}{16} & \cellcolor{gray!6}{9.7} & \cellcolor{gray!6}{1.48} & \cellcolor{gray!6}{2.51} & \cellcolor{gray!6}{375.2} & \cellcolor{gray!6}{631.0} & \cellcolor{gray!6}{6.4} & \cellcolor{gray!6}{7.7}\\
cdm94 & 44.0 & 94 & 110.0 & 40 & 1903 & 16 & 10.1 & 1.53 & 2.87 & 377.2 & 551.2 & 5.5 & 9.1\\
\cellcolor{gray!6}{cdn46} & \cellcolor{gray!6}{72.6} & \cellcolor{gray!6}{46} & \cellcolor{gray!6}{110.0} & \cellcolor{gray!6}{66} & \cellcolor{gray!6}{1997} & \cellcolor{gray!6}{46} & \cellcolor{gray!6}{10.6} & \cellcolor{gray!6}{1.25} & \cellcolor{gray!6}{2.34} & \cellcolor{gray!6}{404.2} & \cellcolor{gray!6}{691.2} & \cellcolor{gray!6}{10.1} & \cellcolor{gray!6}{14.2}\\
cdn62 & 72.6 & 62 & 110.0 & 66 & 1274 & 21 & 6.7 & 1.13 & 2.26 & 369.6 & 583.2 & 8.6 & 9.5\\
\cellcolor{gray!6}{cdn78} & \cellcolor{gray!6}{72.6} & \cellcolor{gray!6}{78} & \cellcolor{gray!6}{110.0} & \cellcolor{gray!6}{66} & \cellcolor{gray!6}{1709} & \cellcolor{gray!6}{18} & \cellcolor{gray!6}{9.0} & \cellcolor{gray!6}{1.38} & \cellcolor{gray!6}{1.46} & \cellcolor{gray!6}{295.2} & \cellcolor{gray!6}{310.1} & \cellcolor{gray!6}{7.2} & \cellcolor{gray!6}{9.0}\\
cdn94 & 72.6 & 94 & 110.0 & 66 & 1733 & 17 & 9.2 & 1.06 & 1.72 & 219.1 & 310.0 & 6.0 & 8.3\\
\cellcolor{gray!6}{cdo46} & \cellcolor{gray!6}{88.0} & \cellcolor{gray!6}{46} & \cellcolor{gray!6}{110.0} & \cellcolor{gray!6}{80} & \cellcolor{gray!6}{1466} & \cellcolor{gray!6}{40} & \cellcolor{gray!6}{7.8} & \cellcolor{gray!6}{1.20} & \cellcolor{gray!6}{1.88} & \cellcolor{gray!6}{381.9} & \cellcolor{gray!6}{381.9} & \cellcolor{gray!6}{8.6} & \cellcolor{gray!6}{8.6}\\
cdo62 & 88.0 & 62 & 110.0 & 80 & 1413 & 10 & 7.5 & 1.05 & 2.42 & 344.5 & 616.6 & 8.6 & 8.6\\
\cellcolor{gray!6}{cdo78} & \cellcolor{gray!6}{88.0} & \cellcolor{gray!6}{78} & \cellcolor{gray!6}{110.0} & \cellcolor{gray!6}{80} & \cellcolor{gray!6}{1707} & \cellcolor{gray!6}{11} & \cellcolor{gray!6}{9.0} & \cellcolor{gray!6}{0.92} & \cellcolor{gray!6}{1.38} & \cellcolor{gray!6}{291.0} & \cellcolor{gray!6}{291.0} & \cellcolor{gray!6}{6.4} & \cellcolor{gray!6}{9.6}\\
cdo94 & 88.0 & 94 & 110.0 & 80 & 2060 & 71 & 10.9 & 1.03 & 2.39 & 302.6 & 537.9 & 6.3 & 10.5\\
\cellcolor{gray!6}{cdp46} & \cellcolor{gray!6}{110.0} & \cellcolor{gray!6}{46} & \cellcolor{gray!6}{110.0} & \cellcolor{gray!6}{100} & \cellcolor{gray!6}{1524} & \cellcolor{gray!6}{83} & \cellcolor{gray!6}{8.1} & \cellcolor{gray!6}{1.27} & \cellcolor{gray!6}{1.63} & \cellcolor{gray!6}{477.6} & \cellcolor{gray!6}{477.6} & \cellcolor{gray!6}{10.3} & \cellcolor{gray!6}{10.5}\\
cdp62 & 110.0 & 62 & 110.0 & 100 & 1422 & 9 & 7.5 & 1.12 & 2.05 & 296.4 & 431.7 & 5.8 & 13.2\\
\cellcolor{gray!6}{cdp78} & \cellcolor{gray!6}{110.0} & \cellcolor{gray!6}{78} & \cellcolor{gray!6}{110.0} & \cellcolor{gray!6}{100} & \cellcolor{gray!6}{1812} & \cellcolor{gray!6}{18} & \cellcolor{gray!6}{9.6} & \cellcolor{gray!6}{1.11} & \cellcolor{gray!6}{1.87} & \cellcolor{gray!6}{333.9} & \cellcolor{gray!6}{481.1} & \cellcolor{gray!6}{7.3} & \cellcolor{gray!6}{7.3}\\
cdp94 & 110.0 & 94 & 110.0 & 100 & 2324 & 18 & 12.3 & 1.37 & 2.83 & 314.5 & 592.9 & 6.7 & 10.0\\*
\end{longtable}
\endgroup{}
\end{landscape}

\endgroup

\hypertarget{chpt4Discussion}{%
\section{Discussion}\label{chpt4Discussion}}

\hypertarget{thermo-kinematic-controls-on-rock-recovery}{%
\subsection{Thermo-Kinematic Controls on Rock Recovery}\label{thermo-kinematic-controls-on-rock-recovery}}

While the combined distribution of markers recovered from all numerical experiments shows appreciable deviations from \gls{pt} estimates compiled by Penniston-Dorland et al. (\protect\hyperlink{ref-penniston2015}{2015}) and Agard et al. (\protect\hyperlink{ref-agard2018}{2018}), markers recovered from simulations with the youngest oceanic plates (32.6-55 Ma) and the slowest convergence velocities (40 km/Ma) begin to resemble the distribution of exhumed \gls{hp} rocks (compare Figure \ref{fig:marxComp} with \ref{fig:olderSlow} \& \ref{fig:youngSlow}) with respect to gradients and absolute \gls{pt} conditions. Slower subduction of younger plates strongly shifts marker recovery down the plate interface (strong correlation with the deepest and warmest PT modes, Figure \ref{fig:corrComp}). The correlations in Figure \ref{fig:corrComp} also suggests this shift towards deeper recovery along the subduction interface should be complemented by thin upper-plates---implying systems with thin upper-plates, slow convergence, and young oceanic plates should be most consistent with the distribution of rock recovery implied by pd15 and ag18 (compare Figure \ref{fig:marxComp} with \ref{fig:olderSlow} \& \ref{fig:youngSlow}). This correspondence \DIFdelbegin \DIFdel{would }\DIFdelend \DIFaddbegin \DIFadd{might }\DIFaddend appear consistent with arguments that the rock record is comprised primarily of rock bodies exhumed from ``warm'' subduction settings (\protect\hyperlink{ref-abers2017}{Abers et al., 2017}; \protect\hyperlink{ref-vankeken2018}{van Keken et al., 2018})\DIFaddbegin \DIFadd{, except that rock recovery does not correlate with subducting plate age or velocity, and recovery is poorer for thinner, warmer, upper plates (Figure \ref{fig:corrComp})}\DIFaddend .

\DIFdelbegin \DIFdel{However}\DIFdelend \DIFaddbegin \DIFadd{Moreover}\DIFaddend , other kinematic characteristics correlate with age and velocity. For example, simulations with young oceanic plates (models: cda, cdb, cdc, cdd), especially at slow convergence velocities (model: cda) show high subduction angles (see Appendix \ref{vis}). High subduction angles tend to form thicker subduction interfaces that allow more markers to subduct to deeper, and thus warmer, conditions compared to other experiments that form narrow interfaces with shallow choke points (e.g.~model: cdj, see Appendix \ref{vis}). Observationally, the angle of subduction does not correlate significantly with plate age or velocity, but rather inversely with the duration of subduction (\protect\hyperlink{ref-hu2020}{Hu \& Gurnis, 2020}). Thus, the rock record might indicate preferential exhumation during the earlier stages of subduction when angles were steeper (although not necessarily during subduction initiation), even for older or faster subducting plates. More generally, differences in plate flexibility, overall subduction geometry, and velocity of plate motions strongly affect \gls{pt} distributions of rock recovery (\protect\hyperlink{ref-monie2009}{Monie \& Agard, 2009})---rather than strictly ``warm'' versus ``cool'' subduction settings \emph{per se}\DIFdelbegin \DIFdel{(e. g. }%DIFDELCMD < \protect\hyperlink{ref-syracuse2010}{Syracuse et al., 2010}%%%
\DIFdel{). }\DIFdelend \DIFaddbegin \DIFadd{. }\DIFaddend That is, thermo-kinematic boundary conditions typically inferred to strictly promote \emph{thermal} effects (e.g.~young oceanic plates fostering warmer thermal gradients) may indeed be promoting more \emph{kinematic} effects (e.g.~young flexible oceanic plates rolling back and supporting deeper subduction of material along thicker interfaces) that are subsequently \emph{observed} as thermal effects (average increase in marker \glspl{pt}).



\begin{figure}[htbp]

{\centering \includegraphics[width=1\linewidth,]{assets/figs/chpt4/marx_comp_older_slow} 

}

\caption[Marker recovery from older-slow subduction simulations]{Recovered markers from simulations with 55 Ma old oceanic plates and the slowest convergence velocities (40 km/Ma, models: cde46, cde62, cde78, cde94). (a) Pressure vs.~temperature diagram showing the densities of recovered markers (point cloud and blue Tanaka contours) in comparison with the pd15 (solid contours, \protect\hyperlink{ref-penniston2015}{Penniston-Dorland et al., 2015}) and ag18 (filled contours, \protect\hyperlink{ref-agard2018}{Agard et al., 2018}) datasets. Thin lines are thermal gradients labeled in \(^{\circ}\)C/km. Reaction boundaries for eclogitization of oceanic crust and antigorite dehydration are from Ito \& Kennedy (\protect\hyperlink{ref-ito1971}{1971}) and Schmidt \& Poli (\protect\hyperlink{ref-schmidt1998}{1998}), respectively. (b-d) Cumulative probability diagrams showing more uniform marker recovery with respect to pressure and considerable discrepancies among marker temperatures, \gls{pt} gradients, and natural samples.}\label{fig:olderSlow}
\end{figure}



\begin{figure}[htbp]

{\centering \includegraphics[width=1\linewidth,]{assets/figs/chpt4/marx_comp_young_slow} 

}

\caption[Marker recovery from young-slow subduction simulations]{Recovered markers from simulations with the youngest oceanic plates (32.6 Ma) and slowest convergence velocities (40 km/Ma, models: cda46, cda62, cda78, cda94). (a) Pressure vs.~temperature diagram showing the densities of recovered markers (point cloud and blue Tanaka contours) in comparison with the pd15 (solid contours, \protect\hyperlink{ref-penniston2015}{Penniston-Dorland et al., 2015}) and ag18 (filled contours, \protect\hyperlink{ref-agard2018}{Agard et al., 2018}) datasets. Thin lines are thermal gradients labeled in \(^{\circ}\)C/km. Reaction boundaries for eclogitization of oceanic crust and antigorite dehydration are from Ito \& Kennedy (\protect\hyperlink{ref-ito1971}{1971}) and Schmidt \& Poli (\protect\hyperlink{ref-schmidt1998}{1998}), respectively. (b-d) Cumulative probability diagrams showing the closest agreement between recovered markers and natural samples in this study. Notice considerable recovery gaps (markers) and discrepancies remain among marker temperatures, \gls{pt} gradients, and natural samples.}\label{fig:youngSlow}
\end{figure}

\hypertarget{comparison-with-other-numerical-experiments}{%
\subsection{Comparison with other Numerical Experiments}\label{comparison-with-other-numerical-experiments}}

Marker \gls{pt} distributions and their correlations with thermo-kinematic boundary conditions presented above are determined directly from large samples of recovered material evolving dynamically in a deforming subduction interface (analogous to reconstructing thermal gradients from large random samples of exhumed \gls{hp} rocks). In contrast, other studies investigating thermal responses to variable boundary conditions typically determine \gls{pt} gradients statically along discrete surfaces representing megathrust faults (e.g. \protect\hyperlink{ref-abers2006}{Abers et al., 2006}; \protect\hyperlink{ref-currie2004}{Currie et al., 2004}; \protect\hyperlink{ref-davies1999b}{Davies, 1999a}; \protect\hyperlink{ref-furukawa1993}{Furukawa, 1993}; \protect\hyperlink{ref-gao2014}{Gao \& Wang, 2014}; \protect\hyperlink{ref-mckenzie1969}{McKenzie, 1969}; \protect\hyperlink{ref-molnar1990}{Molnar \& England, 1990}; \protect\hyperlink{ref-peacock1999b}{Peacock \& Wang, 1999}; \protect\hyperlink{ref-syracuse2010}{Syracuse et al., 2010}; \protect\hyperlink{ref-vankeken2011}{van Keken et al., 2011}, \protect\hyperlink{ref-vankeken2019}{2019}; \protect\hyperlink{ref-wada2009}{Wada \& Wang, 2009}) or dynamically by ``finding'' the plate interface heuristically at each timestep (e.g. \protect\hyperlink{ref-arcay2017}{Arcay, 2017}; \protect\hyperlink{ref-holt2021}{Holt \& Condit, 2021}; \protect\hyperlink{ref-ruh2015}{Ruh et al., 2015}). Other studies using similar geodynamic codes have traced many fewer markers (typically dozens vs.~\textgreater{} 100,000\DIFdelbegin \DIFdel{, }\DIFdelend \DIFaddbegin \DIFadd{; }\DIFaddend \protect\hyperlink{ref-faccenda2008}{Faccenda et al., 2008}; \protect\hyperlink{ref-gerya2002}{Gerya et al., 2002}; \protect\hyperlink{ref-sizova2010}{Sizova et al., 2010}; \protect\hyperlink{ref-yamato2007}{Yamato et al., 2007}, \protect\hyperlink{ref-yamato2008}{2008}), so have less statistical rigor. This study stresses the importance of large sample sizes because individual marker \gls{pt} paths can vary considerably within a single simulation, yet important modes of recovery become apparent from density peaks as more markers are traced. Furthermore, most other studies make no attempt to discriminate \emph{recovery} (with some exceptions, e.g. \protect\hyperlink{ref-roda2012}{Roda et al., 2012}, \protect\hyperlink{ref-roda2020}{2020}), so marker \gls{pt} paths are less analogous to \gls{pt} paths determined by applying petrologic modelling, for example.

\hypertarget{geophysComp}{%
\subsection{Comparison with Geophysical Observations}\label{geophysComp}}

The locations of important recovery modes determined from numerical experiments correspond closely with the depths of important mechanical transitions inferred from seismic imaging studies and surface heat flow observations. For example, the dominant recovery mode common among all numerical experiments at about 1 GPa and 350 \(^{\circ}\)C (Table \ref{tab:recSummary} \& Figure \ref{fig:marxComp}) is consistent with a layer of low seismic velocities and high \(V_p/V_s\) ratios observed at numerous subduction zones between 20-50 km depth (\protect\hyperlink{ref-bostock2013}{Bostock, 2013}). While considerable unknowns persist about the nature of deformation in this region (\protect\hyperlink{ref-bostock2013}{Bostock, 2013}; \protect\hyperlink{ref-tewksbury2021b}{Tewksbury-Christle \& Behr, 2021}), the low-velocity layer, accompanied by low-frequency and slow-slip seismic events, is often interpreted as a transitional brittle-ductile shear zone actively accommodating underplating of subducted material and/or formation of a tectonic mélange around the base of the continental Moho (\protect\hyperlink{ref-audet2016}{Audet \& Kim, 2016}; \protect\hyperlink{ref-audet2018}{Audet \& Schaeffer, 2018}; \protect\hyperlink{ref-bostock2013}{Bostock, 2013}; \protect\hyperlink{ref-calvert2011}{Calvert et al., 2011}, \protect\hyperlink{ref-calvert2020}{2020}; \protect\hyperlink{ref-delph2021}{Delph et al., 2021}).

Formation of low-velocity layers and their geophysical properties are generally attributed to high pore-fluid pressures caused by metamorphic reactions relating to the dehydration of oceanic crust (\protect\hyperlink{ref-hacker2008}{Hacker, 2008}; \protect\hyperlink{ref-rondenay2008}{Rondenay et al., 2008}; \protect\hyperlink{ref-vankeken2011}{van Keken et al., 2011}). Surprisingly, despite implementation of a simple model for dehydration of oceanic crust (\protect\hyperlink{ref-ito1971}{Ito \& Kennedy, 1971}; \protect\hyperlink{ref-kerswell2021}{Kerswell et al., 2021}), and a simple visco-plastic rheologic model (\protect\hyperlink{ref-gerya2003}{Gerya \& Yuen, 2003}; \protect\hyperlink{ref-kerswell2021}{Kerswell et al., 2021}), the primary mode of marker recovery at 1.15 \(\pm\) 0.47 GPa (2 \(\sigma\), Table \ref{tab:recSummary}) coincides closely with the expected region for shallow underplating according to geophysical constraints (35 \(\pm\) 15 km or 1.0 \(\pm\) 0.4 GPa). The size of the markers dataset (n = 122,236) and prevalence of marker recovery from 1 GPa and 350 \(^{\circ}\)C suggest that although dehydration may indeed trigger detachment of subducting rocks, other factors---notably the compositional and mechanical transition in the upper-plate across the Moho---also influence detachment at this depth.

The termination of the low-velocity layer at depths beyond the continental Moho marks another important mechanical transition. This second transition is often interpreted as the onset of mechanical plate coupling near 80 km (or 2.3 GPa) and coincides well with the deeper recovery modes determined from recovered markers at 2.2 \(\pm\) 1 GPa (2 \(\sigma\), Table \ref{tab:recSummary}). Between these two modes of recovery at \textasciitilde40 and \textasciitilde80 km lies a gap that coincides with the densest recovery of natural rocks (pd15 \& ag18, Figure \ref{fig:marxComp}). This recovery gap is discussed in the following section.

\hypertarget{the-marker-recovery-gap}{%
\subsection{The Marker Recovery Gap}\label{the-marker-recovery-gap}}

Although at least a few markers are recovered across the range of the pd15 and ag18 datasets, the differences between recovered markers and distributions of natural samples are numerous, including: (1) an obvious lack of markers recovered from \(\geq\) 15 \(^{\circ}\)C/km (9\%) compared to pd15 and ag18 (37-48\%, Figure \ref{fig:marxComp}d), (2) recovery of markers from a single dominant mode near 1 GPa and 350 \(^{\circ}\)C compared to more uniform recovery across \gls{pt} space for natural samples (Figure \ref{fig:marxComp}a-c), (3) a general shift towards lower temperatures and cooler thermal gradients for markers compared to natural samples, and (4) a remarkable gap in marker recovery near 2 GPa and 550 \(^{\circ}\)C that coincides with the \DIFdelbegin \DIFdel{highest-density }\DIFdelend \DIFaddbegin \DIFadd{highest density }\DIFaddend of natural samples (Figure \ref{fig:marxComp}a). In fact, across 64 numerical experiments with wide-ranging initial conditions less than 1\% (0.71\%) of markers are recovered from between 1.8-2.2 GPa and 500-625 \(^{\circ}\)C. Why might this gap occur? Four possibilities are evident:

\begin{quote}
\uline{Explanations for nonoverlapping densities at 2 GPa and 550 $^{\circ}$C:}\\
\strut \\
\hspace*{0.333em}1. Simple rheologic models preclude certain recovery mechanisms\\
\hspace*{0.333em}\hspace*{0.333em}\hspace*{0.333em}\hspace*{0.333em}(poor implementation or modelling bias, Section \ref{geophysComp})\\
\strut \\
\hspace*{0.333em}2. Peak metamorphic conditions are systematically misinterpreted\\
\hspace*{0.333em}\hspace*{0.333em}\hspace*{0.333em}\hspace*{0.333em}(unknown \gls{pt} paths or petrologic bias, \protect\hyperlink{ref-penniston2015}{Penniston-Dorland et al., 2015})\\
\strut \\
\hspace*{0.333em}3. Rocks are frequently (re)sampled from the same unique conditions\\
\hspace*{0.333em}\hspace*{0.333em}\hspace*{0.333em}\hspace*{0.333em}(nonrandom sampling or scientific bias, \protect\hyperlink{ref-agard2018}{Agard et al., 2018})\\
\strut \\
\hspace*{0.333em}4. Rocks are recovered during short-lived events not implemented in\\
\hspace*{0.333em}\hspace*{0.333em}\hspace*{0.333em}\hspace*{0.333em}models (e.g.~subduction of seamounts, \protect\hyperlink{ref-agard2009}{Agard et al., 2009})
\end{quote}

\hypertarget{modelling-bias}{%
\subsubsection{Modelling Bias}\label{modelling-bias}}

Simplifying assumptions in the numerical experiments could potentially influence rock recovery from the subducting slab. Substantially lower temperatures and thermal gradients in numerical experiments compared to natural samples (Figure \ref{fig:marxComp}) possibly indicate imperfect implementation of heat generation and transfer (\protect\hyperlink{ref-kohn2018}{Kohn et al., 2018}; \protect\hyperlink{ref-penniston2015}{Penniston-Dorland et al., 2015}). Use of a simple serpentine rheology in the numerical experiments creates a weak interface. A stronger rheology (e.g.~quartz or a mixed melange zone, \protect\hyperlink{ref-beall2019}{Beall et al., 2019}; \protect\hyperlink{ref-ioannidi2021}{Ioannidi et al., 2021}) would yield greater heating and higher temperatures (\protect\hyperlink{ref-kohn2018}{Kohn et al., 2018}). In principle, a stronger rheology might shift the overall \gls{pt} distribution of markers to higher temperatures and help fill in the gap, or, possibly, change flow to extract rocks more uniformly along the plate interface. Although the effects of different interface rheologies on thermal structure or rock recovery were not explicitly explored in this study, even numerical simulations with the smallest \gls{pt} discrepancies between markers and natural samples (youngest oceanic plates and slowest convergence velocities, Figures \ref{fig:olderSlow} \& \ref{fig:youngSlow}) exhibit the same sizeable gap in marker recovery. Thus, higher temperatures alone would not seem to close the gap.

\hypertarget{petrologic-bias}{%
\subsubsection{Petrologic Bias}\label{petrologic-bias}}

Petrologic bias implies decades of field observations, conventional thermobarometry (e.g. \protect\hyperlink{ref-spear1983}{Spear \& Selverstone, 1983}), phase equilibria modelling (e.g. \protect\hyperlink{ref-connolly2005}{Connolly, 2005}), trace element thermometry (e.g. \protect\hyperlink{ref-ferry2007}{Ferry \& Watson, 2007}; \protect\hyperlink{ref-kohn2020}{Kohn, 2020}), and Raman Spectrometry of Carbonaceous Material thermometry (\protect\hyperlink{ref-beyssac2002}{Beyssac et al., 2002}) from many independent localities worldwide (e.g. \protect\hyperlink{ref-agard2009}{Agard et al., 2009}, \protect\hyperlink{ref-agard2018}{2018}; \protect\hyperlink{ref-angiboust2009}{Angiboust et al., 2009}, \protect\hyperlink{ref-angiboust2012a}{2012a}, \protect\hyperlink{ref-angiboust2016}{2016}; \protect\hyperlink{ref-avigad1991}{Avigad \& Garfunkel, 1991}; \protect\hyperlink{ref-monie2009}{Monie \& Agard, 2009}; \protect\hyperlink{ref-plunder2013}{Plunder et al., 2013}, \protect\hyperlink{ref-plunder2015}{2015}) have systematically misinterpreted the prograde and retrograde histories of exhumed \gls{hp} rocks. However, estimated uncertainties are generally too small for this argument to be viable (\protect\hyperlink{ref-penniston2015}{Penniston-Dorland et al., 2015}).

\hypertarget{scientific-biases}{%
\subsubsection{Scientific Biases}\label{scientific-biases}}

Two main factors lead to scientific bias. First, thermobarometric analysis is easier for certain rock types and mineral assemblages (e.g.~eclogite-facies metabasites and metapelitic schists) than for others (e.g.~quartzites, metagraywackes). Second, certain subduction complexes expose more rocks than others. These facts lead to sampling bias, both in the rocks that are selected for analysis and which subduction complexes contribute most to compilations. For example, a \gls{pt} condition of c.~550 \(^{\circ}\)C, 2 GPa typically yields assemblages that are both recognizable in the field (eclogites, \emph{sensu stricto}, and kyanite- or chloritoid-schists) and amenable to thermobarometric calculations and petrologic modelling. This fact may lead to oversampling of the rocks that yield these \gls{pt} conditions and the subduction zones that expose these rocks. In Penniston-Dorland et al. (\protect\hyperlink{ref-penniston2015}{2015}), the western and central European Alps, which contain many rocks that equilibrated near this \gls{pt} condition, represented \textasciitilde90 samples across \textless{} 1000 km (\textasciitilde1 sample per 100 km), whereas the Himalaya and Andes, which contained more diverse \gls{pt} conditions, represented only \textasciitilde1 sample per 300-400 km. Some subduction zones are not represented at all in these datasets (e.g.~central and western Aleutians, Kamchatka, Izu-Bonin-Marianas, Philippines, Indonesia, etc.), either because metamorphic rocks are not exposed or rock types are not amenable to petrologic investigation. Correcting for this type of bias is challenging because it would require large random samples of exhumed \gls{hp} rocks from localities worldwide and development of new techniques for quantifying \gls{pt} conditions in diverse rock types.

\hypertarget{short-lived-events}{%
\subsubsection{Short-lived Events}\label{short-lived-events}}

Detachment of rocks from the subducting slab might not occur continuously, but rather in response to specific events, such as subduction of asperities or seamounts (e.g. \protect\hyperlink{ref-agard2009}{Agard et al., 2009}) or abrupt fluid events. Yet no numerical models have attempted to model these events (insofar as determined from the available literature). In the case of seamounts, high surface roughness correlates with higher coefficients of friction (\protect\hyperlink{ref-gao2014}{Gao \& Wang, 2014}). Higher friction increases heating and temperatures, driving subduction interface thermal gradients into the field of \gls{pt} conditions defined by the pd15 and ag18 datasets (\protect\hyperlink{ref-kohn2018}{Kohn et al., 2018}). If asperities become mechanically unstable at depths of \textasciitilde50-70 km, preferential detachment would create an ``overabundance'' of recorded \gls{pt} conditions at moderate temperature (c.~550 \(^{\circ}\)C) at \textasciitilde2 GPa, as observed.

Alternatively, although fluid release is modelled as continuous, it may occur sporadically. Two dehydration reactions \DIFdelbegin \DIFdel{are }\DIFdelend \DIFaddbegin \DIFadd{along the subduction interface are particularly }\DIFaddend relevant: the transformation of lawsonite to epidote, and the transformation of chlorite (plus quartz) to garnet. Although dehydration of lawsonite is nearly discontinuous in \gls{pt} space, few rocks show clear evidence for lawsonite immediately prior to peak metamorphism (although such evidence can be subtle). In the context of equilibrium thermodynamics, chlorite dehydration should occur continuously below depths of \textasciitilde35 km, consistent with assumptions of many numerical geodynamic models. However, recent research suggests substantial overstepping of this reaction, resulting in the abrupt formation of abundant garnet and release of water (\protect\hyperlink{ref-castro2017}{Castro \& Spear, 2017}). Direct geochronology of garnet growth rates in subduction complexes also suggests abrupt growth and water release (\protect\hyperlink{ref-dragovic2015}{Dragovic et al., 2015}). Because fluids are thought to help trigger brittle failure (earthquakes) that could detach rocks from the subducting slab surface, abrupt release at a depth of \textasciitilde50-70 km might again result in an ``overabundance'' of recorded \gls{pt} conditions at pressures of \textasciitilde2 GPa. This mechanism would require relatively consistent degrees of overstepping in rocks of similar bulk composition and would not directly explain higher temperatures, however.

\hypertarget{chpt4Conclusions}{%
\section{Conclusion}\label{chpt4Conclusions}}

This study traces \gls{pt} paths of more than \DIFdelbegin \DIFdel{one-million }\DIFdelend \DIFaddbegin \DIFadd{one million }\DIFaddend markers from 64 subduction simulations representing a large range of presently-active subduction zones worldwide. Marker recovery is identified by implementing a ``soft'' clustering algorithm, and \gls{pt} distributions of recovered markers are compared among models and with the rock record. Such a large dataset presents a statistically-robust portrait of important recovery modes (where most markers are detached) along the plate interface. The three most important findings are as follows:

\begin{enumerate}
\def\labelenumi{\arabic{enumi}.}
\item
  Numerical simulations with simplified (de)hydration models and visco-plastic rheologies \DIFdelbegin \DIFdel{sufficiently }\DIFdelend simulate important recovery mechanisms near the base of the continental Moho around 1 GPa and 350 \(^{\circ}\)C (underplating and/or formation of tectonic mélanges) and near the depth of mechanical plate coupling around 2.5 GPa and 625 \(^{\circ}\)C\DIFaddbegin \DIFadd{.
}\DIFaddend \item
  Subduction systems with young oceanic plate ages, slow convergence velocities, and thin upper-plate lithospheres are most consistent with the rock record, but more likely because of kinematic effects (young flexible oceanic plates with high subduction angles accommodating deeper subduction of material) rather than strictly because of thermal effects (young oceanic plates promoting high thermal gradients)\DIFaddbegin \DIFadd{. Recovery rate does not correlate with either plate age or subduction rate, and warmer subduction zones show poorer recovery.
}\DIFaddend \item
  A gap in marker recovery near 2 GPa and 550 \(^{\circ}\)C \DIFdelbegin \DIFdel{, }\DIFdelend closely coincides with the \DIFdelbegin \DIFdel{highest-densities }\DIFdelend \DIFaddbegin \DIFadd{highest densities }\DIFaddend of natural samples and suggests an \DIFdelbegin \DIFdel{overabundance }\DIFdelend \DIFaddbegin \DIFadd{``overabundance'' }\DIFaddend of samples are studied from this \gls{pt} region. Explanations for \DIFdelbegin \DIFdel{an }\DIFdelend \DIFaddbegin \DIFadd{this }\DIFaddend ``overabundance'' \DIFdelbegin \DIFdel{of rocks from this }%DIFDELCMD < \gls{pt} %%%
\DIFdel{condition }\DIFdelend might include scientific bias, reaction overstepping (abrupt release of water), or processes such as subduction of seamounts that are not included in numerical simulations. Future work investigating natural samples and refining numerical geodynamic models might help resolve this discrepancy.
\end{enumerate}

\cleardoublepage

\hypertarget{chpt5}{%
\chapter{Conclusion}\label{chpt5}}

\markboth{Chapter 5: Conclusion}{Chapter 5: Conclusion}

This work uses three computational approaches---simulation, interpolation, and applied statistics (machine learning)---to address the following questions. How does plate interface mechanics change across a range of subduction zones and how can it be quantified with currently available petrologic and geophysical datasets?

Plate coupling observed in numerical simulations from Chapter \ref{chpt2} demonstrate steady-state mechanical behavior regulated self-consistently by feedbacks involving heat transfer and metamorphic dehydration of ultramafic sheet silicates in the upper-plate mantle. Coupling depth is only weakly correlated with \(\Phi\) but strongly correlated with upper-plate lithospheric thickness. Thus uniform coupling depths are expected if subduction zones have uniformly thick upper-plates, which is indeed the case when considering averaged backarc surface heat flow for 13 presently active subduction zones (\protect\hyperlink{ref-currie2004}{Currie et al., 2004}; \protect\hyperlink{ref-currie2006}{Currie \& Hyndman, 2006}; \protect\hyperlink{ref-hyndman2005}{Hyndman et al., 2005}; \protect\hyperlink{ref-wada2009}{Wada \& Wang, 2009}).

However, surface heat flow interpolations in Chapter \ref{chpt3} show a Kaleidoscope of upper-plate surface heat flow patterns---some implying continuous thermal structure, others implying discontinuous thermal structure---for \DIFdelbegin \DIFdel{17 }\DIFdelend \DIFaddbegin \DIFadd{13 }\DIFaddend presently active subduction zones. While Kriging methods yield some spurious results for segments with low observational densities, both Similarity and Kriging methods indicate comparable accuracy rates on average. Thus both methods are ostensibly suitable for subduction zone research if carefully applied. Further, differences between Similarity and Kriging predictions highlight anomalies and point towards effective sampling strategies for future surveys.

Chapter \ref{chpt4} takes a different approach by considering what \gls{pt} distributions of exhumed \gls{hp} metamorphic rocks imply about mechanical variability among subduction zones with respect to detachment (recovery) of subducted materials. A large (122,236) dataset of recovered markers shows marker \gls{pt} distributions and recovery rates correlate with certain thermo-kinematic boundary conditions (oceanic plate age, convergence velocity, and upper-plate thickness) and indicate a range of plausible conditions for reproducing the rock record. A sizeable gap in marker recovery around 2 GPa and 550 \(^\circ\)C, coinciding with the highest density of natural samples, implies biases (including imperfect implementation of recovery mechanisms in numerical experiments, sampling biases, and reaction overstepping) may be strongly affecting the rock record and/or numerical geodynamic models.

Future work may focus on refining and improving numerical geodynamic codes (to better implement heat generation/transfer and recovery mechanisms), surface heat flow datasets (to improve interpolation accuracies), and petrologic datasets (to enable sampling and modelling of more diverse rock types) by exploring and addressing the potential biases identified in the above studies.

\cleardoublepage

\hypertarget{references}{%
\chapter*{References}\label{references}}
\addcontentsline{toc}{chapter}{References}

\markboth{References}{References}

\hypertarget{refs_main}{}
\begin{CSLReferences}{1}{1}
\leavevmode\vadjust pre{\hypertarget{ref-abers2005}{}}%
Abers, G. (2005). Seismic low-velocity layer at the top of subducting slabs: Observations, predictions, and systematics. \emph{Physics of the Earth and Planetary Interiors}, \emph{149}(1-2), 7--29.

\leavevmode\vadjust pre{\hypertarget{ref-abers2006}{}}%
Abers, G., Keken, P. van, Kneller, E., Ferris, A., \& Stachnik, J. (2006). The thermal structure of subduction zones constrained by seismic imaging: Implications for slab dehydration and wedge flow. \emph{Earth and Planetary Science Letters}, \emph{241}(3-4), 387--397.

\leavevmode\vadjust pre{\hypertarget{ref-abers2017}{}}%
Abers, G., van Keken, P., \& Hacker, B. (2017). The cold and relatively dry nature of mantle forearcs in subduction zones. \emph{Nature Geoscience}, \emph{10}(5), 333--337.

\leavevmode\vadjust pre{\hypertarget{ref-abers2020}{}}%
Abers, G., Keken, P. van, \& Wilson, C. (2020). Deep decoupling in subduction zones: Observations and temperature limits. \emph{Geosphere}, \emph{16}(6), 1408--1424.

\leavevmode\vadjust pre{\hypertarget{ref-afonso2005}{}}%
Afonso, J., Ranalli, G., \& Fernàndez, M. (2005). Thermal expansivity and elastic properties of the lithospheric mantle: Results from mineral physics of composites. \emph{Physics of the Earth and Planetary Interiors}, \emph{149}(3-4), 279--306.

\leavevmode\vadjust pre{\hypertarget{ref-agard2021}{}}%
Agard, P. (2021). Subduction of oceanic lithosphere in the alps: Selective and archetypal from (slow-spreading) oceans. \emph{Earth-Science Reviews}, \emph{214}, 103517.

\leavevmode\vadjust pre{\hypertarget{ref-agard2009}{}}%
Agard, P., Yamato, P., Jolivet, L., \& Burov, E. (2009). Exhumation of oceanic blueschists and eclogites in subduction zones: Timing and mechanisms. \emph{Earth-Science Reviews}, \emph{92}(1-2), 53--79.

\leavevmode\vadjust pre{\hypertarget{ref-agard2016}{}}%
Agard, P., Yamato, P., Soret, M., Prigent, C., Guillot, S., Plunder, A., et al. (2016). Plate interface rheological switches during subduction infancy: Control on slab penetration and metamorphic sole formation. \emph{Earth and Planetary Science Letters}, \emph{451}, 208--220.

\leavevmode\vadjust pre{\hypertarget{ref-agard2018}{}}%
Agard, P., Plunder, A., Angiboust, S., Bonnet, G., \& Ruh, J. (2018). The subduction plate interface: Rock record and mechanical coupling (from long to short time scales). \emph{Lithos}, \emph{320-321}, 537--566.

\leavevmode\vadjust pre{\hypertarget{ref-agard2020}{}}%
Agard, P., Prigent, C., Soret, M., Dubacq, B., Guillot, S., \& Deldicque, D. (2020). Slabitization: Mechanisms controlling subduction development and viscous coupling. \emph{Earth-Science Reviews}, \emph{208}, 103259.

\leavevmode\vadjust pre{\hypertarget{ref-agrusta2013}{}}%
Agrusta, R., Arcay, D., Tommasi, A., Davaille, A., Ribe, N., \& Gerya, T. (2013). Small-scale convection in a plume-fed low-viscosity layer beneath a moving plate. \emph{Geophysical Journal International}, \emph{194}(2), 591--610.

\leavevmode\vadjust pre{\hypertarget{ref-angiboust2009}{}}%
Angiboust, S., Agard, P., Jolivet, L., \& Beyssac, O. (2009). The zermatt-saas ophiolite: The largest (60-km wide) and deepest (c. 70--80 km) continuous slice of oceanic lithosphere detached from a subduction zone? \emph{Terra Nova}, \emph{21}(3), 171--180.

\leavevmode\vadjust pre{\hypertarget{ref-angiboust2011}{}}%
Angiboust, S., Agard, P., Raimbourg, H., Yamato, P., \& Huet, B. (2011). Subduction interface processes recorded by eclogite-facies shear zones (monviso, w. alps). \emph{Lithos}, \emph{127}(1-2), 222--238.

\leavevmode\vadjust pre{\hypertarget{ref-angiboust2012a}{}}%
Angiboust, S., Langdon, R., Agard, P., Waters, D., \& Chopin, C. (2012a). Eclogitization of the monviso ophiolite (w. Alps) and implications on subduction dynamics. \emph{Journal of Metamorphic Geology}, \emph{30}(1), 37--61.

\leavevmode\vadjust pre{\hypertarget{ref-angiboust2012b}{}}%
Angiboust, S., Wolf, S., Burov, E., Agard, P., \& Yamato, P. (2012b). Effect of fluid circulation on subduction interface tectonic processes: Insights from thermo-mechanical numerical modelling. \emph{Earth and Planetary Science Letters}, \emph{357}, 238--248.

\leavevmode\vadjust pre{\hypertarget{ref-angiboust2014a}{}}%
Angiboust, S., Pettke, T., De Hoog, J., Caron, B., \& Oncken, O. (2014a). Channelized fluid flow and eclogite-facies metasomatism along the subduction shear zone. \emph{Journal of Petrology}, \emph{55}(5), 883--916.

\leavevmode\vadjust pre{\hypertarget{ref-angiboust2014b}{}}%
Angiboust, S., Glodny, J., Oncken, O., \& Chopin, C. (2014b). In search of transient subduction interfaces in the dent blanche--sesia tectonic system (w. alps). \emph{Lithos}, \emph{205}, 298--321.

\leavevmode\vadjust pre{\hypertarget{ref-angiboust2015}{}}%
Angiboust, S., Kirsch, J., Oncken, O., Glodny, J., Monié, P., \& Rybacki, E. (2015). Probing the transition between seismically coupled and decoupled segments along an ancient subduction interface. \emph{Geochemistry, Geophysics, Geosystems}, \emph{16}(6), 1905--1922.

\leavevmode\vadjust pre{\hypertarget{ref-angiboust2016}{}}%
Angiboust, S., Agard, P., Glodny, J., Omrani, J., \& Oncken, O. (2016). Zagros blueschists: Episodic underplating and long-lived cooling of a subduction zone. \emph{Earth and Planetary Science Letters}, \emph{443}, 48--58.

\leavevmode\vadjust pre{\hypertarget{ref-arcay2017}{}}%
Arcay, D. (2017). Modelling the interplate domain in thermo-mechanical simulations of subduction: Critical effects of resolution and rheology, and consequences on wet mantle melting. \emph{Physics of the Earth and Planetary Interiors}, \emph{269}, 112--132.

\leavevmode\vadjust pre{\hypertarget{ref-arcay2006}{}}%
Arcay, D., Doin, M., Tric, E., Bousquet, R., \& de Capitani, C. (2006). Overriding plate thinning in subduction zones: Localized convection induced by slab dehydration. \emph{Geochemistry, Geophysics, Geosystems}, \emph{7}(2).

\leavevmode\vadjust pre{\hypertarget{ref-arcay2007}{}}%
Arcay, D., Tric, E., \& Doin, M. (2007). Slab surface temperature in subduction zones: Influence of the interplate decoupling depth and upper plate thinning processes. \emph{Earth and Planetary Science Letters}, \emph{255}(3-4), 324--338.

\leavevmode\vadjust pre{\hypertarget{ref-archinal2018}{}}%
Archinal, B., Acton, C., A'hearn, M., Conrad, A., Consolmagno, G., Duxbury, T., et al. (2018). Report of the IAU working group on cartographic coordinates and rotational elements: 2015. \emph{Celestial Mechanics and Dynamical Astronomy}, \emph{130}(3), 1--46.

\leavevmode\vadjust pre{\hypertarget{ref-audet2016}{}}%
Audet, P., \& Kim, Y. (2016). Teleseismic constraints on the geological environment of deep episodic slow earthquakes in subduction zone forearcs: A review. \emph{Tectonophysics}, \emph{670}, 1--15.

\leavevmode\vadjust pre{\hypertarget{ref-audet2018}{}}%
Audet, P., \& Schaeffer, A. (2018). Fluid pressure and shear zone development over the locked to slow slip region in cascadia. \emph{Science Advances}, \emph{4}(3), eaar2982.

\leavevmode\vadjust pre{\hypertarget{ref-avigad1991}{}}%
Avigad, D., \& Garfunkel, Z. (1991). Uplift and exhumation of high-pressure metamorphic terrains: The example of the cycladic blueschist belt (aegean sea). \emph{Tectonophysics}, \emph{188}(3-4), 357--372.

\leavevmode\vadjust pre{\hypertarget{ref-banfield1993}{}}%
Banfield, J., \& Raftery, A. (1993). Model-based gaussian and non-gaussian clustering. \emph{Biometrics}, 803--821.

\leavevmode\vadjust pre{\hypertarget{ref-bardossy1997}{}}%
Bárdossy, A. (1997). Introduction to geostatistics. \emph{Institute of Hydraulic Engineering, University of Stuttgart}.

\leavevmode\vadjust pre{\hypertarget{ref-barlow1989}{}}%
Barlow, H. (1989). Unsupervised learning. \emph{Neural Computation}, \emph{1}(3), 295--311.

\leavevmode\vadjust pre{\hypertarget{ref-batchelor1953}{}}%
Batchelor, G. (1953). \emph{The theory of homogeneous turbulence}. Cambridge university press.

\leavevmode\vadjust pre{\hypertarget{ref-beall2019}{}}%
Beall, A., Fagereng, Å., \& Ellis, S. (2019). Strength of strained two-phase mixtures: Application to rapid creep and stress amplification in subduction zone m{é}lange. \emph{Geophysical Research Letters}, \emph{46}(1), 169--178.

\leavevmode\vadjust pre{\hypertarget{ref-bebout2007}{}}%
Bebout, G. (2007). Metamorphic chemical geodynamics of subduction zones. \emph{Earth and Planetary Science Letters}, \emph{260}(3-4), 373--393.

\leavevmode\vadjust pre{\hypertarget{ref-bebout2002}{}}%
Bebout, G., \& Barton, M. (2002). Tectonic and metasomatic mixing in a high-t, subduction-zone m{é}lange---insights into the geochemical evolution of the slab--mantle interface. \emph{Chemical Geology}, \emph{187}(1-2), 79--106.

\leavevmode\vadjust pre{\hypertarget{ref-behr2018}{}}%
Behr, W., Kotowski, A., \& Ashley, K. (2018). Dehydration-induced rheological heterogeneity and the deep tremor source in warm subduction zones. \emph{Geology}, \emph{46}(5), 475--478.

\leavevmode\vadjust pre{\hypertarget{ref-beyssac2002}{}}%
Beyssac, O., Goffé, B., Chopin, C., \& Rouzaud, J. (2002). Raman spectra of carbonaceous material in metasediments: A new geothermometer. \emph{Journal of Metamorphic Geology}, \emph{20}(9), 859--871.

\leavevmode\vadjust pre{\hypertarget{ref-bird2003}{}}%
Bird, P. (2003). An updated digital model of plate boundaries. \emph{Geochemistry, Geophysics, Geosystems}, \emph{4}(3).

\leavevmode\vadjust pre{\hypertarget{ref-bittner1995}{}}%
Bittner, D., \& Schmeling, H. (1995). Numerical modelling of melting processes and induced diapirism in the lower crust. \emph{Geophysical Journal International}, \emph{123}(1), 59--70.

\leavevmode\vadjust pre{\hypertarget{ref-bostock2013}{}}%
Bostock, M. (2013). The moho in subduction zones. \emph{Tectonophysics}, \emph{609}, 547--557.

\leavevmode\vadjust pre{\hypertarget{ref-boussinesq1897}{}}%
Boussinesq, J. (1897). \emph{Th{é}orie de l'{é}coulement tourbillonnant et tumultueux des liquides dans les lits rectilignes a grande section} (Vol. 1). Gauthier-Villars.

\leavevmode\vadjust pre{\hypertarget{ref-burg2005}{}}%
Burg, J., \& Gerya, T. (2005). The role of viscous heating in barrovian metamorphism of collisional orogens: Thermomechanical models and application to the lepontine dome in the central alps. \emph{Journal of Metamorphic Geology}, \emph{23}(2), 75--95.

\leavevmode\vadjust pre{\hypertarget{ref-burov2001}{}}%
Burov, E., Jolivet, L., Le Pourhiet, L., \& Poliakov, A. (2001). A thermomechanical model of exhumation of high pressure (HP) and ultra-high pressure (UHP) metamorphic rocks in alpine-type collision belts. \emph{Tectonophysics}, \emph{342}(1-2), 113--136.

\leavevmode\vadjust pre{\hypertarget{ref-burov2014}{}}%
Burov, E., François, T., Agard, P., Le Pourhiet, L., Meyer, B., Tirel, C., et al. (2014). Rheological and geodynamic controls on the mechanisms of subduction and HP/UHP exhumation of crustal rocks during continental collision: Insights from numerical models. \emph{Tectonophysics}, \emph{631}, 212--250.

\leavevmode\vadjust pre{\hypertarget{ref-byerlee1978}{}}%
Byerlee, J. (1978). Friction of rocks. In \emph{Rock friction and earthquake prediction} (Vol. 116, pp. 615--626). Birkh{ä}user Basel.

\leavevmode\vadjust pre{\hypertarget{ref-byrne2020}{}}%
Byrne, K. (2020). Borah: Dell HPC intel (high performance computing cluster).

\leavevmode\vadjust pre{\hypertarget{ref-calvert2011}{}}%
Calvert, A., Preston, L., \& Farahbod, A. (2011). Sedimentary underplating at the cascadia mantle-wedge corner revealed by seismic imaging. \emph{Nature Geoscience}, \emph{4}(8), 545--548.

\leavevmode\vadjust pre{\hypertarget{ref-calvert2020}{}}%
Calvert, A., Bostock, M., Savard, G., \& Unsworth, M. (2020). Cascadia low frequency earthquakes at the base of an overpressured subduction shear zone. \emph{Nature Communications}, \emph{11}(1), 1--10.

\leavevmode\vadjust pre{\hypertarget{ref-carlson2003}{}}%
Carlson, R., \& Miller, D. (2003). Mantle wedge water contents estimated from seismic velocities in partially serpentinized peridotites. \emph{Geophysical Research Letters}, \emph{30}(5).

\leavevmode\vadjust pre{\hypertarget{ref-castro2017}{}}%
Castro, A., \& Spear, F. (2017). Reaction overstepping and re-evaluation of peak p--t conditions of the blueschist unit sifnos, greece: Implications for the cyclades subduction zone. \emph{International Geology Review}, \emph{59}(5-6), 548--562.

\leavevmode\vadjust pre{\hypertarget{ref-celeux1995}{}}%
Celeux, G., \& Govaert, G. (1995). Gaussian parsimonious clustering models. \emph{Pattern Recognition}, \emph{28}(5), 781--793.

\leavevmode\vadjust pre{\hypertarget{ref-chapman1975}{}}%
Chapman, D., \& Pollack, H. (1975). Global heat flow: A new look. \emph{Earth and Planetary Science Letters}, \emph{28}(1), 23--32.

\leavevmode\vadjust pre{\hypertarget{ref-chiles2009}{}}%
Chiles, J., \& Delfiner, P. (2009). \emph{Geostatistics: Modeling spatial uncertainty} (Vol. 497). John Wiley \& Sons.

\leavevmode\vadjust pre{\hypertarget{ref-cizkova2013}{}}%
Čížková, H., \& Bina, C. (2013). Effects of mantle and subduction-interface rheologies on slab stagnation and trench rollback. \emph{Earth and Planetary Science Letters}, \emph{379}, 95--103.

\leavevmode\vadjust pre{\hypertarget{ref-clauser1995}{}}%
Clauser, C., \& Huenges, E. (1995). Thermal conductivity of rocks and minerals. \emph{Rock Physics and Phase Relations: A Handbook of Physical Constants}, \emph{3}, 105--126.

\leavevmode\vadjust pre{\hypertarget{ref-cleveland1988}{}}%
Cleveland, W., \& Devlin, S. (1988). Locally weighted regression: An approach to regression analysis by local fitting. \emph{Journal of the American Statistical Association}, \emph{83}(403), 596--610.

\leavevmode\vadjust pre{\hypertarget{ref-cloos1988}{}}%
Cloos, M., \& Shreve, R. (1988). Subduction-channel model of prism accretion, melange formation, sediment subduction, and subduction erosion at convergent plate margins: 1. Background and description. \emph{Pure and Applied Geophysics}, \emph{128}(3), 455--500.

\leavevmode\vadjust pre{\hypertarget{ref-coffin1997}{}}%
Coffin, M., Gahagan, L., \& Lawver, L. (1997). \emph{Present-day plate boundary digital data compilation}. Institute for Geophysics.

\leavevmode\vadjust pre{\hypertarget{ref-connolly2005}{}}%
Connolly, J. (2005). Computation of phase equilibria by linear programming: A tool for geodynamic modeling and its application to subduction zone decarbonation. \emph{Earth and Planetary Science Letters}, \emph{236}(1-2), 524--541.

\leavevmode\vadjust pre{\hypertarget{ref-cressie2015}{}}%
Cressie, N. (2015). \emph{Statistics for spatial data}. John Wiley \& Sons.

\leavevmode\vadjust pre{\hypertarget{ref-currie2006}{}}%
Currie, C., \& Hyndman, R. (2006). The thermal structure of subduction zone back arcs. \emph{Journal of Geophysical Research: Solid Earth}, \emph{111}(B8), 1--22.

\leavevmode\vadjust pre{\hypertarget{ref-currie2004}{}}%
Currie, C., Wang, K., Hyndman, R., \& He, J. (2004). The thermal effects of steady-state slab-driven mantle flow above a subducting plate: The cascadia subduction zone and backarc. \emph{Earth and Planetary Science Letters}, \emph{223}(1-2), 35--48.

\leavevmode\vadjust pre{\hypertarget{ref-davies1999b}{}}%
Davies, J. (1999a). Simple analytic model for subduction zone thermal structure. \emph{Geophysical Journal International}, \emph{139}(3), 823--828.

\leavevmode\vadjust pre{\hypertarget{ref-davies1999a}{}}%
Davies, J. (1999b). The role of hydraulic fractures and intermediate depth earthquakes in generating suduction zone magmatism. \emph{Nature}, \emph{417}(March), 142--145.

\leavevmode\vadjust pre{\hypertarget{ref-davies2013}{}}%
Davies, J. (2013). Global map of solid earth surface heat flow. \emph{Geochemistry, Geophysics, Geosystems}, \emph{14}(10), 4608--4622.

\leavevmode\vadjust pre{\hypertarget{ref-davis1990}{}}%
Davis, E., Hyndman, R., \& Villinger, H. (1990). Rates of fluid expulsion across the northern cascadia accretionary prism: Constraints from new heat row and multichannel seismic reflection data. \emph{Journal of Geophysical Research: Solid Earth}, \emph{95}(B6), 8869--8889.

\leavevmode\vadjust pre{\hypertarget{ref-delph2021}{}}%
Delph, J., Thomas, A., \& Levander, A. (2021). Subcretionary tectonics: Linking variability in the expression of subduction along the cascadia forearc. \emph{Earth and Planetary Science Letters}, \emph{556}, 116724.

\leavevmode\vadjust pre{\hypertarget{ref-dempster1977}{}}%
Dempster, A., Laird, N., \& Rubin, D. (1977). Maximum likelihood from incomplete data via the EM algorithm. \emph{Journal of the Royal Statistical Society: Series B (Methodological)}, \emph{39}(1), 1--22.

\leavevmode\vadjust pre{\hypertarget{ref-dragovic2012}{}}%
Dragovic, B., Samanta, L., Baxter, E., \& Selverstone, J. (2012). Using garnet to constrain the duration and rate of water-releasing metamorphic reactions during subduction: An example from sifnos, greece. \emph{Chemical Geology}, \emph{314}, 9--22.

\leavevmode\vadjust pre{\hypertarget{ref-dragovic2015}{}}%
Dragovic, B., Baxter, E., \& Caddick, M. (2015). Pulsed dehydration and garnet growth during subduction revealed by zoned garnet geochronology and thermodynamic modeling, sifnos, greece. \emph{Earth and Planetary Science Letters}, \emph{413}, 111--122.

\leavevmode\vadjust pre{\hypertarget{ref-dragovic2020}{}}%
Dragovic, B., Angiboust, S., \& Tappa, M. (2020). Petrochronological close-up on the thermal structure of a paleo-subduction zone (w. alps). \emph{Earth and Planetary Science Letters}, \emph{547}, 116446.

\leavevmode\vadjust pre{\hypertarget{ref-dy2004}{}}%
Dy, J., \& Brodley, C. (2004). Feature selection for unsupervised learning. \emph{Journal of Machine Learning Research}, \emph{5}(Aug), 845--889.

\leavevmode\vadjust pre{\hypertarget{ref-efron1992}{}}%
Efron, B. (1992). Bootstrap methods: Another look at the jackknife. In \emph{Breakthroughs in statistics} (pp. 569--593). Springer.

\leavevmode\vadjust pre{\hypertarget{ref-efron1994}{}}%
Efron, B., \& Tibshirani, R. (1994). \emph{An introduction to the bootstrap}. CRC press.

\leavevmode\vadjust pre{\hypertarget{ref-engdahl1998}{}}%
Engdahl, E., van der Hilst, R., \& Buland, R. (1998). Global teleseismic earthquake relocation with improved travel times and procedures for depth determination. \emph{Bulletin of the Seismological Society of America}, \emph{88}(3), 722--743.

\leavevmode\vadjust pre{\hypertarget{ref-england2010}{}}%
England, P., \& Katz, R. (2010). Melting above the anhydrous solidus controls the location of volcanic arcs. \emph{Nature}, \emph{467}(7316), 700--703.

\leavevmode\vadjust pre{\hypertarget{ref-england2004}{}}%
England, P., Engdahl, R., \& Thatcher, W. (2004). Systematic variation in the depths of slabs beneath arc volcanoes. \emph{Geophysical Journal International}, \emph{156}(2), 377--408.

\leavevmode\vadjust pre{\hypertarget{ref-faccenda2008}{}}%
Faccenda, M., Gerya, T., \& Chakraborty, S. (2008). Styles of post-subduction collisional orogeny: Influence of convergence velocity, crustal rheology and radiogenic heat production. \emph{Lithos}, \emph{103}(1-2), 257--287.

\leavevmode\vadjust pre{\hypertarget{ref-faccenda2009}{}}%
Faccenda, M., Gerya, T., \& Burlini, L. (2009). Deep slab hydration induced by bending-related variations in tectonic pressure. \emph{Nature Geoscience}, \emph{2}(11), 790--793.

\leavevmode\vadjust pre{\hypertarget{ref-ferris2003}{}}%
Ferris, A., Abers, G., Christensen, D., \& Veenstra, E. (2003). High resolution image of the subducted pacific (?) Plate beneath central alaska, 50--150 km depth. \emph{Earth and Planetary Science Letters}, \emph{214}(3-4), 575--588.

\leavevmode\vadjust pre{\hypertarget{ref-ferry2007}{}}%
Ferry, J., \& Watson, E. (2007). New thermodynamic models and revised calibrations for the ti-in-zircon and zr-in-rutile thermometers. \emph{Contributions to Mineralogy and Petrology}, \emph{154}(4), 429--437.

\leavevmode\vadjust pre{\hypertarget{ref-figueiredo2002}{}}%
Figueiredo, M., \& Jain, A. (2002). Unsupervised learning of finite mixture models. \emph{IEEE Transactions on Pattern Analysis and Machine Intelligence}, \emph{24}(3), 381--396.

\leavevmode\vadjust pre{\hypertarget{ref-fisher2000}{}}%
Fisher, A., \& Becker, K. (2000). Channelized fluid flow in oceanic crust reconciles heat-flow and permeability data. \emph{Nature}, \emph{403}(6765), 71--74.

\leavevmode\vadjust pre{\hypertarget{ref-fisher2021}{}}%
Fisher, D., Hooker, J., Smye, A., \& Chen, T. (2021). Insights from the geological record of deformation along the subduction interface at depths of seismogenesis. \emph{Geosphere}, \emph{17}(6), 1686--1703.

\leavevmode\vadjust pre{\hypertarget{ref-fourier1827}{}}%
Fourier, J. (1827). Mémoire sur les températures du globe terrestre et des espaces planétaires. \emph{Mémoires de l'Académie Royale Des Sciences de l'Institut de France}, \emph{7}, 570--604.

\leavevmode\vadjust pre{\hypertarget{ref-fraley2002}{}}%
Fraley, C., \& Raftery, A. (2002). Model-based clustering, discriminant analysis, and density estimation. \emph{Journal of the American Statistical Association}, \emph{97}(458), 611--631.

\leavevmode\vadjust pre{\hypertarget{ref-furlong2013}{}}%
Furlong, K., \& Chapman, D. (2013). Heat flow, heat generation, and the thermal state of the lithosphere. \emph{Annual Review of Earth and Planetary Sciences}, \emph{41}(1), 385--410.

\leavevmode\vadjust pre{\hypertarget{ref-furukawa1993}{}}%
Furukawa, Y. (1993). Depth of the decoupling plate interface and thermal structure under arcs. \emph{Journal of Geophysical Research: Solid Earth}, \emph{98}(B11), 20005--20013.

\leavevmode\vadjust pre{\hypertarget{ref-gao2014}{}}%
Gao, X., \& Wang, K. (2014). Strength of stick-slip and creeping subduction megathrusts from heat flow observations. \emph{Science}, \emph{345}(6200), 1038--1041.

\leavevmode\vadjust pre{\hypertarget{ref-gao2017}{}}%
Gao, X., \& Wang, K. (2017). Rheological separation of the megathrust seismogenic zone and episodic tremor and slip. \emph{Nature}, \emph{543}(7645), 416--419.

\leavevmode\vadjust pre{\hypertarget{ref-gerya2014}{}}%
Gerya, T. (2014). Precambrian geodynamics: Concepts and models. \emph{Gondwana Research}, \emph{25}(2), 442--463.

\leavevmode\vadjust pre{\hypertarget{ref-gerya2019}{}}%
Gerya, T. (2019). \emph{Introduction to numerical geodynamic modelling}. Cambridge University Press.

\leavevmode\vadjust pre{\hypertarget{ref-gerya2011}{}}%
Gerya, T., \& Meilick, F. (2011). Geodynamic regimes of subduction under an active margin: Effects of rheological weakening by fluids and melts. \emph{Journal of Metamorphic Geology}, \emph{29}(1), 7--31.

\leavevmode\vadjust pre{\hypertarget{ref-gerya2006}{}}%
Gerya, T., \& Stöckhert, B. (2006). Two-dimensional numerical modeling of tectonic and metamorphic histories at active continental margins. \emph{International Journal of Earth Sciences}, \emph{95}(2), 250--274.

\leavevmode\vadjust pre{\hypertarget{ref-gerya2003}{}}%
Gerya, T., \& Yuen, D. (2003). Characteristics-based marker-in-cell method with conservative finite-differences schemes for modeling geological flows with strongly variable transport properties. \emph{Physics of the Earth and Planetary Interiors}, \emph{140}(4), 293--318.

\leavevmode\vadjust pre{\hypertarget{ref-gerya2002}{}}%
Gerya, T., Stöckhert, B., \& Perchuk, A. (2002). Exhumation of high-pressure metamorphic rocks in a subduction channel: A numerical simulation. \emph{Tectonics}, \emph{21}(6), 6--1.

\leavevmode\vadjust pre{\hypertarget{ref-gerya2008}{}}%
Gerya, T., Connolly, J., \& Yuen, D. (2008). Why is terrestrial subduction one-sided? \emph{Geology}, \emph{36}(1), 43--46.

\leavevmode\vadjust pre{\hypertarget{ref-goldberg1989}{}}%
Goldberg, D. (1989). Genetic algorithms in search. \emph{Optimization, and MachineLearning}.

\leavevmode\vadjust pre{\hypertarget{ref-gonzalez2016}{}}%
Gonzalez, C., Gorczyk, W., \& Gerya, T. (2016). Decarbonation of subducting slabs: Insight from petrological--thermomechanical modeling. \emph{Gondwana Research}, \emph{36}, 314--332.

\leavevmode\vadjust pre{\hypertarget{ref-goodchild2004}{}}%
Goodchild, M. (2004). The validity and usefulness of laws in geographic information science and geography. \emph{Annals of the Association of American Geographers}, \emph{94}(2), 300--303.

\leavevmode\vadjust pre{\hypertarget{ref-goovaerts1997}{}}%
Goovaerts, P. (1997). \emph{Geostatistics for natural resources evaluation}. Oxford University Press on Demand.

\leavevmode\vadjust pre{\hypertarget{ref-gorbatov1997}{}}%
Gorbatov, A., \& Kostoglodov, V. (1997). Maximum depth of seismicity and thermal parameter of the subducting slab: General empirical relation and its application. \emph{Tectonophysics}, \emph{277}(1-3), 165--187.

\leavevmode\vadjust pre{\hypertarget{ref-gorczyk2007}{}}%
Gorczyk, W., Willner, A., Gerya, T., Connolly, J., \& Burg, J. (2007). Physical controls of magmatic productivity at pacific-type convergent margins: Numerical modelling. \emph{Physics of the Earth and Planetary Interiors}, \emph{163}(1-4), 209--232.

\leavevmode\vadjust pre{\hypertarget{ref-goutorbe2011}{}}%
Goutorbe, B., Poort, J., Lucazeau, F., \& Raillard, S. (2011). Global heat flow trends resolved from multiple geological and geophysical proxies. \emph{Geophysical Journal International}, \emph{187}(3), 1405--1419.

\leavevmode\vadjust pre{\hypertarget{ref-graler2016}{}}%
Gräler, B., Pebesma, E., \& Heuvelink, G. (2016). Spatio-temporal interpolation using gstat. \emph{The R Journal}, \emph{8}, 204--218. Retrieved from \url{https://journal.r-project.org/archive/2016/RJ-2016-014/index.html}

\leavevmode\vadjust pre{\hypertarget{ref-groppo2016}{}}%
Groppo, C., Rolfo, F., Sachan, H., \& Rai, S. (2016). Petrology of blueschist from the western himalaya (ladakh, NW india): Exploring the complex behavior of a lawsonite-bearing system in a paleo-accretionary setting. \emph{Lithos}, \emph{252}, 41--56.

\leavevmode\vadjust pre{\hypertarget{ref-grove2009}{}}%
Grove, T., Till, C., Lev, E., Chatterjee, N., \& Médard, E. (2009). Kinematic variables and water transport control the formation and location of arc volcanoes. \emph{Nature}, \emph{459}(7247), 694--697.

\leavevmode\vadjust pre{\hypertarget{ref-grove2012}{}}%
Grove, T., Till, C., \& Krawczynski, M. (2012). The role of \(H_2O\) in subduction zone magmatism. \emph{Annual Review of Earth and Planetary Sciences}, \emph{40}(1), 413--439.

\leavevmode\vadjust pre{\hypertarget{ref-hacker1996}{}}%
Hacker, B. (1996). Eclogite formation and the rheology, buoyancy, seismicity, and h\textasciitilde{} 2O content of oceanic crust. \emph{GEOPHYSICAL MONOGRAPH-AMERICAN GEOPHYSICAL UNION}, \emph{96}, 337--346.

\leavevmode\vadjust pre{\hypertarget{ref-hacker2006}{}}%
Hacker, B. (2006). Pressures and temperatures of ultrahigh-pressure metamorphism: Implications for UHP tectonics and H2O in subducting slabs. \emph{International Geology Review}, \emph{48}(12), 1053--1066.

\leavevmode\vadjust pre{\hypertarget{ref-hacker2008}{}}%
Hacker, B. (2008). H2O subduction beyond arcs. \emph{Geochemistry, Geophysics, Geosystems}, \emph{9}(3).

\leavevmode\vadjust pre{\hypertarget{ref-hacker2003}{}}%
Hacker, B., Abers, G., \& Peacock, S. (2003). Subduction factory 1. Theoretical mineralogy, densities, seismic wave speeds, and \(H_2O\) contents. \emph{Journal of Geophysical Research: Solid Earth}, \emph{108}(B1).

\leavevmode\vadjust pre{\hypertarget{ref-harlow1962}{}}%
Harlow, F. (1962). \emph{The particle-in-cell method for numerical solution of problems in fluid dynamics}. Los Alamos Scientific Lab., N. Mex.

\leavevmode\vadjust pre{\hypertarget{ref-harlow1964}{}}%
Harlow, F. (1964). The particle-in-cell computing method for fluid dynamics. \emph{Methods Comput. Phys.}, \emph{3}, 319--343.

\leavevmode\vadjust pre{\hypertarget{ref-harlow1965}{}}%
Harlow, F., \& Welch, J. (1965). Numerical calculation of time-dependent viscous incompressible flow of fluid with free surface. \emph{The Physics of Fluids}, \emph{8}(12), 2182--2189.

\leavevmode\vadjust pre{\hypertarget{ref-hasterok2013}{}}%
Hasterok, D. (2013). A heat flow based cooling model for tectonic plates. \emph{Earth and Planetary Science Letters}, \emph{361}, 34--43.

\leavevmode\vadjust pre{\hypertarget{ref-hasterok2008}{}}%
Hasterok, D., \& Chapman, D. (2008). Global heat flow: A new database and a new approach. In \emph{AGU fall meeting abstracts} (Vol. 2008, pp. T21c--1985).

\leavevmode\vadjust pre{\hypertarget{ref-hasterok2011}{}}%
Hasterok, D., Chapman, D., \& Davis, E. (2011). Oceanic heat flow: Implications for global heat loss. \emph{Earth and Planetary Science Letters}, \emph{311}(3-4), 386--395.

\leavevmode\vadjust pre{\hypertarget{ref-hilairet2007}{}}%
Hilairet, N., Reynard, B., Wang, Y., Daniel, I., Merkel, S., Nishiyama, N., \& Petitgirard, S. (2007). High-pressure creep of serpentine, interseismic deformation, and initiation of subduction. \emph{Science}, \emph{318}(5858), 1910--1913.

\leavevmode\vadjust pre{\hypertarget{ref-hirauchi2010}{}}%
Hirauchi, K., Katayama, I., Uehara, S., Miyahara, M., \& Takai, Y. (2010). Inhibition of subduction thrust earthquakes by low-temperature plastic flow in serpentine. \emph{Earth and Planetary Science Letters}, \emph{295}(3-4), 349--357.

\leavevmode\vadjust pre{\hypertarget{ref-holt2021}{}}%
Holt, A., \& Condit, C. (2021). Slab temperature evolution over the lifetime of a subduction zone. \emph{Geochemistry, Geophysics, Geosystems}, e2020GC009476.

\leavevmode\vadjust pre{\hypertarget{ref-hu2020}{}}%
Hu, J., \& Gurnis, M. (2020). Subduction duration and slab dip. \emph{Geochemistry, Geophysics, Geosystems}, \emph{21}(4), e2019GC008862.

\leavevmode\vadjust pre{\hypertarget{ref-hutnak2008}{}}%
Hutnak, M., Fisher, A., Harris, R., Stein, C., Wang, K., Spinelli, G., et al. (2008). Large heat and fluid fluxes driven through mid-plate outcrops on ocean crust. \emph{Nature Geoscience}, \emph{1}(9), 611--614.

\leavevmode\vadjust pre{\hypertarget{ref-hyndman2003}{}}%
Hyndman, R., \& Peacock, S. (2003). Serpentinization of the forearc mantle. \emph{Earth and Planetary Science Letters}, \emph{212}(3-4), 417--432.

\leavevmode\vadjust pre{\hypertarget{ref-hyndman1993}{}}%
Hyndman, R., \& Wang, K. (1993). Thermal constraints on the zone of major thrust earthquake failure: The cascadia subduction zone. \emph{Journal of Geophysical Research: Solid Earth}, \emph{98}(B2), 2039--2060.

\leavevmode\vadjust pre{\hypertarget{ref-hyndman2005}{}}%
Hyndman, R., Currie, C., \& Mazzotti, S. (2005). Subduction zone backarcs, mobile belts, and orogenic heat. \emph{GSA Today}, \emph{15}(2), 4--10.

\leavevmode\vadjust pre{\hypertarget{ref-ioannidi2020}{}}%
Ioannidi, P., Angiboust, S., Oncken, O., Agard, P., Glodny, J., \& Sudo, M. (2020). Deformation along the roof of a fossil subduction interface in the transition zone below seismogenic coupling: The austroalpine case and new insights from the malenco massif (central alps). \emph{Geosphere}, \emph{16}(2), 510--532.

\leavevmode\vadjust pre{\hypertarget{ref-ioannidi2021}{}}%
Ioannidi, P., Le Pourhiet, L., Agard, P., Angiboust, S., \& Oncken, O. (2021). Effective rheology of a two-phase subduction shear zone: Insights from numerical simple shear experiments and implications for subduction zone interfaces. \emph{Earth and Planetary Science Letters}, \emph{566}, 116913.

\leavevmode\vadjust pre{\hypertarget{ref-ito1971}{}}%
Ito, K., \& Kennedy, G. (1971). An experimental study of the basalt-garnet granulite-eclogite transition. \emph{The Structure and Physical Properties of the Earth's Crust}, \emph{14}, 303--314.

\leavevmode\vadjust pre{\hypertarget{ref-jennings2021}{}}%
Jennings, S., \& Hasterok, D. (2021). HeatFlow.org. \emph{Heatflow.org}. Retrieved from \url{http://heatflow.org/}

\leavevmode\vadjust pre{\hypertarget{ref-jull2001}{}}%
Jull, M., \& Kelemen, P. (2001). On the conditions for lower crustal convective instability. \emph{Journal of Geophysical Research: Solid Earth}, \emph{106}(b4), 6423--6446.

\leavevmode\vadjust pre{\hypertarget{ref-karato1993}{}}%
Karato, S., \& Wu, P. (1993). Rheology of the upper mantle: A synthesis. \emph{Science}, \emph{260}(5109), 771--778.

\leavevmode\vadjust pre{\hypertarget{ref-kelvin1863}{}}%
Kelvin, W. (1863). On the secular cooling of the earth. \emph{Transactions of the Royal Society of Edinburgh}, \emph{23}, 157--170.

\leavevmode\vadjust pre{\hypertarget{ref-kerswell2021}{}}%
Kerswell, B., Kohn, M., \& Gerya, T. (2021). Backarc lithospheric thickness and serpentine stability control slab-mantle coupling depths in subduction zones. \emph{Geochemistry, Geophysics, Geosystems}, \emph{22}(6), e2020GC009304.

\leavevmode\vadjust pre{\hypertarget{ref-kirby1991}{}}%
Kirby, S., Durham, W., \& Stern, L. (1991). Mantle phase changes and deep-earthquake faulting in subducting lithosphere. \emph{Science}, \emph{252}(5003), 216--225.

\leavevmode\vadjust pre{\hypertarget{ref-kitamura2012}{}}%
Kitamura, Y., \& Kimura, G. (2012). Dynamic role of tectonic m{é}lange during interseismic process of plate boundary mega earthquakes. \emph{Tectonophysics}, \emph{568}, 39--52.

\leavevmode\vadjust pre{\hypertarget{ref-kohavi1995}{}}%
Kohavi, R. (1995). A study of cross-validation and bootstrap for accuracy estimation and model selection. In \emph{Ijcai} (Vol. 14, pp. 1137--1145). Montreal, Canada.

\leavevmode\vadjust pre{\hypertarget{ref-kohn2020}{}}%
Kohn, M. (2020). A refined zirconium-in-rutile thermometer. \emph{American Mineralogist: Journal of Earth and Planetary Materials}, \emph{105}(6), 963--971.

\leavevmode\vadjust pre{\hypertarget{ref-kohn2017}{}}%
Kohn, M., Engi, M., \& Lanari, P. (2017). Petrochronology. \emph{Methods and Applications, Mineralogical Society of America Reviews in Mineralogy and Geochemistry}, \emph{83}, 575.

\leavevmode\vadjust pre{\hypertarget{ref-kohn2018}{}}%
Kohn, M., Castro, A., Kerswell, B., Ranero, C. R., \& Spear, F. (2018). Shear heating reconciles thermal models with the metamorphic rock record of subduction. \emph{Proceedings of the National Academy of Sciences}, \emph{115}(46), 11706--11711.

\leavevmode\vadjust pre{\hypertarget{ref-korgen1971}{}}%
Korgen, B., Bodvarsson, G., \& Mesecar, R. (1971). Heat flow through the floor of cascadia basin. \emph{Journal of Geophysical Research}, \emph{76}(20), 4758--4774.

\leavevmode\vadjust pre{\hypertarget{ref-kotowski2019}{}}%
Kotowski, A., \& Behr, W. (2019). Length scales and types of heterogeneities along the deep subduction interface: Insights from exhumed rocks on syros island, greece. \emph{Geosphere}, \emph{15}(4), 1038--1065.

\leavevmode\vadjust pre{\hypertarget{ref-krige1951}{}}%
Krige, D. (1951). A statistical approach to some basic mine valuation problems on the witwatersrand. \emph{Journal of the Southern African Institute of Mining and Metallurgy}, \emph{52}(6), 119--139.

\leavevmode\vadjust pre{\hypertarget{ref-laurent2018}{}}%
Laurent, V., Lanari, P., Naïr, I., Augier, R., Lahfid, A., \& Jolivet, L. (2018). Exhumation of eclogite and blueschist (cyclades, greece): Pressure--temperature evolution determined by thermobarometry and garnet equilibrium modelling. \emph{Journal of Metamorphic Geology}, \emph{36}(6), 769--798.

\leavevmode\vadjust pre{\hypertarget{ref-lawver2018}{}}%
Lawver, L., Dalziel, I., Norton, I., Gahagan, L., \& Davis, J. (2018). The PLATES 2014 atlas of plate reconstructions (550 ma to present day), PLATES progress report no. 374-0215. \emph{University of Texas Institute for Geophysics Technical Reports}.

\leavevmode\vadjust pre{\hypertarget{ref-lee1965}{}}%
Lee, W., \& Uyeda, S. (1965). Review of heat flow data. \emph{Terrestrial Heat Flow}, \emph{8}, 87--190.

\leavevmode\vadjust pre{\hypertarget{ref-li2018}{}}%
Li, Z., Zhang, X., Clarke, K., Liu, G., \& Zhu, R. (2018). An automatic variogram modeling method with high reliability fitness and estimates. \emph{Computers \& Geosciences}, \emph{120}, 48--59.

\leavevmode\vadjust pre{\hypertarget{ref-locatelli2019}{}}%
Locatelli, M., Federico, L., Agard, P., \& Verlaguet, A. (2019). Geology of the southern monviso metaophiolite complex (w-alps, italy). \emph{Journal of Maps}, \emph{15}(2), 283--297.

\leavevmode\vadjust pre{\hypertarget{ref-lucazeau2019}{}}%
Lucazeau, F. (2019). Analysis and mapping of an updated terrestrial heat flow data set. \emph{Geochemistry, Geophysics, Geosystems}, \emph{20}(8), 4001--4024.

\leavevmode\vadjust pre{\hypertarget{ref-mann2022}{}}%
Mann, M., Abers, G., Daly, K., \& Christensen, D. (2022). Subduction of an oceanic plateau across southcentral alaska: Scattered-wave imaging. \emph{Journal of Geophysical Research: Solid Earth}, e2021JB022697.

\leavevmode\vadjust pre{\hypertarget{ref-matheron1963}{}}%
Matheron, G. (1963). Principles of geostatistics. \emph{Economic Geology}, \emph{58}(8), 1246--1266.

\leavevmode\vadjust pre{\hypertarget{ref-matheron2019}{}}%
Matheron, G. (2019). \emph{Matheron's theory of regionalized variables}. International Association for.

\leavevmode\vadjust pre{\hypertarget{ref-maunder2019}{}}%
Maunder, B., Hunen, J. van, Bouilhol, P., \& Magni, V. (2019). Modeling slab temperature: A reevaluation of the thermal parameter. \emph{Geochemistry, Geophysics, Geosystems}, \emph{20}(2), 673--687.

\leavevmode\vadjust pre{\hypertarget{ref-mckenzie1969}{}}%
McKenzie, D. (1969). Speculations on the consequences and causes of plate motions. \emph{Geophysical Journal International}, \emph{18}(1), 1--32.

\leavevmode\vadjust pre{\hypertarget{ref-minami2022}{}}%
Minami, H., Okada, C., Saito, K., \& Ohara, Y. (2022). Evidence of an active rift zone in the northern okinawa trough. \emph{Marine Geology}, \emph{443}, 106666.

\leavevmode\vadjust pre{\hypertarget{ref-molnar1990}{}}%
Molnar, P., \& England, P. (1990). Temperatures, heat flux, and frictional stress near major thrust faults. \emph{Journal of Geophysical Research: Solid Earth}, \emph{95}(B4), 4833--4856.

\leavevmode\vadjust pre{\hypertarget{ref-molnar1979}{}}%
Molnar, P., Freedman, D., \& Shih, J. (1979). Lengths of intermediate and deep seismic zones and temperatures in downgoing slabs of lithosphere. \emph{Geophysical Journal International}, \emph{56}(1), 41--54.

\leavevmode\vadjust pre{\hypertarget{ref-monie2009}{}}%
Monie, P., \& Agard, P. (2009). Coeval blueschist exhumation along thousands of kilometers: Implications for subduction channel processes. \emph{Geochemistry, Geophysics, Geosystems}, \emph{10}(7).

\leavevmode\vadjust pre{\hypertarget{ref-moresi2003}{}}%
Moresi, L., Dufour, F., \& Mühlhaus, H. (2003). A lagrangian integration point finite element method for large deformation modeling of viscoelastic geomaterials. \emph{Journal of Computational Physics}, \emph{184}(2), 476--497.

\leavevmode\vadjust pre{\hypertarget{ref-morishige2020}{}}%
Morishige, M., \& Kuwatani, T. (2020). Bayesian inversion of surface heat flow in subduction zones: A framework to refine geodynamic models based on observational constraints. \emph{Geophysical Journal International}, \emph{222}(1), 103--109.

\leavevmode\vadjust pre{\hypertarget{ref-naif2015}{}}%
Naif, S., Key, K., Constable, S., \& Evans, R. (2015). Water-rich bending faults at the m iddle a merica t rench. \emph{Geochemistry, Geophysics, Geosystems}, \emph{16}(8), 2582--2597.

\leavevmode\vadjust pre{\hypertarget{ref-nyblade1993}{}}%
Nyblade, A., \& Pollack, H. (1993). A global analysis of heat flow from precambrian terrains: Implications for the thermal structure of archean and proterozoic lithosphere. \emph{Journal of Geophysical Research: Solid Earth}, \emph{98}(B7), 12207--12218.

\leavevmode\vadjust pre{\hypertarget{ref-okay1989}{}}%
Okay, A. (1989). Alpine-himalayan blueschists. \emph{Annual Review of Earth and Planetary Sciences}, \emph{17}(1), 55--87.

\leavevmode\vadjust pre{\hypertarget{ref-parsons1977}{}}%
Parsons, B., \& Sclater, J. (1977). An analysis of the variation of ocean floor bathymetry and heat flow with age. \emph{Journal of Geophysical Research}, \emph{82}(5), 803--827.

\leavevmode\vadjust pre{\hypertarget{ref-patankar2018}{}}%
Patankar, S. (2018). \emph{Numerical heat transfer and fluid flow}. Taylor \& Francis.

\leavevmode\vadjust pre{\hypertarget{ref-peacock1990}{}}%
Peacock, S. (1990). Fluid processes in subduction zones. \emph{Science}, \emph{248}(4953), 329--337.

\leavevmode\vadjust pre{\hypertarget{ref-peacock1991}{}}%
Peacock, S. (1991). Numerical simulation of subduction zone pressure-temperature-time paths: Constraints on fluid production and arc magmatism. \emph{Philosophical Transactions of the Royal Society of London. Series A: Physical and Engineering Sciences}, \emph{335}(1638), 341--353.

\leavevmode\vadjust pre{\hypertarget{ref-peacock1993}{}}%
Peacock, S. (1993). The importance of blueschist \(\rightarrow\) eclogite dehydration reactions in subducting oceanic crust. \emph{Geological Society of America Bulletin}, \emph{105}(5), 684--694.

\leavevmode\vadjust pre{\hypertarget{ref-peacock1996}{}}%
Peacock, S. (1996). Thermal and petrologic structure of subduction zones. \emph{Subduction: Top to Bottom}, \emph{96}, 119--133.

\leavevmode\vadjust pre{\hypertarget{ref-peacock1999a}{}}%
Peacock, S., \& Hyndman, R. (1999). Hydrous minerals in the mantle wedge and the maximum depth of subduction thrust earthquakes. \emph{Geophysical Research Letters}, \emph{26}(No. 16), 2517--2520.

\leavevmode\vadjust pre{\hypertarget{ref-peacock1999b}{}}%
Peacock, S., \& Wang, K. (1999). Seismic consequences of warm versus cool subduction metamorphism: Examples from southwest and northeast japan. \emph{Science}, \emph{286}(5441), 937--939.

\leavevmode\vadjust pre{\hypertarget{ref-peacock1994}{}}%
Peacock, S., Rushmer, T., \& Thompson, A. (1994). Partial melting of subducting oceanic crust. \emph{Earth and Planetary Science Letters}, \emph{121}(1-2), 227--244.

\leavevmode\vadjust pre{\hypertarget{ref-pebesma2004}{}}%
Pebesma, E. (2004). Multivariable geostatistics in {S}: The gstat package. \emph{Computers \& Geosciences}, \emph{30}, 683--691.

\leavevmode\vadjust pre{\hypertarget{ref-pebesma2018}{}}%
Pebesma, E. (2018). Simple features for r: Standardized support for spatial vector data. \emph{The R Journal}, \emph{10}(1), 439--446. \url{https://doi.org/10.32614/rj-2018-009}

\leavevmode\vadjust pre{\hypertarget{ref-penniston2015}{}}%
Penniston-Dorland, S., Kohn, M., \& Manning, C. (2015). The global range of subduction zone thermal structures from exhumed blueschists and eclogites: Rocks are hotter than models. \emph{Earth and Planetary Science Letters}, \emph{428}, 243--254.

\leavevmode\vadjust pre{\hypertarget{ref-platt1986}{}}%
Platt, J. (1986). Dynamics of orogenic wedges and the uplift of high-pressure metamorphic rocks. \emph{Geological Society of America Bulletin}, \emph{97}(9), 1037--1053.

\leavevmode\vadjust pre{\hypertarget{ref-plumper2017}{}}%
Plümper, O., John, T., Podladchikov, Y., Vrijmoed, J., \& Scambelluri, M. (2017). Fluid escape from subduction zones controlled by channel-forming reactive porosity. \emph{Nature Geoscience}, \emph{10}(2), 150--156.

\leavevmode\vadjust pre{\hypertarget{ref-plunder2013}{}}%
Plunder, A., Agard, P., Chopin, C., \& Okay, A. (2013). Geodynamics of the tav{ş}anl{ı} zone, western turkey: Insights into subduction/obduction processes. \emph{Tectonophysics}, \emph{608}, 884--903.

\leavevmode\vadjust pre{\hypertarget{ref-plunder2015}{}}%
Plunder, A., Agard, P., Chopin, C., Pourteau, A., \& Okay, A. (2015). Accretion, underplating and exhumation along a subduction interface: From subduction initiation to continental subduction (tav{ş}anl{ı} zone, w. turkey). \emph{Lithos}, \emph{226}, 233--254.

\leavevmode\vadjust pre{\hypertarget{ref-plunder2018}{}}%
Plunder, A., Thieulot, C., \& Van Hinsbergen, D. (2018). The effect of obliquity on temperature in subduction zones: Insights from 3-d numerical modeling. \emph{Solid Earth}, \emph{9}(3), 759--776.

\leavevmode\vadjust pre{\hypertarget{ref-pollack1977}{}}%
Pollack, H., \& Chapman, D. (1977). On the regional variation of heat flow, geotherms, and lithospheric thickness. \emph{Tectonophysics}, \emph{38}(3-4), 279--296.

\leavevmode\vadjust pre{\hypertarget{ref-pollack1993}{}}%
Pollack, H., Hurter, S., \& Johnson, J. (1993). Heat flow from the earth's interior: Analysis of the global data set. \emph{Reviews of Geophysics}, \emph{31}(3), 267--280.

\leavevmode\vadjust pre{\hypertarget{ref-powell1994}{}}%
Powell, M. (1994). A direct search optimization method that models the objective and constraint functions by linear interpolation. In \emph{Advances in optimization and numerical analysis} (pp. 51--67). Springer.

\leavevmode\vadjust pre{\hypertarget{ref-proj2021}{}}%
PROJ contributors. (2021). \emph{{PROJ} coordinate transformation software library}. Open Source Geospatial Foundation. Retrieved from \url{https://proj.org/}

\leavevmode\vadjust pre{\hypertarget{ref-ranalli1995}{}}%
Ranalli, G. (1995). \emph{Rheology of the earth}. Springer Science \& Business Media.

\leavevmode\vadjust pre{\hypertarget{ref-reesjones2018}{}}%
Rees Jones, D., Katz, R., Tian, M., \& Rudge, J. (2018). Thermal impact of magmatism in subduction zones. \emph{Earth and Planetary Science Letters}, \emph{481}, 73--79.

\leavevmode\vadjust pre{\hypertarget{ref-reynard2013}{}}%
Reynard, B. (2013). Serpentine in active subduction zones. \emph{Lithos}, \emph{178}, 171--185.

\leavevmode\vadjust pre{\hypertarget{ref-reynolds2009}{}}%
Reynolds, D. (2009). Gaussian mixture models. \emph{Encyclopedia of Biometrics}, \emph{741}, 659--663.

\leavevmode\vadjust pre{\hypertarget{ref-roda2010}{}}%
Roda, M., Marotta, A., \& Spalla, M. (2010). Numerical simulations of an ocean-continent convergent system: Influence of subduction geometry and mantle wedge hydration on crustal recycling. \emph{Geochemistry, Geophysics, Geosystems}, \emph{11}(5).

\leavevmode\vadjust pre{\hypertarget{ref-roda2012}{}}%
Roda, M., Spalla, M., \& Marotta, A. (2012). Integration of natural data within a numerical model of ablative subduction: A possible interpretation for the alpine dynamics of the austroalpine crust. \emph{Journal of Metamorphic Geology}, \emph{30}(9), 973--996.

\leavevmode\vadjust pre{\hypertarget{ref-roda2020}{}}%
Roda, M., Zucali, M., Regorda, A., \& Spalla, M. (2020). Formation and evolution of a subduction-related m{é}lange: The example of the rocca canavese thrust sheets (western alps). \emph{Bulletin}, \emph{132}(3-4), 884--896.

\leavevmode\vadjust pre{\hypertarget{ref-rondenay2008}{}}%
Rondenay, S., Abers, G., \& van Keken, P. (2008). Seismic imaging of subduction zone metamorphism. \emph{Geology}, \emph{36}(4), 275--278.

\leavevmode\vadjust pre{\hypertarget{ref-rudnick1998}{}}%
Rudnick, R., McDonough, W., \& O'Connell, R. (1998). Thermal structure, thickness and composition of continental lithosphere. \emph{Chemical Geology}, \emph{145}(3-4), 395--411.

\leavevmode\vadjust pre{\hypertarget{ref-ruh2015}{}}%
Ruh, J., Le Pourhiet, L., Agard, P., Burov, E., \& Gerya, T. (2015). Tectonic slicing of subducting oceanic crust along plate interfaces: Numerical modeling. \emph{Geochemistry, Geophysics, Geosystems}, \emph{16}(10), 3505--3531.

\leavevmode\vadjust pre{\hypertarget{ref-schmidt1998}{}}%
Schmidt, M., \& Poli, S. (1998). Experimentally based water budgets for dehydrating slabs and consequences for arc magma generation. \emph{Earth and Planetary Science Letters}, \emph{163}(1-4), 361--379.

\leavevmode\vadjust pre{\hypertarget{ref-schwarz1978}{}}%
Schwarz, G. (1978). Estimating the dimension of a model. \emph{Annals of Statistics}, \emph{6}(2), 461--464.

\leavevmode\vadjust pre{\hypertarget{ref-sclater1970}{}}%
Sclater, J., \& Francheteau, J. (1970). The implications of terrestrial heat flow observations on current tectonic and geochemical models of the crust and upper mantle of the earth. \emph{Geophysical Journal International}, \emph{20}(5), 509--542.

\leavevmode\vadjust pre{\hypertarget{ref-scrucca2013}{}}%
Scrucca, L. (2013). {GA}: A package for genetic algorithms in {R}. \emph{Journal of Statistical Software}, \emph{53}(4), 1--37. Retrieved from \url{https://www.jstatsoft.org/v53/i04/}

\leavevmode\vadjust pre{\hypertarget{ref-scrucca2017}{}}%
Scrucca, L. (2017). On some extensions to {GA} package: Hybrid optimisation, parallelisation and islands evolution. \emph{The R Journal}, \emph{9}(1), 187--206. Retrieved from \url{https://journal.r-project.org/archive/2017/RJ-2017-008/}

\leavevmode\vadjust pre{\hypertarget{ref-scrucca2016}{}}%
Scrucca, L., Fop, M., Murphy, T., \& Raftery, A. (2016). Mclust 5: Clustering, classification and density estimation using gaussian finite mixture models. \emph{The R Journal}, \emph{8}(1), 289.

\leavevmode\vadjust pre{\hypertarget{ref-shapiro2004}{}}%
Shapiro, N., \& Ritzwoller, M. (2004). Inferring surface heat flux distributions guided by a global seismic model: Particular application to antarctica. \emph{Earth and Planetary Science Letters}, \emph{223}(1-2), 213--224.

\leavevmode\vadjust pre{\hypertarget{ref-shen2015}{}}%
Shen, T., Hermann, J., Zhang, L., Lü, Z., Padrón-Navarta, J., Xia, B., \& Bader, T. (2015). UHP metamorphism documented in ti-chondrodite-and ti-clinohumite-bearing serpentinized ultramafic rocks from chinese southwestern tianshan. \emph{Journal of Petrology}, \emph{56}(7), 1425--1458.

\leavevmode\vadjust pre{\hypertarget{ref-shepard1968}{}}%
Shepard, D. (1968). A two-dimensional interpolation function for irregularly-spaced data. In \emph{Proceedings of the 1968 23rd ACM national conference} (pp. 517--524).

\leavevmode\vadjust pre{\hypertarget{ref-shreve1986}{}}%
Shreve, R., \& Cloos, M. (1986). Dynamics of sediment subduction, melange formation, and prism accretion. \emph{Journal of Geophysical Research: Solid Earth}, \emph{91}(B10), 10229--10245.

\leavevmode\vadjust pre{\hypertarget{ref-sizova2010}{}}%
Sizova, E., Gerya, T., Brown, M., \& Perchuk, L. (2010). Subduction styles in the precambrian: Insight from numerical experiments. \emph{Lithos}, \emph{116}(3-4), 209--229.

\leavevmode\vadjust pre{\hypertarget{ref-sobolev2005}{}}%
Sobolev, S., \& Babeyko, A. (2005). What drives orogeny in the andes? \emph{Geology}, \emph{33}(8), 617--620.

\leavevmode\vadjust pre{\hypertarget{ref-spear1983}{}}%
Spear, F., \& Selverstone, J. (1983). Quantitative PT paths from zoned minerals: Theory and tectonic applications. \emph{Contributions to Mineralogy and Petrology}, \emph{83}(3), 348--357.

\leavevmode\vadjust pre{\hypertarget{ref-spear2019}{}}%
Spear, F., \& Wolfe, O. (2019). Implications of overstepping of garnet nucleation for geothermometry, geobarometry and p--t path calculations. \emph{Chemical Geology}, \emph{530}, 119323.

\leavevmode\vadjust pre{\hypertarget{ref-spear2014}{}}%
Spear, F., Thomas, J., \& Hallett, B. (2014). Overstepping the garnet isograd: A comparison of QuiG barometry and thermodynamic modeling. \emph{Contributions to Mineralogy and Petrology}, \emph{168}(3), 1--15.

\leavevmode\vadjust pre{\hypertarget{ref-stehman1997}{}}%
Stehman, S. (1997). Selecting and interpreting measures of thematic classification accuracy. \emph{Remote Sensing of Environment}, \emph{62}(1), 77--89.

\leavevmode\vadjust pre{\hypertarget{ref-stein1992}{}}%
Stein, C., \& Stein, S. (1992). A model for the global variation in oceanic depth and heat flow with lithospheric age. \emph{Nature}, \emph{359}(6391), 123--129.

\leavevmode\vadjust pre{\hypertarget{ref-stein1994}{}}%
Stein, C., \& Stein, S. (1994). Constraints on hydrothermal heat flux through the oceanic lithosphere from global heat flow. \emph{Journal of Geophysical Research: Solid Earth}, \emph{99}(B2), 3081--3095.

\leavevmode\vadjust pre{\hypertarget{ref-stockhert2002}{}}%
Stöckhert, B. (2002). Stress and deformation in subduction zones: Insight from the record of exhumed metamorphic rocks. \emph{Geological Society, London, Special Publications}, \emph{200}(1), 255--274.

\leavevmode\vadjust pre{\hypertarget{ref-syracuse2006}{}}%
Syracuse, E., \& Abers, G. (2006). Global compilation of variations in slab depth beneath arc volcanoes and implications. \emph{Geochemistry, Geophysics, Geosystems}, \emph{7}(5).

\leavevmode\vadjust pre{\hypertarget{ref-syracuse2010}{}}%
Syracuse, E., van Keken, P., Abers, G., Suetsugu, D., Bina, C., Inoue, T., et al. (2010). The global range of subduction zone thermal models. \emph{Physics of the Earth and Planetary Interiors}, \emph{183}(1-2), 73--90.

\leavevmode\vadjust pre{\hypertarget{ref-tewksbury2021b}{}}%
Tewksbury-Christle, C., \& Behr, W. (2021). Constraints from exhumed rocks on the seismic signature of the deep subduction interface. \emph{Geophysical Research Letters}, \emph{48}(18).

\leavevmode\vadjust pre{\hypertarget{ref-tewksbury2021a}{}}%
Tewksbury-Christle, C., Behr, W., \& Helper, M. (2021). Tracking deep sediment underplating in a fossil subduction margin: Implications for interface rheology and mass and volatile recycling. \emph{Geochemistry, Geophysics, Geosystems: G (3)}, \emph{22}(3).

\leavevmode\vadjust pre{\hypertarget{ref-turcotte2002}{}}%
Turcotte, D., \& Schubert, G. (2002). \emph{Geodynamics}. Cambridge university press.

\leavevmode\vadjust pre{\hypertarget{ref-vankeken2011}{}}%
van Keken, P., Hacker, B., Syracuse, E., \& Abers, G. (2011). Subduction factory: 4. Depth-dependent flux of \(H_2O\) from subducting slabs worldwide. \emph{Journal of Geophysical Research}, \emph{116}(b1), b01401.

\leavevmode\vadjust pre{\hypertarget{ref-vankeken2018}{}}%
van Keken, P., Wada, I., Abers, G., Hacker, B., \& Wang, K. (2018). Mafic high-pressure rocks are preferentially exhumed from warm subduction settings. \emph{Geochemistry, Geophysics, Geosystems}, \emph{19}(9), 2934--2961.

\leavevmode\vadjust pre{\hypertarget{ref-vankeken2019}{}}%
van Keken, P., Wada, I., Sime, N., \& Abers, G. (2019). Thermal structure of the forearc in subduction zones: A comparison of methodologies. \emph{Geochemistry, Geophysics, Geosystems}, \emph{20}(7), 3268--3288.

\leavevmode\vadjust pre{\hypertarget{ref-vermeesch2018}{}}%
Vermeesch, P. (2018). IsoplotR: A free and open toolbox for geochronology. \emph{Geoscience Frontiers}, \emph{9}(5), 1479--1493.

\leavevmode\vadjust pre{\hypertarget{ref-vignaroli2005}{}}%
Vignaroli, G., Rossetti, F., Bouybaouene, M., Massonne, H.-J., Theye, T., Faccenna, C., \& Funiciello, R. (2005). A counter-clockwise p--t path for the voltri massif eclogites (ligurian alps, italy). \emph{Journal of Metamorphic Geology}, \emph{23}(7), 533--555.

\leavevmode\vadjust pre{\hypertarget{ref-vogt2012}{}}%
Vogt, K., Gerya, T., \& Castro, A. (2012). Crustal growth at active continental margins: Numerical modeling. \emph{Physics of the Earth and Planetary Interiors}, \emph{192}, 1--20.

\leavevmode\vadjust pre{\hypertarget{ref-wada2009}{}}%
Wada, I., \& Wang, K. (2009). Common depth of slab-mantle decoupling: Reconciling diversity and uniformity of subduction zones. \emph{Geochemistry, Geophysics, Geosystems}, \emph{10}(10).

\leavevmode\vadjust pre{\hypertarget{ref-wada2008}{}}%
Wada, I., Wang, K., He, J., \& Hyndman, R. (2008). Weakening of the subduction interface and its effects on surface heat flow, slab dehydration, and mantle wedge serpentinization. \emph{Journal of Geophysical Research: Solid Earth}, \emph{113}(4), 1--15.

\leavevmode\vadjust pre{\hypertarget{ref-wada2012}{}}%
Wada, I., Behn, M., \& Shaw, A. (2012). Effects of heterogeneous hydration in the incoming plate, slab rehydration, and mantle wedge hydration on slab-derived \(H_2O\) flux in subduction zones. \emph{Earth and Planetary Science Letters}, \emph{353-354}, 60--71.

\leavevmode\vadjust pre{\hypertarget{ref-wakabayashi2015}{}}%
Wakabayashi, J. (2015). Anatomy of a subduction complex: Architecture of the franciscan complex, california, at multiple length and time scales. \emph{International Geology Review}, \emph{57}(5-8), 669--746.

\leavevmode\vadjust pre{\hypertarget{ref-wang1995}{}}%
Wang, K., Mulder, T., Rogers, G., \& Hyndman, R. (1995). Case for very low coupling stress on the cascadia ssubduction fault. \emph{Journal of Geophysical Research: Solid Earth}, \emph{100}(B7), 12907--12918.

\leavevmode\vadjust pre{\hypertarget{ref-wilkinson2016}{}}%
Wilkinson, M., Dumontier, M., Aalbersberg, I., Appleton, G., Axton, M., Baak, A., et al. (2016). The FAIR guiding principles for scientific data management and stewardship. \emph{Scientific Data}, \emph{3}(1), 1--9.

\leavevmode\vadjust pre{\hypertarget{ref-wilson2014}{}}%
Wilson, C., Spiegelman, M., van Keken, P., \& Hacker, B. (2014). Fluid flow in subduction zones: The role of solid rheology and compaction pressure. \emph{Earth and Planetary Science Letters}, \emph{401}, 261--274.

\leavevmode\vadjust pre{\hypertarget{ref-wilson1966}{}}%
Wilson, J. (1966). Did the atlantic close and then re-open? \emph{Nature}, \emph{211}(5050), 676--681.

\leavevmode\vadjust pre{\hypertarget{ref-yamato2007}{}}%
Yamato, P., Agard, P., Burov, E., Le Pourhiet, L., Jolivet, L., \& Tiberi, C. (2007). Burial and exhumation in a subduction wedge: Mutual constraints from thermomechanical modeling and natural p-t-t data (schistes lustr{é}s, western alps). \emph{Journal of Geophysical Research: Solid Earth}, \emph{112}(B7).

\leavevmode\vadjust pre{\hypertarget{ref-yamato2008}{}}%
Yamato, P., Burov, E., Agard, P., Le Pourhiet, L., \& Jolivet, L. (2008). HP-UHP exhumation during slow continental subduction: Self-consistent thermodynamically and thermomechanically coupled model with application to the western alps. \emph{Earth and Planetary Science Letters}, \emph{271}(1-4), 63--74.

\leavevmode\vadjust pre{\hypertarget{ref-ypma2014}{}}%
Ypma, J. (2014). Introduction to nloptr: An r interface to NLopt. \emph{R Package}, \emph{2}.

\leavevmode\vadjust pre{\hypertarget{ref-zack2007}{}}%
Zack, T., \& John, T. (2007). An evaluation of reactive fluid flow and trace element mobility in subducting slabs. \emph{Chemical Geology}, \emph{239}(3-4), 199--216.

\leavevmode\vadjust pre{\hypertarget{ref-zhu2018}{}}%
Zhu, A., Lu, G., Liu, J., Qin, C., \& Zhou, C. (2018). Spatial prediction based on third law of geography. \emph{Annals of GIS}, \emph{24}(4), 225--240.

\end{CSLReferences}

\cleardoublepage

\hypertarget{tglobe}{%
\chapter*{ThermoGlobe References}\label{tglobe}}
\addcontentsline{toc}{chapter}{ThermoGlobe References}

\markboth{ThermoGlobe References}{ThermoGlobe References}

\hypertarget{refs_tglobe}{}
\begin{CSLReferences}{1}{1}
\leavevmode\vadjust pre{\hypertarget{ref-abbott1984}{}}%
Abbott, D., Menke, W., Hobart, M., Anderson, R. N., \& Embley, R. W. (1984). Correlated sediment thickness, temperature gradient and excess pore pressure in a marine fault block basin. \emph{Geophys. Res. Lett.}, \emph{11}, 485--488.

\leavevmode\vadjust pre{\hypertarget{ref-abbott1986b}{}}%
Abbott, D. H., Hobart, M. A., \& Embley, R. W. (1986). Heat flow and mass wasting in the {Wilmington Canyon} region: {U.S.} Continental margin. \emph{Geo-Marine Lett.}, \emph{6}, 131--138.

\leavevmode\vadjust pre{\hypertarget{ref-abbott1986a}{}}%
Abbott, Dallas H., Morton, J. L., \& Holmes, M. L. (1986). Heat flow measurements on a hydrothermally-active, slow-spreading ridge: The escanaba trough. \emph{Geophysical Research Letters}, \emph{13}, 678--680. \url{https://doi.org/10.1029/GL013i007p00678}

\leavevmode\vadjust pre{\hypertarget{ref-akhmedzyanov2012}{}}%
Akhmedzyanov, V. R., Ermakov, A. V., \& Khutorskoy, M. D. (2012). New data on heat flow in the north atlantic region. \emph{Doklady Earth Sciences}, \emph{442}(1), 91--96. \url{https://doi.org/10.1134/s1028334x12010011}

\leavevmode\vadjust pre{\hypertarget{ref-albert1979}{}}%
Albert-Beltran, J. F. (1979). Heat flow and temperture gradient data from {Spain}. In \emph{Terrestrial heat flow in europe} (pp. 261--266). Springer Verlag.

\leavevmode\vadjust pre{\hypertarget{ref-alexandrino2008}{}}%
Alexandrino, C. H., \& Hamza, V. M. (2008). Estimates of heat flow and heat production and a thermal model of the são francisco craton. \emph{International Journal of Earth Sciences}, \emph{97}(2), 289--306. \url{https://doi.org/10.1007/s00531-007-0291-y}

\leavevmode\vadjust pre{\hypertarget{ref-aliev1979}{}}%
Aliev, S. A., Ashirov, T., Lipsits, Yu. M., Sopiev, V. A., \& Sudakov, N. P. (1979). Novye dannye o teplovom potoke cherez dno kaspiiskogo morya (russ.). \emph{Izvestiya An Turkm. Ssr, Ser. Fiziko-Tekhnicheskikh, Khimiches- Kikh I Geologicheskikh Nauk}, \emph{2}, 124--126.

\leavevmode\vadjust pre{\hypertarget{ref-allis1975}{}}%
Allis, R. G. (1975). \emph{Geothermal measurements in five small lakes of northwestern {Ontario, Canada}} (Master's thesis).

\leavevmode\vadjust pre{\hypertarget{ref-allis1979}{}}%
Allis, R. G., \& Garland, G. D. (1979). Heat flow measurements under some lakes in the superior province of the canadian shield. \emph{Canadian Journal of Earth Sciences}, \emph{16}, 1954--1961. \url{https://doi.org/10.1139/e80-112}

\leavevmode\vadjust pre{\hypertarget{ref-anderson1940}{}}%
Anderson, E. M. (1940). Loss of heat by conduction from the {Earth's} crust. \emph{Proc. R. Soc. Edinb.}, \emph{60}, 192--209.

\leavevmode\vadjust pre{\hypertarget{ref-anderson1975}{}}%
Anderson, R. N. (1975). Heat flow in the mariana marginal basin. \emph{Journal of Geophysical Research}, \emph{80}, 4043--4048. \url{https://doi.org/10.1029/JB080i029p04043}

\leavevmode\vadjust pre{\hypertarget{ref-anderson1976}{}}%
Anderson, R. N., \& Hobart, M. A. (1976). The relation between heat flow, sediment thickness, and age in the eastern pacific. \emph{Journal of Geophysical Research}, \emph{81}, 2968--2989. \url{https://doi.org/10.1029/JB081i017p02968}

\leavevmode\vadjust pre{\hypertarget{ref-anderson1991}{}}%
Anderson, R. N., \& Larue, D. K. (1991). Wellbore heat flow from the {Toa Baja} scientific drillhole, {Puerto Rico}. \emph{Geophysical Research Letters}, \emph{18}, 537--540. \url{https://doi.org/10.1029/91gl00391}

\leavevmode\vadjust pre{\hypertarget{ref-anderson1978}{}}%
Anderson, R. N., \& Von Herzen, R. P. (1978). Heat flow on the pacific-antarctic ridge. \emph{Earth and Planetary Science Letters}, \emph{41}(4), 451--460. \url{https://doi.org/10.1016/0012-821x(78)90176-0}

\leavevmode\vadjust pre{\hypertarget{ref-anderson1976a}{}}%
Anderson, R. N., Moore, G. F., Schilt, S. S., Cardwell, R. C., \& Tréhu, A. (1976). Heat flow near a fossil ridge on the north flank of the {Galapagos} spreading center. \emph{J. Geophys. Res.}, \emph{81}, 1828--1838.

\leavevmode\vadjust pre{\hypertarget{ref-anderson1977}{}}%
Anderson, R. N., Langseth, M. G., \& Sclater, J. G. (1977). The mechanism of heat transfer through the floor of the {Indian Ocean}. \emph{J. Geophys. Res.}, \emph{82}, 3391--3490.

\leavevmode\vadjust pre{\hypertarget{ref-anderson1978a}{}}%
Anderson, R. N., Hobart, M. A., Von Herzen, R. P., \& Fornari, D. J. (1978). Geophysical surveys on the {East Pacific Rise--Galapagos} rise system. \emph{Geophys. J. Roy. Astr. Soc.}, \emph{54}, 141--166.

\leavevmode\vadjust pre{\hypertarget{ref-anderson1979}{}}%
Anderson, R. N., Hobart, M. A., \& Langseth, M. G. (1979). Geothermal convection through oceanic crust and sediments in the {Indian Ocean}. \emph{Science}, \emph{204}, 828--832.

\leavevmode\vadjust pre{\hypertarget{ref-andreescu1989}{}}%
Andreescu, M., Burst, D., Demetrescu, C., Ene, M., \& Polonic, G. (1989). On the geothermal regime of the {Moesian Platform and Getic Depression}. \emph{Tectonophysics}, \emph{164}, 281--286.

\leavevmode\vadjust pre{\hypertarget{ref-andrews1984}{}}%
Andrews-Speed, C. P., Oxburgh, E. R., \& Cooper, B. A. (1984). Temperatures and depth dependent heat flow in western {North Sea}. \emph{Bull. Am. Ass. Petrol. Geol.}, \emph{68}, 1764--1781.

\leavevmode\vadjust pre{\hypertarget{ref-arnaiz2013}{}}%
Arnaiz-Rodríguez, M. S., \& Orihuela, N. (2013). Curie point depth in venezuela and the eastern caribbean. \emph{Tectonophysics}, \emph{590}(0), 38--51. \url{https://doi.org/10.1016/j.tecto.2013.01.004}

\leavevmode\vadjust pre{\hypertarget{ref-arney1982}{}}%
Arney, B. H. (1982). Evidence of form higher temperatures from alteration minerals, {Bostic 1-A Well, Mountain Home, Idaho}. \emph{Geothermal Res. Council Trans.}, \emph{6}, 3--6.

\leavevmode\vadjust pre{\hypertarget{ref-arshavskaya1984}{}}%
Arshavskaya, N. I., Galdin, N. E., Karus, E. V., Kuznetsov, O. L., Lubimova, E. A., Milanovskii, S. Y., et al. (1984). Teplovye svoistva porod. In \emph{Kolskaya sverkhglubo- kaya. Issledovanie glubinnogo stroeniya kontinentalnoi kory s po- moshchyyu bureniya kolskoi sverkhglubokoi skvazhiny. (Pod red. Koz- lovskii e.a.)} (pp. 341--348).

\leavevmode\vadjust pre{\hypertarget{ref-artemenko1986}{}}%
Artemenko, V. I., Selyaninov, V. G., Smirnova, L. A., \& Strygin, V. N. (1986). Avtonom- nyi tsifrovoi termozond dlya morskikh geotermicheskikh issledovanii (atstm-1) (russ.). \emph{Okeanologiya}, \emph{T.26, Vyp.6}, 1033--1038.

\leavevmode\vadjust pre{\hypertarget{ref-ashirov1984}{}}%
Ashirov, T. A. (1984). Geotermicheskoe pole turkmenii. - moskva: nauka.

\leavevmode\vadjust pre{\hypertarget{ref-ashirov1985}{}}%
Ashirov, T. O. (1985). Teplovom pole v predelakh zapadnogo borta yuzhno- kaspiiskoi depressii. - izvestiya an turkm. Ssr, ser. Fiziko-tekh- nicheskikh, khimicheskikh i geologicheskikh nauk. (russ.), \emph{2}, 70--74.

\leavevmode\vadjust pre{\hypertarget{ref-atroshchenko1975}{}}%
Atroshchenko, P. P. (1975). Geotermicheskie usloviya severnoi chasti pri- pyatskoi vpadiny (russ.). \emph{Minsk Nauka I Tekhnika}, \emph{104}.

\leavevmode\vadjust pre{\hypertarget{ref-avetisyyants1974}{}}%
Avetisyyants, A. A. (1974a). Teplovoe pole geosinklinalnogo obramleniya vostochno-evropeiskoi platformy. Armeniya i sopredelnye territorii (russ.). \emph{Glubinnyi Teplovoi Potok Evropeiskoi Chasti SSSR. Kiev, Naukova Dumka}, \emph{V}, 90--95.

\leavevmode\vadjust pre{\hypertarget{ref-avetisyyants1974a}{}}%
Avetisyyants, A. A. (1974b). Teplovoi potok v armenii (russ.). \emph{Geotermiya. Otchety Po Geotermicheskim Issledovaniyam V SSSR. Vypusk 1-2. Ot- Chety Za 1971-1972 Gg. Moskva}, 44--47.

\leavevmode\vadjust pre{\hypertarget{ref-avetisyyants1979}{}}%
Avetisyyants, A. A. (1979). Geotermicheskie usloviya nedr armenii (russ.). \emph{Moskva Nauka}, \emph{88}.

\leavevmode\vadjust pre{\hypertarget{ref-avetisyyants1968}{}}%
Avetisyyants, A. A., Ananyan, A. L., \& Igumnov, V. A. (1968). Teplovoi potok po skvazhine kadzharan - 480. - doklady an arm. Ssr. 1968. \emph{T. 46}.

\leavevmode\vadjust pre{\hypertarget{ref-baikal1985}{}}%
Baikal. (1985). Katalog dannykh po teplovomu potoku sibiri (1966-1984) (p. 82). Institut Geologii I Geofisiki So An Sssr (Russ.).

\leavevmode\vadjust pre{\hypertarget{ref-balabashin1987}{}}%
Balabashin, V. I., \& Koptev, A. A. (2004). Results of the 6th cruise of r/v "academic lavrentiev" in 1987 (personal communication). In \emph{CD rom: Geothermal gradient and heat flow data in and around japan} (p. --). Geological Survey of Japan, AIST, 2004.

\leavevmode\vadjust pre{\hypertarget{ref-balkan2019}{}}%
Balkan-Pazvantoğlu, E., \& Erkan, K. (2019). Temperature-depth curves and heat flow in central part of {Anatolia}, {Turkey}. \emph{Tectonophysics}, \emph{757}, 24--34. \url{https://doi.org/10.1016/j.tecto.2019.02.019}

\leavevmode\vadjust pre{\hypertarget{ref-ballard1987}{}}%
Ballard, S. I. I. I., Pollack, H. N., \& Skinner, N. J. (1987). Terrestrial heat flow in botswana and namibia. \emph{Journal of Geophysical Research}, \emph{92}, 6291--6300. \url{https://doi.org/10.1029/JB092iB07p06291}

\leavevmode\vadjust pre{\hypertarget{ref-balling1979}{}}%
Balling, N. (1979). Subsurface temperatures and heat flow estimates in {Denmark}. In \emph{Terrestrial heat flow in europe} (pp. 1161--1171). Springer Verlag.

\leavevmode\vadjust pre{\hypertarget{ref-balling1986}{}}%
Balling, N. (1986). \emph{Temperature of geothermal reservoirs in denmark. Report to commission of the european communities.}

\leavevmode\vadjust pre{\hypertarget{ref-balling1991}{}}%
Balling, N. (1991). Catalogue of heat flow density data: denmark. In \emph{Geothermal atlas of europe} (pp. 111--112). Hermann Haack Verlagsgesellschaft mbH.

\leavevmode\vadjust pre{\hypertarget{ref-balling1984}{}}%
Balling, N., Kristiansen, J. I., \& Saxov, S. (1984). Geothermal measurements from the vestmanna-1 and lopra-1 boreholes. In \emph{The deep drilling project 1980-1981 in the faeroe islands} (Vol. Supplementum IX Vol., pp. 137--148). F{o}roya Fróõskaparfelag.

\leavevmode\vadjust pre{\hypertarget{ref-balling2006}{}}%
Balling, N., Breiner, N., \& Waagstein, R. (2006). Thermal structure of the deep lopra-1/1A borehole in the faroe islands. \emph{Geological Survey of Denmark and Greenland Bulletin}, \emph{9}, 91--107.

\leavevmode\vadjust pre{\hypertarget{ref-balobaev1982}{}}%
Balobaev, V. N., \& Deviatkin, V. N. (1982a). Merzlotno-geotermicheskie usloviya zapadnoy jakutii v svyasi s neftegasonosnostiu (russ.). \emph{Gidrogeologiya Neftegasonosnykh Oblastey Sibirskoy Platformy. Novosibirsk: Igig So an SSSR}, 18--22.

\leavevmode\vadjust pre{\hypertarget{ref-balobaev1978}{}}%
Balobaev, V. T. (1978). Reconstructsiya paleoklimata po sovremennym geotermicheskim dannym (russ.).

\leavevmode\vadjust pre{\hypertarget{ref-balobaev1982a}{}}%
Balobaev, V. T., \& Deviatkin, V. N. (1982b). Geothermics and geothermal. energy (pp. 107--110). E. Schweizerbartische Verlagsbuch - Handlung, Stutgart.

\leavevmode\vadjust pre{\hypertarget{ref-balobaev1978a}{}}%
Balobaev, V. T., \& Levchenko, A. I. (1978). Geotermicheskie osobennosti i merz- laya zona hr.suntar-khayata (na primere nezhdaninskogo mestorozh- deniya). \emph{Geoteplofizicheskie Issledovaniya V Sibiri. No- Vosibirsk: Nauka}, 129--142.

\leavevmode\vadjust pre{\hypertarget{ref-banda1991}{}}%
Banda, E., Albert-Bertran, J. F., Fernàndez, M., \& Garcia de la Noceda, C. (1991). Catalogue of heat flow density data: spain. In \emph{Geothermal atlas of europe} (p. 124). Hermann Haack Verlagsgesellschaft mbH.

\leavevmode\vadjust pre{\hypertarget{ref-barr1979}{}}%
Barr, S. M., Ratanasathien, B., Breen, D., Ramingwong, T., \& Sertsrivanit, S. (1979). Hot springs and geothermal gradient in northern {Thailand}. \emph{Geothermics}, \emph{8}, 85--95.

\leavevmode\vadjust pre{\hypertarget{ref-batir2016}{}}%
Batir, J. F., Blackwell, D. D., \& Richards, M. C. (2016). Heat flow and temperature-depth curves throughout alaska: Finding regions for future geothermal exploration. \emph{Journal of Geophysics and Engineering}, \emph{13}(3), 366. \url{https://doi.org/10.1088/1742-2132/13/3/366}

\leavevmode\vadjust pre{\hypertarget{ref-bauer1986}{}}%
Bauer, M. S., \& Chapman, D. S. (1986). Thermal regime at the upper {Stillwater Dam} site, {Uinta Mountains, Utah}: Implications for terrain, microclimate and structural corrections in heat flow studies. \emph{Tectonophysics}, \emph{128}, 1--20.

\leavevmode\vadjust pre{\hypertarget{ref-bayer1982}{}}%
Bayer, R., Couturie, J. P., \& Vasseur, G. (1982). Données geophysique recentes sur le {Massif de la Margeride}. \emph{Ann. Geophys.}, \emph{38}, 431--447.

\leavevmode\vadjust pre{\hypertarget{ref-beach1987}{}}%
Beach, R. D. W., Jones, F. W., \& Majorowicz, J. A. (1987). Heat flow and heat generation estimates for the {Churchill} basement of the western {Canadian Basin} in {Alberta, Canada}. \emph{Geothermics}, \emph{16}, 1--16.

\leavevmode\vadjust pre{\hypertarget{ref-beamish2015}{}}%
Beamish, D., \& Busby, J. (2015). The cornubian geothermal province: Heat production and flow in SW england. \emph{Geophysical Journal International}, \emph{submitted}.

\leavevmode\vadjust pre{\hypertarget{ref-beardsmore2004}{}}%
Beardsmore, G. R. (2004). The influence of basement on surface heat flow in the {Cooper Basin}. \emph{Explor. Geophys.}, \emph{35}, 223--235.

\leavevmode\vadjust pre{\hypertarget{ref-beardsmore2005}{}}%
Beardsmore, G. R. (2005). High-resolution heat-flow measurements in the southern {Carnarvon Basin, Western Australia}. \emph{Explor. Geophys.}, \emph{36}, 206--215.

\leavevmode\vadjust pre{\hypertarget{ref-beardsmore2002}{}}%
Beardsmore, G. R., \& Altmann, M. J. (2002). A heat flow map of the {Dampier} sub-basin.

\leavevmode\vadjust pre{\hypertarget{ref-beck1977}{}}%
Beck, A. E. (1977). Climaticially perturbed temperature gradients and their effect on regional and continental heat-flow means. \emph{Tectonophysics}, \emph{41}, 17--39.

\leavevmode\vadjust pre{\hypertarget{ref-beck1964}{}}%
Beck, A. E., \& Logis, Z. (1964). Terrestrial flow of heat in the {Brent} crater. \emph{Nature}, \emph{201}, 383.

\leavevmode\vadjust pre{\hypertarget{ref-beck1972}{}}%
Beck, A. E., \& Mustonen, E. (1972). Preliminary heat flow data from ghana. \emph{Nature Physical Science}, \emph{235}, 172--174. \url{https://doi.org/10.1038/physci235172a0}

\leavevmode\vadjust pre{\hypertarget{ref-beck1968}{}}%
Beck, A. E., \& Neophytou, J. P. (1968). Heat flow and underground water flow in the {Coronation} mine area. In \emph{Symposium on the geology of coronation mine, saskatchewan} (pp. 229--239). Geol. Surv. Can.

\leavevmode\vadjust pre{\hypertarget{ref-beck1966}{}}%
Beck, A. E., \& Sass, J. H. (1966). A preliminary value of heat flow at the {Muskox} intrusion near {Coppermine, N.W.T., Canada}. \emph{Earth Planet. Sci. Lett.}, \emph{1}, 123--129.

\leavevmode\vadjust pre{\hypertarget{ref-beck1976}{}}%
Beck, A. E., Hamza, V. M., \& Chang, C. C. (1976). Analysis of heat flow data---correlation of thermal resistivity and shock metamorphic grade and its use as evidence for an impact origin of the {Brent Crater}. \emph{Canadian Journal of Earth Sciences}, \emph{13}, 929--936.

\leavevmode\vadjust pre{\hypertarget{ref-becker1968}{}}%
Becker, D., \& Meincke, W. (1968). Der waermefluss zwischen harz und priegnitz. \emph{Z. F. Angew. Geol.}, \emph{14}, 291--297.

\leavevmode\vadjust pre{\hypertarget{ref-becker1981}{}}%
Becker, K. (1981). \emph{Heat flow studies of spreading center hydrothermal processes} (PhD thesis).

\leavevmode\vadjust pre{\hypertarget{ref-becker1991}{}}%
Becker, K., \& Fisher, A. T. (1991). A brief review of heat-flow studies in the {Guaymas Basin, Gulf of California}. In \emph{The gulf and peninsular province of the californias} (Vol. 47, pp. 709--720). Am. Assoc. Petrol. Geol.

\leavevmode\vadjust pre{\hypertarget{ref-becker1983b}{}}%
Becker, K., \& Von Herzen, R. P. (1983a). Heat flow on the western flank of the east pacific rise at 21\textdegree{}n. \emph{Journal of Geophysical Research}, \emph{88}, 1057--1066. \url{https://doi.org/10.1029/JB088iB02p01057}

\leavevmode\vadjust pre{\hypertarget{ref-becker1983a}{}}%
Becker, K., \& Von Herzen, R. P. (1983b). Heat transfer through the sediments of the mounds hydrothermal area {Galapagos} spreading center at 86\(^\circ\)w. \emph{J. Geophys. Res.}, \emph{88}, 995--1008.

\leavevmode\vadjust pre{\hypertarget{ref-becker1996}{}}%
Becker, K., \& Von Herzen, R. P. (1996). Pre-drilling observations of conductive heat flow at the TAG active mound using DSV {Alvin}. \emph{Initial Reports ODP}, \emph{158}, 23--29.

\leavevmode\vadjust pre{\hypertarget{ref-becker1983}{}}%
Becker, K., Langseth, M. G., \& Von Herzen, R. P. (1983). Deep crustal geothermal measurements. Hole 504B. Legs 69 and 70. \emph{Initial Reports DSDP}, \emph{69}.

\leavevmode\vadjust pre{\hypertarget{ref-benavraham1987}{}}%
Ben-Avraham, Z., \& Von Herzen, R. P. (1987). Heat flow and continental breakup: The {Gulf of Elat (Aqaba)}. \emph{J. Geophys. Res.}, \emph{92}, 1407--1416.

\leavevmode\vadjust pre{\hypertarget{ref-benfield1939}{}}%
Benfield, A. E. (1939). Terrestrial heat flow in {Britain}. \emph{Proc. Roy. Soc. London A}, \emph{173}, 430--450.

\leavevmode\vadjust pre{\hypertarget{ref-benfield1947}{}}%
Benfield, A. E. (1947). A heat flow value for a well in {California}. \emph{Am. J. Sci.}, \emph{245}, 1--18.

\leavevmode\vadjust pre{\hypertarget{ref-bentkowski1989}{}}%
Bentkowski, W. H., \& Lewis, T. J. (1989). \emph{Thermal measurements in cordillera boreholes of opportunity 1984-1987} (No. 2048) (p. 30p.).

\leavevmode\vadjust pre{\hypertarget{ref-bentkowski1994}{}}%
Bentkowski, W. H., \& Lewis, T. J. (1994). \emph{Heat flow determinations in the cordillera: 1988-1992} (No. 298).

\leavevmode\vadjust pre{\hypertarget{ref-berthier1984}{}}%
Berthier, F., Fabriol, R., \& Puvilland, P. (1984). \emph{Évaluation des ressources géothermiques basse énergie en {République} de {Haiti}. {Recherche} d'un projet type: Synthèse des travaux de terrain (géologie, géochimie, géophysique)} (No. 84 Sgn 206 Gth).

\leavevmode\vadjust pre{\hypertarget{ref-birch1947}{}}%
Birch, F. (1947). Temperature and heat flow in a well near {Colorado Springs}. \emph{Am. J. Sci.}, \emph{245}, 733--753.

\leavevmode\vadjust pre{\hypertarget{ref-birch1950}{}}%
Birch, F. (1950). Flow of heat in the {Front Range, Colorado}. \emph{Geol. Soc. Am. Bull.}, \emph{61}, 567--630.

\leavevmode\vadjust pre{\hypertarget{ref-birch1954}{}}%
Birch, F. (1954). Thermal conductivity, climatic variation, and heat flow near {Calumet, Michigan}. \emph{Am. J. Sci.}, \emph{252}, 1--25.

\leavevmode\vadjust pre{\hypertarget{ref-birch1956}{}}%
Birch, F. (1956). Heat flow at {Eniwetok Atoll}. \emph{Bulletin of Geological Society of America}, \emph{67}, 941--942. \url{https://doi.org/10.1130/0016-7606(1956)67\%5B941:hfaea\%5D2.0.co;2}

\leavevmode\vadjust pre{\hypertarget{ref-birch1964}{}}%
Birch, F. S. (1964). \emph{Some heat flow measurements in the {Atlantic Ocean}} (Master's thesis).

\leavevmode\vadjust pre{\hypertarget{ref-birch1965}{}}%
Birch, F. S. (1965). Heat flow near the {New England} seamounts. \emph{Journal of Geophysical Research}, \emph{70}, 5223--5226. \url{https://doi.org/10.1029/JZ070i020p05223}

\leavevmode\vadjust pre{\hypertarget{ref-birch1970}{}}%
Birch, F. S. (1970). The barracuda fault zone in the western north atlantic- geological and geophysical studies. \emph{Deep Sea Res.}, \emph{17}, 841--849. \url{https://doi.org/10.1016/0011-7471(70)90002-1}

\leavevmode\vadjust pre{\hypertarget{ref-birch1966}{}}%
Birch, F. S., \& Halunen, A. (1966). Heat flow measurements in the {Atlantic Ocean, Indian Ocean, Mediterranean Sea, and Red Sea}. \emph{J. Geophys. Res.}, \emph{71}, 583--586.

\leavevmode\vadjust pre{\hypertarget{ref-black1983}{}}%
Black, G. L., Blackwell, D. D., \& Steele, J. L. (1983). Heat flow in the {Oregon Cascades}. In \emph{Geology and geothermal resources of the central oregon cascade range} (Vol. 15, p. 123). Oregon Dept. Geology Mineral Industries.

\leavevmode\vadjust pre{\hypertarget{ref-blackman1987}{}}%
Blackman, D. K., Von Herzen, R. P., \& Lawver, L. A. (1987). Heat flow and tectonics in the western {Ross Sea}, {Antarctica}. In \emph{The antarctic continental margin: Geology and geophysics of the western ross sea} (Vol. 5b). Circum-Pacific Council for Energy; Mineral Resources.

\leavevmode\vadjust pre{\hypertarget{ref-blackwell1967}{}}%
Blackwell, D. D. (1967). \emph{Terrestrial heat flow determinations in the northwestern united states} (PhD thesis).

\leavevmode\vadjust pre{\hypertarget{ref-blackwell1969}{}}%
Blackwell, D. D. (1969). Heat flow determinations in the northwestern {United States}. \emph{Journal of Geophysical Research}, \emph{74}, 992--1007. \url{https://doi.org/10.1029/JB074i004p00992}

\leavevmode\vadjust pre{\hypertarget{ref-blackwell1974}{}}%
Blackwell, D. D. (1974). Terrestrial heat flow and its implications on the location of geothermal reservoirs in {Washington} (pp. 21--33). Washington Division of Mines; Geology.

\leavevmode\vadjust pre{\hypertarget{ref-blackwell1980}{}}%
Blackwell, D. D. (1980). \emph{Heat flow and geothermal gradient measurements in washington to 1979 and temperature-depth data collected during 1979} (No. 80--89).

\leavevmode\vadjust pre{\hypertarget{ref-blackwell1973}{}}%
Blackwell, D. D., \& Baag, C. (1973). Heat flow in a blind geothermal area near marysville, montana. \emph{Geophysics}, \emph{38}, 941--956. \url{https://doi.org/10.1190/1.1440384}

\leavevmode\vadjust pre{\hypertarget{ref-blackwell1988}{}}%
Blackwell, D. D., \& Baker, S. L. (1988). Thermal analysis of the breitenbush geothermal system. \emph{Geothermal Resources Council Trans.}, \emph{12}, 221--226.

\leavevmode\vadjust pre{\hypertarget{ref-blackwell1989}{}}%
Blackwell, D. D., \& Carter, L. S. (1989). Thermal aspect data. In \emph{Decade of north american geology}. NOAA, Geophys. Data Center.

\leavevmode\vadjust pre{\hypertarget{ref-blackwell1977}{}}%
Blackwell, D. D., \& Chapman, D. S. (1977). Interpretation of geothermal gradient and heat flow data for {Basin and Range} geothermal systems. \emph{Geothermal Res. Council Trans.}, \emph{1}, 19--20.

\leavevmode\vadjust pre{\hypertarget{ref-blackwell2004}{}}%
Blackwell, D. D., \& Richards, M. (2004). \emph{Geothermal map of north america}. \url{https://doi.org/10.1130/dnag-csms-v6.1}

\leavevmode\vadjust pre{\hypertarget{ref-blackwell1987}{}}%
Blackwell, D. D., \& Steele, J. L. (1987). Geothermal data from deep holes in the oregon cascade range. \emph{Geothermal Resources Council Trans.}, \emph{11}, 317--322.

\leavevmode\vadjust pre{\hypertarget{ref-blackwell1975}{}}%
Blackwell, D. D., Holdaway, M. J., Morgan, P., Petefish, D., Rape, T., Steele, J. L., et al. (1975). Results and analysis of exploration and deep drilling at {Marysville} geothermal area. In \emph{The {Marysville, Montana Geothermal Project} final report}. Battelle Pacific NW Laboratories.

\leavevmode\vadjust pre{\hypertarget{ref-blackwell1978}{}}%
Blackwell, D. D., Hull, D. A., Bowen, R. G., \& Steele, J. L. (1978). \emph{Heat flow of oregon, oregon} (No. 4) (p. 42p.). Retrieved from \url{http://www.oregongeology.org/pubs/OG/OGv65n01.pdf}

\leavevmode\vadjust pre{\hypertarget{ref-blackwell1982}{}}%
Blackwell, D. D., Bowen, R. G., Hull, D. A., Riccio, J., \& Steele, J. L. (1982). Heat flow, arc volcanism, and subduction in northern {Oregon}. \emph{J. Geophys. Res.}, \emph{87}, 8735--8754.

\leavevmode\vadjust pre{\hypertarget{ref-blackwell1986}{}}%
Blackwell, D. D., Kelley, S. A., \& Edmiston, R. C. (1986). Analysis and interpretation of thermal data from the borax lake geothermal project, oregon. \emph{Geothermal Resources Council Trans.}, \emph{10}, 169--174.

\leavevmode\vadjust pre{\hypertarget{ref-blackwell1990}{}}%
Blackwell, D. D., Steele, J. L., Kelley, S. A., \& Korosec, M. A. (1990). Heat flow in the state of {Washington} and the {Cascade} thermal conditions. \emph{J. Geophys. Res.}, \emph{95}, 19495--19516.

\leavevmode\vadjust pre{\hypertarget{ref-boccaletti1977}{}}%
Boccaletti, M., Fazzuoli, M., Loddo, M., \& Mongelli, F. (1977). Heat flow measurements on the northern apennines arc. \emph{Tectonophysics}, \emph{41}, 101--112. \url{https://doi.org/10.1016/0040-1951(77)90182-2}

\leavevmode\vadjust pre{\hypertarget{ref-bodell1982}{}}%
Bodell, J., \& Chapman, D. S. (1982). Heat flow in the north-central {Colorado Plateau}. \emph{J. Geophys. Res.}, \emph{87}, 2869--2884.

\leavevmode\vadjust pre{\hypertarget{ref-bodmer1984}{}}%
Bodmer, P., \& Rybach, L. (1984). Geothermal map of switzerland (heat flow density). \emph{Geophysique}, \emph{22}, 46--47.

\leavevmode\vadjust pre{\hypertarget{ref-bodmer1982}{}}%
Bodmer, P. H. (1982). \emph{Beitragen zur geotermie der schweiz} (PhD thesis).

\leavevmode\vadjust pre{\hypertarget{ref-bodmer1983}{}}%
Bodmer, P. H. (1983). \emph{Heat flow density calculations}.

\leavevmode\vadjust pre{\hypertarget{ref-bogomolov1970a}{}}%
Bogomolov, Yu. G. (1970). Dannye o teplovom rezhime zemnoi kory yugo-zapada BSSR (russ.). \emph{Doklady an BSSR}, \emph{14}(1), 57--60.

\leavevmode\vadjust pre{\hypertarget{ref-bojadgieva2008}{}}%
Bojadgieva, K. (2008). Pers. comm.

\leavevmode\vadjust pre{\hypertarget{ref-boldizsar1956}{}}%
Boldizsar, T. (1956). Terrestrial heat flow in {Hungary}. \emph{Geofisica Pura e Applicata}, \emph{34}, 66--70.

\leavevmode\vadjust pre{\hypertarget{ref-boldizsar1975}{}}%
Boldizsar, T. (1975). Research and development of geothermal energy production in {Hungary}. \emph{Geothermics}, \emph{4}, 44--50.

\leavevmode\vadjust pre{\hypertarget{ref-boldizsar1959}{}}%
Boldizsár, T. (1959). Terrestrial heat flow in the {Nagylengyel} oilfield. \emph{Publ. Min. Fak. Sopron.}, \emph{20}, 27--34.

\leavevmode\vadjust pre{\hypertarget{ref-boldizsar1963}{}}%
Boldizsár, T. (1963). Terrestrial heat flow in the natural steam field at {Larderello}. \emph{Geofis.pura Appl.}, \emph{56}, 115--122.

\leavevmode\vadjust pre{\hypertarget{ref-boldizsar1964b}{}}%
Boldizsár, T. (1964). Geothermal measurements in the twin shaft of {Hosszuheteny}. \emph{Acta Techn. Acad. Sci. Hungry}, \emph{47}(3-4), 293--308.

\leavevmode\vadjust pre{\hypertarget{ref-boldizsar1965}{}}%
Boldizsár, T. (1965). Heat flow in {Oligocene} sediments at {Szentendre}. \emph{Pure and Applied Geophysics}, \emph{61}, 127--138. \url{https://doi.org/10.1007/bf00875769}

\leavevmode\vadjust pre{\hypertarget{ref-boldizsar1966}{}}%
Boldizsár, T. (1966). Heat flow in the natural gas field of {Hadjuszoboszlo}. \emph{Pure and Applied Geophysics}, \emph{64}, 121--125. \url{https://doi.org/10.1007/bf00875537}

\leavevmode\vadjust pre{\hypertarget{ref-boldizsar1967}{}}%
Boldizsár, T. (1967). Terrestrial heat flow in {Hungarian Permian strata}. \emph{Pure and Applied Geophysics}, \emph{67}, 128--132. \url{https://doi.org/10.1007/bf00880570}

\leavevmode\vadjust pre{\hypertarget{ref-boldizsar1968}{}}%
Boldizsár, T. (1968). Geothermal data from the {Vienna Basin}. \emph{Journal of Geophysical Research}, \emph{73}(2), 613--618. \url{https://doi.org/10.1029/JB073i002p00613}

\leavevmode\vadjust pre{\hypertarget{ref-bonneville1997}{}}%
Bonneville, A., Von Herzen, R. P., \& Lucazeau, F. (1997). Heat flow over {Reunion} hot spot track: Additional evidence for thermal rejuvenation of oceanic lithosphere. \emph{J. Geophys. Res.}, \emph{102}, 22731--22747.

\leavevmode\vadjust pre{\hypertarget{ref-bookman1972}{}}%
Bookman, C. A., Malone, I., \& Langseth, M. G. (1972). \emph{Sea floor geothermal measurements from conrad cruise 13} (No. 5-cu-5-72, Ntis Ad749983) (Vol. 5--Cu--5--72, Ntis Ad749983, p. --).

\leavevmode\vadjust pre{\hypertarget{ref-bookman1973}{}}%
Bookman, C. A., Malone, I., \& Langseth, M. G. (1973). \emph{Sea floor geothermal measurements from {Vema} cruise 26} (No. 7-cu-7-73).

\leavevmode\vadjust pre{\hypertarget{ref-borel1995}{}}%
Borel, R. A. (1995). \emph{Geothermics of the gypsy site, northcentral oklahoma} (Master's thesis).

\leavevmode\vadjust pre{\hypertarget{ref-bossolasco1965}{}}%
Bossolasco, L., \& Paulau, C. (1965). Il flusso geotermico sotto il {Monte Bianco}. \emph{Geof. E Meteorol.}, \emph{14}, 135--138.

\leavevmode\vadjust pre{\hypertarget{ref-bossolasco1967}{}}%
Bossolasco, M., \& Palau, C. (1967). Il flusso geotermico sotto il {Monte Bianco}. \emph{Geofis. Meteorol.}, \emph{14}, 135--138.

\leavevmode\vadjust pre{\hypertarget{ref-bott1972}{}}%
Bott, M. H. P., Johnson, G. A. L., Mansfield, J., \& Wheildon, J. (1972). Terrestrial heat flow in north-east {England}. \emph{Geophys. J. Roy. Astr. Soc.}, \emph{27}, 277--288.

\leavevmode\vadjust pre{\hypertarget{ref-boulos1987}{}}%
Boulos, F. K. (1987). Geothermal gradients inside water wells of east {Oweinat} area, south western dester of {Egypt}. \emph{Revista Brasileira de Geofisica}, \emph{5}, 165--172.

\leavevmode\vadjust pre{\hypertarget{ref-bowen1973}{}}%
Bowen, R. G. (1973). Geothermal activity in 1972. \emph{Ore Bin,} \emph{35}(1), 4--7.

\leavevmode\vadjust pre{\hypertarget{ref-bowen1977}{}}%
Bowen, R. G., Blackwell, D. D., \& Hull, D. A. (1977). \emph{Geothermal exploration studies in oregon} (No. 19) (Vol. 1977, p. 50p.).

\leavevmode\vadjust pre{\hypertarget{ref-bowin1980}{}}%
Bowin, C., Purdy, G. M., Johnston, C., Shor, G., Lawver, L., Hartono, H. M. S., \& Jezek, P. (1980). Arc-continent collision in {Banda Sea} region. \emph{AAPG Bull.}, \emph{64}, 868--915.

\leavevmode\vadjust pre{\hypertarget{ref-boyce1981}{}}%
Boyce, R. E. (1981). Electrical resistivity, sound velocity, thermal conductivity, density-porosity, and temperature, obtained by laboratory techniques and well logs: D site 462 in the {Naru Basin of hte Pacific Ocean}. \emph{Initial Reports DSDP}, \emph{61}, 849--853.

\leavevmode\vadjust pre{\hypertarget{ref-bram1979}{}}%
Bram, K. (1979). Heat flow measurements in the {Federal Republic of Germany}. In \emph{Terrestrial heat flow in europe} (pp. 191--196). Springer Verlag.

\leavevmode\vadjust pre{\hypertarget{ref-bram1980}{}}%
Bram, K. (1980). New heat flow observations on the reykjanes ridge. \emph{Journal of Geophysics}, \emph{47}, 86--90.

\leavevmode\vadjust pre{\hypertarget{ref-brewster1976}{}}%
Brewster, D., \& Pollack, H. N. (1976). Continued heat flow investigations in the {Michigan} basin deep borehole. \emph{EOS Trans. AGU}, \emph{57}, 760.

\leavevmode\vadjust pre{\hypertarget{ref-brigaud1985}{}}%
Brigaud, F., Lucazeau, F., Ly, S., \& Sauvage, J. F. (1985). Heat flow from the west african shield. \emph{Geophysical Research Letters}, \emph{12}(9), 549--552. \url{https://doi.org/10.1029/GL012i009p00549}

\leavevmode\vadjust pre{\hypertarget{ref-brock1989}{}}%
Brock, A. (1989). Heat flow measurements in ireland. \emph{Tectonophysics}, \emph{164}(2-4), 231--236. \url{https://doi.org/10.1016/0040-1951(89)90016-4}

\leavevmode\vadjust pre{\hypertarget{ref-brock1984}{}}%
Brock, A., \& Barton, K. J. (1984). \emph{Equilibrium temperature and heat flow density measurements in ireland} (No. AGV Report AGR 84-1).

\leavevmode\vadjust pre{\hypertarget{ref-brott1976}{}}%
Brott, C. A., Blackwell, D. D., \& Mitchell, J. C. (1976). \emph{Heat flow study of the snake river plain region, idaho. Geothermal investigations in idaho, water information bull. 30, part 8} (No. 30) (p. --). Idaho department of water resources.

\leavevmode\vadjust pre{\hypertarget{ref-brott1978}{}}%
Brott, C. A., Blackwell, D. D., \& Mitchell, J. C. (1978). Tectonic implications of the heat flow western {Snake River, Idaho}. \emph{Geol. Soc. Am. Bull.}, \emph{89}, 1697--1707.

\leavevmode\vadjust pre{\hypertarget{ref-brunnerova1975}{}}%
Brunnerova, Z., Skorepa, J., \& Simanek, V. (1975). Bituminous indications in the roblin RO-1 borehole in the barrandian, to the SW of prague. \emph{Vestnik U Str. Ust. Geol.}, \emph{50}, 217--229.

\leavevmode\vadjust pre{\hypertarget{ref-buachidze1980}{}}%
Buachidze, I. M., Buachidze, G. I., Goderzishvili, N. A., Mkheidze, B. S., \& Shaorshadze, M. P. (1980). Geotermicheskie usloviya i termalnye vody gruzii. Tbilisi, sabchota sakartvelo. 206 s. (russ.).

\leavevmode\vadjust pre{\hypertarget{ref-bucher1980}{}}%
Bucher, G. J. (1980). \emph{Heat flow and radioactivity studies in the {Ross Island--Dry Valley} area, {Antarctica}} (PhD thesis).

\leavevmode\vadjust pre{\hypertarget{ref-buecker2001}{}}%
Bücker, C. J., Jarrard, R. D., \& Wonik, T. (2001). Downhole temperature, radiogenic heat production, and heat flow from the {CRP-3} drillhole, {Victoria Land Basin, Antarctica}. \emph{Terra Antarctica}, \emph{8}, 151--159.

\leavevmode\vadjust pre{\hypertarget{ref-buffler1984}{}}%
Buffler, R. T. (1984). \emph{Initial Reports DSDP}, \emph{77}, 234--238.

\leavevmode\vadjust pre{\hypertarget{ref-bugge2002}{}}%
Bugge, T., Elvebakk, G., Fanavoll, S., Mangerud, G., Smelror, M., Weiss, H. M., et al. (2002). Shallow stratigraphic drilling applied in hydrocarbon exploration of the nordkapp basin, barents sea. \emph{Marine and Petroleum Geology}, \emph{19}(1), 13--37. \url{https://doi.org/10.1016/s0264-8172(01)00051-4}

\leavevmode\vadjust pre{\hypertarget{ref-bulashevich1983}{}}%
Bulashevich, Yu. P., \& Shchapov, V. A. (1983). Geotermicheskaya kharakteristika urala (russ.). \emph{Primenenie Geotermii V Regionalnykh I Poiskovo-Raz- Vedochnykh Issledovaniyakh. Svedrlovsk, Uralskii Nauchnyi Tsentr.}, 3--17.

\leavevmode\vadjust pre{\hypertarget{ref-bullard1939}{}}%
Bullard, E. C. (1939). Heat flow in {South Africa}. \emph{Proceeding of the Royal Society London Serie A}, \emph{173}, 474--502. \url{https://doi.org/10.1098/rspa.1939.0159}

\leavevmode\vadjust pre{\hypertarget{ref-bullard1954}{}}%
Bullard, E. C. (1954). The flow of heat through the floor of the {Atlantic Ocean}. \emph{Proceeding of the Royal Society London Serie A}, \emph{222}, 408--429. \url{https://doi.org/10.1098/rspa.1954.0085}

\leavevmode\vadjust pre{\hypertarget{ref-bullard1961}{}}%
Bullard, E. C., \& Day, A. (1961). The flow of heat throught the floor of the {Atlantic Ocean}. \emph{Geophys. J.}, \emph{4}, 282--292.

\leavevmode\vadjust pre{\hypertarget{ref-bullard1951}{}}%
Bullard, E. C., \& Niblett, E. R. (1951). Terrestrial heat flow in {England}. \emph{Mon. Not. R. Astr. Soc.}, \emph{4}, 309--312.

\leavevmode\vadjust pre{\hypertarget{ref-bullard1958}{}}%
Bullard, E. C., Maxwell, A. E., \& Revelle, R. (1958). Heat flow through the deep sea floor. \emph{Advances in Geophysics}, \emph{3}, 153--181.

\leavevmode\vadjust pre{\hypertarget{ref-burch1981}{}}%
Burch, T. K., \& Langseth, M. G. (1981). Heat flow determination in three DSDP boreholes near the japan trench. \emph{Journal of Geophysical Research}, \emph{86}, 9411--9419. \url{https://doi.org/10.1029/JB086iB10p09411}

\leavevmode\vadjust pre{\hypertarget{ref-burgassi1970}{}}%
Burgassi, P. D., Ceron, P., Ferara, G. S., Sestini, G., \& Toro, B. (1970). Geothermal gradient and heat flow in the radicofani region (east of monte amiata, italy). \emph{Geothermics}, \emph{sp.issue 2}(2), 443--449. \url{https://doi.org/10.1016/0375-6505(70)90042-8}

\leavevmode\vadjust pre{\hypertarget{ref-burgess1983}{}}%
Burgess, M. M. (1983). \emph{Summary of heat flow studies in the {Sohm} abyssal plain: {C.S.S. Hudson Cruise} 80-016}.

\leavevmode\vadjust pre{\hypertarget{ref-burkhardt1989}{}}%
Burkhardt, H., Haack, U., Hahn, A., Honarmand, H., Jäger, K., Stiefel, A., et al. (1989). Geothermal investigations at the {KTB locations Oberpfalz and Schwarzwald}. In \emph{The german continental deep drilling program KTB, site selection studies in the oberpfalz and schwarzwald} (pp. 433--480). Springer Verlag.

\leavevmode\vadjust pre{\hypertarget{ref-burns1964}{}}%
Burns, R. E. (1964). Sea bottom heat-flow measurements in the {Adaman Sea}. \emph{Journal of Geophysical Research}, \emph{69}, 4918--4919. \url{https://doi.org/10.1029/JZ069i022p04918}

\leavevmode\vadjust pre{\hypertarget{ref-burns1970}{}}%
Burns, R. E. (1970). Heat flow operations at holes 35.0 and 35.1. \emph{Initial Reports DSDP}, \emph{5}, 551--554.

\leavevmode\vadjust pre{\hypertarget{ref-burns1967}{}}%
Burns, R. E., \& Grim, P. J. (1967). Heat flow in the {Pacific Ocean} off central {California}. \emph{J. Geophys. Res.}, \emph{72}, 6239--6247.

\leavevmode\vadjust pre{\hypertarget{ref-burrus1986}{}}%
Burrus, J., \& Foucher, J. P. (1986). Contribution to the thermal regime of the provencal basin based on FLUMED heat flow surveys and previous investigations. \emph{Tectonophysics}, \emph{128}, 303--334. \url{https://doi.org/10.1016/0040-1951(86)90299-4}

\leavevmode\vadjust pre{\hypertarget{ref-cabal1995}{}}%
Cabal, J., \& Fernàndez, M. (1995). Heat flow and regional uplift at the north-eastern border of the {Ebro basin, NE Spain}. \emph{Geophys. J. Int.}, \emph{121}, 393--403.

\leavevmode\vadjust pre{\hypertarget{ref-camelo1987}{}}%
Camelo, S. M. L. (1987). Analysis of bottom---hole temperature and preliminary estimation of heat flow in {Portugese} sedimentary basins. \emph{Revista Brasileira de Geofisica}, \emph{5}, 139--142.

\leavevmode\vadjust pre{\hypertarget{ref-cande1987}{}}%
Cande, S. C., Leslie, R. B., Parra, J. C., \& Hobart, M. A. (1987). Interaction between the chile ridge and chile trench: Geophysical and geothermal evidence. \emph{Journal of Geophysical Research}, \emph{92}, 495--520. \url{https://doi.org/10.1029/JB092iB01p00495}

\leavevmode\vadjust pre{\hypertarget{ref-cardoso2014}{}}%
Cardoso, R. A., \& Hamza, V. M. (2014). Heat flow in the campos sedimentary basin and thermal history of the continental margin of southeast brazil. \emph{ISRN Geophysics}, \emph{2014}, 19 pp. \url{https://doi.org/10.1155/2014/384752}

\leavevmode\vadjust pre{\hypertarget{ref-carrier1979}{}}%
Carrier, D. L. (1979). \emph{Heat flow in twin peak} (Master's thesis).

\leavevmode\vadjust pre{\hypertarget{ref-carte1954}{}}%
Carte, A. E. (1954). Heat flow in the {Transvaal} and the {Orange Free State}. \emph{Proc. Phys. Soc. B.}, \emph{67}, 664--672. \url{https://doi.org/10.1088/0370-1301/67/9/302}

\leavevmode\vadjust pre{\hypertarget{ref-carte1969}{}}%
Carte, A. E., \& van Rooyen, A. I. M. (1969). Further measurements of heat flow in {South Africa}. In \emph{Proc. Nat. Upper mantle project symposium} (pp. 445--448).

\leavevmode\vadjust pre{\hypertarget{ref-carter1998}{}}%
Carter, L. S., Kelley, S. A., Blackwell, D. D., \& Naeser, N. D. (1998). Heat flow and thermal history of the {Anadarko Basin, Oklahoma}. \emph{AAPG Bull.}, \emph{82}, 291--316.

\leavevmode\vadjust pre{\hypertarget{ref-carvalho1977}{}}%
Carvalho, H. D. S., \& Vacquier, V. (1977). Method for determining terrestrial heat flow in oil fields. \emph{Geophysics}, \emph{42}(3(April)), 584--593. \url{https://doi.org/10.1190/1.1440729}

\leavevmode\vadjust pre{\hypertarget{ref-carvalho1980}{}}%
Carvalho, H. D. S., Purwoko, Siswoyo, Thamrin, M., \& Vacquier, V. (1980). Terrestrial heat flow in the {Tertiary} basin of central {Sumatra}. \emph{Tectonophysics}, \emph{69}, 163--188.

\leavevmode\vadjust pre{\hypertarget{ref-cermak1967b}{}}%
Cermak, V. (1967a). Heat flow in the {Kladno--Rakovnik} coal basin. \emph{Gerlands Beitrage Zur Geophysik}, \emph{76}, 461--466.

\leavevmode\vadjust pre{\hypertarget{ref-cermak1967a}{}}%
Cermak, V. (1967b). Heat flow near {Teplice} in northern {Bohemia}. \emph{Geophysical Journal of the Royal Astronomical Society}, \emph{13}, 547--549. \url{https://doi.org/10.1111/j.1365-246X.1967.tb02306.x}

\leavevmode\vadjust pre{\hypertarget{ref-cermak1968e}{}}%
Cermak, V. (1967c). Terrestrial heat flow in eastern {Slovakia}. \emph{Travaux in St. Geophys. Acad. Tchecosl. Sci.}, \emph{275}, 305--319.

\leavevmode\vadjust pre{\hypertarget{ref-cermak1968d}{}}%
Cermak, V. (1968a). Heat flow in the upper {Silesian} coal basin. \emph{Pure and Applied Geophysics}, \emph{69}, 119--130.

\leavevmode\vadjust pre{\hypertarget{ref-cermak1968b}{}}%
Cermak, V. (1968b). Heat flow in the {Zacler--Svatonovice} basin. \emph{Acta Geophys. Pol.,} \emph{16}, 3--9.

\leavevmode\vadjust pre{\hypertarget{ref-cermak1968c}{}}%
Cermak, V. (1968c). Terrestrial heat flow in {Czechoslovakia} and its relation to some geological features. \emph{Proc. 23rd Int. Geol. Congr., Praha}, \emph{5}, 75--85.

\leavevmode\vadjust pre{\hypertarget{ref-cermak1968a}{}}%
Cermak, V. (1968d). Terrestrial heat flow in the {Alpine-Carpathian} foredeep in {South Moravia}. \emph{J. Geophys. Res.}, \emph{73}, 820--821.

\leavevmode\vadjust pre{\hypertarget{ref-cermak1975a}{}}%
Cermak, V. (1975a). Combined heat flow and heat generation measurements in the bohemian massif. \emph{Geothermics}, \emph{4}(1-4), 19--26. \url{https://doi.org/10.1016/0375-6505(75)90005-x}

\leavevmode\vadjust pre{\hypertarget{ref-cermak1975b}{}}%
Cermak, V. (1975b). Terrestrial heat flow in the neogene foredeep and the flysch zone of the czechoslovak carpathians. \emph{Geothermics}, \emph{4}(1-4), 8--13. \url{https://doi.org/10.1016/0375-6505(75)90003-6}

\leavevmode\vadjust pre{\hypertarget{ref-cermak1976b}{}}%
Cermak, V. (1976a). High heat flow measured in the ostrava-karvinà coal basin. \emph{Stud. Geophys.et Geod.}, \emph{20}, 64--71.

\leavevmode\vadjust pre{\hypertarget{ref-cermak1976c}{}}%
Cermak, V. (1976b). Terrestrial heat flow in two deep holes in the ostrava-karvinà coal basin. \emph{Vestnìk Ústr. Úst.geol. (In Czech)}, \emph{51}, 75--84.

\leavevmode\vadjust pre{\hypertarget{ref-cermak1976a}{}}%
Cermak, V. (1976c). Zemskü tepelnü tok ve vrtu lidecko-1 v magurském flysi ve vnejsìch karpatech. \emph{Casop.miner.geol. (In Czech)}, \emph{21}, 193--198.

\leavevmode\vadjust pre{\hypertarget{ref-cermak1977}{}}%
Cermak, V. (1977a). Geothermal measurements in {Palaeogene, Cretaceous and Permocarboniferous} sediments in northern {Bohemia}. \emph{Geophys. J. Roy. Astr. Soc.}, \emph{148}, 537--541.

\leavevmode\vadjust pre{\hypertarget{ref-cermak1977a}{}}%
Cermak, V. (1977b). Heat flow measured in five holes in eastern and central slovakia. \emph{Earth and Planetary Science Letters}, \emph{34}, 67--70. \url{https://doi.org/10.1016/0012-821x(77)90106-6}

\leavevmode\vadjust pre{\hypertarget{ref-cermak1979}{}}%
Cermak, V. (1979). Tepelny tok v csr (in czech). In \emph{Mozn osti vyuziti zemskeho tepla suchych hornin v csr}. Ustr. Ust. Geol.

\leavevmode\vadjust pre{\hypertarget{ref-cermak1971}{}}%
Cermak, V., \& Jessop, A. M. (1971). Heat flow, heat generation and crustal temperatures in the {Kapuskasing} area of the {Canadian shield}. \emph{Tectonophysics}, \emph{11}, 287--303.

\leavevmode\vadjust pre{\hypertarget{ref-cermak1985}{}}%
Cermak, V., \& Jetel, J. (1985). Heat flow and ground water movement in the {Bohemian Cretaceous} basin {(Czechoslovakia)}. \emph{J. Geodynamics}, \emph{4}, 285--303.

\leavevmode\vadjust pre{\hypertarget{ref-cermak1967}{}}%
Cermak, V., \& Krcmar, B. (1967). Tepelny tok ve vrtu NV-1 (nova ves u ch ynova) (in czech). \emph{Vestnik Ustr. Ust. Geol.}, \emph{42}, 445--448.

\leavevmode\vadjust pre{\hypertarget{ref-cermak1967c}{}}%
Cermak, V., \& Krcmàr, B. (1967). Tepelny tok ve vrtu NV-1 (nova ves u ch ynova) (in czech). \emph{Vestnìk Ústr. Úst.geol. (In Czech)}, \emph{42}, 445--448.

\leavevmode\vadjust pre{\hypertarget{ref-cermak1968}{}}%
Cermak, V., \& Krcmàr, B. (1968). Merenì tepelného toku ve dvou sachtàch v zàpadnìch a jiznìch cechàch. \emph{Vestnìk Ústr. Úst.geol. (In Czech)}, \emph{43}, 415--422. \url{https://doi.org/10.1016/s0012-821x(68)80032-9}

\leavevmode\vadjust pre{\hypertarget{ref-cermak1982}{}}%
Cermak, V., \& Safanda, J. (1982). \emph{Mapa tepelneho toku na uzemi {Ceskoslovenska} (1:1 000 000) (in czech)} (pp. 20 pp). Zprava o cinnosti, Geophys. Inst. Praha.

\leavevmode\vadjust pre{\hypertarget{ref-cermak1991}{}}%
Cermak, V., Kral, M., Kubik, J., Safanda, J., Kresl, M., Kucerova, L., et al. (1991). Catalogue of heat flow density data: czechoslovakia. In \emph{Geothermal atlas of europe} (pp. 110--111). Hermann Haack Verlagsgesellschaft mbH.

\leavevmode\vadjust pre{\hypertarget{ref-cermak1996}{}}%
Cermak, V., Kresl, M., Kucerová, L., Safanda, J., Frasheri, A., Kapedani, N., et al. (1996). Heat flow in albania. \emph{Geothermics}, \emph{25}(1), 91--102. \url{https://doi.org/10.1016/0375-6505(95)00036-4}

\leavevmode\vadjust pre{\hypertarget{ref-cermak1984}{}}%
Čermák, V., Krešl, M., Šafanda, J., Nápoles-Pruna, M., Tenreyro-Perez, R., Torres-Paz, L. M., \& Valdés, J. J. (1984). First heat flow density assessments in cuba. \emph{Tectonophysics}, \emph{103}(1--4), 283--296. \url{https://doi.org/10.1016/0040-1951(84)90090-8}

\leavevmode\vadjust pre{\hypertarget{ref-cermak1991b}{}}%
Čermák, V., Krešl, M., Šafanda, J., Bodri, L., Nápoles-Pruna, M., \& Tenreyro-Perez, R. (1991). Terrestrial heat flow in cuba. \emph{Physics of the Earth and Planetary Interior}, \emph{65}, 207--209. \url{https://doi.org/10.1016/0031-9201(91)90128-5}

\leavevmode\vadjust pre{\hypertarget{ref-chadwick1956}{}}%
Chadwick, P. (1956). Heat flow from the {Earth} at {Cambridge}. \emph{Nature}, \emph{178}, 105--106. \url{https://doi.org/10.1038/178105a0}

\leavevmode\vadjust pre{\hypertarget{ref-chapman1974}{}}%
Chapman, D. S., \& Pollack, H. N. (1974). Cold spot in west {Africa}---anchoring the {African Plate}. \emph{Nature}, \emph{250}, 477--478.

\leavevmode\vadjust pre{\hypertarget{ref-chapman1977}{}}%
Chapman, D. S., \& Pollack, H. N. (1977). Heat flow and heat production in zambia: Evidence for lithospheric thinning in central africa. \emph{Tectonophysics}, \emph{41}, 79--100. \url{https://doi.org/10.1016/0040-1951(77)90181-0}

\leavevmode\vadjust pre{\hypertarget{ref-chapman1978}{}}%
Chapman, D. S., Blackwell, D. D., Parry, W. T., Sill, W. R., Ward, S. H., \& Whelan, J. A. (1978). \emph{Regional heat flow and geochemical studies in southwest utah} (No. 14-08-0001-g-341) (p. 120p.). University of Utah, Department of Geology; Geophysics, Final Report, v. 2, contract no. 14-08-0001-G-341.

\leavevmode\vadjust pre{\hypertarget{ref-chapman1981}{}}%
Chapman, D. S., Clement, M. D., \& Mase, C. W. (1981). Thermal regime of the escalante desert, utah, with an analysis of the newcastle geothermal system. \emph{Journal of Geophysical Research}, \emph{86}, 11735--11746.

\leavevmode\vadjust pre{\hypertarget{ref-chen1991a}{}}%
Chen, M.-X., \& Xia, S.-G. (1991). Geothermal study in the leizhou panisulase china (in chinese). \emph{Scientia Geologica Sinica}, \emph{4}, 369--383.

\leavevmode\vadjust pre{\hypertarget{ref-cheremenskii1979}{}}%
Cheremenskii, G. A. (1979). Vliyanie treshchinovatosti v fundamente na plotnost teplovogo potoka na yugo-vostochnoi okraine baltiiskogo shchita (russ.). \emph{Sovetskaya Geologiya}, \emph{9}, 90--95.

\leavevmode\vadjust pre{\hypertarget{ref-choi1990}{}}%
Choi, D. R., Liu, Y. S. B., \& Cull, J. P. (1990). Heat flow and sediment thickness in the {Queensland Trough, western Coral Sea}. \emph{J. Geophys. Res.}, \emph{95}, 21399--21411.

\leavevmode\vadjust pre{\hypertarget{ref-chukwueke1987}{}}%
Chukwueke, C. (1987). \emph{Mesure du flux de chaleur à ririwai, delta du niger (nigéria)} (PhD thesis).

\leavevmode\vadjust pre{\hypertarget{ref-chukwueke1990}{}}%
Chukwueke, C. (1990). Notes on heat flow at {Ririwai, Nigeria}. \emph{J. Afr. Earth Sci.}, \emph{10}, 503--507.

\leavevmode\vadjust pre{\hypertarget{ref-chukwueke1992}{}}%
Chukwueke, C., Thomas, G., \& Delfaud, J. (1992). Sedimentary processes, eustatism, subsidence and heat flow in the distal part of the niger delta. \emph{Bulletin Des Centres de Recherches Exploration-Production}, \emph{16}, 137--186.

\leavevmode\vadjust pre{\hypertarget{ref-chung1969}{}}%
Chung, Y., Bell, M. L., Sclater, J. G., \& Corry, C. (1969). \emph{Temperature data from the {Pacific Abyssal Water}} (No. Ref. 69-17) (Vol. 69--17, p. --). Scripps Inst. Oceangr.

\leavevmode\vadjust pre{\hypertarget{ref-clark1957}{}}%
Clark Jr., S. P. (1957). Heat flow at {Grass Valley, California}. \emph{Trans. Am. Geophys. Union}, \emph{38}, 239--244.

\leavevmode\vadjust pre{\hypertarget{ref-clark1961}{}}%
Clark Jr., S. P. (1961). Heat flow in the {Austrian Alps}. \emph{Geophysical Journal of the Royal Astronomy Society}, \emph{6}, 54--63. \url{https://doi.org/10.1111/j.1365-246X.1961.tb02961.x}

\leavevmode\vadjust pre{\hypertarget{ref-clark1956}{}}%
Clark, S. P., \& Niblett, E. R. (1956). Terrestrial heat flow in the {Swiss Alps}. \emph{Mon. Not. R. Astr. Soc. Geophys. Suppl.}, \emph{7}, 176--195.

\leavevmode\vadjust pre{\hypertarget{ref-clark1978}{}}%
Clark, T. F., Korgen, B. J., \& Best, D. M. (1978). Heat flow measurements made in a traverse across the eastern {Carribean}. \emph{J. Geophys. Res.}, \emph{83}, 5883--5898.

\leavevmode\vadjust pre{\hypertarget{ref-clement1980}{}}%
Clement, M. D. (1980). \emph{Heat flow in escalante desert} (Master's thesis).

\leavevmode\vadjust pre{\hypertarget{ref-cochran1981}{}}%
Cochran, J. R. (1981). Simple models of diffuse extension and the pre-seafloor spreading development of the continental margin of the northeastern gulf of aden. \emph{Oceanologica Acta}, \emph{sp.}, 155--165.

\leavevmode\vadjust pre{\hypertarget{ref-coleno1986}{}}%
Coleno, B. (1986). \emph{Diagraphie thermique et distribution du champ de température dans le bassin de paris} (PhD thesis).

\leavevmode\vadjust pre{\hypertarget{ref-collette1968}{}}%
Collette, R. J., Lagaay, R. A., Van Lenner, A. P., Schouten, J. A., \& Schiling, R. D. (1968). Some heat-flow measurements in the {North Atlantic Ocean}. \emph{{Nederlandse Akademie van Wetenschappen, Amsterdam}, Proc. Sect. Sci. Series B, Phys. Sci.}, \emph{71}, 203--208.

\leavevmode\vadjust pre{\hypertarget{ref-collins1985}{}}%
Collins, W. H. (1985). \emph{Thermal anomalies in the mississippi embayent and tectonic implications} (Master's thesis).

\leavevmode\vadjust pre{\hypertarget{ref-combs1970}{}}%
Combs, J. B. (1970). \emph{Terrestrial heat flow in north central united states} (PhD thesis).

\leavevmode\vadjust pre{\hypertarget{ref-combs1971}{}}%
Combs, J. B. (1971). Heat flow and geothermal resource estimates for the {Imperial Valley}. In \emph{Cooperative geological--geophysical--geochemical investigations of geothermal resources in the {Imperial Valley Area of California, Riverside}} (pp. 5--27). Education Research Service.

\leavevmode\vadjust pre{\hypertarget{ref-combs1980}{}}%
Combs, J. B. (1980). Heat flow in the coso geothermal area, inyo county, california. \emph{Journal of Geophysical Research}, \emph{85}, 2411--2424. \url{https://doi.org/10.1029/JB085iB05p02411}

\leavevmode\vadjust pre{\hypertarget{ref-combs1973}{}}%
Combs, J. B., \& Simmons, G. (1973). Terrestrial heat flow in the north central united states. \emph{Journal of Geophysical Research}, \emph{78}, 441--461. \url{https://doi.org/10.1029/JB078i002p00441}

\leavevmode\vadjust pre{\hypertarget{ref-sgc}{}}%
Company, S. G. (n.d.). \emph{Compilation}.

\leavevmode\vadjust pre{\hypertarget{ref-corry1990}{}}%
Corry, C. E., Herrin, E., McDowell, F. W., \& Phillips, K. A. (1990). \emph{Geology of the solitario, trans-pecos, texas}. Geol. Soc. Am.

\leavevmode\vadjust pre{\hypertarget{ref-corry1998}{}}%
Corry, D., \& Brown, C. (1998). Temperature and heat flow in the {Celtic Sea} basins. \emph{Petroleum Geoscience}, \emph{4}, 317--326.

\leavevmode\vadjust pre{\hypertarget{ref-costain1987}{}}%
Costain, J. K., \& Decker, E. R. (1987). Heat flow at the proposed ultradeep core hole (ADCOH) site: Tectonic implications. \emph{Geophysical Research Letters}, \emph{14}, 252--255.

\leavevmode\vadjust pre{\hypertarget{ref-costain1973}{}}%
Costain, J. K., \& Wright, P. M. (1973). Heat flow at spor mountain, jordan valley, bingham, and la sal, utah. \emph{Journal of Geophysical Research}, \emph{78}(b5), 8687--8698. \url{https://doi.org/10.1029/JB078i035p08687}

\leavevmode\vadjust pre{\hypertarget{ref-costain1986}{}}%
Costain, J. K., Speer, J. A., Glover, L., Perry, L. D., Dashevsky, S., \& McKinney, M. (1986). Heat flow in the piedmont and atlantic coastal plain of the southeastern united states. \emph{Journal of Geophysical Research}, \emph{91}(b2), 2123--2135. \url{https://doi.org/10.1029/JB091iB02p02123}

\leavevmode\vadjust pre{\hypertarget{ref-coster1947}{}}%
Coster, H. P. (1947). Terrestrial heat flow in {Persia}. \emph{Mon. Not. R. Astr. Soc. Geophys. Suppl.}, \emph{5}, 131--145. \url{https://doi.org/10.1111/j.1365-246X.1947.tb00349.x}

\leavevmode\vadjust pre{\hypertarget{ref-courtney1986}{}}%
Courtney, R. C., \& Recq, M. (1986). Anomalous heat flow near the {Crozet Plateau} and mantle convection. \emph{Earth Planet. Sci. Lett.}, \emph{79}, 373.

\leavevmode\vadjust pre{\hypertarget{ref-crane1982}{}}%
Crane, K., Eldholm, O., Myhre, A. M., \& Sundvor, E. (1982). Thermal implications for the evolution of the spitsbergen transform fault. \emph{Tectonophysics}, \emph{89}(1-3), 1--32.

\leavevmode\vadjust pre{\hypertarget{ref-crane1988}{}}%
Crane, K., Sundvor, E., Foucher, J. P., Hobart, M. A., Myhre, A. M., \& Le Douaran, S. (1988). Thermal evolution of the western svalbard. \emph{Marine Geophysical Research}, \emph{9}(2), 165--194. \url{https://doi.org/10.1007/bf00369247}

\leavevmode\vadjust pre{\hypertarget{ref-crane1991}{}}%
Crane, K., Sundvor, E., Buck, R., \& Martinez, F. (1991). Riftin in the northern {Norwegian--Greenland Sea}: Thermal tests of asymmetric rifing. \emph{J. Geophys. Res.}, \emph{96}, 14529--14550.

\leavevmode\vadjust pre{\hypertarget{ref-cranganu1998}{}}%
Cranganu, C., Lee, Y., \& Deming, D. (1998). Heat flow in {Oklahoma} and the south central {United States}. \emph{J. Geophys. Res.}, \emph{103}, 27107--27121.

\leavevmode\vadjust pre{\hypertarget{ref-creutzburg1964}{}}%
Creutzburg, H. (1964). Untersuchungen über den {Wärmestrom} der {Erde in Westdeutchland}. \emph{Kali Steinsalz}, \emph{4}, 73--108.

\leavevmode\vadjust pre{\hypertarget{ref-crowe1981}{}}%
Crowe, J. (1981). \emph{Mechanisms of heat transport through the floor of the equatorial pacific ocean} (PhD thesis). \url{https://doi.org/10.1575/1912/3214}

\leavevmode\vadjust pre{\hypertarget{ref-cui2004}{}}%
Cui, J.-P. (2004). Study on the thermal evolution and reservoir history in hailar basin. \emph{Xi'an Northwestern University}.

\leavevmode\vadjust pre{\hypertarget{ref-cull1980}{}}%
Cull, J. P. (1980). Geothermal records of climatic change in new south wales. \emph{Search}, \emph{11}, 201--203.

\leavevmode\vadjust pre{\hypertarget{ref-cull1982}{}}%
Cull, J. P. (1982). An appraisal of australian heat-flow data. \emph{BMR Journal of Australian Geology and Geophysics}, \emph{7}(1), 11--21.

\leavevmode\vadjust pre{\hypertarget{ref-cull1991}{}}%
Cull, J. P. (1991). In \emph{Terrestrial heat flow and lithospheric structure}. Springer-Verlag.

\leavevmode\vadjust pre{\hypertarget{ref-cull1979}{}}%
Cull, J. P., \& Denham, D. (1979). Regional variations in australian heat flow. \emph{Bureau of Mineral Resources Journal of Australian Geology and Geophysics}, \emph{4}, 1--13.

\leavevmode\vadjust pre{\hypertarget{ref-dahl1998}{}}%
Dahl-Jensen, D., Mosegaard, K., Gundestrup, N., Clow, G. D., Johnsen, S. J., Hansen, A. W., \& Balling, N. (1998). Past temperatures directly from the greenland ice sheet. \emph{Science}, \emph{282}(5387), 268--271. \url{https://doi.org/10.1126/science.282.5387.268}

\leavevmode\vadjust pre{\hypertarget{ref-daignieres1979}{}}%
Daignières, M., \& Vasseur, G. (1979). Détermination et interprétation du flux géotermique à bournac, haute loire. \emph{Annales Géophysiques}, \emph{35}(1), 31--39.

\leavevmode\vadjust pre{\hypertarget{ref-dao1995}{}}%
Dao, D. V., \& Huyen, T. (1995). Heat flow in the oil basins of vietnam. \emph{CCOP Tech. Bull.}, \emph{25}, 55--61. Retrieved from \url{http://www.gsj.jp/en/publications/ccop-bull/ccop-vol25.html}

\leavevmode\vadjust pre{\hypertarget{ref-davis1984}{}}%
Davis, E. E., \& Lewis, T. J. (1984). Heat flow in a back-arc environment: Intermontane and omineca crystalline belts, southern canadian cordillera. \emph{Canadian Journal of Earth Sciences}, \emph{21}, 715--726.

\leavevmode\vadjust pre{\hypertarget{ref-davis1977}{}}%
Davis, E. E., \& Lister, C. R. B. (1977). Heat flow measured over the juan de fuca ridge: D evidence for widespread hydrothermal circulation in a highly heat transportive crust. \emph{Journal of Geophysical Research}, \emph{82}, 4845--4860. \url{https://doi.org/10.1029/JB082i030p04845}

\leavevmode\vadjust pre{\hypertarget{ref-davis1982}{}}%
Davis, E. E., \& Riddihough, R. P. (1982). The {Winona Basin}: Structure and tectonics. \emph{Can. J Earth Sci.}, \emph{19}, 767--788.

\leavevmode\vadjust pre{\hypertarget{ref-davis1991}{}}%
Davis, E. E., \& Villinger, H. (1991). Tectonic and thermal structure of the {Middle Valley} sedimented rift, northern {Juan de Fuca Ridge}. \emph{Proc. ODP Initial Reports}, \emph{139}.

\leavevmode\vadjust pre{\hypertarget{ref-davis1980}{}}%
Davis, E. E., Lister, C. R. B., Wade, U. S., \& Hyndman, R. D. (1980). Detailed heat flow measurements over the juan de fuca ridge system. \emph{Journal of Geophysical Research}, \emph{85}, 299--310. \url{https://doi.org/10.1029/JB085iB01p00299}

\leavevmode\vadjust pre{\hypertarget{ref-davis1990}{}}%
Davis, E. E., Hyndman, R. D., \& Villinger, H. (1990). Rates of fluid expulsion across the {Northern Cascadia} accretionary prism: Constraints from new heat flow and multichannel seismic reflection data. \emph{J. Geophys. Res.}, \emph{95}, 8869--8889.

\leavevmode\vadjust pre{\hypertarget{ref-davis1992}{}}%
Davis, E. E., Chapman, N. R., Mottl, M. J., Bentkowski, W. J., Dadey, K., Forster, C. B., et al. (1992). Flankflux: An experiment to study the nature of hydrothermal circulation in young oceanic crust. \emph{Canadian Journal of Earth Sciences}, \emph{29}(5), 925--952. \url{https://doi.org/10.1139/e92-078}

\leavevmode\vadjust pre{\hypertarget{ref-davis1997a}{}}%
Davis, E. E., Fisher, A. T., Firth, J. V., \& Shipboard Scientific Party. (1997a). 1. Introduction and summary: Hydrothermal circulation in the ocean crust and its consequences on the eastern flank of the {Juan de Fuca} ridge. \emph{Proc. Ocean Drilling Prog., Init. Rep.}, \emph{168}, 7--21.

\leavevmode\vadjust pre{\hypertarget{ref-davis1997}{}}%
Davis, E. E., Chapman, D. S., Villinger, H., Robinson, S., Grigel, J., Rosenberger, A., \& Pribnow, D. (1997b). Seafloor heat flow on the eastern flank of the {Juan de Fuca} ridge: Data from {``FLANKFLUX''} studies through 1995. \emph{Proc. Ocean Drilling Prog. Init. Rep.}, \emph{168}, 23--33.

\leavevmode\vadjust pre{\hypertarget{ref-davis2003}{}}%
Davis, E. E., Wang, K., Becker, K., Thompson, R. E., \& Yashayaev, I. (2003). Deep-ocean temperature variations and implications for errors in seafloor heat flow determinations. \emph{J. Geophys. Res.}, \emph{108}, doi:10.1029/2001JB001695.

\leavevmode\vadjust pre{\hypertarget{ref-davis2004}{}}%
Davis, E. E., Becker, K., \& He, J. (2004). Costa rica rift revisited: Constraints on shallow and deep hydrothermal circulation in young oceanic crust. \emph{Earth and Planetary Science Letters}, \emph{222}(3-4), 863--879. \url{https://doi.org/10.1016/j.epsl.2004.03.032}

\leavevmode\vadjust pre{\hypertarget{ref-derito1989}{}}%
De Rito, R. F., Lachenbruch, A. H., Moses, T. H., \& Munroe, R. J. (1989). Heat flow and thermotectonic problems of the central ventura basin, southern california. \emph{Journal of Geophysical Research}, \emph{94}(b1), 681--699. \url{https://doi.org/10.1029/JB094iB01p00681}

\leavevmode\vadjust pre{\hypertarget{ref-decker1988}{}}%
Decker, E., Heasler, H. P., Buelow, K. L., Baker, K. H., \& Hallin, J. S. (1988). Significance of past and recent heat-flow and radioactivity studies in the southern {Rocky Mountains} region. \emph{Geol. Soc. Am. Bull.}, \emph{100}, 1971--1980.

\leavevmode\vadjust pre{\hypertarget{ref-decker1969}{}}%
Decker, E. R. (1969). Heat flow in {Colorado and New Mexico}. \emph{Journal of Geophysical Research}, \emph{74}, 550--559. \url{https://doi.org/10.1029/JB074i002p00550}

\leavevmode\vadjust pre{\hypertarget{ref-decker1987}{}}%
Decker, E. R. (1987). Heat flow and basement radioactivity in {Maine}: First-order results and preliminary interpretations. \emph{Geophys. Res. Lett.}, \emph{14}, 256--259.

\leavevmode\vadjust pre{\hypertarget{ref-decker1974}{}}%
Decker, E. R., \& Birch, F. S. (1974). \emph{Basic heat flow data from colorado, minnesota, new mexico and texas} (No. 74--79) (p. --). U.S. Geol. Surv. open-file report.

\leavevmode\vadjust pre{\hypertarget{ref-decker1979}{}}%
Decker, E. R., \& Bucher, G. J. (1979). \emph{Thermal gradients and heat flow data in {Colorado and Wyoming}}.

\leavevmode\vadjust pre{\hypertarget{ref-decker1983}{}}%
Decker, E. R., \& Bucher, G. J. (1983). Geothermal studies in the {Ross Island--Dry Valley} region. In \emph{Antarctic geoscience} (pp. 887--894). Univ. Wisconsin.

\leavevmode\vadjust pre{\hypertarget{ref-decker1975}{}}%
Decker, E. R., \& Smithson, S. B. (1975). Heat flow and gravity interpretation across the rio grande rift in southern new mexico and west texas. \emph{Journal of Geophysical Research}, \emph{80}, 2542--2552. \url{https://doi.org/10.1029/JB080i017p02542}

\leavevmode\vadjust pre{\hypertarget{ref-decker1980}{}}%
Decker, E. R., Baker, K. R., Bucher, G. J., \& Heasler, H. P. (1980). Preliminary heat flow and radioactivity studies in wyoming. \emph{Journal of Geophysical Research}, \emph{85}, 311--321. \url{https://doi.org/10.1029/JB085iB01p00311}

\leavevmode\vadjust pre{\hypertarget{ref-degens1971}{}}%
Degens, E. T., Von Herzen, R. P., \& Wong, H. (1971). Lake {Tanganyika}---water chemistry, sediments, geological structure. \emph{Naturwissenschaeten}, \emph{58}, 229--240.

\leavevmode\vadjust pre{\hypertarget{ref-degens1973}{}}%
Degens, E. T., Von Herzen, R. P., Wong, H.-K., Deuser, W. G., \& Jannasch, H. W. (1973). Lake kivu --- structure, chemistry, and biology of an {East African Rift} lake. \emph{Geol. Rundshau}, \emph{62}, 245--277.

\leavevmode\vadjust pre{\hypertarget{ref-delisle1994}{}}%
Delisle, G. (1994). Measurement of terrestrial heat flow in glaciated terrain. \emph{Terra Antarctica}, \emph{1}, 527--528.

\leavevmode\vadjust pre{\hypertarget{ref-delisle2011}{}}%
Delisle, G. (2011). Positive geothermal anomalies in oceanic crust of cretaceous age offshore kamchatka. \emph{Solid Earth}, \emph{2}(2), 191--198. \url{https://doi.org/10.5194/se-2-191-2011}

\leavevmode\vadjust pre{\hypertarget{ref-delisle2002}{}}%
Delisle, G., \& Ladage, S. (2002). New heat flow data from the chilean coast between ~36\textdegree{} and 40\textdegree{}. In \emph{Final report SO-161 leg 2, 3 \& 5 SPOC subduction processes off chile} (pp. 1--13). Bundestalt für Geoweissenschaften und Rohstoffe (BGR).

\leavevmode\vadjust pre{\hypertarget{ref-delisle2007}{}}%
Delisle, G., \& Zeibig, M. (2007). Marine heat flow measurements in hard ground offshore sumatra. \emph{EOS Trans. AGU}, \emph{88}(4), 38--39. \url{https://doi.org/10.1029/2007eo040004}

\leavevmode\vadjust pre{\hypertarget{ref-delisle1998}{}}%
Delisle, G., Beiersdorf, H., Neben, S., \& Steinmann, D. (1998). The geothermal field of the north sulawesi accretionary wedge and a model on BSR migration in unstable depositional environments. In \emph{Gas hydrates: Relevance to world margin stability and climate change} (Vol. 137, pp. 267--274). Geological Society of London.

\leavevmode\vadjust pre{\hypertarget{ref-dellavedova1979}{}}%
Della Vedova, B., \& Pellis, G. (1979). Risultati delle misure di flusso di calore esequite nel tirreno sud-orientale. In \emph{Atti del convegno scientifico nazionale, p.f. Oceanografia e fondi marini, roma, 5--7 march 1979} (pp. 693--712).

\leavevmode\vadjust pre{\hypertarget{ref-dellavedova1983}{}}%
Della Vedova, B., \& Pellis, G. (1983). \emph{Dati di flusso di calore nei mari italiani} (p. --). Cnr.

\leavevmode\vadjust pre{\hypertarget{ref-dellavedova1987a}{}}%
Della Vedova, B., \& Pellis, G. (1987). Risulti delle misure di flusso di calore nel mare di sardegna. In \emph{Atti del 5\textdegree{} convegno} (Vol. Ii, pp. 1141--1155). Consiglio Nazionale delle Ricerche.

\leavevmode\vadjust pre{\hypertarget{ref-dellavedova1984}{}}%
Della Vedova, B., Pellis, G., Foucher, J. P., \& Rehault, J. P. (1984). Geothermal structure of the tyrrhenian sea. \emph{Marine Geology}, \emph{55}, 271--289. \url{https://doi.org/10.1016/0025-3227(84)90072-0}

\leavevmode\vadjust pre{\hypertarget{ref-dellavedova1992}{}}%
Della Vedova, B., Pellis, G., Lawver, L. A., \& Brancolini, G. (1992). Heat flow and tectonics of the western ross sea. In \emph{Recent progress in antarctic earth science} (pp. 627--637). Terra Sci. Pub. Co.

\leavevmode\vadjust pre{\hypertarget{ref-demetrescu1979}{}}%
Demetrescu, C. (1979). Heat flow values for some tectonic units in {Romania}. \emph{St. Cerc. Geol, Geofiz., Geogr., Geofizica}, \emph{17}, 35--46.

\leavevmode\vadjust pre{\hypertarget{ref-demetrescu1981a}{}}%
Demetrescu, C., Ene, M., \& Andreescu, M. (1981a). Geothermal profile in the {Central Moesian Platform}. \emph{Stud. Cercet. Fiz.}, \emph{33}, 1015--1021.

\leavevmode\vadjust pre{\hypertarget{ref-demetrescu1981}{}}%
Demetrescu, C., Ene, M., \& Andreescu, M. (1981b). On the geothermal regime of the transylvanian depression. \emph{St. Cerc. Geol., Geofiz., Geogr., Geofizica,} \emph{19}(6), 11--71.

\leavevmode\vadjust pre{\hypertarget{ref-demetrescu1983}{}}%
Demetrescu, C., Ene, M., \& Andreescu, M. (1983). New heat flow data for the {Romanian Territory}. \emph{An. Inst. Geol. Geophys.}, \emph{63}, 45--57.

\leavevmode\vadjust pre{\hypertarget{ref-demetrescu1991}{}}%
Demetrescu, C., Veliciu, S., \& Burst, A. D. (1991). Catalogue of heat flow density data: romania. In \emph{Geothermal atlas of europe} (pp. 123--124). Hermann Haack Verlagsgesellschaft mbH.

\leavevmode\vadjust pre{\hypertarget{ref-demetrescu2001}{}}%
Demetrescu, C., Nielsen, S. B., Ene, M., Serban, D. Z., Polonic, G., Andreescu, M., et al. (2001). Lithosphere thermal structure and evolution of the {Transylvanian Depression} --- insights from geothermal measurements and modelling results. \emph{Phys. Earth Planet. Int.}, \emph{126}, 249--267.

\leavevmode\vadjust pre{\hypertarget{ref-demetrescu2007}{}}%
Demetrescu, C., Wilhelm, H., Tumanian, M., Nielson, S. B., Damian, A., Dobrica, V., \& M. Ene, M. (2007). Time-dependent thermal state of the lithosphere in the foreland of the eastern {Carpathians} bend. Insights from new geothermal measurements and modelling results. \emph{Geophys. J. Int.}, \emph{170}, 896--912.

\leavevmode\vadjust pre{\hypertarget{ref-deming1988}{}}%
Deming, D., \& Chapman, D. S. (1988). Heat flow in the {Utah--Wyoming} thrust belt from analysis of bottom-hole temperature data measured in oil and gas wells. \emph{J. Geophys. Res.}, \emph{93}, 13657--13672.

\leavevmode\vadjust pre{\hypertarget{ref-deming1992}{}}%
Deming, D., Sass, J. H., Lachenbruch, A. H., \& De Rito, R. F. (1992). Heat flow and subsurface temperature as evidence for basin-scale ground-water flow {North Slope of Alaska}. \emph{Geol. Soc. Am. Bull.}, \emph{104}, 528--542.

\leavevmode\vadjust pre{\hypertarget{ref-detrick1986}{}}%
Detrick, R. S., Von Herzen, R. P., Parsons, B., Sandwell, D., \& Doughrty, M. (1986). Heat flow observations on the {Bermuda Rise} and thermal models of mid-plate swells: \emph{J. Geophys. Res.}, \emph{91}, 3701--3723.

\leavevmode\vadjust pre{\hypertarget{ref-deviatkin1973}{}}%
Deviatkin, V. N. (1973). Metodika izucheniya geotermicheskikh parametrov v oblasti rasprostraneniya mnogoletnemerzlykh (russ.). \emph{Porod. - Moskva}, 17 pp.

\leavevmode\vadjust pre{\hypertarget{ref-deviatkin1975}{}}%
Deviatkin, V. N. (1975). Rezultaty opredeleniya glubinnogo teplovogo potoka na territorii jakutii (russ.). \emph{Regionalnye I Tematicheskie Geokriolo- Gicheskie Issledovaniya. Novosibirsk Nauka}, 148--150.

\leavevmode\vadjust pre{\hypertarget{ref-deviatkin1981}{}}%
Deviatkin, V. N. (1981). Geotermicheskie usloviya basseinov rek kurungyuruakh i hatat (zapadnaya jakutiya) (russ.). \emph{Stroenie I Teplovoy Rezhim Merzlykh Porod. Novosibirsk: Naua}, 78--80.

\leavevmode\vadjust pre{\hypertarget{ref-deviatkin1982}{}}%
Deviatkin, V. N. (1982). O geotermicheskoi anomalii leno-ust-vilyuiskogo gazonosnogo raiona (russ.). \emph{Termika Pochv I Gornykh Porod V Kholodnykh Regionakh. Jakutsk: Institut Merzlotovedeniya so an SSSR}, 111--117.

\leavevmode\vadjust pre{\hypertarget{ref-deviatkin1981a}{}}%
Deviatkin, V. N., \& Gavriliev, R. I. (1981). Geotermiya vmeshchayushchikh porod ka- riera mir (zapadnaya jakutiya) (russ.). \emph{Stroenie I Teplovoy Rezhim Merzlykh Porod. Novosibirsk: Nauka}, 76--78.

\leavevmode\vadjust pre{\hypertarget{ref-deviatkin1982a}{}}%
Deviatkin, V. N., \& Rusakov, V. G. (1982). Geotermicheskie parametry v predelakh yugo-vostoka sibirskoi platformy (russ.). \emph{Termika Pochv I Gornykh Porod V Kholodnykh Regionakh. Jakutsk: Institut Merzlotovedeniya so an SSSR}, 117--122.

\leavevmode\vadjust pre{\hypertarget{ref-deviatkin1978}{}}%
Deviatkin, V. N., \& Shamshurin, V. Y. (1978). Geotermicheskaya kharakteristika mestorozhdeniya sytykan (russ.). \emph{Geoteplofizicheskie Issledovaniya V Sibiri. Novosibirsk: Nauka}, 142--148.

\leavevmode\vadjust pre{\hypertarget{ref-deviatkin1980}{}}%
Deviatkin, V. N., \& Shamshurin, V. Yu. (1980). Geotermicheskie usloviya kimberlitovoi trubki yubileynaya (russ.). \emph{Merzlotnye Issledovaniya v Osvaivaemykh Regionakh SSSR. Novosibirsk: Nauka}, 79--82.

\leavevmode\vadjust pre{\hypertarget{ref-deville2006}{}}%
Deville, E., Guerlais, S.-H., Callec, Y., Griboulard, R., Huyghe, P., Lallemant, S., et al. (2006). Liquefied vs stratified sediment mobilization processes: Insight from the south of the barbados accretionary prism. \emph{Tectonophysics}, \emph{428}(1--4), 33--47. \url{https://doi.org/10.1016/j.tecto.2006.08.011}

\leavevmode\vadjust pre{\hypertarget{ref-diment1963}{}}%
Diment, W. H., \& Robertson, E. C. (1963). Temperature, thermal conductivity, and heat flow in a drilled hole near {Oak Ridge, Tennessee}. \emph{J. Geophys. Res.}, \emph{68}, 5035--5047.

\leavevmode\vadjust pre{\hypertarget{ref-diment1964}{}}%
Diment, W. H., \& Werre, R. W. (1964). Terrestrial heat flow near {Washington, D.C.} \emph{J. Geophys. Res.}, \emph{69}, 2143--2149.

\leavevmode\vadjust pre{\hypertarget{ref-diment1965}{}}%
Diment, W. H., Marine, I. W., Neiheisel, J., \& Siple, G. E. (1965a). Subsurface temperature, thermal conductivity, and heat flow near {Aiken, South Carolina}. \emph{J. Geophys. Res.}, \emph{70}, 5635--5644.

\leavevmode\vadjust pre{\hypertarget{ref-diment1965b}{}}%
Diment, W. H., Raspet, R., Mayhew, M. A., \& Werre, R. W. (1965b). Terrestrial heat flow near {Alberta, Virginia}. \emph{J. Geophys. Res.}, \emph{70}, 923--929.

\leavevmode\vadjust pre{\hypertarget{ref-dorofeeva1983}{}}%
Dorofeeva, R. P. (1983). Rezultaty izucheniya teplofizicheskikh svoistv gornykh porod dlya tselei geologicheskogo kartirovaniya. - v kn.: Primenenie geotermii v regionalnykh i poiskovo-razvedochnykh issledovaniyakh.

\leavevmode\vadjust pre{\hypertarget{ref-dorofeeva1992}{}}%
Dorofeeva, R. P. (1992). Geothermal studies in siberia and mongolia. \emph{Proc. 14th New Zealand Geothrmal Workshop}, 237--240.

\leavevmode\vadjust pre{\hypertarget{ref-dorfeeva1995}{}}%
Dorofeeva, R. P., \& Duchkov, A. D. (1995). A new geothermal study in underwater boreholes on {Lake Baikal} (continental rift zone).

\leavevmode\vadjust pre{\hypertarget{ref-dougherty1986}{}}%
Dougherty, M. E., Herzen, R. P. V., \& Barker, P. F. (1986). Anomalous heat flow from a miocene ridge crest-trench collision, antarctic peninsula. \emph{Antarctic J. U.S.}, \emph{21}, 151--153.

\leavevmode\vadjust pre{\hypertarget{ref-dovenyi1983}{}}%
Dovenyi, P., Horvath, F., Liebe, P., Gafi, J., \& Erki, I. (1983). Geothermal conditions of {Hungary}. \emph{Geophys. Trans.}, \emph{29}, 1--114.

\leavevmode\vadjust pre{\hypertarget{ref-dowgiallo1987}{}}%
Dowgiallo, J. (1987). Preblematyka hydrogeotericzna regionu sudeckiego. \emph{Prozeglad Geologiczny}, \emph{6}, 321--327.

\leavevmode\vadjust pre{\hypertarget{ref-downorowicz1983}{}}%
Downorowicz, S. (1983). \emph{Gerotermika zloza rud miedzi monokliny przedsudeckiej} (pp. pp. 88). Prace Instytutu Geologicznego. CVI Wyd. Geol. Warszawa.

\leavevmode\vadjust pre{\hypertarget{ref-drachev2003}{}}%
Drachev, S. S., Kaul, N., \& Beliaev, V. N. (2003). Eurasia spreading basin to laptev shelf transition:structural pattern and heat flow. \emph{Geophysical Journal International}, \emph{152}, 688--698. \url{https://doi.org/10.1046/j.1365-246X.2003.01882.x}

\leavevmode\vadjust pre{\hypertarget{ref-drury1985}{}}%
Drury, M. J. (1985). Heat flow and heat generation in the churchill province of the canadian shield, and their palaeotectonic significance. \emph{Tectonophysics}, \emph{115}(1-2), 25--44. \url{https://doi.org/10.1016/0040-1951(85)90097-6}

\leavevmode\vadjust pre{\hypertarget{ref-drury1991}{}}%
Drury, M. J. (1991). Heat flow in the {Canadian} shield and its relation to other geophysical parameters. In \emph{Terrestrial heat flow and the lithosphere structure} (pp. 338--380). Springer Verlag.

\leavevmode\vadjust pre{\hypertarget{ref-drury1983}{}}%
Drury, M. J., \& Lewis, T. J. (1983). Water movement within lac du bonnet batholith as revealed by detailed thermal studies of three closely spaced boreholes. \emph{Tectonophysics}, \emph{95}, 337--351. \url{https://doi.org/10.1016/0040-1951(83)90077-x}

\leavevmode\vadjust pre{\hypertarget{ref-drury1987}{}}%
Drury, M. J., Jessop, A. M., \& Lewis, T. J. (1987). The thermal nature of the canadian appalachian crust. \emph{Tectonophysics}, \emph{133}, 1--14. \url{https://doi.org/10.1016/0040-1951(87)90276-9}

\leavevmode\vadjust pre{\hypertarget{ref-drwiega1980}{}}%
Drwiega, Z., \& Myśko, A. (1980). Wyniki badań ziemskiego strumienia ciepla obszaru lubelskiego na tle jego tektoniki. \emph{Pub. Of the Inst. Geophys.}, \emph{A-8}, 169--180.

\leavevmode\vadjust pre{\hypertarget{ref-duchikov2004}{}}%
Duchkov, A. D. (2004). Personal communication. In \emph{CD rom: Geothermal gradient and heat flow data in and around japan} (p. --). Geological Survey of Japan, AIST, 2004.

\leavevmode\vadjust pre{\hypertarget{ref-duchkov1984}{}}%
Duchkov, A. D., \& Kazantsev, S. A. (1984). Rezultaty izucheniya teplovogo potoka cherez dno ozer. - v kn.: Teoreticheskie i eksperimentalnye issle- dovaniya po geotermike morey i okeanov. Moskva: nauka.

\leavevmode\vadjust pre{\hypertarget{ref-duchkov1985}{}}%
Duchkov, A. D., \& Kazantsev, S. A. (1985). Teplovoi potok cherez dno zapadnoi chernogo morya (in russian). \emph{Geologiya i Geofizika}, \emph{8}, 113--123.

\leavevmode\vadjust pre{\hypertarget{ref-duchkov1988}{}}%
Duchkov, A. D., \& Kazantsev, S. A. (1988). Teplovoi potok vpadiny chernogo morya. \emph{Geofizicheskie Polya Atlanticheskogo Okeana. Moskva, Mezh- Duvedomstvennyi Geofizicheskii Komitet Pri Prezidiume An SSSR. , S. (Russ.).}, 121--130.

\leavevmode\vadjust pre{\hypertarget{ref-duchkov1974}{}}%
Duchkov, A. D., \& Sokolova, L. S. (1974). Teplovoy potok tsentralnykh rayonov altae-sayanskoy oblasti (russ.). \emph{Geologiya I Geofizika}, \emph{8}, 114--123.

\leavevmode\vadjust pre{\hypertarget{ref-duchkov1976}{}}%
Duchkov, A. D., Kazantsev, S. A., \& Golubev, V. A. (1976). I dr. Teplovoi potok v predelakh ozera baikal. - geologia i geofizika (russ.), \emph{4}, 112--121.

\leavevmode\vadjust pre{\hypertarget{ref-duchkov1977}{}}%
Duchkov, A. D., Kazantsev, S. A., Golubev, V. A., \& Lysak, S. V. (1977). Geotermicheskie issledovaniya na ozere baikal (russ.). \emph{Geologiya I Geofizika}, \emph{6}, 126--130.

\leavevmode\vadjust pre{\hypertarget{ref-duchkov1979}{}}%
Duchkov, A. D., Kazantsev, S. A., \& Velinskii, V. V. (1979). Teplovoi potok ozera baikal. - geologiya i geofizika, (russ.), \emph{9}, 137--141.

\leavevmode\vadjust pre{\hypertarget{ref-duchkov1992}{}}%
Duchkov, A. D., Jen, N. C., Toam, D. V., \& Bak, C. V. (1992). First estimates of heat flow in {North Vietnam}. \emph{Sov. Geol. Geophys.}, \emph{33}, 92--96.

\leavevmode\vadjust pre{\hypertarget{ref-duennebier1987}{}}%
Duennebier, F., Cessaro, R. K., \& Harris, D. (1987). Temperature and tilt variation measured for 64 days in hole 581C. \emph{Initial Reports DSDP}, \emph{88}, 161--165.

\leavevmode\vadjust pre{\hypertarget{ref-duque1993}{}}%
Duque, M. R., \& Mendes-Victor, L. A. (1993). Heat flow and deep temperature in south {Portugal}. \emph{Studia Geoph. Et Geod.}, \emph{37}, 279--292.

\leavevmode\vadjust pre{\hypertarget{ref-dzhamalova1967}{}}%
Dzhamalova, A. S. (1967). O teplovom rezhime nedr v raione russkogo khutora ravninnogo dagestana. - v kn.: Regionalnaya geotermiya i raspredele- nie termalnykh vod v sssr. moskva. \emph{Nauka}.

\leavevmode\vadjust pre{\hypertarget{ref-dzhamalova1969}{}}%
Dzhamalova, A. S. (1969). Glubinnyi teplovoi potok na territorii dagesta- na. moskva. \emph{Nauka. 1969}.

\leavevmode\vadjust pre{\hypertarget{ref-dzhamalova1972}{}}%
Dzhamalova, A. S. (1972). Radioaktivnyi raspad v osadochnoi tolshche i ego rol v formirovanii glubinnogo teplovogo potoka na territorii da- gestana (russ.). \emph{Energetika Geologicheskikh I Geofizicheskikh Pro- Tsessov, Moskva Nauka}, 88--89.

\leavevmode\vadjust pre{\hypertarget{ref-ebinger1987}{}}%
Ebinger, C. J., Rosendahl, B. R., \& Reynolds, D. J. (1987). Tectonic model of the malaŵi rift, africa. \emph{Tectonophysics}, \emph{141}(1-3), 215--235. \url{https://doi.org/10.1016/0040-1951(87)90187-9}

\leavevmode\vadjust pre{\hypertarget{ref-eckstein1978}{}}%
Eckstein, Y., \& Simmons, G. (1978). Measurements and interpretation of terrestrial heat flow in israel. \emph{Geothermics}, \emph{6}, 117--142. \url{https://doi.org/10.1016/0375-6505(77)90023-2}

\leavevmode\vadjust pre{\hypertarget{ref-eckstein1982}{}}%
Eckstein, Y., Heimlich, R. A., Palmer, D. F., \& Shannon Jr., S. S. (1982). \emph{Geothermal investigations in ohio and pennsylvania} (No. La-9223-hdr) (p. --). Retrieved from \url{http://epic.awi.de/32508/5/Eckstein-etal/_/\%281982/\%29/_WHFDC.pdf}

\leavevmode\vadjust pre{\hypertarget{ref-edwards1978}{}}%
Edwards, C. L., Reiter, M. A., Shearer, C., \& Young, W. (1978). Terrestrial heat flow and crustal radioactivity in northeastern new mexico and southeastern colorado. \emph{Geological Society of America Bulletin}, \emph{89}(9), 1341--1350. \url{https://doi.org/10.1130/0016-7606(1978)89\%3C1341:thfacr\%3E2.0.co;2}

\leavevmode\vadjust pre{\hypertarget{ref-eggleston1984}{}}%
Eggleston, R. E., \& Reiter, M. A. (1984). Terrestrial heat flow estimates from petroleum bottom-hole temperature data in the colorado plateau and the eastern basin and range province. \emph{Geological Society of America Bulletin}, \emph{95}(9), 1027--1034. \url{https://doi.org/10.1130/0016-7606(1984)95\%3C1027:thefpb\%3E2.0.co;2}

\leavevmode\vadjust pre{\hypertarget{ref-ehara1979}{}}%
Ehara, S. (1979). Heat flow in the {Hokkaido--Okhotsk} region and its tectonic implications. \emph{J. Phys. Earth}, \emph{27}, s125--s139.

\leavevmode\vadjust pre{\hypertarget{ref-ehara1984}{}}%
Ehara, S. (1984). Terrestrial heat flow determinations in central kyushu, japan. \emph{Bulletin of Volcanic Society of Japan}, \emph{29}, 75--94.

\leavevmode\vadjust pre{\hypertarget{ref-ehara1985}{}}%
Ehara, S., \& Sakmoto, M. (1985). Terrestrial heat flow determinations in southern kyushu, japan. \emph{Bulletin of Volcanic Society of Japan}, \emph{30}, 253--271.

\leavevmode\vadjust pre{\hypertarget{ref-ehara1971}{}}%
Ehara, S., \& Yokoyama, I. (1971). Measurements of terrestrial heat flow in hokkaido (part 2). \emph{Geophysical Bulletin Hokkaido University}, \emph{26}(in japanese with english abstract), 67--84.

\leavevmode\vadjust pre{\hypertarget{ref-ehara1980}{}}%
Ehara, S., Yuhara, K., \& Shigematsu, A. (1980). Heat flow measurements in the submarine calderas, southern kyushu, japan - preliminary report. \emph{Bulletin of Volcanic Society of Japan}, \emph{25}, 51--61.

\leavevmode\vadjust pre{\hypertarget{ref-ehara1989}{}}%
Ehara, S., Jin, K., \& Yuhara, K. (1989). Determination of heat flow values in the two granitic rock regions of japan - houfu area in yamaguchi prefecture and kunisaki area in oita prefecture, southwest japan. \emph{J. Geotherm. Res. Soc. Japan}, \emph{11}.

\leavevmode\vadjust pre{\hypertarget{ref-eldholm1987}{}}%
Eldholm, O., Thiede, J., Taylor, E., \& Shipboard Scientific Party. (1987). \emph{Norwegian sea} (Vol. 104). Ocean Drilling Program.

\leavevmode\vadjust pre{\hypertarget{ref-eldholm1999}{}}%
Eldholm, O., Sundvor, E., Vogt, P. R., Hjelstuen, B. O., Crane, K., Nilsen, A. K., \& Gladczenko, T. P. (1999). SW barents sea continental margin heat flow and hakon mosby mud volcano. \emph{Geo-Marine Letters}, \emph{19}, 29--37. \url{https://doi.org/10.1007/s003670050090}

\leavevmode\vadjust pre{\hypertarget{ref-eliasson1991}{}}%
Eliasson, T., Eriksson, K. G., Lindquist, G., Malmquist, D., \& Parasnis, D. (1991). Catalogue of heat flow density data: sweden. In \emph{Geothermal atlas of europe} (pp. 124--125). Hermann Haack Verlagsgesellschaft mbH.

\leavevmode\vadjust pre{\hypertarget{ref-embley1983}{}}%
Embley, R. W., Hobart, M. A., Anderson, R. N., \& Abbott, Dallas H. (1983). Anomalous heat flow in the northwest atlantic: A case for continued hydrothermal circulation in 80 my crust. \emph{Journal of Geophysical Research}, \emph{88}(b2), 1067--1074. \url{https://doi.org/10.1029/JB088iB02p01067}

\leavevmode\vadjust pre{\hypertarget{ref-england1980}{}}%
England, P. C., Oxburgh, E. R., \& Richardson, S. W. (1980). Heat refraction and heat production in and around granite plutons in north-east england. \emph{Geophys. J. Roy. Astr. Soc.}, \emph{62}, 439--455.

\leavevmode\vadjust pre{\hypertarget{ref-epp1970}{}}%
Epp, D., Grim, P. J., \& Langseth, M. G. (1970). Heat flow in the caribbean and gulf of mexico. \emph{Journal of Geophysical Research}, \emph{75}, 5655--5669. \url{https://doi.org/10.1029/JB075i029p05655}

\leavevmode\vadjust pre{\hypertarget{ref-erickson1974}{}}%
Erickson, A., \& Simmons, G. (1974). Enviromnetal and geophysical interpreation of heat-flow measurements in the {Black Sea}. In \emph{The black sea --- geology, chemistry and biology} (Vol. 20, pp. 50--62). Am. Assoc. Petrol. Geol.

\leavevmode\vadjust pre{\hypertarget{ref-erickson1970}{}}%
Erickson, A. J. (1970). \emph{The measurement and interpretation of heat flow in the mediterranean and black seas} (PhD thesis).

\leavevmode\vadjust pre{\hypertarget{ref-erickson1973}{}}%
Erickson, A. J. (1973). Initial report on downhole temperature and shipboard thermal conductivity measurements {Leg} 19. \emph{Initial Reports DSDP}, \emph{19}, 643--656.

\leavevmode\vadjust pre{\hypertarget{ref-erickson1978}{}}%
Erickson, A. J., \& Hyndman, R. D. (1978). Downhole temperature measurements and thermal conductivities of samples, site 396. \emph{Initial Reports DSDP}, \emph{46}, 389.

\leavevmode\vadjust pre{\hypertarget{ref-erickson1969}{}}%
Erickson, A. J., \& Simmons, G. (1969). Thermal measurements in the {Red Sea} hot brine pools. In \emph{D hot brines and heavy metal deposits in the {Red Sea}---a geochemical and geophysical account} (pp. 114--121). Springer-Verlag.

\leavevmode\vadjust pre{\hypertarget{ref-erickson1978b}{}}%
Erickson, A. J., \& Von Herzen, R. P. (1978). Downhole temperature measurements and heat flow data in the {Black Sea} --- {DSDP Leg 42B}. \emph{Initial Reports DSDP}, \emph{42-2}, 1085--1103.

\leavevmode\vadjust pre{\hypertarget{ref-erickson1972}{}}%
Erickson, A. J., Helsley, C. E., \& Simmons, G. (1972). Heat flow and continuous seismic profiles in the cayman trough and yucatan basin. \emph{Bulletin Geololgical Society of America}, \emph{83}, 1242--1260. \url{https://doi.org/10.1130/0016-7606}

\leavevmode\vadjust pre{\hypertarget{ref-erickson1977}{}}%
Erickson, A. J., Simmons, G., \& Ryan, W. B. F. (1977). Review of heat flow data from the {Mediterranean and Aegean Seas}. In \emph{International symposium on structural history of the mediterranean basins. Split (yugoslavia) 25-29 october 1976} (pp. 263--280). Editions Technip.

\leavevmode\vadjust pre{\hypertarget{ref-erickson1979}{}}%
Erickson, A. J., Avera, W. E., \& Byrne, R. (1979). Heat-flow results, DSDP {Leg} 48. \emph{Initial Reports DSDP}, \emph{48}, 277.

\leavevmode\vadjust pre{\hypertarget{ref-eriksson1979}{}}%
Eriksson, K. G., \& Malmqvist, D. (1979). A review of the past and the present investigations of heat flow in sweden. In \emph{Terresrial heat flow in europe} (pp. 267--277). Springer Verlag. \url{https://doi.org/10.1007/978-3-642-95357-6/_28}

\leavevmode\vadjust pre{\hypertarget{ref-erki1984}{}}%
Erki, I., Kolios, N., \& Stegena, L. (1984). Heat flow density determination in the {Strymon} basin, {NE Greece}. \emph{J. Geophys.}, \emph{54}, 106--109.

\leavevmode\vadjust pre{\hypertarget{ref-evans1976}{}}%
Evans, T. R. (1976). \emph{Terrestrial heat flow studies in eastern africa and the north sea} (PhD thesis).

\leavevmode\vadjust pre{\hypertarget{ref-evans1974}{}}%
Evans, T. R., \& Tammemagi, H. Y. (1974). Heat flow and heat production in northeast africa. \emph{Earth and Planetary Science Letters}, \emph{23}(3), 349--356. \url{https://doi.org/10.1016/0012-821x(74)90124-1}

\leavevmode\vadjust pre{\hypertarget{ref-fanelli1974}{}}%
Fanelli, M., Loddo, M., Mongelli, F., \& Squarci, P. (1974). Terrestrial heat flow measurements near {Rosignano Solvey (Tuscany)}. \emph{Geothermics}, \emph{3}, 65--73.

\leavevmode\vadjust pre{\hypertarget{ref-feinstein1996}{}}%
Feinstein, S., Kohn, B. P., Steckler, M. S., \& Eyal, M. (1996). Thermal history of the eastern margin of the {Gulf of Suez, I}. Reconstruction from borehole temperature and organic maturity measurements. \emph{Tectonophysics}, \emph{266}, 203--220.

\leavevmode\vadjust pre{\hypertarget{ref-feng2009}{}}%
Feng, C.-G., Liu, S.-W., Wang, L.-S., \& Li, C. (2009). Present-day geothermal regime in-plane tarim basin, northwest china. \emph{Chinese Journal of Geophysics}, \emph{52}(6, 11), 2752--2762. \url{https://doi.org/10.1002/cjg2.1450}

\leavevmode\vadjust pre{\hypertarget{ref-fernandez1992}{}}%
Fernàndez, M., \& Cabal, J. (1992). Heat-flow data and shallow thermal regime on {Mallorca and Menorca} (western {Mediterranean}). \emph{Tectonophysics}, \emph{203}, 133--143.

\leavevmode\vadjust pre{\hypertarget{ref-fernandez1998}{}}%
Fernàndez, M., Marzán, I., Correia, A., \& Ramalho, E. (1998). Heat flow, heat production, and lithospheric thermal regime in the {Iberian Peninsula}. \emph{Tectonophysics}, \emph{291}, 29--53.

\leavevmode\vadjust pre{\hypertarget{ref-finckh1981}{}}%
Finckh, P. (1981). Heat flow measurements in 17 perialpine lakes. \emph{Bull Geol. Soc. Am., Part II}, \emph{92}, 452--514.

\leavevmode\vadjust pre{\hypertarget{ref-firsov1979}{}}%
Firsov, F. V. (1979). Teplovoe pole na yuzhnom urale (russ.). \emph{Eksperimental- Noe I Teoreticheskoe Izuchenie Teplovykh Potokov. Moskva, Nauka}, 217--221.

\leavevmode\vadjust pre{\hypertarget{ref-fisher2001}{}}%
Fisher, A. T., Giambalvo, E., Sclater, J. G., Kastner, M., Ransom, B., Weinstein, Y., \& Lonsdale, P. (2001). Heat flow, sediment and pore fluid chemistry, and hydrothermal circulation on the east flank of alarcon ridge, gulf of california. \emph{Earth and Planetary Science Letters}, \emph{188}, 521--534. \url{https://doi.org/10.1016/s0012-821x(01)00310-7}

\leavevmode\vadjust pre{\hypertarget{ref-fisher1981}{}}%
Fisher, M. A., \& Gardner, M. C. (1981). \emph{Temperature-gradient and heat-flow data, {Panther Canyon, Nevada}} (No. Report Nv-lch-amn-9 For Sunco Energy Development Co.).

\leavevmode\vadjust pre{\hypertarget{ref-flovenz1991}{}}%
Flovenz, O. G., \& Saemundsson, K. (1991). Catalogue of heat flow density data: iceland. In \emph{Geothermal atlas of europe} (pp. 118--119). Hermann Haack Verlagsgesellschaft mbH.

\leavevmode\vadjust pre{\hypertarget{ref-flovenz1993}{}}%
Flóvenz, Ó. G., \& Saemundsson, K. (1993). Heat flow and geothermal processes in iceland. \emph{Tectonophysics}, \emph{225}(1--2), 123--138. \url{https://doi.org/10.1016/0040-1951(93)90253-g}

\leavevmode\vadjust pre{\hypertarget{ref-foerster2000}{}}%
Förster, A., \& Förster, H.-J. (2000). Crustal compostion and mantle heat flow: Implications from surface heat flow and radiogenic heat production in the {Variscan Erzgebirge (Germany)}. \emph{J. Geophys. Res.}, \emph{105}, 27917--27938.

\leavevmode\vadjust pre{\hypertarget{ref-foerster1997}{}}%
Förster, A., \& Merriam, D. F. (1997). Heat flow in the {Cretaceous} of northwestern {Kansas} and implications for regional hydrology. In \emph{Kansas geological survey bulletin} (Vol. 240, pp. 1--11).

\leavevmode\vadjust pre{\hypertarget{ref-foerster2007}{}}%
Förster, A., Förster, H.-J., Masarweh, R., Masri, A., Tarawneh, K., \& Group, D. (2007). The surface heat flow of the {Arabian Shield in Jordan}. \emph{J. Asian Earth Sci.}, \emph{30}, 271--284.

\leavevmode\vadjust pre{\hypertarget{ref-foster1974}{}}%
Foster, S. E., Simmons, G., \& Lamb, W. (1974). Heat-flow near a {North Atlantic} fracture zone. \emph{Geothermics}, \emph{3}, 3.

\leavevmode\vadjust pre{\hypertarget{ref-foster1962}{}}%
Foster, T. D. (1962). Heat-flow measurements in the northeast {Pacific} and in the {Bering Sea}. \emph{Journal of Geophysical Research}, \emph{67}, 2991--2993. \url{https://doi.org/10.1029/JZ067i007p02991}

\leavevmode\vadjust pre{\hypertarget{ref-foster1978}{}}%
Foster, T. D. (1978). The temperature and salinity fields under the {Ross Ice Shelf}. \emph{EOS Trans.}, \emph{59}, 308.

\leavevmode\vadjust pre{\hypertarget{ref-foster1983}{}}%
Foster, T. D. (1983). The temperature and salinity finestructure of the ocean under the {Ross Ice Shelf}. \emph{J. Geophys. Res.}, \emph{88}, 2556--2564.

\leavevmode\vadjust pre{\hypertarget{ref-fotiadi1969}{}}%
Fotiadi, A. A., Moiseenko, U. I., \& Sokolova, L. S. (1969). O teplovom pole zapadno- sibirskoy plity.- doklady an sssr.

\leavevmode\vadjust pre{\hypertarget{ref-fou1969}{}}%
Fou, J. T. K. (1969). \emph{Thermal conductivity and heat flow at {St. Jerome, Quebec}} (Master's thesis).

\leavevmode\vadjust pre{\hypertarget{ref-foucher1985}{}}%
Foucher, J. P., Chenet, P. Y., Montadert, L., \& Roux, J. M. (1985). Geothermal measurements during deep sea drilling project {Leg 80}. \emph{Initial Reports DSDP}, \emph{80}, 423--436.

\leavevmode\vadjust pre{\hypertarget{ref-foucher1990}{}}%
Foucher, J. P., Le Pichon, X., Lallemant, S., Hobart, M. A., Henry, P., Benedetti, M., et al. (1990). Heat flow, tectonics and fluid circulation at the toe of the {Barbados Ridge} accretionary prism. \emph{J. Geophys. Res.}, \emph{95}, 8859--8867.

\leavevmode\vadjust pre{\hypertarget{ref-foucher1992}{}}%
Foucher, J. P., Mauffret, A., Steckler, M., Brunet, M. F., Maillard, A., Rehault, J. P., et al. (1992). Heat flow in the valencia trough: Geodynamic implications. \emph{Tectonophysics}, \emph{203}(1-4), 77--97. \url{https://doi.org/10.1016/0040-1951(92)90216-s}

\leavevmode\vadjust pre{\hypertarget{ref-foucher1979}{}}%
Foucher, J.-P., \& Sibuet, J.-C. (1979). Thermal regime of the northern {Bay of Biscay} continental margin in the vicinity of DSDP sites 400 to 402. \emph{Initial Reports DSDP}, \emph{48}, 289--296.

\leavevmode\vadjust pre{\hypertarget{ref-francheteau1984}{}}%
Francheteau, J., Jaupart, C., Jie, S. X., Wen-Hua, K., De-Lu, L., Jia-Chi, B., et al. (1984). High heat flow in southern {Tibet}. \emph{Nature}, \emph{307}, 32--36.

\leavevmode\vadjust pre{\hypertarget{ref-fujii1981}{}}%
Fujii, N. (1981). Down-hole temperature measurements and heat flow at {Hess Rise}. \emph{Initial Reports DSDP}, \emph{62}, 1009.

\leavevmode\vadjust pre{\hypertarget{ref-funnell1996}{}}%
Funnell, R., Chapman, D., Allis, R., \& Armstrong, P. (1996). Thermal state of the {Taranaki Basin, New Zealand}. \emph{J. Geophys. Res.}, \emph{101}, 25197--25215.

\leavevmode\vadjust pre{\hypertarget{ref-furukawa1998}{}}%
Furukawa, Y., Shinjoe, H., \& Nishimura, S. (1998). Heat flow in the southwest japan arc and its implication for thermal processes under arcs. \emph{Geophysical Research Letters}, \emph{25}(7), 1087--1090. \url{https://doi.org/10.1029/98gl00545}

\leavevmode\vadjust pre{\hypertarget{ref-fytikas1979}{}}%
Fytikas, M. D., \& Kolios, N. P. (1979). Preliminary heat flow map of greece. In \emph{Terrestrial heat flow in europe} (pp. 197--205). Springer Verlag. \url{https://doi.org/10.1007/978-3-642-95357-6/_20}

\leavevmode\vadjust pre{\hypertarget{ref-gable1980}{}}%
Gable, R. (1980). Terrestrial heat flow in france. In \emph{Advances in european geothermal research} (pp. 466--473).

\leavevmode\vadjust pre{\hypertarget{ref-gable1979}{}}%
Gable, R., \& Watremez, P. (1979). Premières estimations du flux de chaleur dans le massif armoricain. \emph{Bulletin BRGM}, \emph{17}(1), 35--38.

\leavevmode\vadjust pre{\hypertarget{ref-galanis1986}{}}%
Galanis, S. P., Sass, J. H., Munroe, R. J., \& Abu-Ajamieh, M. (1986). \emph{Heat flow at zerqa ma'in and zara and a geothermal reconnaissance of jordan} (No. 86-63).

\leavevmode\vadjust pre{\hypertarget{ref-gallagher1987}{}}%
Gallagher, K. (1987). Thermal conductivity and heat flow in the southern {Cooper Basin}. \emph{Explor. Geophys.}, \emph{18}, 62--67.

\leavevmode\vadjust pre{\hypertarget{ref-gallagher1990}{}}%
Gallagher, K. (1990). Some strategies for estimating present day heat flow from exploration wells, with examples. \emph{Explor. Geophys.}, \emph{21}, 145--159.

\leavevmode\vadjust pre{\hypertarget{ref-galson1981}{}}%
Galson, D. A., \& Von Herzen, R. P. (1981). A heat flow survey on anomaly M0 south of the bermuda rise. \emph{Earth and Planetary Science Letters}, \emph{53}, 296--306. \url{https://doi.org/10.1016/0012-821x(81)90035-2}

\leavevmode\vadjust pre{\hypertarget{ref-garcia2001}{}}%
Garcia-Estrada, G., Lopez-Hernandez, A., \& Prol-Ledesma, R. M. (2001). Temperature--depth relationships based on log data from the los azufres geothermal field, mexico. \emph{Geothermics}, \emph{30}(1), 111--132. \url{https://doi.org/10.1016/s0375-6505(00)00039-0}

\leavevmode\vadjust pre{\hypertarget{ref-garland1962}{}}%
Garland, G. D., \& Lennox, D. H. (1962). Heat flow in western {Canada}. \emph{Geophys. J. R. Astron. Soc.}, \emph{6}, 245--262.

\leavevmode\vadjust pre{\hypertarget{ref-gebski1987}{}}%
Gebski, J. S., Wheildon, J., \& Thomas-Betts, A. (1987). \emph{Detailed investigation of the UK heat flow field 1984-87. Investigation of the geothermal potential of the UK}.

\leavevmode\vadjust pre{\hypertarget{ref-geli2008}{}}%
Géli, L., Lee, T. C., Cochran, J. R., Francheteau, J., Abbott, D., Labails, C., \& Appriou, D. (2008). Heat flow from the {Southeast Indian Ridge} flanks between 80\(\circ\)e and 140\(\circ\)e: Data review and analysis. \emph{J. Geophys. Res.}, \emph{113}, b01101, doi:10.1029/2007JB005001.

\leavevmode\vadjust pre{\hypertarget{ref-geller1983}{}}%
Geller, C. A., Weissel, J. K., \& Anderson, R. N. (1983). Heat transfer and intraplate deformation in the central indian ocean. \emph{Journal of Geophysical Research}, \emph{88}, 1018--1032. \url{https://doi.org/10.1029/JB088iB02p01018}

\leavevmode\vadjust pre{\hypertarget{ref-gerard1962}{}}%
Gerard, R., Langseth, M. G., \& Ewing, M. (1962). Thermal gradient measurements in the water and bottom sediment of the western {Atlantic}. \emph{J. Geophys. Res.}, \emph{67}, 785--803.

\leavevmode\vadjust pre{\hypertarget{ref-gettings1986}{}}%
Gettings, M., Jr., H. B., Mooney, W., \& Healey, J. (1986). Crustal structure of southwestern {Saudi Arabia}. \emph{J. Geophys. Res.}, \emph{91}, 6491--6512.

\leavevmode\vadjust pre{\hypertarget{ref-ginsburg2004}{}}%
Ginsburg, G. D., \& Soloviev, V. A. (2004). Personal communication. In \emph{CD rom: Geothermal gradient and heat flow data in and around japan} (p. --). Geological Survey of Japan, AIST, 2004.

\leavevmode\vadjust pre{\hypertarget{ref-girdler1970}{}}%
Girdler, R. W. (1970). A review of red sea heat flow. \emph{Philosophical Transaction of the Royal Astronomy Society, Ser. A}, \emph{267}, 191--203. \url{https://doi.org/10.1098/rsta.1970.0032}

\leavevmode\vadjust pre{\hypertarget{ref-girdler1974}{}}%
Girdler, R. W., Erickson, A. J., \& Von Herzen, R. P. (1974). Downhole temperature and shipboard thermal conductivity measurements abroad the d.v. Glomar challenger in the {Red Sea}. \emph{Initial Reports DSDP}, \emph{23}, 679--786.

\leavevmode\vadjust pre{\hypertarget{ref-glaeser1982}{}}%
Gläser, S., \& Hurtig, E. (1982). \emph{Interner bericht}.

\leavevmode\vadjust pre{\hypertarget{ref-goff1992}{}}%
Goff, S. J., Goff, F., \& Janik, C. J. (1992). {Tecuamburro Volcano, Guatemala}: Exploration geothermal gradient drilling and results. \emph{Geothermics}, \emph{21}, 483--502.

\leavevmode\vadjust pre{\hypertarget{ref-golovanova2001}{}}%
Golovanova, I. V., Harris, R. N., Selezniova, G. V., \& Stulc, P. (2001). Evidence of climatic warming in the southern urals region derived from borehole temperatures and meteorological data. \emph{Global Planetary Change}, \emph{29}, 167--188.

\leavevmode\vadjust pre{\hypertarget{ref-golubev1995}{}}%
Golubev, V., \& Poort, J. (1995). Local heat flow anomalies along the western shore of the north {Baikal} basin. \emph{Russian Geology and Geophysics}, \emph{36}, 175--186.

\leavevmode\vadjust pre{\hypertarget{ref-golubev1978}{}}%
Golubev, V. A. (1978). Geotermicheskie issledovaniya na baikale s ispolzovaniem kabelnogo zonda - termometra (russ.). \emph{Izvestiya Akademii Nauk SSSR, Fizika Zemli}, \emph{3}, 106--109.

\leavevmode\vadjust pre{\hypertarget{ref-golubev1982}{}}%
Golubev, V. A. (1982). Geotermiya baikala. - novosibirsk: Nauka (russ.) (p. 150).

\leavevmode\vadjust pre{\hypertarget{ref-golubev1992}{}}%
Golubev, V. A. (1992). Heat flow through the bottom of {Khubsugul Lake} and the bordering mountains {(Mongolia)}. \emph{Izv. Akad. Nauk SSSR, Fizika Zemli}, \emph{1}, 48--60.

\leavevmode\vadjust pre{\hypertarget{ref-golubev1980}{}}%
Golubev, V. A., \& Osokina, S. V. (1980). Raspredelenie teplovogo potoka i priroda ego lokalnych anomaliy v raione ozera baikal (russ.). \emph{Izvestiya Akademii Nauk SSSR, Fizika Zemli}, \emph{4}, 63--75.

\leavevmode\vadjust pre{\hypertarget{ref-gomes2005}{}}%
Gomes, A. J. L., \& Hamza, V. M. (2005). Geothermal gradient and heat flow in the state of {Rio de Janeiro}. \emph{Revista Brasileira de Geofisica}, \emph{23}, 325--347.

\leavevmode\vadjust pre{\hypertarget{ref-gong2003}{}}%
Gong, Y., Wang, L., \& Liu, S. (2003). Distribution of geothermal heat flow in jiyang depression. \emph{Science of China (Series D)}, \emph{33}, 384--391.

\leavevmode\vadjust pre{\hypertarget{ref-gordienko1972}{}}%
Gordienko, V. V. (1972). Novi dani pro teplovii potik krimu ta prichorno- mor'ya (ukrain.). \emph{Dopovidni An USSR, Ser. B}, \emph{8}, 711--713.

\leavevmode\vadjust pre{\hypertarget{ref-gordienko1968}{}}%
Gordienko, V. V., \& Kutas, R. I. (1968). Teplovoe pole radyanskikh karpat i su- sidnikh teritorii. - dopovidni an ursr. \emph{Ser. B. 1968}.

\leavevmode\vadjust pre{\hypertarget{ref-gordienko1970}{}}%
Gordienko, V. V., \& Kutas, R. I. (1970). Teplovii potik dneprovsko-donetskoi zapadini ta donbasu (ukrain.). \emph{Dopovidni An USSR, Ser. B}, \emph{1}, 56--59.

\leavevmode\vadjust pre{\hypertarget{ref-gordienko1971}{}}%
Gordienko, V. V., \& Kutas, R. I. (1971). Novi dani pro teplovii potik ukrains- kogo shchita (ukrain.). \emph{Dopovidni An USSR, Ser. B.}, \emph{6}, 541--542.

\leavevmode\vadjust pre{\hypertarget{ref-gordienko1982}{}}%
Gordienko, V. V., \& Zavgordnyaya, O. V. (1982). Novye opredeleniya i karta teplovogo potoka kryma. - geofizicheskii zhurnal. T. 4, no 3, (russ.)., 56--62.

\leavevmode\vadjust pre{\hypertarget{ref-gordienko1980}{}}%
Gordienko, V. V., \& Zavgorodnyaya, O. V. (1980). Izmerenie teplovogo potoka zemli u poverkhnosti. Kiev (russ.). \emph{Naukova Dumka}, \emph{104}.

\leavevmode\vadjust pre{\hypertarget{ref-gordienko1985}{}}%
Gordienko, V. V., \& Zavgorodnyaya, O. V. (1985). Opredelenie teplovogo potoka na vostochno-evropeiskoi platforme. - doklady an ussr. \emph{Ser. B. 1985}.

\leavevmode\vadjust pre{\hypertarget{ref-gordienko1987}{}}%
Gordienko, V. V., \& Zavgorodnyaya, O. V. (1987). Anomalii teplovogo potoka v moskovskoi i baltiiskoi sineklizakh. - doklady an ussr. \emph{Ser. B. 1987}.

\leavevmode\vadjust pre{\hypertarget{ref-gordienko1988}{}}%
Gordienko, V. V., \& Zavgorodnyaya, O. V. (1988). Yavorovskaya anomaliya teplovogo potoka. - geofizicheskii zhurnal. 1988. \emph{T.10}.

\leavevmode\vadjust pre{\hypertarget{ref-gosnold1982}{}}%
Gosnold Jr., W. D., \& Eversoll, D. A. (1982). Geothermal resources of {Nebraska}. National Geophysical; Solar-Terrestrial Data Center, National Oceanic; Atmospheric Administration.

\leavevmode\vadjust pre{\hypertarget{ref-gosnold1984}{}}%
Gosnold, W. D. (1984). \emph{Geothermal resource assessment for north dakota. Final report} (No. Doe/id/12030-t4) (p. --).

\leavevmode\vadjust pre{\hypertarget{ref-gosnold1987}{}}%
Gosnold, W. D. (1987). \emph{Final report geothermal resource assessment, south dakota}.

\leavevmode\vadjust pre{\hypertarget{ref-gosnold1990}{}}%
Gosnold, W. D. (1990). Heat flow in the great plains of the united states. \emph{Journal of Geophysical Research}, \emph{95}, 353--374. \url{https://doi.org/10.1029/JB095iB01p00353}

\leavevmode\vadjust pre{\hypertarget{ref-gosnold1991}{}}%
Gosnold, W. D. (1991). \emph{Stratabound geothermal resources in north dakota and south dakota}.

\leavevmode\vadjust pre{\hypertarget{ref-gosnold1999}{}}%
Gosnold, W. D. (1999). Basin-scale groundwater flow and advective heat flow: An example from the northern {Great Plains}. In \emph{Geothermics in basin analysis} (pp. 99--116). Kluwer Academic/Plenum Publishers.

\leavevmode\vadjust pre{\hypertarget{ref-gough1963}{}}%
Gough, D. T. (1963). Heat flow in the southern {Karroo}. \emph{Proceeding of the Royal Society London Serie A}, \emph{272}, 207--230. \url{https://doi.org/10.1098/rspa.1963.0050}

\leavevmode\vadjust pre{\hypertarget{ref-goutorbe2007}{}}%
Goutorbe, B., Drab, L., Loubet, N., \& Lucazeau, F. (2007). Heat flow of the eastern {Canadian} rifted continental margin revisited. \emph{Terra Nova}, \emph{6}, 381-386; doi: 10.1111/j.1365-3121.2007.00750.x.

\leavevmode\vadjust pre{\hypertarget{ref-goutorbe2008a}{}}%
Goutorbe, B., Lucazeau, F., \& Bonneville, A. (2008a). Surface heat flow and the mantle contribution on the margins of {Australia}. \emph{Geochem. Geophys. Geosys.}, \emph{9}, q05011, doi:10.1029/2007GC001924.

\leavevmode\vadjust pre{\hypertarget{ref-goutorbe2008}{}}%
Goutorbe, B., Lucazeau, F., \& Bonneville, A. (2008b). The thermal regime of south african continental margins. \emph{Earth and Planetary Science Letters}, \emph{267}(1-2), 256--265. \url{https://doi.org/10.1016/j.epsl.2007.11.044}

\leavevmode\vadjust pre{\hypertarget{ref-goy1996}{}}%
Goy, L., Fabre, D., \& Menard, G. (1996). Modelling of rock temperatures for deep alpine tunnel projects, \emph{29}(1), 1--18. \url{https://doi.org/10.1007/bf01019936}

\leavevmode\vadjust pre{\hypertarget{ref-green1980}{}}%
Green, K. E. (1980). \emph{Geothermal processes at the galapagos spreading center} (PhD thesis).

\leavevmode\vadjust pre{\hypertarget{ref-green1981}{}}%
Green, K. E., Von Herzen, Richard P. ., \& Williams, D. L. (1981). The galapagos spreading centre at 86\textdegree{}w: A detailed geothermal field study. \emph{Journal of Geophysical Research}, \emph{86}, 979--986. \url{https://doi.org/10.1029/JB086iB02p00979}

\leavevmode\vadjust pre{\hypertarget{ref-grevemeyer1999}{}}%
Grevemeyer, I., Kaul, N., Villinger, H., \& Weigel, W. (1999). Hydrothermal activity and the evolution of the seismic properties of upper oceanic crust. \emph{Journal of Geophysical Research}, \emph{104}(b3), 5069--5079. \url{https://doi.org/10.1029/1998jb900096}

\leavevmode\vadjust pre{\hypertarget{ref-grevemeyer2003}{}}%
Grevemeyer, I., Diaz-Naveas, J. L., Ranero, C. R., Villinger, H. W., \& ODP Leg 202 Scientific Party. (2003). Heat flow over the descending {Nazca} plate in central {Chile}, 32\(\circ\)s to 41\(\circ\)s: Observations from {ODP Leg 2002} and the occruance of natural gas hydrates. \emph{Earth Planet. Sci. Lett.}, \emph{213}, 285--298.

\leavevmode\vadjust pre{\hypertarget{ref-grevemeyer2004}{}}%
Grevemeyer, I., Kopf, A. J., Fekete, N., Kaul, N., Villinger, H. W., Heesemann, M., et al. (2004). Fluid flow through active mud dome {Mound Culebra} offshore {Nicoya Peninsula, Costa Rica}: Evidence from heat flow surveying. \emph{Marine Geology}, \emph{207}, 145--157.

\leavevmode\vadjust pre{\hypertarget{ref-grevemeyer2005}{}}%
Grevemeyer, I., Kaul, N., Diaz-Naveas, J. L., Villinger, H. W., Ranero, C. R., \& Reichert, C. (2005). Heat flow and bending-related faulting at subduction zone trenches: Case studies offshore of {Nicaragua and Central Chile}. \emph{Earth Planet. Sci. Lett.}, \emph{236}, 238--248.

\leavevmode\vadjust pre{\hypertarget{ref-grevemeyer2006}{}}%
Grevemeyer, I., Kaul, N., \& Diaz-Naveas, J. L. (2006). Geothermal evidence for fluid flow through the gas hydrate stability field off {Central Chile}--transient flow related to large subduction zone earthquakes? \emph{Geophysical Journal International}, \emph{166}, 461--468. \url{https://doi.org/10.1111/j.1365-246X.2006.02940.x}

\leavevmode\vadjust pre{\hypertarget{ref-grevemeyer2009}{}}%
Grevemeyer, I., Kaul, N., \& Kopf, A. (2009). Heat flow anomalies in the {Gulf of Cadiz} and off {Cape San Vincente, Portugal}. \emph{Marine and Petroleum Geology}, \emph{in press}, doi:10.1016/j.marpetgeo.2008.08.006.

\leavevmode\vadjust pre{\hypertarget{ref-griffin1977}{}}%
Griffin, G. M., Reel, D. A., \& Pratt, R. W. (1977). Heat flow in {Florida} oil test holes and indications of oceanic crust beneath the southern {Florida--Bahamas}. In \emph{The geothermal nature of the floridan plateau} (pp. 43--63). Florida Bureau of Geology.

\leavevmode\vadjust pre{\hypertarget{ref-grim1969}{}}%
Grim, P. J. (1969). Heat flow measurements in the {Tasman Sea}. \emph{Journal of Geophysical Research}, \emph{74}, 3933--3934. \url{https://doi.org/10.1029/JB074i015p03933}

\leavevmode\vadjust pre{\hypertarget{ref-groenlie1977}{}}%
Gronlie, G., Heier, K. S., \& Swanberg, C. A. (1977). Terrestrial heat flow determinations from {Norway}. \emph{Norsk Geologisk Tidsskrift}, \emph{56}, 153--162.

\leavevmode\vadjust pre{\hypertarget{ref-guillou1994}{}}%
Guillou, L., Mareschal, J.-C., Jaupart, C., Gariépy, C., Bienfait, G., \& Lapointe, R. (1994). Heat flow, gravity and structure of the {Abitibi} belt, {Superior Province, Canada}: Implications for mantle heat flow. \emph{Earth Planet. Sci. Lett.}, \emph{122}, 103--123.

\leavevmode\vadjust pre{\hypertarget{ref-guilloufrottier1995}{}}%
Guillou-Frottier, L., Mareschal, J.-C., Jaupart, C., Gariepy, C., Lapointe, R., \& Bienfait, Gérard. (1995). Heat flow variations in the grenville province, canada. \emph{Earth and Planetary Science Letters}, \emph{136}(3-4), 447--460. \url{https://doi.org/10.1016/0012-821x(95)00187-h}

\leavevmode\vadjust pre{\hypertarget{ref-guillou1996}{}}%
Guillou-Frottier, L., Jaupart, C., Mareschal, J. C., Gariépy, C., \& Bienfait, G. (1996). High heat flow in the {Trans-Hudson} orogen, central {Canadian} shield. \emph{Geophys. Res. Lett.}, \emph{23}, 3027--3030.

\leavevmode\vadjust pre{\hypertarget{ref-gupta1981}{}}%
Gupta, M. L. (1981). Surface heat flow and igneous intrusion in the cambay basin, india. \emph{Journal of Volcanic and Geothermal Research}, \emph{10}, 279--292. \url{https://doi.org/10.1016/0377-0273(81)90080-9}

\leavevmode\vadjust pre{\hypertarget{ref-gupta1988}{}}%
Gupta, M. L. (1988). Pers. comm.

\leavevmode\vadjust pre{\hypertarget{ref-gupta1967}{}}%
Gupta, M. L., Verma, R. K., Hamza, V. M., Rao, G. V., \& Rao, R. U. M. (1967). Terrestrial heat flow in {Khetri Copper Belt}, {Rahasthan, India}. \emph{J. Geophys. Res.}, \emph{72}, 4215--4220.

\leavevmode\vadjust pre{\hypertarget{ref-gupta1970}{}}%
Gupta, M. L., Verma, R. K., Hamza, V. M., Rao, G. V., \& Rao, R. U. M. (1970). Terrestrial heat flow and tectonics of the cambay basin, gujarat state (india). \emph{Tectonophysics}, \emph{10}, 147--163. \url{https://doi.org/10.1016/0040-1951(70)90104-6}

\leavevmode\vadjust pre{\hypertarget{ref-gupta1987}{}}%
Gupta, M. L., Sharma, S. R., Sundar, A., \& Singh, S. B. (1987). Geothermal studies in the {Hyderabad} granitic region and the crustal thermal structure of the {Southern Indian Shield}. \emph{Tectonophysics}, \emph{140}, 257--264.

\leavevmode\vadjust pre{\hypertarget{ref-gupta1991}{}}%
Gupta, M. L., Sundar, A., \& Sharma, S. R. (1991). Heat flow and heat generation in the {Archean Dharwar} cratons and implications for the {Southern Indian Shield} geotherm and lithospheric thickness. \emph{Tectonophysics}, \emph{194}, 107--122.

\leavevmode\vadjust pre{\hypertarget{ref-gupta1993}{}}%
Gupta, M. L., Sundar, A., Sharma, S. R., \& Singh, S. B. (1993). Heat flow in the {Bastar Craton}, central {Indian Shield}: Implications for thermal characteristics of {Proterozoic} cratons. \emph{Phys. Earth Planet. Int.}, \emph{78}, 23--31.

\leavevmode\vadjust pre{\hypertarget{ref-haenel1970}{}}%
Haenel, R. (1970). Eine neue methode zür bestimmung der terrestrischen waermestromdichte in binnenseen. \emph{Z. Geophys.}, \emph{36}, 725--742.

\leavevmode\vadjust pre{\hypertarget{ref-haenel1971b}{}}%
Haenel, R. (1971). Heat flow measurements and a first heat flow map of {Germany}. \emph{Z. Geophys.}, \emph{37}, 975--992.

\leavevmode\vadjust pre{\hypertarget{ref-haenel1972b}{}}%
Haenel, R. (1972a). Heat flow measurements in the ionian sea with a new heat flow probe. \emph{Meteor. Forschungsergebn.}, \emph{c11}, 105--108.

\leavevmode\vadjust pre{\hypertarget{ref-haenel1972a}{}}%
Haenel, R. (1972b). Heat flow measurements in the red sea and the gulf of aden. \emph{Zeitschrift Für Geophysik}, \emph{38}, 1035--1047.

\leavevmode\vadjust pre{\hypertarget{ref-haenel1974}{}}%
Haenel, R. (1974a). Heat flow measurements in northern italy and heat flow maps of europe. \emph{Zeitschrift Für Geophys.}, \emph{40}, 367--380.

\leavevmode\vadjust pre{\hypertarget{ref-haenel1974b}{}}%
Haenel, R. (1974b). Heat flow measurements in the {Norwegian Sea}. \emph{Meteor. Forschungsergebn.}, \emph{c17}, 74--78.

\leavevmode\vadjust pre{\hypertarget{ref-haenel1973}{}}%
Haenel, R., \& Zoth, G. (1973). Heat flow measurements in austria and heat flow maps of central europe. \emph{Zeitschrift Für Geophysik}, \emph{39}, 425--439.

\leavevmode\vadjust pre{\hypertarget{ref-haenel1979}{}}%
Haenel, R., Gronlie, G., \& Heier, K. S. (1979). Terrestrial heat flow determination in {Norway} and an attempted interpretation. In \emph{Terrestrial heat flow in europe} (pp. 232--239). Springer Verlag.

\leavevmode\vadjust pre{\hypertarget{ref-haenel1988}{}}%
Haenel, R., Staroste, E., et al. (1988). Atlas of geothermal resources in the european community, austria and switzerland.

\leavevmode\vadjust pre{\hypertarget{ref-halunen1973}{}}%
Halunen, A. J., \& Von Herzen, R. P. (1973). Heat flow in the western equatorial pacific ocean. \emph{Journal of Geophysical Research}, \emph{78}, 5195--5208. \url{https://doi.org/10.1029/JB078i023p05195}

\leavevmode\vadjust pre{\hypertarget{ref-hamamoto2005}{}}%
Hamamoto, H., Yamano, M., \& Goto, S. (2005). Heat flow measurement in shallow seas through long-term temperature monitoring. \emph{Geophys. Res. Lett.}, \emph{32}, l21311, doi:10.1029/2005GL024138.

\leavevmode\vadjust pre{\hypertarget{ref-hamamoto2011}{}}%
Hamamoto, H., Yamano, M., Goto, S., Kinoshita, M., Fujino, K., \& Wang, K. (2011). Heat flow distribution and the thermal structure of the {Nankai} subduction zone off the {Kii Peninsula}. \emph{Geochem. Geophys. Geosys.}, \emph{12}, q0ad20. \url{https://doi.org/10.1029/2011gc003623}

\leavevmode\vadjust pre{\hypertarget{ref-hamza1982}{}}%
Hamza, V. M. (1982). Terrestrial heat flow in the alkaline intrusive complex of poços de caldas, brazil. \emph{Tectonophysics}, \emph{83}, 45--62. \url{https://doi.org/10.1016/0040-1951(82)90006-3}

\leavevmode\vadjust pre{\hypertarget{ref-hamza1981}{}}%
Hamza, V. M., \& Eston, S. M. (1981). Assessment of geothermal resources of brazil. \emph{Zbl. Geol. Palaontol. Teil}, \emph{1983}(1/2), 128--155.

\leavevmode\vadjust pre{\hypertarget{ref-hamza1983}{}}%
Hamza, V. M., \& Eston, S. M. (1983). Assessment of geothermal resources of {Brazil} --- 1981. \emph{Zbl. Geol. Paläontol. Teil}, \emph{1}, 128--155.

\leavevmode\vadjust pre{\hypertarget{ref-hamza2005}{}}%
Hamza, V. M., Silva Dias, F. J. S., Gomes, A. J. L., \& Delgadilho Terceros, Z. G. (2005). Numerical and functional representations of regional heat flow in {South America}. \emph{Phys. Earth Planet Int.}, \emph{152}, 223--256.

\leavevmode\vadjust pre{\hypertarget{ref-han1979}{}}%
Han, U. (1979). \emph{Heat flow in south korea} (Master's thesis).

\leavevmode\vadjust pre{\hypertarget{ref-haenel1971a}{}}%
Hänel, R. (1971). Bestimmung der {Terrestrischen Wärmestromdichte in Deutschland}. \emph{Z. Geophys.}, \emph{37}, 119--134.

\leavevmode\vadjust pre{\hypertarget{ref-haenel1983}{}}%
Hänel, R. (1983). Geothermal investigations in the {Rheinish Massif}. In \emph{Plateau uplift, the rhenish shield -- a case history} (pp. 228--246). Springer Verlag.

\leavevmode\vadjust pre{\hypertarget{ref-haenel1977}{}}%
Hänel, R., \& Bram, K. (1977). Das {Geotermische Feld des Nördlinger Ries}. \emph{Geol. Bavarica}, \emph{75}, 373--380.

\leavevmode\vadjust pre{\hypertarget{ref-harder1995}{}}%
Harder, S. H., Toan, D. V., Yem, N. T., Bac, T. V., Vu, N. G., Mauri, S. J., et al. (1995). Preliminary heat flow results from the hanoi basin, vietnam. In \emph{Terrestrial heat flow and geothemal energy in asia} (pp. 163--172). Science Publ.

\leavevmode\vadjust pre{\hypertarget{ref-harris2000}{}}%
Harris, R. N., Von Herzen, R. P., McNutt, M. K., Garven, G., \& Jordahl, K. (2000). Submarine hydrogeology of the {Hawaiian} archipelagic apron {1. Heat flow patterns north of Oaho and Maro Reef}. \emph{J. Geophys. Res.}, \emph{105}, 21353--21369.

\leavevmode\vadjust pre{\hypertarget{ref-harris2010}{}}%
Harris, R. N., Grevemeyer, I., Ranero, C. R., Villinger, H., Barckhausen, U., Henke, T., et al. (2010). The thermal regime of the {Costa Rican} convergent margin 1: Along strike variations in heat flow from probe measurements and estimated from bottom simulating reflectors. \emph{Geochem. Geophys. Geosys.}, submitted.

\leavevmode\vadjust pre{\hypertarget{ref-harris2011}{}}%
Harris, R. N., Schmidt-Schierhorn, F., \& Spinelli, G. (2011). Heat flow along the NanTroSEIZE transect: Results from IODP expeditions 315 and 316 offshore the kii peninsula, japan. \emph{Geochemistry Geophysics Geosystems}, \emph{12}, q0ad16. \url{https://doi.org/10.1029/2011gc003593}

\leavevmode\vadjust pre{\hypertarget{ref-harrison2012}{}}%
Harrison, B., Taylor, D., Tingate, P., \& Sandiford, M. (2012). Heat flow modelling and thermal history of the onshore {Gippsland Basin}: Upside potential for unconventional gas and geothermal resources.

\leavevmode\vadjust pre{\hypertarget{ref-hart1965}{}}%
Hart, S. R., \& Steinhart, J. S. (1965). Terrestrial heat flow--measurement in lake bottoms. \emph{Science}, \emph{149}, 1499--1501.

\leavevmode\vadjust pre{\hypertarget{ref-hart1968}{}}%
Hart, S. R., Steinhart, J. S., \& Smith, T. J. (1968). Heat flow. \emph{Yearbook Carnegie Inst. Washington}, \emph{67}, 360--367.

\leavevmode\vadjust pre{\hypertarget{ref-hart1994}{}}%
Hart, S. R., Steinhart, J. S., \& Smith, T. J. (1994). Terrestrial heat flow in lake superior. \emph{Can. J. Earth Sci.}, \emph{31}, 698--708.

\leavevmode\vadjust pre{\hypertarget{ref-hass2016}{}}%
Hass, B., \& Harris, R. N. (2016). Heat flow along the costa rica seismogenesis project drilling transect: Implications for hydrothermal and seismic processes. \emph{Geochemistry, Geophysics, Geosystems}, \emph{17}(6), 2110--2127. \url{https://doi.org/10.1002/2016gc006314}

\leavevmode\vadjust pre{\hypertarget{ref-hayashi1997}{}}%
Hayashi, T. (1997). \emph{Thermal structure and tectonic history of the derugin basin, sea of okhotsk (in japanese with english abstract)} (Master's thesis).

\leavevmode\vadjust pre{\hypertarget{ref-he2014}{}}%
He, J., Wang, J., Tan, F., Chen, M., Li, Z., Sun, T., et al. (2014). A comparative study between present and palaeo-heat flow in the qiangtang basin, northern tibet, china. \emph{Marine and Petroleum Geology}, \emph{57}, 345--358. https://doi.org/\url{http://dx.doi.org/10.1016/j.marpetgeo.2014.05.020}

\leavevmode\vadjust pre{\hypertarget{ref-he2002}{}}%
He, L., Xiong, L., \& Wang, J. (2002). Heat flow and thermal modeling of the yinggehai basin, south china sea. \emph{Tectonophysics}, \emph{351}, 245--253. \url{https://doi.org/10.1016/s0040-1951(02)00160-9}

\leavevmode\vadjust pre{\hypertarget{ref-he2008}{}}%
He, L., Hu, S., Huang, S., Yang, W., Wang, J., Yuan, Y., \& Yang, S. (2008). Heat flow study at the {Chinese Continental Scientific Drilling} site: Borehole temperature, thermal conductivity and radiogenic heat production. \emph{J. Geophys. Res.}, \emph{113}, b02404, doi:10.1029/2007JB004958.

\leavevmode\vadjust pre{\hypertarget{ref-he2009}{}}%
He, L., Wang, J., Xu, X., Liang, J., Wang, H., \& Zhang, G. (2009). Disparity between measured and {BSR heat flow in the Xisha Trough of the South China Sea} and its implications for the methane hydrate. \emph{J. Asian Earth Sci.}, \emph{34}, 771--780. \url{https://doi.org/10.1016/j.jseaes.2008.11.004}

\leavevmode\vadjust pre{\hypertarget{ref-heasler1982}{}}%
Heasler, H. P., Decker, E. R., \& Buelow, K. L. (1982). Heat flow studies in {Wyoming}: 1979--1981. In \emph{Geothermal direct heat program roundup technical conference proceedings} (pp. 292--312). Earth Science Laboratory, Univ. of Utah.

\leavevmode\vadjust pre{\hypertarget{ref-henderson1983}{}}%
Henderson, J., \& Davis, E. E. (1983). An estimate of heat flow in the western north {Atlantic}. \emph{Initial Reports DSDP}, \emph{76}.

\leavevmode\vadjust pre{\hypertarget{ref-henrikson2000}{}}%
Henrikson, A. (2000). \emph{New heat flow determinations from oil and gas wells in the colorado plateau and basin and range of utah} (Master's thesis).

\leavevmode\vadjust pre{\hypertarget{ref-henry1988}{}}%
Henry, S. G., \& Pollack, H. N. (1988). Terrestrial heat flow overlying the {Andean} subduction zone in {Bolivia and Peru}. \emph{J. Geophys. Res.}, \emph{93}, 15153--15162.

\leavevmode\vadjust pre{\hypertarget{ref-hentinger1970}{}}%
Hentinger, R., \& Jolivet, J. (1970). Nouvelles déterminations du flux géothermique en france. \emph{Tectonophysics}, \emph{10}, 127--146. \url{https://doi.org/10.1016/0040-1951(70)90103-4}

\leavevmode\vadjust pre{\hypertarget{ref-henyey1968}{}}%
Henyey, T. L. (1968). \emph{Heat flow near major strike-slip faults in central and southern california} (PhD thesis).

\leavevmode\vadjust pre{\hypertarget{ref-henyey1973}{}}%
Henyey, T. L., \& Bischoff, J. L. (1973). Tectonic elements of the northern part of the {Gulf of California}. \emph{Geol. Soc. Am. Bull.}, \emph{84}, 315--330.

\leavevmode\vadjust pre{\hypertarget{ref-henyey1976}{}}%
Henyey, T. L., \& Lee, T. C. (1976). Heat flow in the {Lake Tahoe, California--Nevada}, and the {Sierra Nevada--Basin and Range} transition. \emph{Geol. Soc. Am. Bull.}, \emph{87}, 1179--1187.

\leavevmode\vadjust pre{\hypertarget{ref-henyey1971}{}}%
Henyey, T. L., \& Wasserburg, G. J. (1971). Heat flow near major strike-slip faults in california. \emph{Journal of Geophysical Research}, \emph{76}(32), 7924--7946. \url{https://doi.org/10.1029/JB076i032p07924}

\leavevmode\vadjust pre{\hypertarget{ref-herman1977}{}}%
Herman, B. M., Langseth, M. G., \& Hobart, M. A. (1977). Heat flow in the oceanic crust bounding western africa. \emph{Tectonophysics}, \emph{41}(1-3), 61--77. \url{https://doi.org/10.1016/0040-1951(77)90180-9}

\leavevmode\vadjust pre{\hypertarget{ref-herman1978}{}}%
Herman, B. M., Anderson, R. N., \& Truchan, M. (1978). Extensional tectonics in the okinawa trough: Convergent margins. In \emph{Geological and geophysical investigations of continental margins} (Vol. 29, pp. 199--208). Am. Assoc. Pet. Geol. memoir 29.

\leavevmode\vadjust pre{\hypertarget{ref-herrin1956}{}}%
Herrin, E. T., \& Clark, S. P. (1956). Heat flow in {West Texas and eastern New Mexico}. \emph{Geophysics}, \emph{21}, 1087--1099.

\leavevmode\vadjust pre{\hypertarget{ref-vonherzen1966}{}}%
Herzen, R. P. V., \& Vacquier, V. (1966). Heat flow and magnetic profiles on the {mid-Indian Ocean}. \emph{Phil. Trans. R. Soc. A}, \emph{259}, 262--270.

\leavevmode\vadjust pre{\hypertarget{ref-hobart1974}{}}%
Hobart, M. A., Udintsov, G. B., \& Popova, A. K. (1974). Heat-flow measurements in the east-central atlantic ocean and near the atlantis fracture zone. In \emph{Problems of oceanic rift zone} (p. --). Nauka press.

\leavevmode\vadjust pre{\hypertarget{ref-hobart1975}{}}%
Hobart, M. A., Bunce, E. T., \& Sclater, J. G. (1975). Bottom water flow through the kane gap, sierra leone rise, atlantic ocean. \emph{Journal of Geophysical Research}, \emph{80}, 5083--5088. \url{https://doi.org/10.1029/JC080i036p05083}

\leavevmode\vadjust pre{\hypertarget{ref-hobart1985}{}}%
Hobart, M. A., Langseth, M. G., \& Anderson, R. N. (1985). A geothermal and geophysical survey on the south flank of the {Costa Rican Rift}: Sites 504 and 505. \emph{Initial Reports DSDP}, \emph{83}, 379--404.

\leavevmode\vadjust pre{\hypertarget{ref-honda1979}{}}%
Honda, S., Matsubara, Y., Watanabe, T., Uyeda, S., Shiazaki, K., Nomura, K., \& Fujii, N. (1979). Compilation of eleven new heat flow measurements on {Japanese Islands}. \emph{Bull. Earthquake Res. Inst.}, \emph{54}, 45--73.

\leavevmode\vadjust pre{\hypertarget{ref-horai1964}{}}%
Horai, K. (1964). Studies of the thermal state of the {Earth} the 13th paper: {Terrestristrial Heat Flow in Japan}. \emph{Bulletin of the Earthquake Research Institute, University of Tokyo}, \emph{42}, 93--132.

\leavevmode\vadjust pre{\hypertarget{ref-horai1985}{}}%
Horai, K., \& Von Herzen, R. P. (1985). Measurement of heat flow on {Leg} 86 of the {Deep Sea Drilling Project}. \emph{Initial Reports DSDP}, \emph{86}, 759--776.

\leavevmode\vadjust pre{\hypertarget{ref-horai1970}{}}%
Horai, K., Chapman, M., \& Simmons, G. (1970). Heat flow measurements on the reykjanes ridge. \emph{Nature}, \emph{225}, 264--265. \url{https://doi.org/10.1038/225264a0}

\leavevmode\vadjust pre{\hypertarget{ref-horai1994}{}}%
Horai, K. I., Sasaki, Y., \& Kobayashi, Y. (1994). A relationship between cutoff depth of seismicity and heat flow in the central japan. \emph{Japan Earth and Planetary Science Joint Meeting}, 273.

\leavevmode\vadjust pre{\hypertarget{ref-horvath1977}{}}%
Horvath, F., Erki, I., Bodri, L., \& Marko, L. (1977). \emph{Heat flow measurements in hungary.}

\leavevmode\vadjust pre{\hypertarget{ref-horvath1979}{}}%
Horváth, F., Bodri, L., \& Ottlik, P. (1979). Geothermics of {Hungary} and the tectonophysics of the {Pannonian Basin} "red spot". In \emph{Terrestrial heat flow in europe} (pp. 206--217). Springer Verlag.

\leavevmode\vadjust pre{\hypertarget{ref-houseman1989}{}}%
Houseman, G. A., Cull, J. P., Muir, P. M., \& Paterson, H. L. (1989). Geothermal signatures and uranium ore deposits on the stuart shelf of south australia. \emph{Geophysics}, \emph{54}(2), 158--170. \url{https://doi.org/10.1190/1.1442640}

\leavevmode\vadjust pre{\hypertarget{ref-howard1964}{}}%
Howard, L. E., \& Sass, J. H. (1964). Terrestrial heat flow in {Australia}. \emph{J. Geophys. Res.}, \emph{69}, 1617--1626.

\leavevmode\vadjust pre{\hypertarget{ref-hu2001}{}}%
Hu, S., O'Sullivan, P. B., Raza, A., \& Kohn, B. P. (2001). Thermal history and tectonic subsidence of the bohai basin, northern china: A cenozoic rifted and local pull-apart basin. \emph{Physics of The Earth and Planetary Interiors}, \emph{126}(3-4), 221--235. \url{https://doi.org/10.1016/s0031-9201(01)00257-6}

\leavevmode\vadjust pre{\hypertarget{ref-hueckel1966}{}}%
Hückel, B., \& Kappelmeyer, O. (1966). Geotermische {Untersuchungen im Saarkarbon}. \emph{Z. Deutsch. Geol. Ges.}, \emph{117}, 280--311.

\leavevmode\vadjust pre{\hypertarget{ref-hull1977}{}}%
Hull, D. A., Blackwell, D. D., Bowen, R. G., \& Peterson, N. V. (1977). \emph{Heat flow study of the brothers fault zone, oregon} (No. O-77-03) (p. 38p.). Retrieved from \url{http://www.oregongeology.org/pubs/OG/OGv65n01.pdf}

\leavevmode\vadjust pre{\hypertarget{ref-hurter2002}{}}%
Hurter, S., \& Hänel, R. (2002). \emph{Atlas of geothermal resources in europe} (pp. 92 pp.). European Commission.

\leavevmode\vadjust pre{\hypertarget{ref-hurter1996}{}}%
Hurter, S. J., \& Pollack, H. N. (1996). Terrestrial heat flow in the paraná basin, southern {Brazil}. \emph{J. Geophys. Res.}, \emph{101}, 8659--8671.

\leavevmode\vadjust pre{\hypertarget{ref-hurtig1991}{}}%
Hurtig, E., \& Rockel, W. (1991). \emph{Geothermal atlas of europe} (p. 115). Hermann Haack Verlagsgesellschaft mbH.

\leavevmode\vadjust pre{\hypertarget{ref-hutchison1981}{}}%
Hutchison, I., Louden, K. E., White, R. S., \& Von Herzen, R. P. (1981). Heat flow and age of the gulf of oman. \emph{Earth and Planetary Science Letters}, \emph{56}, 252--262. \url{https://doi.org/10.1016/0012-821x(81)90132-1}

\leavevmode\vadjust pre{\hypertarget{ref-hutchison1985}{}}%
Hutchison, I., Herzen, R. P. V., Louden, K. E., Sclater, J. G., \& Jemsek, S. (1985). Heat flow in the {Belaric and Tyrrhenian basins, western Mediterranean}. \emph{J. Geophys. Res.}, \emph{90}, 685--701.

\leavevmode\vadjust pre{\hypertarget{ref-hutnak2008}{}}%
Hutnak, M., Fisher, A. T., Harris, R., Stein, C., Wang, K., Spinelli, G., et al. (2008). Large heat and fluid flux driven through mid-plate outcrops on ocean crust. \emph{Nature Geoscience}, \emph{1}, 611--614. \url{https://doi.org/10.1038/ngeo264}

\leavevmode\vadjust pre{\hypertarget{ref-hyndman1967}{}}%
Hyndman, R. D. (1967). Heat flow in {Queensland and Northern Territory, Australia}. \emph{J. Geophys. Res.}, \emph{72}, 527--539.

\leavevmode\vadjust pre{\hypertarget{ref-hyndman1976}{}}%
Hyndman, R. D. (1976). Heat flow measurements in the inlets of southwestern {British Columbia}. \emph{J. Geophys. Res.}, \emph{81}, 337--349.

\leavevmode\vadjust pre{\hypertarget{ref-hyndman1968}{}}%
Hyndman, R. D., \& Everett, J. E. (1968). Heat flow measurements in a low radioactivity area of the western australian precambrian shield. \emph{Geophysical Journal of the Royal Astronomical Society}, \emph{14}, 479--486. \url{https://doi.org/0.1111/j.1365-246X.1967.tb06267.x}

\leavevmode\vadjust pre{\hypertarget{ref-hyndman1999}{}}%
Hyndman, R. D., \& Lewis, T. J. (1999). Geophysical consequences of the {Cordillera-Craton} thermal transition in southwestern {Canada}. \emph{Tectonophysics}, \emph{306}, 397--422.

\leavevmode\vadjust pre{\hypertarget{ref-hyndman1972}{}}%
Hyndman, R. D., \& Rankin, D. S. (1972). The {Mid-Atlantic Ridge} near 45\(\circ\)n. XVIII. Heat flow measurements. \emph{Can. J Earth Sci.}, \emph{8}, 664--670.

\leavevmode\vadjust pre{\hypertarget{ref-hyndman1966}{}}%
Hyndman, R. D., \& Sass, J. H. (1966). Geothermal measurements at {Mount Isa, Queensland}. \emph{J. Geophys. Res.}, \emph{71}, 587--601.

\leavevmode\vadjust pre{\hypertarget{ref-hyndman1969}{}}%
Hyndman, R. D., Jaeger, J. C., \& Sass, J. H. (1969). Heat flow measurements on the southeast coast of {Australia}. \emph{Earth Planet. Sci. Lett.}, \emph{7}, 12--16.

\leavevmode\vadjust pre{\hypertarget{ref-hyndman1974}{}}%
Hyndman, R. D., Muecke, G. K., \& Aumento, F. (1974a). Deep drill 1972. Heat flow and heat production in bermuda. \emph{Canadian Journal of Earth Sciences}, \emph{11}, 809--818. \url{https://doi.org/10.1139/e74-081}

\leavevmode\vadjust pre{\hypertarget{ref-hyndman1974b}{}}%
Hyndman, R. D., Erickson, A. J., \& Von Herzen, R. P. (1974b). Geothermal measurements on DSDP leg 26. \emph{Initial Reports DSDP}, \emph{26}, 451--463.

\leavevmode\vadjust pre{\hypertarget{ref-hyndman1978}{}}%
Hyndman, R. D., Rogers, G. C., Bone, M. N., Lister, C. R. B., Wade, U. S., Barrett, D. L., et al. (1978). Geophysical measurements in the region of the explorer ridge offwestern canada. \emph{Canadian Journal of Earth Sciences}, \emph{15}, 1508--1525. \url{https://doi.org/10.1139/e78-156}

\leavevmode\vadjust pre{\hypertarget{ref-hyndman1979}{}}%
Hyndman, R. D., Jessop, A. M., Judge, A. S., \& Rankin, D. S. (1979). Heat flow in the maritime provinces of canada. \emph{Canadian Journal of Earth Sciences}, \emph{16}, 1154--1165. \url{https://doi.org/10.1139/e79-102}

\leavevmode\vadjust pre{\hypertarget{ref-hyndman1982}{}}%
Hyndman, R. D., Lewis, T. J., Wright, J. A., Burgess, M., Chapman, D. S., \& Yamano, M. (1982). Queen charlotte fault zone: Heat flow measurements. \emph{Canadian Journal of Earth Sciences}, \emph{19}, 1657--1669. \url{https://doi.org/10.1139/e82-141}

\leavevmode\vadjust pre{\hypertarget{ref-hyndman1984}{}}%
Hyndman, R. D., Langseth, M. G., \& Von Herzen, R. P. (1984). A review of {Deep Sea Drilling Project} geothermal measurements through {Leg 71}. \emph{Initial Reports DSDP}, \emph{78b}, 813--823.

\leavevmode\vadjust pre{\hypertarget{ref-icerman1984}{}}%
Icerman, L., Swanberg, C. A., Lohse, R. L., Hunter, J. C., \& Gross, J. T. (1984). \emph{Regional geothermal exploration in north central new mexico} (No. Nmerdi 2-69-2208).

\leavevmode\vadjust pre{\hypertarget{ref-imperialcollege}{}}%
Imperial College Heat Flow Group, Univ. of L., Dept. of Geol. (n.d.). \emph{Unpubl. data}.

\leavevmode\vadjust pre{\hypertarget{ref-ingebritsen1989}{}}%
Ingebritsen, S. E., Sherrod, D. R., \& Mariner, R. H. (1989). Heat flow and hydrothermal circulation in the {Cascade Range, north-central Oregon}. \emph{Science}, \emph{243}, 1458--1462.

\leavevmode\vadjust pre{\hypertarget{ref-ingebritsen1993}{}}%
Ingebritsen, S. E., Scholl, M. A., \& Sherrod, D. R. (1993). Heat flow from four new research drill hole in the western {Cascades, Oregon, U.S.A.} \emph{Geothermics}, \emph{22}, 151--163.

\leavevmode\vadjust pre{\hypertarget{ref-ingebritsen1994b}{}}%
Ingebritsen, S. E., Mariner, R. H., \& Sherrod, D. R. (1994). \emph{Hydrothermal systems of the cascade range, north-central oregon} (No. 1044-l).

\leavevmode\vadjust pre{\hypertarget{ref-isaksen2001}{}}%
Isaksen, K., Holmlund, P., Sollid, J. L., \& Harris, C. (2001). Three deep alpine-permafrost boreholes in svalbard and scandinavia. \emph{Permafrost and Periglacial Processes}, \emph{12}(1), 13--25. \url{https://doi.org/10.1002/ppp.380}

\leavevmode\vadjust pre{\hypertarget{ref-ismail1985}{}}%
Ismail, W., \& Yousoff, W. (1985). Heat flow study in the {Malay basin}.

\leavevmode\vadjust pre{\hypertarget{ref-jackson1984}{}}%
Jackson, H. R., Johnson, G. L., Sundvor, E., \& Myhre, A. M. (1984). The yermak plateau: Formed at a triple junction. \emph{Journal of Geophysical Research}, \emph{89}, 3223--3232. \url{https://doi.org/10.1029/JB089iB05p03223}

\leavevmode\vadjust pre{\hypertarget{ref-jaeger1970}{}}%
Jaeger, J. C. (1970). Heat flow and radioactivity in australia. \emph{Earth and Planetary Science Letters}, \emph{8}, 285--292. \url{https://doi.org/10.1016/0012-821x(70)90114-7}

\leavevmode\vadjust pre{\hypertarget{ref-jaeger1964}{}}%
Jaeger, J. C., \& Sass, J. H. (1963). Lees topographic correction in heat flow and the geothermal flux in {Tasmania}. \emph{Geofisica Pura e Applicata}, \emph{54}, 53--63.

\leavevmode\vadjust pre{\hypertarget{ref-jansen1996}{}}%
Jansen, E., Raymo, M. E., \& Blum, P. (1996). \emph{North atlantic--arctic gateways II} (Vol. 162). Ocean Drilling Program.

\leavevmode\vadjust pre{\hypertarget{ref-gsj1997}{}}%
Japan, G. S. of. (1997). \emph{Heat flow map of east and southeast asia}. Geol. Surv. Japan.

\leavevmode\vadjust pre{\hypertarget{ref-jaervimaeki1979}{}}%
Järvimäki, P., \& Puranen, M. (1979). Heat flow measurements in {Finland}. In \emph{Terrestrial heat flow in europe} (pp. 172--178). Springer Verlag.

\leavevmode\vadjust pre{\hypertarget{ref-jaupart1982}{}}%
Jaupart, C., Mann, J. R., \& Simmons, G. (1982). A detailed study of the distribution of heat flow and radioactivity in {New Hampshire}. \emph{Earth Planet. Sci. Lett.}, \emph{59}, 267--287.

\leavevmode\vadjust pre{\hypertarget{ref-jaupart2014}{}}%
Jaupart, C., Mareschal, J.-C., Bouquerel, H., \& Phaneuf, C. (2014). The building and stabilization of an {Archean} craton in the {Superior Province, Canada}, from a heat flow perspective. \emph{Journal of Geophysical Research: Solid Earth}, \emph{119}(12), 9130--9155. \url{https://doi.org/10.1002/2014jb011018}

\leavevmode\vadjust pre{\hypertarget{ref-jemsek1985}{}}%
Jemsek, J., Von Herzen, R. P., Rehault, J.-P., Williams, D. L., \& Sclater, J. (1985). Heat flow and the lithosphere thinning in the {Ligurian Basin, N.W. Mediterranean}. \emph{Geophys. Res. Lett.}, \emph{12}, 693--696.

\leavevmode\vadjust pre{\hypertarget{ref-jemsek1988}{}}%
Jemsek, J. P. (1988). \emph{Heat flow and tectonics of the {Ligurian Sea} basin and margin} (PhD thesis).

\leavevmode\vadjust pre{\hypertarget{ref-jessop1968}{}}%
Jessop, A. M. (1968). Three measurements of heat flow in eastern {Canada}. \emph{Can. J. Earth Sci.}, \emph{5}, 61--68.

\leavevmode\vadjust pre{\hypertarget{ref-jessop1971}{}}%
Jessop, A. M., \& Judge, A. S. (1971). Five measurements of heat flow in southern canada. \emph{Canadian Journal of Earth Sciences}, \emph{8}, 711--716. \url{https://doi.org/10.1139/e71-069}

\leavevmode\vadjust pre{\hypertarget{ref-jessop1978}{}}%
Jessop, A. M., \& Lewis, T. J. (1978). Heat flow and heat generation in the superior province of the canadian shield. \emph{Tectonophysics}, \emph{50}, 55--77. \url{https://doi.org/10.1016/0040-1951(78)90199-3}

\leavevmode\vadjust pre{\hypertarget{ref-jessop1984a}{}}%
Jessop, A. M., Souther, J. G., Lewis, T. J., \& Judge, A. S. (1984). Geothermal measurements in northern british columbia and the southern yukon territory. \emph{Canadian Journal of Earth Sciences}, \emph{21}(5), 599--608.

\leavevmode\vadjust pre{\hypertarget{ref-jiang2016a}{}}%
Jiang, G., Gao, P., Rao, S., Zhang, L.-Y., Tang, X.-Y., Huang, F., \& Zhao, P. (2016a). Compilation of heat flow data in the continental area of {China} (4th edition). \emph{Chinese Journal of Geophysics - Chinese Edition}, 2892--2910. \url{https://doi.org/10.6038/cjg20160815}

\leavevmode\vadjust pre{\hypertarget{ref-jiang2016}{}}%
Jiang, G.-Z., Tang, X.-Y., Rao, S., Gao, P., Zhang, L.-Y., Zhao, P., \& Hu, S.-B. (2016b). High-quality heat flow determination from the crystalline basement of the south-east margin of {North China Craton}. \emph{Journal of Asian Earth Sciences}, \emph{118}, 1--10. \url{https://doi.org/10.1016/j.jseaes.2016.01.009}

\leavevmode\vadjust pre{\hypertarget{ref-jiyang1981}{}}%
Jiyang, C. W. (1981). \emph{Geothermal Studies in China}.

\leavevmode\vadjust pre{\hypertarget{ref-johnson1993}{}}%
Johnson, H. P., Becker, K., \& Herzen, R. P. V. (1993). Near-axis heat flow measurements on the northern {Juan de Fuca} ridge: Implications for fluid circulation in oceanic crust. \emph{Geophys. Res. Lett.}, \emph{20}, 1875--1878.

\leavevmode\vadjust pre{\hypertarget{ref-johnson2010}{}}%
Johnson, H. P., Tivey, M. A., Bjorklund, T. A., \& Salmi, M. S. (2010). Hydrothermal circulation within the {Endeavour Segment, Juan de Fuca Ridge}. \emph{Geochemistry Geophysics Geosystems}, \emph{11}(5), q05002--. \url{https://doi.org/10.1029/2009gc002957}

\leavevmode\vadjust pre{\hypertarget{ref-johnson1997}{}}%
Johnson, P., \& Hutnak, M. (1997). Conductive heat loss in recent eruptions at mid-oceans ridges. \emph{Geophysical Research Letters}, \emph{24}, 3089--3092. \url{https://doi.org/10.1029/97gl02998}

\leavevmode\vadjust pre{\hypertarget{ref-jolivet1989}{}}%
Jolivet, J., Bienfait, G., Vigneresse, J. L., \& Cuney, M. (1989). Heat flow and heat production in {Brittany (western France)}. \emph{Tectonophysics}, \emph{159}, 61--72.

\leavevmode\vadjust pre{\hypertarget{ref-jones1989}{}}%
Jones, F. W., Majorowicz, J. A., \& Embry, A. F. (1989). A heat flow profile across the {Sverdrup Basin}, {Canadian Arctic Islands}. \emph{Geophysics}, \emph{54}, 171--180.

\leavevmode\vadjust pre{\hypertarget{ref-jones1990}{}}%
Jones, F. W., Majorowicz, J. A., Embry, A. F., \& Jessop, A. M. (1990). Geothermal gradients and terrestrial heat flow along a south-north profile in the {Sverdrup Basin, Canadian Arctic Archipelago}. \emph{Geophysics}, \emph{55}, 1105--1107.

\leavevmode\vadjust pre{\hypertarget{ref-jones1987}{}}%
Jones, M. Q. W. (1987). Heat flow and heat production in the namaqua mobile belt, south africa. \emph{Journal of Geophysical Research}, \emph{92}, 6273--6289. \url{https://doi.org/10.1029/JB092iB07p06273}

\leavevmode\vadjust pre{\hypertarget{ref-jones1988}{}}%
Jones, M. Q. W. (1988). Heat flow in the {Witwatersrand Basin} and environs, and its significance for the {South African} shield geotherm and lithospheric thickness. \emph{J. Geophys. Res.}, \emph{93}, 3243--3260.

\leavevmode\vadjust pre{\hypertarget{ref-jones1992}{}}%
Jones, M. Q. W. (1992). Heat flow anomaly in {Lesotho}: Implications for the southern boundary of the {Kaapvaal} craton. \emph{Geophys. Res. Lett.}, \emph{19}, 2031--2034.

\leavevmode\vadjust pre{\hypertarget{ref-jongsma1974}{}}%
Jongsma, D. (1974). Heat flow in the aegean sea. \emph{Geophysical Journal of the Royal Astronomical Society}, \emph{37}, 337--346. \url{https://doi.org/10.1111/j.1365-246X.1974.tb04087.x}

\leavevmode\vadjust pre{\hypertarget{ref-joshima1984}{}}%
Joshima, M. (1984). Heat flow measurement in the GH80-5 area. \emph{Geol. Surv. Japan Cruise Rep.}, \emph{20}, 53--66.

\leavevmode\vadjust pre{\hypertarget{ref-joshima1994}{}}%
Joshima, M. (1994). Heat flow measurements in the eastern japan sea during GH93 cruise, in 1994.

\leavevmode\vadjust pre{\hypertarget{ref-joshima1996}{}}%
Joshima, M. (1996). Heat flow measurements off shakotan peninsula during the r/v hakurei-maru GH95 cruise (pp. 662--662).

\leavevmode\vadjust pre{\hypertarget{ref-joshima1987}{}}%
Joshima, M., \& Honza, E. (1987). Age estimation of the {Soloman Sea} based on heat flow data. \emph{Geo-Marine Lett.}, \emph{6}, 211--217.

\leavevmode\vadjust pre{\hypertarget{ref-joshima1999}{}}%
Joshima, M., \& Kuramoto, S. (1999). Heat flow measurements in the off tokai area. \emph{Geological Survey of Japan Cruise Report}, \emph{24}, 81--86.

\leavevmode\vadjust pre{\hypertarget{ref-judge1967}{}}%
Judge, A. S., \& Beck, A. E. (1967). Anomalous heat flow layer at {London, Ontario}. \emph{Earth Planet. Sci. Lett.}, \emph{3}, 167--170.

\leavevmode\vadjust pre{\hypertarget{ref-logachev2000}{}}%
\emph{k2K cruise report}. (2000).

\leavevmode\vadjust pre{\hypertarget{ref-kappelmeyer1967}{}}%
Kappelmeyer, O. (1967). The geothermal field of the upper {Rhinegraben}. In \emph{The rhinegraben progress report} (Vol. 6, pp. 101--103). Abh. geol. Landesamt.

\leavevmode\vadjust pre{\hypertarget{ref-kasameyer1972}{}}%
Kasameyer, P. W., Von Herzen, R. P., \& Simmons, G. (1972). Heat flow, bathymetry, and the {Mid-Atlantic} ridge at 43\(^\circ\)n. \emph{J. Geophys. Res.}, \emph{77}, 2535--2542.

\leavevmode\vadjust pre{\hypertarget{ref-kashkai1974}{}}%
Kashkai, M. A., \& Aliev, S. A. (1974). Teplovoi potok v kurinskoi depressii (russ.). \emph{Glubinnyi Teplovoi Potok Evropeiskoi Chasti Sssr. Kiev, Naukova Dumka}, 95--109.

\leavevmode\vadjust pre{\hypertarget{ref-kaul2000}{}}%
Kaul, N., Rosenberger, A., \& Villinger, H. (2000). Comparison of measured and {BSR-derived} heat flow values, {Makran} accretionary prism, {Pakistan}. \emph{Marine Geology}, \emph{164}(1-2), 37--51. \url{https://doi.org/10.1016/s0025-3227(99)00125-5}

\leavevmode\vadjust pre{\hypertarget{ref-kaul2006}{}}%
Kaul, N., Foucher, J.-P., \& Heesemann, M. (2006). Estimating mud expulsion rates from temperature measurements on {Hakon Mosby Mud Volcano, SW Barents Sea}. \emph{Marine Geology}, \emph{229}, 1--14.

\leavevmode\vadjust pre{\hypertarget{ref-khutorskoi1979}{}}%
Khutorskoi, M. D. (1979). Termicheskaya razedka mestorozhdenii v usloviyakh strukturno-geologicheskikh neodnorodnostei (in russian). In \emph{Teplovoe pole zemli (trudy vsesoyuznoi konferentsii "narodnokhozyaistvennye i metodicheskie problemy geotermii) T2} (Vol. S, pp. 12--21). Makhachkala.

\leavevmode\vadjust pre{\hypertarget{ref-khutorskoi2003}{}}%
Khutorskoi, M. D., Podgornykh, L. V., Gramberg, I. S., \& Leonov, Y. G. (2003). Thermal tomography of the west arctic basin. \emph{Geotectonics}, \emph{37}, 245--260.

\leavevmode\vadjust pre{\hypertarget{ref-khutorskoi2009}{}}%
Khutorskoi, M. D., Leonov, Yu. G., Ermakov, A. Y., \& Akhmedzyanov, V. R. (2009). Abnormal heat flow and the trough's nature in the northern {Svalbard} plate. \emph{Dokl. Acak. Nauk. SSSR}, \emph{424}, 227-233 (English Trans. 29-35).

\leavevmode\vadjust pre{\hypertarget{ref-khutorskoi1982}{}}%
Khutorskoy, M. D. (1982). Teplovoi potok v oblastyakh strukturno-geologicheskikh neodnorodnostei (russ.). \emph{Trudy Geologicheskogo Instituta An SSSR}, \emph{353}, 78.

\leavevmode\vadjust pre{\hypertarget{ref-khutorskoy1996}{}}%
Khutorskoy, M. D. (1996). \emph{Geothermics of the central-asian fold belt (in russian)} (pp. 332 pp.). RUDN Publ.

\leavevmode\vadjust pre{\hypertarget{ref-khutorskoy1989}{}}%
Khutorskoy, M. D., \& Yarmoluk, V. V. (1989). Heat flow, structure and evolution of the lithosphere of {Mongolia}. \emph{Tectonophysics}, \emph{164}, 315--322.

\leavevmode\vadjust pre{\hypertarget{ref-khutorskoy1986}{}}%
Khutorskoy, M. D., Golubev, V. A., Kozlovtseva, S. V., \& Timareva, S. V. (1986). Glubinny teplovoy potok v mnr (in russian). \emph{Dokl. An. SSSR}, \emph{791}, 939--944.

\leavevmode\vadjust pre{\hypertarget{ref-khutorskoy1990}{}}%
Khutorskoy, M. D., Fernandez, R., Kononov, V. I., Polyak, B. G., Matveev, V. G., \& Rot, A. A. (1990). Heat flow through the sea bottom around the {Yucatan Peninsula}. \emph{J. Geophys. Res.}, \emph{95}, 1223--1237.

\leavevmode\vadjust pre{\hypertarget{ref-kido1993}{}}%
Kido, M., Kinoshita, H., \& Seno, T. (1993). Heat flow measurements in the ayu trough. In \emph{Preliminary report of the hakuho-maru cruise KH 92-1} (pp. 99--105). Ocean Res. Inst., Univ. Tokyo.

\leavevmode\vadjust pre{\hypertarget{ref-kim2007}{}}%
Kim, H. C., \& Lee, Y. (2007). Heat flow in the {Republic of Korea}. \emph{J. Geophys. Res.}, \emph{112}, doi:10.1029/2006JB004266.

\leavevmode\vadjust pre{\hypertarget{ref-kim2010}{}}%
Kim, Y.-G., Lee, S.-M., \& Matsubayashi, O. (2010). New heat flow measurements in the ulleung basin, east sea (sea of japan): Relationship to local BSR depth, and implications for regional heat flow distribution. \emph{Geo-Mar Lett}, --. \url{https://doi.org/10.1007/s00367-010-0207-x}

\leavevmode\vadjust pre{\hypertarget{ref-kimura1997}{}}%
Kimura, G., Silver, E., Blum, P., \& Party, S. S. (1997). Leg 170. In \emph{Proceedings of the ocean drilling program, initial reports} (Vol. 170, pp. 7--17).

\leavevmode\vadjust pre{\hypertarget{ref-kinoshita1986}{}}%
Kinoshita, H., \& Yamano, M. (1986). The heat flow anomaly in the {Nankai Trough} area. \emph{Initial Reports DSDP}, \emph{87}, 737--743.

\leavevmode\vadjust pre{\hypertarget{ref-kinoshita1989}{}}%
Kinoshita, H., Kasumi, Y., \& Baba, H. (1989). Report on DELP 1987 cruises in the ogasawara area. Part VI: Heat flow measurements. \emph{Bulletin of the Earthquake Research Institute, University of Tokyo}, \emph{64}, 223--232.

\leavevmode\vadjust pre{\hypertarget{ref-kinoshita1987}{}}%
Kinoshita, M. (1987). \emph{Heat flow measurements in some western pacific trench-arc-backarc systems and their interpretation} (Master's thesis).

\leavevmode\vadjust pre{\hypertarget{ref-kinoshita2004}{}}%
Kinoshita, M. (2004). Personal communication. In \emph{CD rom: Geothermal gradient and heat flow data in and around japan} (p. --). Geological Survey of Japan, AIST, 2004.

\leavevmode\vadjust pre{\hypertarget{ref-kinoshita1995}{}}%
Kinoshita, M., \& Yamano, M. (1995). Heat flow distribution in the nankai trough region. In \emph{Geology and geophysics of the philippine sea} (pp. 77--86). Terrapub.

\leavevmode\vadjust pre{\hypertarget{ref-kinoshita1997}{}}%
Kinoshita, M., \& Yamano, M. (1997). Hydrothermal regime and constraints on reservoir depth of the {Jade} site in the {Mid-Okinawa Trough} inferred from heat flow measurements. \emph{J. Geophys. Res.}, \emph{102}, 3183--3194.

\leavevmode\vadjust pre{\hypertarget{ref-kinoshita1990}{}}%
Kinoshita, M., Yamano, M., Post, J., \& Halbach, P. (1990). Heat flow measurements in the southern and middle okinawa trough on r/v sonne in 1988. \emph{Bull. Earthq. Res. Inst.}, \emph{65}(3), 571--588. Retrieved from \url{http://ci.nii.ac.jp/naid/120000871865}

\leavevmode\vadjust pre{\hypertarget{ref-kinoshita1991b}{}}%
Kinoshita, M., Yamano, M., \& Makita, S. (1991a). High heat-flow anomaly around {Hatsushima} biological community in the western {Sagami Bay, Japan}. \emph{J. Phys. Earth}, \emph{39}, 553--571.

\leavevmode\vadjust pre{\hypertarget{ref-kinoshita1991}{}}%
Kinoshita, M., Yamano, M., Kasumi, Y., \& Baba, H. (1991b). Report on DELP 1988 cruises in the okinawa trough. Part 8: Heat flow measurements. \emph{Bull. Earthq. Res. Inst.}, \emph{66}, 221--228.

\leavevmode\vadjust pre{\hypertarget{ref-kinoshita2006}{}}%
Kinoshita, M., Kawada, Y., Tanaka, A., \& Urabe, T. (2006). Recharge/discharge interface of a secondary hydrothermal circulation in the {Suiyo Seamount} of the {Izu-Bonin} arc, identified by submersible-operated heat flow measurements. \emph{Earth Planet. Sci. Lett.}, \emph{245}, 498--508.

\leavevmode\vadjust pre{\hypertarget{ref-kirkby2010}{}}%
Kirkby, A., \& Gerner, E. (2010). \emph{Heat flow interpretations for the australian continent: Release 1}. Geoscience Australia. Retrieved from \url{http://pid.geoscience.gov.au/dataset/ga/71211}

\leavevmode\vadjust pre{\hypertarget{ref-kissin1964}{}}%
Kissin, I. G. (1964). Vostochno-predkavkazskii artezianskii bassein. Mosk- va. \emph{Nauka.}

\leavevmode\vadjust pre{\hypertarget{ref-kitajima1997}{}}%
Kitajima, T., Kobayashi, Y., Suzuki, H., Ikeda, R., Omura, K., Kasahara, K., \& Okada, Y. (1997). Thermal structure and earthquakes beneath the kanto district. \emph{Japan Earth and Planetary Science Joint Meeting}, \emph{Abstracts}, 247.

\leavevmode\vadjust pre{\hypertarget{ref-kitajima2001}{}}%
Kitajima, T., Kobayashi, Y., Ikeda, R., Iio, Y., \& Omura, K. (2001). Terrestrial heat flow at hirabayashi on awaji island, south-west japan. \emph{Island Arc}, \emph{10}, 318--325. \url{https://doi.org/10.1111/j.1440-1738.2001.00330.x}

\leavevmode\vadjust pre{\hypertarget{ref-kobolev1993}{}}%
Kobolev, V. P., Kutas, R. I., Tsvyashchenko, V. A., Kravchuk, O. P., \& Bevzyuk, M. I. (1993). Geothermal studies in the {NW Black Sea} (in russian). \emph{Geophys. J.}, \emph{15}, 67--72.

\leavevmode\vadjust pre{\hypertarget{ref-kondyurin1983}{}}%
Kondyurin, A. V., \& Sochelnikov, V. V. (1983). Geotermicheskii potok v zapadnoi chasti chernogo morya. T. 23, vyp. 4,(russ.). \emph{Okeanologiya}, 622--627.

\leavevmode\vadjust pre{\hypertarget{ref-kono1971}{}}%
Kono, Y., \& Kobayashi, Y. (1971). Terrestrial heat flow in hokuriku district, central japan. \emph{Sci. Rep. Kanazawa. Univ.}, \emph{16}, 61--72.

\leavevmode\vadjust pre{\hypertarget{ref-kopf2006}{}}%
Kopf, A., Alves, T., Heesemann, B., Irving, M., Kaul, N. E., Kock, L., et al. (2006). \emph{Report and preliminary results of poseidon cruise P336: Crests - cretan sea tectonics and sedimentation} (No. 253) (p. 140). Retrieved from \url{http://www.geo.uni-bremen.de/FB5/Sensorik/publikationen/P336/_cruisereport.pdf}

\leavevmode\vadjust pre{\hypertarget{ref-korgen1971}{}}%
Korgen, B. J., Bodvarsson, G., \& Mesecar, R. S. (1971). Heat flow through the floor of the cascadia basin. \emph{Journal of Geophysical Research}, \emph{76}, 4758--4774.

\leavevmode\vadjust pre{\hypertarget{ref-kral1985}{}}%
Kral, M., Lizon, I., \& Janci, J. (1985). \emph{Geotermicky vyskrum ssr. Zav. Sprava za roky 1981 az 1985 (in slovak)}.

\leavevmode\vadjust pre{\hypertarget{ref-kubik1986}{}}%
Kubik, J., \& Cermak, V. (1986). Heat flow in the {Upper Silurian} coal basin: Re-evaluation of data with special attention to the lithology. \emph{Studia Geoph. Et Geod.}, \emph{30}, 376--393.

\leavevmode\vadjust pre{\hypertarget{ref-kukkonen1991}{}}%
Kukkonen, I., \& Järvimäki, P. (1991). Catalogue of heat flow density data: finland. In \emph{Geothermal atlas of europe} (p. 112). Hermann Haack Verlagsgesellschaft mbH.

\leavevmode\vadjust pre{\hypertarget{ref-kukkonen1987}{}}%
Kukkonen, I. T. (1987). Vertical variation of apparent and paleoclimatically corrected heat flow densities in the {Central Baltic Shield}. \emph{J. Geodynamics}, \emph{8}, 33--53.

\leavevmode\vadjust pre{\hypertarget{ref-kukkonen1988}{}}%
Kukkonen, I. T. (1988). Terrestrial heat flow and groundwater circulation in the bedrock in the central {Baltic Shield}. \emph{Tectonophysics}, \emph{156}, 59--74. \url{https://doi.org/10.1016/0040-1951(88)90283-1}

\leavevmode\vadjust pre{\hypertarget{ref-kukkonen1989}{}}%
Kukkonen, I. T. (1989a). Terrestrial heat flow and radiogenic heat production in {Finland}, the central {Baltic} shield. \emph{Tectonophysics}, \emph{164}, 219--230.

\leavevmode\vadjust pre{\hypertarget{ref-kukkonen1989b}{}}%
Kukkonen, I. T. (1989b). \emph{Terrestrial heat flow in finland, the central fennoscandian shield} (No. Report YST-68).

\leavevmode\vadjust pre{\hypertarget{ref-kukkonen1993}{}}%
Kukkonen, I. T. (1993). Heat flow map of northern and central parts of the {Fennoscandian} shield based on geochemical surveys of heat producing elements. \emph{Tectonophysics}, \emph{225}, 3--13.

\leavevmode\vadjust pre{\hypertarget{ref-kukkonen1998}{}}%
Kukkonen, I. T., Gosnold, W. D., \& Safanda, J. (1998). Anomalously low heat flow density in eastern karelia, baltic shield: A possible palaeoclimatic signature. \emph{Tectonophysics}, \emph{291}(1-4), 235--249. \url{https://doi.org/10.1016/s0040-1951(98)00043-2}

\leavevmode\vadjust pre{\hypertarget{ref-kurchikov1982}{}}%
Kurchikov, A. R. (1982). Paleogeotermicheskie usloviya formirovaniya zon preimu- shchestvennogo nefte- (russ.). \emph{I Gazonakopleniya V Zapadnoy Sibiri. - Tumen}, 18p.

\leavevmode\vadjust pre{\hypertarget{ref-kurchikov1981}{}}%
Kurchikov, A. R., \& Stavitsky, B. P. (1981). Teplovoy potok v predelakh zapadno-sibir- skoy plity (russ.). \emph{Problemy Nefti I Gaza Tumeny, Tumen}, \emph{51}, 11--14.

\leavevmode\vadjust pre{\hypertarget{ref-kurchikov1987}{}}%
Kurchikov, A. R., \& Stavitsky, B. P. (1987). Geotermiya neftegazonosnykh oblastey zapadnoy sibiri. - moscow. \emph{Izdatelstvo Nedra}.

\leavevmode\vadjust pre{\hypertarget{ref-kutas1970}{}}%
Kutas, R. I., \& Gordienko, V. V. (1970). Teplovoe pole i glubinnoe stroenie vos- tochnykh karpat (russ.). \emph{Geofizicheskii Sbornik}, \emph{34}, 29--41.

\leavevmode\vadjust pre{\hypertarget{ref-kutas1971}{}}%
Kutas, R. I., \& Gordienko, V. V. (1971). Teplovoe pole ukrainy (russ.). \emph{Kiev Naukova Dumka}, \emph{140}.

\leavevmode\vadjust pre{\hypertarget{ref-kutas1973}{}}%
Kutas, R. I., \& Gordienko, V. V. (1973). Novye dannye o teplovom potoke yugo- zapadnoi chasti ukrainy (russ.). \emph{Geofiziczeskii Sbornik}, \emph{56}, 35--40.

\leavevmode\vadjust pre{\hypertarget{ref-kutas1979}{}}%
Kutas, R. I., Bevzyuk, M. I., \& Vygovsky, V. F. (1975). Heat flow and heat transfer conditions in the bottom sediments of equatorial indian ocean. \emph{Geothermics}, \emph{4}, 8--13. \url{https://doi.org/10.1016/0375-6505(79)90064-6}

\leavevmode\vadjust pre{\hypertarget{ref-kutas1992}{}}%
Kutas, R. I., Kobolev, V. P., Tsvyashchenko, V. A., Vasilyev, A. D., \& Kravchuk, O. P. (1992). New determination of heat flow in the bulgarian sector of the black sea (in ukranian). \emph{Dopovidi Akademii Nauk Ukrainy}, \emph{7}, 104--107.

\leavevmode\vadjust pre{\hypertarget{ref-kutas1999}{}}%
Kutas, R. I., Kobolev, V. P., Tsvyashchenko, V. A., Bevzyuk, M. I., \& Kravchuk, O. P. (1999). Results of heat flow determinations in the northwestern {Black Sea} basin (in russian). \emph{Geophys. J.}, \emph{2}, 38--51.

\leavevmode\vadjust pre{\hypertarget{ref-kutas2003}{}}%
Kutas, R. I., Kobolev, V. P., Bevzyuk, M. I., \& Kravchuk, O. P. (2003). New heat flow determinations in the northwestern {Black Sea} (in russian). \emph{Geophys. J.}, \emph{2}, 48--52.

\leavevmode\vadjust pre{\hypertarget{ref-kuzmin1972}{}}%
Kuzmin, V. A., Suzyumov, A. E., \& Bezludov, A. V. (1972). Geothermic soundings on the manihiki plateau and the marcus-necker rise (the pacific ocean). \emph{Okeanologiya}, \emph{12}, 1044--1046.

\leavevmode\vadjust pre{\hypertarget{ref-lachenbruch1957}{}}%
Lachenbruch, A. H. (1957). Thermal effects of the ocean on permafrost. \emph{Bull. Geol. Soc. Am.}, \emph{68}, 1515--1529.

\leavevmode\vadjust pre{\hypertarget{ref-landstroem1980}{}}%
Landström, O., Larson, S. Å., Lind, G., \& Malmqvist, D. (1980). Geothermal investigations in the bohus granite area in southwestern sweden. \emph{Tectonophysics}, \emph{64}(1-2), 131--162. \url{https://doi.org/10.1016/0040-1951(80)90266-8}

\leavevmode\vadjust pre{\hypertarget{ref-langseth1964}{}}%
Langseth, M. G., \& Grim, P. J. (1964). New heat-flow measurements in the {Caribbean and western Atlantic}. \emph{J. Geophys. Res.}, \emph{69}, 4916--4917.

\leavevmode\vadjust pre{\hypertarget{ref-langseth1981}{}}%
Langseth, M. G., \& Herman, B. M. (1981). Heat transfer in the oceanic crust of the {Brazil Basin}. \emph{J. Geophys. Res.}, \emph{86}, 10805--10819.

\leavevmode\vadjust pre{\hypertarget{ref-langseth1976}{}}%
Langseth, M. G., \& Hobart, M. A. (1976). Interpretation of heat flow measurements in the VEMA fracture zone. \emph{Geophysical Research Letters}, \emph{3}, 241--244. \url{https://doi.org/10.1029/GL003i005p00241}

\leavevmode\vadjust pre{\hypertarget{ref-langseth1983}{}}%
Langseth, M. G., \& Ludwig, W. J. (1983). A heat flow measurement on the {Falkland Plateau}. \emph{Initial Reports DSDP}, \emph{71}, 299--303.

\leavevmode\vadjust pre{\hypertarget{ref-langseth1996}{}}%
Langseth, M. G., \& Silver, E. A. (1996). The nicoya convergent margin---a region of exceptionally low heat flow. \emph{Geophys. Res. Lett.}, \emph{23}, 891--894.

\leavevmode\vadjust pre{\hypertarget{ref-langseth1967}{}}%
Langseth, M. G., \& Taylor, P. T. (1967). Recent heat flow measurements in the {Indian Ocean}. \emph{J. Geophys. Res.}, \emph{72}, 6249--6260.

\leavevmode\vadjust pre{\hypertarget{ref-langseth1974}{}}%
Langseth, M. G., \& Zielinski, G. W. (1974). Marine heat flow measurements in the {Norwegian--Greenland Sea} and in the vicinity of {Iceland}. In \emph{Geodynamics of iceland and northern atlantic area: Proceedings of the NATO advanced study institute held in reykjavik, iceland} (pp. 277--295). Reidel.

\leavevmode\vadjust pre{\hypertarget{ref-langseth1965}{}}%
Langseth, M. G., Grim, P. J., \& Ewing, M. (1965). Heat-flow measurements in the {East Pacific Ocean}. \emph{J. Geophys. Res.}, \emph{70}, 367--380.

\leavevmode\vadjust pre{\hypertarget{ref-langseth1966}{}}%
Langseth, M. G., Lepichon, X., \& Ewing, M. (1966). Crustal structure of the mid-ocean ridges. {5. Heat flow through the Atlantic Ocean floor, and convection currents}. \emph{J. Geophys. Res.}, \emph{71}, 5321--5355.

\leavevmode\vadjust pre{\hypertarget{ref-langseth1970}{}}%
Langseth, M. G., Malone, I., \& Berger, D. (1970). \emph{Sea floor geothermal measurements from VEMA cruise 23} (No. 2-cu-2-70, (Ntis Ad 718826)) (Vol. 2--Cu--2--70, (NTIS AD 718826), p. --).

\leavevmode\vadjust pre{\hypertarget{ref-langseth1971}{}}%
Langseth, M. G., Malone, I., \& Berger, D. (1971). \emph{Sea floor geothermal measurements form {Vema} cruise 24} (No. 3-cu-3-71, (Ntis Ad 729682)).

\leavevmode\vadjust pre{\hypertarget{ref-langseth1972}{}}%
Langseth, M. G., Malone, I., \& Berger, D. (1972). \emph{Sea floor geothermal measurements from VEMA cruise 25} (No. 4-cu-4-72, (Ntis Ad 748309)) (Vol. 4--Cu--4--72, (NTIS AD 748309), pp. 168 pp).

\leavevmode\vadjust pre{\hypertarget{ref-langseth1980}{}}%
Langseth, M. G., Hobart, M. A., \& Horai, K. (1980). Heat flow in the bering sea. \emph{Journal of Geophysical Research}, \emph{85}, 3740--3750. \url{https://doi.org/10.1029/JB085iB07p03740}

\leavevmode\vadjust pre{\hypertarget{ref-langseth1988a}{}}%
Langseth, M. G., Westbrook, G. K., \& Hobart, M. A. (1988a). Geophysical survey of a mud volcano seaward of the {Barbados} ridge accretionary complex. \emph{J. Geophys. Res.}, \emph{93}, 1049--1061.

\leavevmode\vadjust pre{\hypertarget{ref-langseth1988b}{}}%
Langseth, M. G., Mottl, M. J., Hobart, M. A., \& Fisher, A. (1988b). The distribution of geothermal and geochemical gradients near site 501/504: Implications for hydrothermal circulation in the oceanic crust. \emph{Proc. ODP Initial Reports (Pt. A)}, \emph{111}, 23--32.

\leavevmode\vadjust pre{\hypertarget{ref-langseth1990}{}}%
Langseth, M. G., Westbrook, G. K., \& Hobart, M. (1990). Contrasting geothermal regimes of the {Barbados Ridge} accretionary complex. \emph{J. Geophys. Res.}, \emph{95}, 8829--8843.

\leavevmode\vadjust pre{\hypertarget{ref-langseth1992}{}}%
Langseth, M. G., Becker, K., Von Herzen, R. P., \& Schultheiss, P. (1992). Heat and fluid flux through sediment on the western flank of the {Mid-Atlantic Ridge}: A hydrogeological study of north pond. \emph{Geophys. Res. Lett.}, \emph{19}, 517--520.

\leavevmode\vadjust pre{\hypertarget{ref-larue1987}{}}%
Larue, B. M., \& Foucher, J. P. (1987). Evidence for injection of hot material during early stages of opening of pull-apart graben: Example from the {Sunda Strait, Indonesia}. In \emph{Proceedings of indonesia--france seminar on sunda strait} (pp. 15--21).

\leavevmode\vadjust pre{\hypertarget{ref-latilbrun1988}{}}%
Latil-Brun, M. V., \& Lucazeau, F. (1988). Subsidence, extension and thermal history of the west african margin in senegal. \emph{Earth and Planetary Science Letters}, \emph{90}(2), 204--220. \url{https://doi.org/10.1016/0012-821x(88)90101-x}

\leavevmode\vadjust pre{\hypertarget{ref-lavenia1967}{}}%
Lavenia, A. (1967). Heat flow measurements through bottom sediments in the southern {Adriatic Sea}. \emph{Boll. Geofis. Teor. Appl.}, \emph{9}(36), 323--332.

\leavevmode\vadjust pre{\hypertarget{ref-law1965}{}}%
Law, L. K., Paterson, W. S. B., \& Whitham, K. (1965). Heat flow determinations in the {Canadian} arctic archipelago. \emph{Can. J. Earth Sci.}, \emph{2}, 59--71.

\leavevmode\vadjust pre{\hypertarget{ref-lawver1987}{}}%
Lawver, L. A., \& Taylor, P. T. (1987). Heat flow off sumatra. In \emph{Marine geophysics : A navy symposium} (pp. 67--76).

\leavevmode\vadjust pre{\hypertarget{ref-lawver1979}{}}%
Lawver, L. A., \& Williams, D. L. (1979). Heat flow in the central gulf of california. \emph{Journal of Geophysical Research}, \emph{84}, 3465--3478. \url{https://doi.org/10.1029/JB084iB07p03465}

\leavevmode\vadjust pre{\hypertarget{ref-lawver1973}{}}%
Lawver, L. A., Sclater, J. G., Henyey, T. L., \& Rogers, J. (1973). Heat flow measurements in the southern portion of the gulf of california. \emph{Earth and Planetary Science Letters}, \emph{19}, 198--208. \url{https://doi.org/10.1016/0012-821x(73)90115-5}

\leavevmode\vadjust pre{\hypertarget{ref-lawver1975}{}}%
Lawver, L. A., Williams, D. L., \& Von Herzen, R. P. (1975). A major geothermal anomaly in the {Gulf of California}. \emph{Nature}, \emph{257}, 23--28.

\leavevmode\vadjust pre{\hypertarget{ref-lawver1982}{}}%
Lawver, L. A., Loy, W., Sclater, J. G., \& Von Herzen, Richard P. (1982). Heat flow in the east scotia sea. \emph{Antarctic Journal}, \emph{16}, 106--107.

\leavevmode\vadjust pre{\hypertarget{ref-lawver1991}{}}%
Lawver, L. A., B.Della Vedova, \& Von Herzen, R. P. (1991). Heat flow in {Jane Basin}, northwest {Weddell Sea}. \emph{J. Geophys. Res.}, \emph{96}, 2019--2038.

\leavevmode\vadjust pre{\hypertarget{ref-lawver1994}{}}%
Lawver, L. A., Williams, T., \& Sloan, B. J. (1994). Seismic stratigraphy and heat flow of powell basin. \emph{Terra Antartica}, \emph{1}, 309--310.

\leavevmode\vadjust pre{\hypertarget{ref-lawver1995}{}}%
Lawver, L. A., Keller, R. A., Fisk, M. R., \& Strelin, J. A. (1995). Bransfield strait, antarctic peninsula: Active extension behind a dead arc. In \emph{Backarc basins: Tectonics and magmatism}. Plenium Press.

\leavevmode\vadjust pre{\hypertarget{ref-ldeo2004}{}}%
Ldeo. (2004). Lamont-doherty earth observatory. In \emph{CD rom: Geothermal gradient and heat flow data in and around japan}. Geological Survey of Japan, AIST.

\leavevmode\vadjust pre{\hypertarget{ref-legal2018}{}}%
Le Gal, V., Lucazeau, F., Cannat, M., Poort, J., Monnin, C., Battani, A., et al. (2018). Heat flow, morphology, pore fluids and hydrothermal circulation in a typical {Mid-Atlantic Ridge} flank near {Oceanographer Fracture Zone}. \emph{Earth and Planetary Science Letters}, \emph{482}, 423--433. \url{https://doi.org/10.1016/j.epsl.2017.11.035}

\leavevmode\vadjust pre{\hypertarget{ref-lemarne1962}{}}%
Le Marne, A. E., \& Sass, J. H. (1962). Heat flow at {Cobar, New South Wales}. \emph{J. Geophys. Res.}, \emph{67}, 3981--3983.

\leavevmode\vadjust pre{\hypertarget{ref-lepichon1971}{}}%
Le Pichon, X., Eittreim, S. L., \& Ludwig, W. J. (1971). Sediment transport and distribution in the {Argentine Basin} - 1 - {Antarctic} bottom current passage through the {Faulkland Fracture Zone}. \emph{Physics and Chem. Earth}, \emph{8}, 3--28.

\leavevmode\vadjust pre{\hypertarget{ref-lebedev1967}{}}%
Lebedev, T. S., Gordienko, V. V., \& Kutas, R. I. (1967). Geotermicheskie usloviya kryma. - geofizicheskii sbornik. 1967. \emph{Vyp.}

\leavevmode\vadjust pre{\hypertarget{ref-lee1986}{}}%
Lee, C. R., \& Cheng, W. T. (1986). Preliminary heat flow measurements in taiwan. \emph{Fourth Circum-Pacific Energy and Mineral Resources Conference}.

\leavevmode\vadjust pre{\hypertarget{ref-lee1975}{}}%
Lee, T. C., \& Henyey, T. L. (1975). Heat flow through the southern california borderland. \emph{Journal of Geophysical Research}, \emph{80}, 3733--3743. \url{https://doi.org/10.1029/JB080i026p03733}

\leavevmode\vadjust pre{\hypertarget{ref-lee1977}{}}%
Lee, T. C., \& Von Herzen, R. P. (1977). A composite {trans-Atlantic} heat flow profile between 20\(^\circ\)s and 35\(^\circ\)s. \emph{Earth Planet. Sci. Lett.}, \emph{35}, 123--133.

\leavevmode\vadjust pre{\hypertarget{ref-lee1999}{}}%
Lee, Y., \& Deming, D. (1999). Heat flow and thermal history of the anadarko basin and the western oklahoma platform. \emph{Tectonophysics}, \emph{313}, 389--410.

\leavevmode\vadjust pre{\hypertarget{ref-lee1996}{}}%
Lee, Y., Deming, D., \& Chen, K. F. (1996). Heat flow and heat production in the arkoma basin and oklahoma platform, southeastern oklahoma. \emph{Journal of Geophysical Research}, \emph{101}(b11), 25387--25401. \url{https://doi.org/10.1029/96jb02532}

\leavevmode\vadjust pre{\hypertarget{ref-leinen1986}{}}%
Leinen, M. (1986). \emph{Initial Reports DSDP}, \emph{92}, 47--53, 108--108, 130--132, 169--173, 197--206.

\leavevmode\vadjust pre{\hypertarget{ref-lekuthai1995}{}}%
Lekuthai, T., Charusirisawad, R., \& Vacher, M. (1995). Heat flow map of the gulf of thailand. \emph{CCOP Tech. Bull.}, \emph{25}, 63--78. Retrieved from \url{http://www.gsj.jp/en/publications/ccop-bull/ccop-vol25.html}

\leavevmode\vadjust pre{\hypertarget{ref-lesquer1983}{}}%
Lesquer, A., Pagel, M., Orsini, J., \& Bonin, B. (1983). Premières déterminations du flux de chaleur et de la production de chaleur en corse. \emph{Compte-Rendus de l'Académie Des Sciences, Série II}, \emph{297}, 491--494.

\leavevmode\vadjust pre{\hypertarget{ref-lesquer1988}{}}%
Lesquer, A., Bourmatte, A., \& Dautria, J. M. (1988). Deep structure of the {Hoggar} domal uplift {(Central Sahara, south Algeria)} from gravity, thermal and petrological data. \emph{Tectonophysics}, \emph{152}, 71--87.

\leavevmode\vadjust pre{\hypertarget{ref-lesquer1989}{}}%
Lesquer, A., Bourmatte, A., Ly, S., \& Daturia, J. M. (1989). First heat flow determination from the central {Sahara}: Relationship with the {Pan-African} belt and {Hoggar} domal uplift. \emph{J. Afr. Earth. Sci.}, \emph{9}, 41--48.

\leavevmode\vadjust pre{\hypertarget{ref-lesquer1991}{}}%
Lesquer, A., Villeneuve, J. C., \& Bronner, G. (1991). Heat flow data from the western margin of the {West African} craton {(Mauritania)}. \emph{Phys. Earth Planet. Int.}, \emph{66}, 320--329.

\leavevmode\vadjust pre{\hypertarget{ref-levchenko1981}{}}%
Levchenko, A. I. (1981). Geotermicheskie usloviya gazokondensatnykh mestorogde- niy severa tumenskoy oblasti. - 2 vsesoyuznaya nauchno-tekhn.konfer. "Problemy gornoy teplofiziki".

\leavevmode\vadjust pre{\hypertarget{ref-levitte1984}{}}%
Levitte, D., Maurath, G., \& Eckstei, Y. (1984). Terrestrial heat flow in a 3.5 km deep borehole in the jordan--dead sea rift valley. In \emph{Ann. Meet. abstr.} (Vol. 16, p. 575). Geol. Soc. Am.

\leavevmode\vadjust pre{\hypertarget{ref-levy2010}{}}%
Lévy, F., Jaupart, C., Mareschal, J.-C., Bienfait, G., \& Limare, A. (2010). Low heat flux and large variations of lithospheric thickness in the canadian shield. \emph{Journal of Geophysical Research}, \emph{115}(b6), b06404--. \url{https://doi.org/10.1029/2009jb006470}

\leavevmode\vadjust pre{\hypertarget{ref-lewis1983}{}}%
Lewis, B. T. R. (1983). Temperatures, heat flow and lithospheric cooling at the mouth of the {Gulf of California}. \emph{Initial Reports DSDP}, \emph{65}, 343.

\leavevmode\vadjust pre{\hypertarget{ref-lewis1981}{}}%
Lewis, J. F., \& Jessop, A. M. (1981). Heat flow in the garibaldi volcanic belt, a possible canadian geothermal resource area. \emph{Canadian Journal of Earth Sciences}, \emph{18}, 366--375. \url{https://doi.org/10.1139/e81-028}

\leavevmode\vadjust pre{\hypertarget{ref-lewis1969}{}}%
Lewis, T. J. (1969). Terrestrial heat flow at {Eldorado, Saskatchewan}. \emph{Canadian Journal of Earth Sciences}, \emph{6}(5), 1191--1197. \url{https://doi.org/10.1139/e69-120}

\leavevmode\vadjust pre{\hypertarget{ref-lewis1984}{}}%
Lewis, T. J. (1984). Geothermal energy from penticton tertiary outlier, british columbia: An initial assessment. \emph{Canadian Journal of Earth Sciences}, \emph{21}, 181--188. \url{https://doi.org/10.1139/e84-019}

\leavevmode\vadjust pre{\hypertarget{ref-lewis1977}{}}%
Lewis, T. J., \& Beck, A. E. (1977). Analysis of heat flow data---detailed observations in many holes in a small area. \emph{Tectonophysics}, \emph{41}, 41--59.

\leavevmode\vadjust pre{\hypertarget{ref-lewis1976}{}}%
Lewis, T. J., \& Hyndman, R. D. (1976). Oceanic heat flow measurements over the continental margins of eastern canada. \emph{Canadian Journal of Earth Sciences}, \emph{13}(8), 1031--1038. \url{https://doi.org/10.1139/e76-106}

\leavevmode\vadjust pre{\hypertarget{ref-lewis1992}{}}%
Lewis, T. J., \& Wang, K. (1992). Influence of terrain on bedrock temperatures. \emph{Palaeogeo. Palaeoclim. Palaeoeco.}, \emph{98}, 87--100.

\leavevmode\vadjust pre{\hypertarget{ref-lewis1985}{}}%
Lewis, T. J., Jessop, A. M., \& Judge, A. S. (1985). Heat flux measurements in southwestern british columbia: The thermal consequences of plate tectonics. \emph{Canadian Journal of Earth Sciences}, \emph{22}(9), 1262--1273. \url{https://doi.org/10.1139/e85-131}

\leavevmode\vadjust pre{\hypertarget{ref-lewis1988}{}}%
Lewis, T. J., Bentkowski, W. H., Davis, E. E., Hyndman, R. D., Souther, J. G., \& Wright, J. A. (1988). Subduction of the juan de fuca plate: Thermal consequences. \emph{Journal of Geophysical Research}, \emph{93}, 15207--15225. \url{https://doi.org/10.1029/JB093iB12p15207}

\leavevmode\vadjust pre{\hypertarget{ref-lewis2003}{}}%
Lewis, T. J., Hyndman, R. D., \& Flü"ck, P. (2003). Heat flow, heat generation, and crustal temperatures in the northern {Canadian Cordillera}: Thermal controls on tectonics. \emph{J. Geophys. Res.}, \emph{108}, doi:10.1029/2002JB002090.

\leavevmode\vadjust pre{\hypertarget{ref-leyden1978}{}}%
Leyden, R., Damuth, J. E., Ongley, L. K., Kostecki, J., \& Van Stevenick, W. (1978). Salt diapirs and {São Paulo Plateau}, southeastern {Brazilian} continental margin. \emph{AAPG Bull.}, \emph{62}, 657--666.

\leavevmode\vadjust pre{\hypertarget{ref-li2014}{}}%
Li, W.-W., Rao, S., Tang, X.-Y., Jiang, G.-Z., Hu, S.-B., Kong, Y.-L., et al. (2014). Borehole temperature logging and temperature field in the xiongxian geothermal field, hebei province. \emph{Scientia Geologica Sinica}, \emph{49}(3), 850--863. \url{https://doi.org/10.3969/j.issn.0563-5020.2014.03.012}

\leavevmode\vadjust pre{\hypertarget{ref-li1989}{}}%
Li, X., Furukawa, Y., Nagao, T., Uyeda, S., \& Suzuki, H. (1989). Heat flow in central japan and its relations to geological and geophysical features. \emph{Bull. Earthq. Res. Inst.}, \emph{64}, 1--36.

\leavevmode\vadjust pre{\hypertarget{ref-li2015a}{}}%
Li, Z.-X., Gao, J., Zheng, C., Liu, C.-L., Ma, Y.-S., \& Zhao, W.-Y. (2015). Present-day heat flow and tectonic-thermal evolution since the late paleozoic time of the qaidam basin. \emph{Chinese Journal Geophysics}, \emph{58}(10), 3687--3705. \url{https://doi.org/10.6038/cjg20151021}

\leavevmode\vadjust pre{\hypertarget{ref-liang1987}{}}%
Liang, S. (1987). \emph{Heat flow values of the 5th ggt in china}.

\leavevmode\vadjust pre{\hypertarget{ref-liangshu2002}{}}%
Liangshu, W., Shaowen, L., Weiyong, X., Cheng, L., Hua, L., Suiping, G., et al. (2002). Distribution features of terrestrial heat flow densities in the {Bohai Basin, east China}. \emph{Chinese Sci. Bull.}, \emph{47}, 857--862.

\leavevmode\vadjust pre{\hypertarget{ref-liao2014}{}}%
Liao, W.-Z., Lin, A. T., Liu, C.-S., Oung, J.-N., \& Wang, Y. (2014). Heat flow in the rifted continental margin of the south china sea near taiwan and its tectonic implications. \emph{Journal of Asian Earth Sciences}, \emph{92}(0), 233--244. \url{https://doi.org/10.1016/j.jseaes.2014.01.003}

\leavevmode\vadjust pre{\hypertarget{ref-lilley1979}{}}%
Lilley, F. E. M., Sloane, M. N., \& Sass, J. H. (1979). Compilation of {Australian} heat flow measurements. \emph{J. Geol. Soc. Austalia}, \emph{24}, 439--45.

\leavevmode\vadjust pre{\hypertarget{ref-lindqvist1984}{}}%
Lindqvist, J. G. (1984). Heat flow density measurements in the sediments of three lakes in northern sweden. \emph{Tectonophysics}, \emph{103}(1-4), 121--140.

\leavevmode\vadjust pre{\hypertarget{ref-lister1963a}{}}%
Lister, C. R. B. (1963a). Geothermal gradient measurement using a deep sea corer. \emph{Geophysical Journal of the Royal Astronomical Society}, \emph{7}, 571--783. \url{https://doi.org/10.1111/j.1365-246X.1963.tb03822.x}

\leavevmode\vadjust pre{\hypertarget{ref-lister1963b}{}}%
Lister, C. R. B. (1963b). Geothermal gratient measurement using a deep sea corer. \emph{Geophys. J. Roy. Astr. Soc.}, \emph{7}, 571--583.

\leavevmode\vadjust pre{\hypertarget{ref-lister1972}{}}%
Lister, C. R. B. (1972). On the thermal balance of a mid-ocean ridge. \emph{Geophysical Journal of the Royal Astronomy Society}, \emph{26}, 515--535. \url{https://doi.org/10.1111/j.1365-246X.1972.tb05766.x}

\leavevmode\vadjust pre{\hypertarget{ref-lister1964}{}}%
Lister, C. R. B., \& Reitzel, J. S. (1964). Some measurements of heat flow through the floor of the north {Atlantic}. \emph{J. Geophys. Res.}, \emph{69}, 2151--2154.

\leavevmode\vadjust pre{\hypertarget{ref-lister1990}{}}%
Lister, C. R. B., Sclater, J. G., Davis, E. E., Villinger, H., \& Nagihara, S. (1990). Heat flow maintained in ocean basins of great age: Investigations in the north-equatorial west {Pacific}. \emph{Geophys. J. Int.}, \emph{102}, 603--630.

\leavevmode\vadjust pre{\hypertarget{ref-liu2015}{}}%
Liu, S., Lei, X., \& Wang, L. (2015). New heat flow determination in northern tarim craton, northwest china. \emph{Geophysical Journal International}, \emph{200}(2), 1194--1204. \url{https://doi.org/10.1093/gji/ggu458}

\leavevmode\vadjust pre{\hypertarget{ref-lizon1978}{}}%
Lizon, I., \& Janci, J. (1978). \emph{Zakladny vyskum priestoroveho rozlozenia zemskeho tepla v zapadnych karpatoch (in slovak)}.

\leavevmode\vadjust pre{\hypertarget{ref-loddo1975}{}}%
Loddo, M., \& Mongelli, F. (1975). Heat flow in southern italy and surrounding seas. \emph{Boll. Geofis. Teor. Appl.}, \emph{16}, 115--122.

\leavevmode\vadjust pre{\hypertarget{ref-loddo1973}{}}%
Loddo, M., Mongelli, F., \& Roda, C. (1973). Heat flow in {Calabria, Italy}. \emph{Nature Phys. Sci.}, \emph{244}, 91--92.

\leavevmode\vadjust pre{\hypertarget{ref-loddo1982}{}}%
Loddo, M., Mongelli, F., Pecorini, G., \& Tramacere, A. (1982). Prime misure di flusso di calore in sardegna. In \emph{Ricerche geotermiche in sardegna: Con particolare riferimento al graben del campidano} (Vol. Cnr--pfe--rf10, pp. 181--209). Cnr-Pfe-Rf10.

\leavevmode\vadjust pre{\hypertarget{ref-lonsdale1985}{}}%
Lonsdale, P., \& Becker, K. (1985). Hydrothermal plumes, hot springs, and conductive heat flow in the southern trough of guaymas basin. \emph{Earth and Planetary Science Letters}, \emph{73}, 211--225. \url{https://doi.org/10.1016/0012-821x(85)90070-6}

\leavevmode\vadjust pre{\hypertarget{ref-loseth1992}{}}%
Loseth, H., Lippard, S. J., Saettem, J., Fanavoll, S., Fjerdingstad, V., Leith, T. L., et al. (1992). Cenozoic uplift and erosion of the barents sea- evidence from the svalis dome area. In \emph{Arctic geology and petroleum potential} (Vol. 2, pp. 643--664). Elsevier.

\leavevmode\vadjust pre{\hypertarget{ref-louden1987}{}}%
Louden, K. E., Wallace, D., \& Courtney, R. C. (1987). Heat flow and depth versus age for the mesozoic NW atlantic ocean: Results from the sohm abyssal plain and implications for the bermuda rise. \emph{Earth and Planetary Science Letters}, \emph{83}, 109--122. \url{https://doi.org/10.1016/0012-821x(87)90055-0}

\leavevmode\vadjust pre{\hypertarget{ref-louden1990}{}}%
Louden, K. E., Leger, G., \& Hamilton, N. (1990). Marine heat flow observations on the canadian arctic continental shelf and slope. \emph{Marine Geology}, \emph{93}, 267--288. \url{https://doi.org/10.1016/0025-3227(90)90087-z}

\leavevmode\vadjust pre{\hypertarget{ref-louden1991}{}}%
Louden, K. E., Sibuet, J. C., \& Foucher, J. P. (1991). Variations in heat flow across the goban spur and galicia bank continental margins. \emph{Journal of Geophysical Research}, \emph{96}(b10), 16131--16150. \url{https://doi.org/10.1029/91jb01453}

\leavevmode\vadjust pre{\hypertarget{ref-louden1997}{}}%
Louden, K. E., Sibuet, J.-C., \& Harmegnies, F. (1997). Variations in heat flow across the ocean--continent transition in the {Iberia} abyssal plain. \emph{Earth Planet. Sci. Lett.}, \emph{151}, 233--254.

\leavevmode\vadjust pre{\hypertarget{ref-lovering1948}{}}%
Lovering, T. S. (1948). Geothermal gradients, recent climatic changes, and rate of sufide oxidation in the {San Manuel} district, {Arizona}. \emph{Economic Geol.}, \emph{43}, 1--20.

\leavevmode\vadjust pre{\hypertarget{ref-lu2005}{}}%
Lu, Q.-Z., Hu, S.-B., \& Guo, T.-L. (2005). The background of the geothermal field for formation of abnormal high pressure in the northeastern sichuan basin. \emph{Journal of Geophysics}, \emph{48}, 1110--1116.

\leavevmode\vadjust pre{\hypertarget{ref-lu1981}{}}%
Lu, R. S., Pan, J. J., \& Lee, T. C. (1981). Heat flow in the southwestern okinawa trough. \emph{Earth and Planetary Science Letters}, \emph{55}(2), 299--310. \url{https://doi.org/10.1016/0012-821x(81)90109-6}

\leavevmode\vadjust pre{\hypertarget{ref-lubimova1964}{}}%
Lubimova, E. A. (1964). Heat flow in the ukrainian shield in relation to recent tectonic movements. \emph{J. Geophys. Res.}, \emph{69}.

\leavevmode\vadjust pre{\hypertarget{ref-lubimova1968}{}}%
Lubimova, E. A. (1968). \emph{Termika zemli i luni (in russian) izd.nauka, moskva} (p. 279).

\leavevmode\vadjust pre{\hypertarget{ref-lubimova1973c}{}}%
Lubimova, E. A., \& Savostin, L. A. (1973). Teplovoi potok v tsentralnoi i vostochnoi chasti chernogo morya (russ.). \emph{Doklady an SSSR}, \emph{212}(2), 349--352.

\leavevmode\vadjust pre{\hypertarget{ref-lubimova1966}{}}%
Lubimova, E. A., \& Shelyagin, V. A. (1966). Teplovoi potok cherez dno ozera baikal. - doklady akademii nauk sssr, 171, n 6, (russ.), 1321--1325.

\leavevmode\vadjust pre{\hypertarget{ref-lubimova1969}{}}%
Lubimova, E. A., Tomara, G. A., Demenitskaya, R. M., \& Karasik, A. M. (1969). Measurement of heat flow across the {Arctic Ocean} floor in the vicinity of the median {Hackel Ridge}. \emph{Dokl. Acak. Nauk. SSSR}, \emph{186}, 1318--1321, 22--24 (AGI English Transl.).

\leavevmode\vadjust pre{\hypertarget{ref-lubimova1972}{}}%
Lubimova, E. A., Gorskov, A. P., Vlasenko, V. I., Efimov, A. V., \& Alexandrov, A. A. (1972). Heat flux measurements near the {Kurile Island Chain} in {Kamchatka}, and the {Kurile Lake}. \emph{Dokl. Acak. Nauk. SSSR}, \emph{207}, 842-845 (AGI English Transl. 24-28).

\leavevmode\vadjust pre{\hypertarget{ref-lubimova1973}{}}%
Lubimova, E. A., Polyak, B. G., Smirnov, Y. B., Kutas, R. I., Firsov, F. V., Sergienko, S. I., \& Luisova, L. N. (1973). \emph{Heat flow on the USSR territory} (p. --). Catalogue of Data.Geophys., Committee Acad. Sci. USSR.

\leavevmode\vadjust pre{\hypertarget{ref-lubimova1976}{}}%
Lubimova, E. A., Nikitina, V. N., \& Tomara, G. A. (1976). \emph{Thermal fields of the u.s.s.r. Inland and marginal seas} (pp. 222 pp.). Nauka.

\leavevmode\vadjust pre{\hypertarget{ref-lucazeau2011a}{}}%
Lucazeau, F. (2011). \emph{Heat flow analysis on EST433, bure}.

\leavevmode\vadjust pre{\hypertarget{ref-lucazeau1989}{}}%
Lucazeau, F., \& Ben Dhia, H. (1989). Preliminary heat flow data from {Tunisia and Pelagian Sea}. \emph{Can. J Earth Sci.}, \emph{26}, 993--1000.

\leavevmode\vadjust pre{\hypertarget{ref-lucazeau2012}{}}%
Lucazeau, F., \& Rolandone, F. (2012). Heat-flow and subsurface temperature history at the site of saraya (eastern senegal). \emph{Solid Earth}, \emph{4}(2), 599--626. \url{https://doi.org/10.5194/sed-4-599-2012}

\leavevmode\vadjust pre{\hypertarget{ref-lucazeau1984}{}}%
Lucazeau, F., Vasseur, G., \& Bayer, R. (1984). Interpretation of heat flow data in the {French Massif Central}. \emph{Tectonophysics}, \emph{103}, 99--119.

\leavevmode\vadjust pre{\hypertarget{ref-lucazeau1991b}{}}%
Lucazeau, F., Cautru, J. P., Maget, P., \& Vasseur, G. (1991a). Catalogue of heat flow density data: france. In \emph{Geothermal atlas of europe} (pp. 112--115). Hermann Haack Verlagsgesellschaft mbH.

\leavevmode\vadjust pre{\hypertarget{ref-lucazeau1991}{}}%
Lucazeau, F., Lesquer, A., \& Vasseur, G. (1991b). Trends of heat flow density from {West Africa}. In \emph{Terrestrial heat flow and the lithosphere structure} (pp. 417--425). Springer Verlag.

\leavevmode\vadjust pre{\hypertarget{ref-lucazeau2004}{}}%
Lucazeau, F., Brigaud, F., \& Bouroullec, J. L. (2004). High resolution heat-flow density in lower congo basin. \emph{Geochem. Geophys. Geosys.}, \emph{5}, q03001, doi:10.1029/2003GC000644.

\leavevmode\vadjust pre{\hypertarget{ref-lucazeau2006}{}}%
Lucazeau, F., Bonneville, A., Escartin, J., Von Herzen, R. P., Gouze, P., Carton, H., et al. (2006). Heat flow variations on a slowly accreting ridge: Constraints on the hydrothermal and conductivive cooling for the {Lucky Strike} segment {(Mid-Atlantic Ridge, 37\(\circ\)N)}. \emph{Geochem. Geophys. Geosys.}, \emph{7}, q07011, doi:10.1029/2005GC001178.

\leavevmode\vadjust pre{\hypertarget{ref-lucazeau2008}{}}%
Lucazeau, F., Leroy, S., Bonneville, A., Goutorbe, B., Rolandone, F., d'Acremont, E., et al. (2008). Persistent thermal activity at the eastern {Gulf of Aden} after continental break-up. \emph{Nature Geoscience}, \emph{1}, doi:10.1038/ngeo359.

\leavevmode\vadjust pre{\hypertarget{ref-lucazeau2009}{}}%
Lucazeau, F., Leroy, S., Autin, J., Bonneville, A., Goutorbe, B., Watremez, L., et al. (2009). Post-rift volcanism and high heat-flow at the ocean-continent transition of the eastern gulf of aden. \emph{Terra Nova}, \emph{21}, 285--292.

\leavevmode\vadjust pre{\hypertarget{ref-lucazeau2010}{}}%
Lucazeau, F., Leroy, S., Rolandone, F., d'Acremont, E., Watremez, L., Bonneville, A., et al. (2010). Heat-flow and hydrothermal circulation at the ocean-continent transition of the eastern {gulf of Aden}. \emph{Earth and Planetary Science Letters}, \emph{295}(3-4), 554--570. \url{https://doi.org/10.1016/j.epsl.2010.04.039}

\leavevmode\vadjust pre{\hypertarget{ref-lucazeau2014d}{}}%
Lucazeau, F., Bouquerel, H., Rolandone, F., Pichot, T., \& Heuret, A. (2014). \emph{Méthodologie et résultats de la campagne ANTITHESIS 2}.

\leavevmode\vadjust pre{\hypertarget{ref-lucazeau2015}{}}%
Lucazeau, F., Armitage, J. J., \& Kadima Kabongo, E. (2015). Thermal regime and evolution of the congo basin as an intracratonic basin. In \emph{Geology and resource potential of the congo basin, regional geology reviews} (pp. 229--244). Springer-Verlag Berlin Heidelberg. \url{https://doi.org/10.1007/978-3-642-29482-2/_12}

\leavevmode\vadjust pre{\hypertarget{ref-luyendyk1969}{}}%
Luyendyk, B. P. (1969). \emph{Geological and geophysical observations in an abyssal hill area using a deeply towed instrument package} (No. Ntis Ad714852) (pp. 69--19).

\leavevmode\vadjust pre{\hypertarget{ref-lysak1976a}{}}%
Lysak, S. V. (1976). Novye dannye o zakonomernostyakh izmeneniya glubinnykh temperatur i teplovom potoke yuga vostochnoi sibiri (russ.). \emph{Geoter- Miya, Ch. 1, Moskva}, 77--86.

\leavevmode\vadjust pre{\hypertarget{ref-lysak1978}{}}%
Lysak, S. V. (1978). Prognoznaya karta glubinnogo teplovogo potoka territorii bam (russ.). \emph{Geologicheskie I Seismicheskie Usloviya Raiona Baikalo- Amurskoi Magistrali. Novosibirsk: Nauka}, 94--99.

\leavevmode\vadjust pre{\hypertarget{ref-lysak1983}{}}%
Lysak, S. V. (1983). Metodika i resultaty geotermicheskogo kartirovaniya terri- torii yuga vostochnoi sibiri. - v kn.: Primenenie geotermii v regional- nykh i poiskovo-razvedochnykh issledovaniyakh.

\leavevmode\vadjust pre{\hypertarget{ref-lysak1976}{}}%
Lysak, S. V., \& Zorin, Yu. A. (1976). Geotermicheskoe pole baikalskoi riftovoi zony (russ.). \emph{Moskva Nauka}, 90p.

\leavevmode\vadjust pre{\hypertarget{ref-lyubimova1966}{}}%
Lyubimova, E. A. (1966). Otsenka raspredeleniya glubinnogo teplovogo potoka dlya yuga evropeiskoi chasti sssr. - v kn.: Problemy glubinnogo tep- lovogo potoka. moskva. \emph{Nauka.}

\leavevmode\vadjust pre{\hypertarget{ref-lyubimova1968}{}}%
Lyubimova, E. A. (1968). Termika zemli i luny. Moskva, nauka. (russ.)., 280.

\leavevmode\vadjust pre{\hypertarget{ref-lyubimova1984}{}}%
Lyubimova, E. A., \& Salman, A. G. (1984). O svyazi teplovogo potoka s geologicheski- mi strukturami dna severnogo ledovitogo okeana. - v kn.: Teoreticheskie i experimentalnye issledovaniya po geotermike morey i okeanov. Moskva: Nauka, (russ.), 52--59.

\leavevmode\vadjust pre{\hypertarget{ref-lyusova1979}{}}%
Lyusova, L. N. (1979). Otsenka teplovykh potokov v tsentralnoi chasti mos- kovskoi sineklizy (russ.). \emph{Eksperimentalnoe I Teoreticheskoe Izu- Chenie Teplovykh Potokov. Moskva, Nauka}, 113--122.

\leavevmode\vadjust pre{\hypertarget{ref-lyusova1973}{}}%
Lyusova, L. N., \& Kutasov, I. M. (1973). Teplovye potoki na territorii krymsko- go poluostrova (russ.). \emph{Teplovye Potoki Iz Kory I Verkhnei Mantii Zemli. Verkhnyaya Mantiya N 12 (Red. Vlodavets V.I., Lyubimova E.A.). Moskva, Nauka}, 58--77.

\leavevmode\vadjust pre{\hypertarget{ref-macdonald2009}{}}%
MacDonald, D. (2009). \emph{Geothermal exploration results -- heat flow 9 april 2009}.

\leavevmode\vadjust pre{\hypertarget{ref-macdonald1973}{}}%
Macdonald, K. C., Luyendyk, B. P., \& Von Herzen, R. P. (1973). Heat flow and plate boundaries in {Melanesia}. \emph{J. Geophys. Res.}, \emph{78}, 2537--2546.

\leavevmode\vadjust pre{\hypertarget{ref-madsen1975}{}}%
Madsen, L. (1975). Approximate geothermal gradients in denmark and the danish north sea sector.

\leavevmode\vadjust pre{\hypertarget{ref-majorowicz1973b}{}}%
Majorowicz, J. (1973a). Heat flow data from {Poland}. \emph{Nature Phys. Sci.}, \emph{241}, 16--17.

\leavevmode\vadjust pre{\hypertarget{ref-majorowicz1975}{}}%
Majorowicz, J. (1975). Strumien cieplny na obszarze nizu {Polski} (in polish). \emph{Acta Geophys. Polon.}, \emph{23}, 259--275.

\leavevmode\vadjust pre{\hypertarget{ref-majorowicz1979}{}}%
Majorowicz, J., \& Plewa, S. (1979). Study of heat flow in {Poland} with special regard to tectonophysical problems. In \emph{Terrestrial heat flow in europe} (pp. 240--252). Springer Verlag.

\leavevmode\vadjust pre{\hypertarget{ref-majorowicz1974}{}}%
Majorowicz, J., Plewa, S., \& Wesierska, M. (1974). \emph{Rozklad pola cieplnegoziemi na obszare {Polski} problem wezlowy 01.1.1 n/.3 (in polish)}.

\leavevmode\vadjust pre{\hypertarget{ref-majorowicz2014}{}}%
Majorowicz, J., Chan, J., Crowell, J., Gosnold, W., Heaman, L. M., Kück, J., et al. (2014). The first deep heat flow determination in crystalline basement rocks beneath the western canadian sedimentary basin. \emph{Geophysical Journal International}, \emph{197}(2), 731--747. \url{https://doi.org/10.1093/gji/ggu065}

\leavevmode\vadjust pre{\hypertarget{ref-majorowicz1973a}{}}%
Majorowicz, J. A. (1973b). Heat flow in poland and its relation to the geological structure. \emph{Geothermics}, \emph{2}(1), 24--28. \url{https://doi.org/10.1016/0375-6505(73)90031-x}

\leavevmode\vadjust pre{\hypertarget{ref-majorowicz1996}{}}%
Majorowicz, J. A. (1996). Anomalous heat flow regime in the western margin of the north american craton, canada. \emph{Journal of Geodynamics}, \emph{21}(2), 123--140. \url{https://doi.org/10.1016/0264-3707(95)00020-2}

\leavevmode\vadjust pre{\hypertarget{ref-majorowicz1998}{}}%
Majorowicz, J. A., \& Embry, A. F. (1998). Present heat flow and paleo-geothermal regime in the canadian arctic margin: Analysis of industrial thermal data and coalification gradients. \emph{Tectonophysics}, \emph{291}(1-4), 141--159.

\leavevmode\vadjust pre{\hypertarget{ref-majorowicz1981}{}}%
Majorowicz, J. A., \& Jessop, A. M. (1981). Regional heat flow patterns in the western canadian sedimentary basin. \emph{Tectonophysics}, \emph{74}, 209--238. \url{https://doi.org/10.1016/0040-1951(81)90191-8}

\leavevmode\vadjust pre{\hypertarget{ref-majorowicz1990}{}}%
Majorowicz, J. A., Jones, F. W., \& Judge, A. S. (1990). Deep subpermafrost thermal regime in the {McKenzie Delta} basin, northern {Canada}---analysis from petroleum bottom-hole temperature data. \emph{Geophysics}, \emph{55}, 362--371.

\leavevmode\vadjust pre{\hypertarget{ref-majorowicz1999}{}}%
Majorowicz, J. A., Garven, G., Jessop, A., \& Jessop, C. (1999). Present heat flow across the western {Canada} sedimentary basin: The extent of hydrodynamic influence. In \emph{Geothermics in basin analysis} (pp. 61--80). Kluwer Academic.

\leavevmode\vadjust pre{\hypertarget{ref-makita1992}{}}%
Makita, S. (1992). \emph{Heat flow measurements around the japanese islands: Interpretation with reference to the tectonics in the okinawa trough (in japanese).} (Master's thesis).

\leavevmode\vadjust pre{\hypertarget{ref-malmqvist1983}{}}%
Malmqvist, D., Larson, S. A., Landstroem, O., \& Lind, G. (1983). Heat flow and heat production from the {Malingsbo} granite, central {Sweden}. \emph{Bull. Geol. Inst. Univ. Uppsala}, \emph{9}, 137--152.

\leavevmode\vadjust pre{\hypertarget{ref-manga2012}{}}%
Manga, M., Hornbach, M. J., Le Friant, A., Ishizuka, O., Stroncik, N., Adachi, T., et al. (2012). Heat flow in the lesser antilles island arc and adjacent back arc grenada basin. \emph{Geochemistry Geophysics Geosystems}, \emph{13}, q08007--. \url{https://doi.org/10.1029/2012gc004260}

\leavevmode\vadjust pre{\hypertarget{ref-marcaillou2012}{}}%
Marcaillou, B., Henry, P., Kinoshita, M., Kanamatsu, T., Screaton, E., Daigle, H., et al. (2012). Seismogenic zone temperatures and heat flow anomalies in the to-nankai margin segment based on temperature data from IODP expedition 333 and thermal model. \emph{Earth and Planetary Science Letters}, \emph{349-350}(0), 171--185. \url{https://doi.org/10.1016/j.epsl.2012.06.048}

\leavevmode\vadjust pre{\hypertarget{ref-mareschal1989}{}}%
Mareschal, J. C., Pinet, C., Gariépy, C., Jaupart, C., Bienfait, G., Dalla Coletta, G., et al. (1989). New heat flow density and radiogenic heat production data in the {Canadian Shield} and {Quebec Appalachians}. \emph{Can. J Earth Sci.}, \emph{26}, 845--852.

\leavevmode\vadjust pre{\hypertarget{ref-mareschal1999}{}}%
Mareschal, J. C., Jaupart, C., Cheng, L. Z., Rolandone, F., Gariepy, C., Bienfait, G., et al. (1999). Heat flow in the trans-hudson orogen of the canadian shield: Implications for proterozoic continental growth. \emph{Journal of Geophysical Research-Solid Earth}, \emph{104}(b12), 29007--29024. \url{https://doi.org/10.1029/1998jb900209}

\leavevmode\vadjust pre{\hypertarget{ref-mareschal2000b}{}}%
Mareschal, J. C., Jaupart, C., Gariépy, C., Cheng, L. Z., Guillou-Frottier, L., Bienfait, G., \& Lapointe, R. (2000a). Heat flow and deep thermal structure near the southeastern edge of the {Candian Shield}. \emph{Can. J Earth Sci.}, \emph{37}, 399--414.

\leavevmode\vadjust pre{\hypertarget{ref-mareschal2000a}{}}%
Mareschal, J. C., Poirier, A., Rolandone, F., Bienfait, G., Gariépy, C., Lapointe, R., \& Jaupart, C. (2000b). Low mantle heat flow at the edge of the {North American} continent, {Voisey Bay, Labrador}. \emph{Geophys. Res. Lett.}, \emph{27}, 823--826.

\leavevmode\vadjust pre{\hypertarget{ref-mareschal2004}{}}%
Mareschal, J. C., Nyblade, A., Perry, H. K. C., Jaupart, C., \& Bienfait, G. (2004). Heat flow and deep lithospheric thermal structure at {Lac de Gras, Slave Province, Canada}. \emph{Geophys. Res. Lett.}, \emph{31}, l12611, doi:10.10299/2004GL020133.

\leavevmode\vadjust pre{\hypertarget{ref-mareschal2005}{}}%
Mareschal, J. C., Jaupart, C., Rolandone, F., Gariépy, C., Fowler, C. M. R., Bienfait, G., et al. (2005). Heat flow, thermal regime, and elastic thickness of the lithosphere in the {Trans-Hudson Orogen}. \emph{Can. J Earth Sci.}, \emph{42}, 517--532.

\leavevmode\vadjust pre{\hypertarget{ref-mareschal2017}{}}%
Mareschal, J.-C., Jaupart, C., Armitage, J., Phaneuf, C., Pickler, C., \& Bouquerel, H. (2017). The sudbury huronian heat flow anomaly, ontario, canada. \emph{Precambrian Research}, \emph{295}, 187--202. \url{https://doi.org/10.1016/j.precamres.2017.04.024}

\leavevmode\vadjust pre{\hypertarget{ref-marshall1974}{}}%
Marshall, B. V., \& Erickson, A. J. (1974). Heat flow and thermal conductivity measurements, {Leg} 25. \emph{Initial Reports DSDP}, \emph{25}, 349.

\leavevmode\vadjust pre{\hypertarget{ref-martinelli1995}{}}%
Martinelli, G., Dongarrø, G., Jones, M. Q. W., \& Rodrigues, A. (1995). Geothermal features of mozambique - country update. In \emph{Proceedings of the world geothermal congress 1995} (Vol. 1, pp. 251--273). International Geothermal Association.

\leavevmode\vadjust pre{\hypertarget{ref-martinez1989}{}}%
Martinez, F., \& Cochran, J. R. (1989). Geothermal measurements in the northern red sea: Implications for lithospheric thermal structure and mode of extension during continental rifting. \emph{Journal of Geophysical Research}, \emph{94}(b9), 12239--12265. \url{https://doi.org/10.1029/JB094iB09p12239}

\leavevmode\vadjust pre{\hypertarget{ref-marusiak1975}{}}%
Marusiak, I., \& Lizon, I. (1975). Vysledky geotermickeho vyskumu v cesko slovenskej casti viedenskej panvy (in slovak). \emph{Geol. Prace, Spravy}, \emph{63}, 191--204.

\leavevmode\vadjust pre{\hypertarget{ref-marzan2000}{}}%
Marzan, I. (2000). \emph{Régimen térmico en la peninsula ibérica. Estructura litosférica a través del macizo ibérico y el margen surportugués.} (PhD thesis).

\leavevmode\vadjust pre{\hypertarget{ref-mas2000}{}}%
Mas, L., Mas, G., \& Bengochea, L. (2000). Heat flow of {Copahue} geothermal field, and its relation with tectonic scheme. In \emph{Proceedings word geothermal congress} (pp. 1419--1424).

\leavevmode\vadjust pre{\hypertarget{ref-masaki2011}{}}%
Masaki, Y., Kinoshita, M., Ingakai, F., Nakagawa, S., \& Takai, K. (2011). Possible kilometer-scale hydrothermal circulation within the {Iheya-North} field, {mid-Okinawa Trough}, as inferred from heat flow data. \emph{JAMSTEC Rep. Rev. Dev.}, \emph{12}, 1--12.

\leavevmode\vadjust pre{\hypertarget{ref-matsubara1981}{}}%
Matsubara, Y. (1981). Heat flow measurements in the bonin arc area. In \emph{Geological investigation of the ogasawara (bonin) and northern mariana arcs, cruise rep.} (Vol. 14, pp. 130--136). Geological Survey of Japan.

\leavevmode\vadjust pre{\hypertarget{ref-matsubara2004}{}}%
Matsubara, Y. (2004). Unpublished data. In \emph{CD rom: Geothermal gradient and heat flow data in and around japan} (p. --). Geological Survey of Japan, AIST, 2004.

\leavevmode\vadjust pre{\hypertarget{ref-matsubara1982}{}}%
Matsubara, Y., Kinoshita, H., Uyeda, S., \& Thienprasert, A. (1982). Development of a new system for shallow sea heat flow measurement and its test application in the {Gulf of Thailand}. \emph{Tectonophysics}, \emph{83}, 13--31.

\leavevmode\vadjust pre{\hypertarget{ref-matsubayashi1982}{}}%
Matsubayashi, O. (1982). Reconnaissance measurements of heat flow in the central pacific. \emph{Geol. Surv. Japan Cruise Rep.}, \emph{18}, 90--94.

\leavevmode\vadjust pre{\hypertarget{ref-matsubayashi1979}{}}%
Matsubayashi, O., Kinoshita, H., Matsubara, Y., \& Matsuda, J. I. (1979). Preliminary report on heat flow in the central part of kagoshima bay, kyushu, japan. \emph{Bull. Geol. Surv. Japan}, \emph{30}, 45--49.

\leavevmode\vadjust pre{\hypertarget{ref-matthews2007}{}}%
Matthews, C., \& Beardsmore, G. (2007). New heat flow data from south-eastern south australia. \emph{Exploration Geophysics}, \emph{38}(4), 260--269. \url{https://doi.org/10.1071/eg07028}

\leavevmode\vadjust pre{\hypertarget{ref-matthews2013}{}}%
Matthews, C., Beardsmore, G., Driscoll, J., \& Pollington, N. (2013). Heat flow data from the southeast of south australia: Distribution and implications for the relationship between current heat flow and the newer volcanics province. \emph{Exploration Geophysics}, \emph{44}(2), 133--144. \url{https://doi.org/10.1071/eg12052}

\leavevmode\vadjust pre{\hypertarget{ref-mathews1972}{}}%
Matthews, W. H. (1972). Geothermal data from the granduc area, northern coast mountains of british columbia. \emph{Canadian Journal of Earth Sciences}, \emph{9}, 1333--1337. \url{https://doi.org/10.1139/e72-117}

\leavevmode\vadjust pre{\hypertarget{ref-matvienko1976a}{}}%
Matvienko, V. N., \& Sergienko, S. I. (1976a). Rezultaty opredeleniya teplovogo potoka v zapadnom predkavkazye (russ.). \emph{Geotermiya. /Geotermiches- Kie Issledovaniya V SSSR}, \emph{1}, 53--58.

\leavevmode\vadjust pre{\hypertarget{ref-matvienko1976}{}}%
Matvienko, V. N., \& Sergienko, S. I. (1976b). Teplovoe pole neftegazonosnykh raionov predkavkazyya (russ.). \emph{Izvestiya an SSSR, Ser. Geologicheskaya}, \emph{2}, 149--155.

\leavevmode\vadjust pre{\hypertarget{ref-maxwell1958}{}}%
Maxwell, A. E. (1958). \emph{The outflow of heat under the {Pacific Ocean}} (PhD thesis).

\leavevmode\vadjust pre{\hypertarget{ref-maystrenko2015}{}}%
Maystrenko, Y. P., Slagstad, T., Elvebakk, H. K., Olesen, O., Ganerød, G. V., \& Rønning, J. S. (2015). New heat flow data from three boreholes near bergen, stavanger and moss, southern norway. \emph{Geothermics}, \emph{56}, 79--92. https://doi.org/\url{http://dx.doi.org/10.1016/j.geothermics.2015.03.010}

\leavevmode\vadjust pre{\hypertarget{ref-medici1995}{}}%
Medici, F., \& Rybach, L. (1995). \emph{Geothermal map of switzerland 1995 (heat flow density)} (No. No. 30).

\leavevmode\vadjust pre{\hypertarget{ref-meert1991}{}}%
Meert, J. G., Smith, D. L., \& Fishkin, L. (1991). Heat flow in the {Ozark Plateau, Arkansas and Missouri}: Relationship to groundwater flow. \emph{J. Volcan. Geothermal Res.}, \emph{47}, 337--347. \url{https://doi.org/10.1016/0377-0273(91)90024-t}

\leavevmode\vadjust pre{\hypertarget{ref-meinke1967}{}}%
Meinke, W., Hurtig, E., \& Werner, J. (1967). Temperaturverteilung, {Wärmemeleitfähigkeit und Wämefluss im Thüringer Becken}. \emph{Geophys. Und Geol.}, \emph{11}, 140--171.

\leavevmode\vadjust pre{\hypertarget{ref-mendesvictor1991}{}}%
Mendes-Victor, L. A., \& Duque, M. R. (1991). Catalogue of heat flow density data: portugal. In \emph{Geothermal atlas of europe} (p. 123). Hermann Haack Verlagsgesellschaft mbH.

\leavevmode\vadjust pre{\hypertarget{ref-mercier2009}{}}%
Mercier, M. (2009). \emph{Relations entre flux de chaleur océanique et zone sismogène : Cas de la subduction de sumatra} (Master's thesis).

\leavevmode\vadjust pre{\hypertarget{ref-merkushov1983}{}}%
Merkushov, V. N., Podgornykh, L. V., \& Smirnov, Ya. B. (1983). I dr. - v kn.: Metodiches- kie i experimentalnye osnovy geotermii. Moskva: Nauka (russ.), 181--185.

\leavevmode\vadjust pre{\hypertarget{ref-middleton1979}{}}%
Middleton, M. F. (1979). Heat flow in {Moomba, Big Lake and Toolachee} gas fields of the {Cooper Basin} and implications for hydrocarbon maturation. \emph{Explor. Geophys.}, \emph{10}, 149--155.

\leavevmode\vadjust pre{\hypertarget{ref-minier1991}{}}%
Minier, J., \& Reiter, M. (1991). Heat flow on the southern {Colorado Plateau}. \emph{Tectonophysics}, \emph{200}, 51--66.

\leavevmode\vadjust pre{\hypertarget{ref-miridzhanyan1983}{}}%
Miridzhanyan, R. T. (1983). Geotermicheskie usloviya uchastka shakhty arpa- sevan. - izvestiya an arm. ssr. \emph{Ser. Nauki o Zemle. 1983}.

\leavevmode\vadjust pre{\hypertarget{ref-misener1955}{}}%
Misener, A. D. (1955). Heat flow and depth of permafrost at {Resolute Bay, Cornwallis Island, N.W.T., Canada}. \emph{Trans. Am. Geophys. Union}, \emph{36}, 1055--1060.

\leavevmode\vadjust pre{\hypertarget{ref-misener1951}{}}%
Misener, A. D., Thompson, L. G. D., \& Uffen, R. J. (1951). Terrestrial heat flow in {Ontario and Quebec}. \emph{Trans. Am. Geophys. Union}, \emph{32}, 729--738.

\leavevmode\vadjust pre{\hypertarget{ref-mizutani1982}{}}%
Mizutani, H., \& Yokokura, T. (1982). Preliminary heat flow study in {Papua New Guinea}. \emph{United Nations ESCAP, CCOP Tech. Bull.}, \emph{15}, 29--43.

\leavevmode\vadjust pre{\hypertarget{ref-mizutani1970}{}}%
Mizutani, H., Baba, K., Kobayashi, N., Chang, C. C., Lee, C. H., \& Kang, Y. S. (1970). Heat flow in {Korea}. \emph{Tectonophysics}, \emph{10}, 183--203.

\leavevmode\vadjust pre{\hypertarget{ref-moiseenko1967}{}}%
Moiseenko, U. I., \& Sokolova, L. S. (1967a). Teplovoi potok po dvum skvazhinam stol- bovskoy struktury vostochnoy kamchatki. - geologiya i geofizika.

\leavevmode\vadjust pre{\hypertarget{ref-moiseenko1967a}{}}%
Moiseenko, U. I., \& Sokolova, L. S. (1967b). Teplovoy potok po skvazhinam yuzhno- minusinskoy vpadiny. - geologiya i geofizika.

\leavevmode\vadjust pre{\hypertarget{ref-mongelli1974}{}}%
Mongelli, F., \& Loddo, M. (1974). The present state of geothermal investigations in {Italy}. \emph{Acta Geodaet., Geophys., Montanist.,} \emph{9}, 449--454.

\leavevmode\vadjust pre{\hypertarget{ref-mongelli1970b}{}}%
Mongelli, F., \& Ricchetti, G. (1970a). Heat flow along the candelaro faul - gargano headland (italy). \emph{Geothermics}, \emph{Sp.issue2}(2), 450--458. \url{https://doi.org/10.1016/0375-6505(70)90043-x}

\leavevmode\vadjust pre{\hypertarget{ref-mongelli1970a}{}}%
Mongelli, F., \& Ricchetti, G. (1970b). The {Earth's} crust and heat flow in {Fossa Bradanica, southern Italy}. \emph{Tectonophysics}, \emph{10}, 103--125.

\leavevmode\vadjust pre{\hypertarget{ref-mongelli1981}{}}%
Mongelli, F., Loddo, M., Tramacere, A., Zito, G., Perusini, P., SquarciI, P., \& Taffi, L. (1981). Contributo alla mappa del flusso geotermico in {Italia}: Misure sulla fascia pre-{Appenninica Marchigiana}. In \emph{Atti del 1. Convegno annuale del gruppo nazionale di geofisica della terra solida} (pp. 427--450). Edizioni Scientifiche Associate.

\leavevmode\vadjust pre{\hypertarget{ref-mongelli1982}{}}%
Mongelli, F., Tramacere, A., Grassi, S., Perusini, P., Squarci, P., \& Taffi, L. (1982). Misure di flusso di calorie. In \emph{Il graben di siena, studi geologici, idrogeologici e geofisici finalizzati alla ricerca di fluidi caldi nel sottosuolo} (Vol. Cnr--pfe--rf9, pp. 150--162).

\leavevmode\vadjust pre{\hypertarget{ref-mongelli1983}{}}%
Mongelli, F., Ciaranfi, N., Tramacere, A., Zito, G., Perusini, P., Squarci, P., \& Taffi, L. (1983). Contributo alla mappa del flusso geotermico in {Italia}: Misure dalle marche alla {Puglia}. In \emph{Atti del 2. Convegno annuale del gruppo nazionale di geofisica della terra solida} (pp. 737--763). Edizioni Scientifiche Associate.

\leavevmode\vadjust pre{\hypertarget{ref-mongelli1991}{}}%
Mongelli, F., Cataldi, R., Celati, R., Della Vedova, B., Fanelli, M., Nuti, S., et al. (1991). Catalogue of heat flow density data: italy. In \emph{Geothermal atlas of europe} (pp. 119--121). Hermann Haack Verlagsgesellschaft mbH.

\leavevmode\vadjust pre{\hypertarget{ref-moore2001}{}}%
Moore, G. F., Taira, \& al., A. K. et. (2001). Leg 190. \emph{Proc. ODP Initial Reports}.

\leavevmode\vadjust pre{\hypertarget{ref-moran1985}{}}%
Moran, J. E. (1985). \emph{Heat flow and the thermal evolution of the {Cascadia Basin}} (Master's thesis).

\leavevmode\vadjust pre{\hypertarget{ref-morgan1973}{}}%
Morgan, P. (1973). \emph{Terrestrial heat flow studies in cyprus and kenya} (PhD thesis).

\leavevmode\vadjust pre{\hypertarget{ref-morgan1975}{}}%
Morgan, P. (1975). Porosity determinations and the thermal conductivity of rock fragments with application to heat flow on cyprus. \emph{Earth and Planetary Science Letters}, \emph{26}, 253--262. \url{https://doi.org/10.1016/0012-821x(75)90093-x}

\leavevmode\vadjust pre{\hypertarget{ref-morgan1979}{}}%
Morgan, P., \& Swanberg, C. A. (1979). Preliminary eastern {Egypt} heat flow values. \emph{Pure Appl. Geophys.}, \emph{117}, 213--226. \url{https://doi.org/10.1007/978-3-642-95357-6/_13}

\leavevmode\vadjust pre{\hypertarget{ref-morgan1976}{}}%
Morgan, P., Blackwell, D. D., \& Boulos, F. K. (1976). Heat flow measurements in {Egypt}. \emph{Trans. AGU}, \emph{57}, 1009.

\leavevmode\vadjust pre{\hypertarget{ref-morgan1977}{}}%
Morgan, P., Blackwell, D. D., Spafford, R. E., \& Smith, R. B. (1977). Heat flow measurements in yellowstone lake and the thermal structure of the yellowstone caldera. \emph{Journal of Geophysical Research}, \emph{82}, 3719--3732.

\leavevmode\vadjust pre{\hypertarget{ref-morgan1983}{}}%
Morgan, P., Boulos, F. K., \& Swanberg, C. A. (1983). Regional geothermal exploration in {Egypt}. \emph{Geophys. Prospecting}, \emph{31}, 361--376.

\leavevmode\vadjust pre{\hypertarget{ref-morgan1985}{}}%
Morgan, P., Boulos, F. K., Hennin, S. F., El-Sherif, A. A., El-Sayed, A. A., Basta, N. Z., \& Melek, Y. S. (1985). Heat flow in eastern egypt: The thermal signature of a continental breakup. \emph{Journal of Geodynamics}, \emph{4}, 107--131. \url{https://doi.org/10.1016/0264-3707(85)90055-9}

\leavevmode\vadjust pre{\hypertarget{ref-morin1986}{}}%
Morin, R. H., \& Von Herzen, R. P. (1986). Geothermal measurements at {Deep Sea Drilling Project} site 587. \emph{Initial Reports DSDP}, \emph{90}, 1317--1324.

\leavevmode\vadjust pre{\hypertarget{ref-morin2010}{}}%
Morin, R. H., Williams, T., Henrys, S. A., Magens, D., Niessen, F., \& Hansaraj, D. (2010). Heat flow and hydrologic characteristics at the {AND-1B} borehole, {ANDRILL McMurdo Ice Shelf Project, Antarctica}. \emph{Geosphere}, \emph{6}(4), 370--378. \url{https://doi.org/10.1130/ges00512.1}

\leavevmode\vadjust pre{\hypertarget{ref-mottaghy2005}{}}%
Mottaghy, D., Schellschmidt, R., Popov, Y. A., Clauser, C., Kukkonen, I. T., Nover, G., et al. (2005). New heat flow data from the immediate vicinity of the kola super-deep borehole: Vertical variation in heat flow confirmed and attributed to advection. \emph{Tectonophysics}, \emph{401}(1-2), 119--142. \url{https://doi.org/10.1016/j.tecto.2005.03.005}

\leavevmode\vadjust pre{\hypertarget{ref-moxiang1988}{}}%
Moxiang, C. C. (1988). \emph{Geothermics of Northern China}.

\leavevmode\vadjust pre{\hypertarget{ref-mullins1957}{}}%
Mullins, R., \& Hinsley, F. B. (1957). Measurement of geothermic gradients in boreholes. \emph{Trans. Instn. Min. Eng.}, \emph{117}, 379--393.

\leavevmode\vadjust pre{\hypertarget{ref-munoz1993}{}}%
Muñoz, M., \& Hamza, V. (1993). Heat flow and temperature gradients in chile. \emph{Studia Geoph. Et Geod.}, \emph{37}, 315--348.

\leavevmode\vadjust pre{\hypertarget{ref-munroe1975}{}}%
Munroe, R. J., Sass, J. H., Milburn, G. T., Jaeger, J. C., \& Tammemagi, H. Y. (1975). \emph{Basic data for some recent australian heat-flow measurements} (No. 76-567). {US} Geological Survey. \url{https://doi.org/10.3133/ofr75567}

\leavevmode\vadjust pre{\hypertarget{ref-muraviev2004}{}}%
Muraviev, A. V., \& Matveev, V. G. (2004). Results of the 42nd cruise of r/v "dmitryi mendeleev" in 1988 (personal communication). In \emph{CD rom: Geothermal gradient and heat flow data in and around japan} (p. --). Geological Survey of Japan, AIST.

\leavevmode\vadjust pre{\hypertarget{ref-muraviev1988}{}}%
Muraviev, A. V., Smirnov, Y. A., \& Sugrobov, V. M. (1988). Heat flow measurements along the philippine sea geotraverse 18{\(^\circ\)}n (in russian). \emph{Dokl. Acak. Nauk. SSSR}, \emph{229}, 189--193.

\leavevmode\vadjust pre{\hypertarget{ref-yasui1963}{}}%
M.Yasui, Horai, K., Uyeda, S., \& Akamatsu, H. (1963). Heat flow measurement in the western {Pacific} during the {JEDS-5} and other cruses in 1962 aboard {M/S Ryofu Maru}. \emph{Oceanogrl. Mag.}, \emph{14}, 147--156.

\leavevmode\vadjust pre{\hypertarget{ref-myhre1995}{}}%
Myhre, A. M., Thiede, J., \& Firth, J. V. (1995). \emph{North atlantic--arctic gateways} (Vol. 151). Ocean Drilling Program.

\leavevmode\vadjust pre{\hypertarget{ref-nagao1986}{}}%
Nagao, T. (1987). \emph{Heat flow measurements in the tohoku-hokkaido regions by some new techniques and their geotectonic interpretation} (PhD thesis).

\leavevmode\vadjust pre{\hypertarget{ref-nagao1983}{}}%
Nagao, T., \& Kaminuma, K. (1983). Heat flow measurements in the {Lützow--Holm Bay, Antarctica}. \emph{Mem. Nat. Inst. Polar Res.}, \emph{28}, 18--26.

\leavevmode\vadjust pre{\hypertarget{ref-nagao1989}{}}%
Nagao, T., \& Uyeda, S. (1989). Heat flow measurements in the northern part of honshu, northeast japan, using shallow holes. \emph{Tectonophysics}, \emph{164}, 301--314.

\leavevmode\vadjust pre{\hypertarget{ref-nagao2002}{}}%
Nagao, T., Saki, T., \& Joshima, M. (2002). Heat flow measurements around the {Antarctica}: Contributions of the r/v hakurei. \emph{Proc. Japan Acad. Ser. B}, \emph{78}, 19--23.

\leavevmode\vadjust pre{\hypertarget{ref-nagaraju2012}{}}%
Nagaraju, P., Ray, L., Ravi, G., Akkiraju, VyasuluV., \& Roy, S. (2012). Geothermal investigations in the upper vindhyan sedimentary rocks of shivpuri area, central india. \emph{Journal of the Geological Society of India}, \emph{80}(1), 39--47. \url{https://doi.org/10.1007/s12594-012-0116-x}

\leavevmode\vadjust pre{\hypertarget{ref-nagasaka1970}{}}%
Nagasaka, K., Francheteau, J., \& Kishii, T. (1970). Terrestrial heat flow in the celebes and sulu seas. \emph{Marine Geophysical Research}, \emph{1}, 99--103. \url{https://doi.org/10.1007/bf00310013}

\leavevmode\vadjust pre{\hypertarget{ref-nagasawa1979}{}}%
Nagasawa, K., \& Komatsu, K. (1979). Thermal structure under the ground in osaka plain, southwest japan. \emph{J. Geosci. Osaka City Univ.}, \emph{22}, 151--166.

\leavevmode\vadjust pre{\hypertarget{ref-nagihara1987}{}}%
Nagihara, S. (1987). \emph{Heat flow and tectonics of the northwestern pacific subduction zones -concerning the yap trench convergence-} (Master's thesis).

\leavevmode\vadjust pre{\hypertarget{ref-nagihara2005}{}}%
Nagihara, S., \& Jones, K. O. (2005). Geothermal heat flow in the northeast margin of the {Gulf of Mexico}. \emph{AAPG Bull.}, \emph{89}, 821--831.

\leavevmode\vadjust pre{\hypertarget{ref-nagihara1989}{}}%
Nagihara, S., Kinoshita, M., Fujimoto, H., Katao, H., Kinoshita, H., \& Tomoda, Y. (1989). Geophysical observations around the northern yap trench: Seismicity, gravity and heat flow. \emph{Tectonophysics}, \emph{163}, 93--104. \url{https://doi.org/10.1016/s0040-1951(96)00251-x}

\leavevmode\vadjust pre{\hypertarget{ref-nagihara1992}{}}%
Nagihara, S., Sclater, J. G., Beckley, L. M., Behrens, E. W., \& Lawver, L. A. (1992). High heat flow anomalies over salt structures on the texas continental slope, gulf of mexico. \emph{Geophysical Research Letters}, \emph{19}(16), 1687--1690. \url{https://doi.org/10.1029/92gl00976}

\leavevmode\vadjust pre{\hypertarget{ref-nagihara1993}{}}%
Nagihara, S., Beckley, L. M., Behrens, E. W., \& Sclater, J. G. (1993). Characteristics of heat flow through diapiric salt structures on the {Texas} continental slope. \emph{Gulf Coast Association of Geological Societies Transactions}, \emph{43}, 269--279.

\leavevmode\vadjust pre{\hypertarget{ref-nagihara1996}{}}%
Nagihara, S., Sclater, J. G., Phillips, J. D., Behrens, E. W., Lewis, T., Lawver, L. A., et al. (1996). Heat flow in the western abyssal plain of the {Gulf of Mexico}: Implications for thermal evolution of the old ocean lithosphere. \emph{J. Geophys. Res.}, \emph{101}, 2895--2913.

\leavevmode\vadjust pre{\hypertarget{ref-nakajin1972}{}}%
Nakajin, T., \& Anma, M. (1972). Heat flow measurements in the {Surga Bay}. In \emph{Izu peninsula} (pp. 287--300). Tokai Univ. Press.

\leavevmode\vadjust pre{\hypertarget{ref-nason1964}{}}%
Nason, R. D., \& Lee, W. H. K. (1964). Heat flow measurements in the north {Atlantic, Caribbean, and Mediterranean}. \emph{J. Geophys. Res.}, \emph{69}, 4875--4883.

\leavevmode\vadjust pre{\hypertarget{ref-negoita1970}{}}%
Negoita, V. (1970). Etude sur la distribution des températures en roumanie. \emph{Rev. Roum. Géol. Géophys. Géogr., Ser. Géophysique}, \emph{14}, 25--30.

\leavevmode\vadjust pre{\hypertarget{ref-negraru2009}{}}%
Negraru, P. T., Blackwell, D., \& Richards, M. (2009). Texas heat flow patterns. \emph{Search and Discovery}, 80048.

\leavevmode\vadjust pre{\hypertarget{ref-negulic2016}{}}%
Negulic, E., \& Louden, K. E. (2016). The thermal structure of the central nova scotia slope (eastern canada): Seafloor heat flow and thermal maturation models, \emph{54}, 146--162. \url{https://doi.org/10.1139/cjes-2016-0060}

\leavevmode\vadjust pre{\hypertarget{ref-negut1982}{}}%
Negut, A. (1982). \emph{Implications of the thermal field structure in {Mutenia} and {Oltenia}} (PhD thesis).

\leavevmode\vadjust pre{\hypertarget{ref-nekrasov1976}{}}%
Nekrasov, I. A. (1976). Kriolitozona severo-vostoka i yuga sibiri i zakonomernos- ti ee razvitiya (russ.). \emph{Jakutsk: Jakutskoe Knizhnoe Izdatelstvo}, 244p.

\leavevmode\vadjust pre{\hypertarget{ref-nekrasov1966}{}}%
Nekrasov, I. A., \& Selivanov, A. A. (1966). Mnogoletnemerzlye porody nizhne-ingama- kitskoi kotloviny. - v kn.: Geokriologicheskie usloviya zabaikalskogo severa. Moskva: nauka.

\leavevmode\vadjust pre{\hypertarget{ref-neprimerov1987}{}}%
Neprimerov, N. N., \& Khodyreva, E. Ya. (1987). Konduktivnye i konvektivnye tep- lovye potoki pripyatskogo neftegazonosnogo basseina. - neftyanaya promyshlennost. Ekspress informatsiya. \emph{Ser. Neftegazovaya Geologiya i Geofizika. 1987}.

\leavevmode\vadjust pre{\hypertarget{ref-newstead1953}{}}%
Newstead, G., \& Beck, A. (1953). Borehole temperature measuring equipment and the geothermal flux in {Tasmania}. \emph{Aust. J. Phys.}, \emph{6}, 480--489.

\leavevmode\vadjust pre{\hypertarget{ref-nicholls1993}{}}%
Nicholls, K. W., \& Paren, J. G. (1993). Extending the antarctic meteorological record using ice-sheet temperature profiles. \emph{Journal of Climate}, \emph{6}(1), 141--150. \url{https://doi.org/10.1175/1520-0442(1993)006\%3C0141:etamru\%3E2.0.co;2}

\leavevmode\vadjust pre{\hypertarget{ref-nied1995}{}}%
NIED, W. area deep observation group of. (1995). Basal structures of the southern kanto district - results of drilling and logging of the chiba, yokohama, edosaki, ichihara and atsugi observation wells.

\leavevmode\vadjust pre{\hypertarget{ref-nishimura1990}{}}%
Nishimura, S. (1990). Thermal gradients of deep wells and their terrestrial heat flows (2). \emph{J. Geotherm. Res. Soc. Japan}, \emph{12}(in Japanese with English abstract), 283--293.

\leavevmode\vadjust pre{\hypertarget{ref-nishimura1986}{}}%
Nishimura, S., Mogi, T., \& Katsura, K. (1986). Thermal gradients of deep wells and their terrestrial heat flows in central and southwest japan. \emph{J. Geotherm. Res. Soc. Japan}, \emph{8}(in Japanese with English abstract), 347--360.

\leavevmode\vadjust pre{\hypertarget{ref-nissen1995}{}}%
Nissen, S. S., Hayes, D. E., Bochu, Y., Weijun, Z., Yongqin, C., \& Xiaupin, N. (1995). Gravity, heat flow, and seismic constraints on the processes of crustal extension: Northern margin of the {South China Sea}. \emph{J. Geophys. Res.}, \emph{100}, 22447--22483.

\leavevmode\vadjust pre{\hypertarget{ref-noel1985}{}}%
Noel, M. (1985). Heat flow, sediment faulting and porewater advection in the {Madeira} abyssal plain. \emph{Earth Planet. Sci. Lett.}, \emph{73}, 398--406.

\leavevmode\vadjust pre{\hypertarget{ref-noel1988}{}}%
Noel, M., \& Hounslow, M. W. (1988). Heat flow evidence for hydrothermal convection in cretaceous crust of the madeira abyssal plain. \emph{Earth and Planetary Science Letters}, \emph{90}, 77--86. \url{https://doi.org/10.1016/0012-821x(88)90113-6}

\leavevmode\vadjust pre{\hypertarget{ref-norden2008}{}}%
Norden, B., Förster, A., \& Balling, N. (2008). Heat flow and lithospheric thermal regime in the {Northeast German Basin}. \emph{Tectonophysics}, \emph{460}, 215--229.

\leavevmode\vadjust pre{\hypertarget{ref-nouze2009}{}}%
Nouzé, H., Cosquer, E., Collot, J., Foucher, J.-P., Klingelhoefer, F., Lafoy, Y., \& Géli, L. (2009). Geophysical characterization of bottom simulating reflectors in the {Fairway Basin} (off {New Caledonia, Southwest Pacific}), based on high resolution seismic profiles and heat flow data. \emph{Marine Geology}, \emph{266}, 80--90.

\leavevmode\vadjust pre{\hypertarget{ref-novak1971}{}}%
Novak, V. (1971). Zemsky tepelny tok v hlubinnych vrtech {Zarosice-1 A 2} v oblasti zdanickeho lesa (in czech). \emph{Vestnik Ustr. Ust. Geol.}, \emph{46}, 277--284.

\leavevmode\vadjust pre{\hypertarget{ref-russ1983}{}}%
Novosibirsk., T. I. K. Sibiri. -. (1983). Izdatelstvo nauka.

\leavevmode\vadjust pre{\hypertarget{ref-nurusman1986}{}}%
Nurusman, S. (1986). (PhD thesis).

\leavevmode\vadjust pre{\hypertarget{ref-nurusman1995}{}}%
Nurusman, S., \& Subono, S. (1995). Heat flow measurements in indonesia. In \emph{Terrestrial heat flow and geothemal energy in asia} (pp. 145--162). Science Publ.

\leavevmode\vadjust pre{\hypertarget{ref-nyblade1997}{}}%
Nyblade, A. A. (1997). Heat flow across the east african plateau. \emph{Geophysical Research Letters}, \emph{24}(16), 2083--2086. \url{https://doi.org/10.1029/97gl01952}

\leavevmode\vadjust pre{\hypertarget{ref-nyblade1990}{}}%
Nyblade, A. A., Pollack, H. N., Jones, D. L., Podmore, F., \& Mushayandebvu, M. (1990). Terrestrial heat flow in east and southern {Africa}. \emph{J. Geophys. Res.}, \emph{95}, 17371--17384.

\leavevmode\vadjust pre{\hypertarget{ref-nyblade1996}{}}%
Nyblade, A. A., Suleiman, I. S., Roy, R. F., Pursell, B., Suleiman, A. S., Doser, D. I., \& Keller, G. R. (1996). Terrestrial heat flow in the sirt basin, libya, and the pattern of heat flow across northern africa. \emph{Journal of Geophysical Research}, \emph{101}(b8), 17--737. \url{https://doi.org/10.1029/96jb01177}

\leavevmode\vadjust pre{\hypertarget{ref-oregan2016}{}}%
O'Regan, M., Preto, P., Stranne, C., Jakobsson, M., \& Koshurnikov, A. (2016). Surface heat flow measurements from the east siberian continental slope and southern lomonosov ridge, arctic ocean. \emph{Geochemistry, Geophysics, Geosystems}, \emph{17}(5), 1608--1622. \url{https://doi.org/10.1002/2016gc006284}

\leavevmode\vadjust pre{\hypertarget{ref-omura1994}{}}%
Omura, K., Ikeda, R., Horai, K. I., \& Kobayashi, Y. (1994). Terrestrial heat flow in an active seismic region: A precise measurement in the ashio 2km deep borehole.

\leavevmode\vadjust pre{\hypertarget{ref-omura1995}{}}%
Omura, K., Horai, K. I., Kobayashi, Y., \& Ikeda, R. (1995). A relationship between the cutoff depth of seismicity and the thermal structure in the crust-measurement of terrestrial heat flow in {Neo, Gifu Prefecture}.

\leavevmode\vadjust pre{\hypertarget{ref-onuoha1999}{}}%
Onuoha, K. M., \& Ekine, A. S. (1999). Subsurface temperature variations and heat flow in the anambra basin, nigeria. \emph{Journal of African Earth Sciences}, \emph{28}(3), 641--652. \url{https://doi.org/10.1016/s0899-5362(99)00036-6}

\leavevmode\vadjust pre{\hypertarget{ref-ostrihansky1980}{}}%
Ostrihansky, L. (1980). The structure of the earth's crust and the heat-flow---heat generation relationship in the {Bohemian Massif}. \emph{Tectonophysics}, \emph{68}, 325--337.

\leavevmode\vadjust pre{\hypertarget{ref-oxburgh1982}{}}%
Oxburgh, E. R. (1982). \emph{Compilation of heat flow data measured by the university of OXford heat flow group for the department of energy}.

\leavevmode\vadjust pre{\hypertarget{ref-oxburgh1977}{}}%
Oxburgh, E. R., Richardson, S. W., Bloomer, J. R., Martin, A., \& Wright, S. (1977). Sub-surface temperatures from heat flow studies in the {United Kingdom}. \emph{Semin. Geothermal Energy (Commission of the European Communities)}, \emph{1}, 155--173.

\leavevmode\vadjust pre{\hypertarget{ref-oxburgh1980}{}}%
Oxburgh, E. R., Richardson, S. W., Wright, S. M., Jones, M. O. R., Penney, S. R., Watson, S. A., \& Bloomer, J. R. (1980). Heat flow pattern of the united kingdom. D. Reidel Publishing.

\leavevmode\vadjust pre{\hypertarget{ref-palmason1967}{}}%
Pálmason, G. (1967). On heat flow in {Iceland} in relation to the {Mid-Atlantic Ridge}. In \emph{Iceland and mid-ocean ridges} (Vol. Rit. 38, pp. 111--127). Soc. Sci. Islandica.

\leavevmode\vadjust pre{\hypertarget{ref-palmason1971}{}}%
Pálmason, G. (1971). \emph{Crustal structure of iceland from explosion seismology} (Vol. Rit. 40, pp. 187 pp.). Soc. Sci. Islandica.

\leavevmode\vadjust pre{\hypertarget{ref-palmason1973}{}}%
Pálmason, G. (1973). Kinematics and heat flow in a volcanic rift zone, with application to {Iceland}. \emph{Geophys. J. Roy. Astr. Soc.}, \emph{33}, 451--481.

\leavevmode\vadjust pre{\hypertarget{ref-pandey1981a}{}}%
Pandey, O. P. (1981). Terrestrial heat flow in the north island of new zealand. \emph{Journal of Volcanology and Geothermal Research}, \emph{10}(4), 309--316. \url{https://doi.org/10.1016/0377-0273(81)90083-4}

\leavevmode\vadjust pre{\hypertarget{ref-pandey1991}{}}%
Pandey, O. P. (1991). Terrestrial heat flow and lithospheric geothermal structure in {New Zealand}. In \emph{Terrestrial heat flow and the lithosphere structure} (pp. 338--380). Springer Verlag.

\leavevmode\vadjust pre{\hypertarget{ref-parasnis1982}{}}%
Parasnis, D. S. (1982). Geothermal flow and phenomena in two swedish localities north of the arctic circle. \emph{Geophysical Journal of the Royal Astronomical Society}, \emph{71}, 545--554. \url{https://doi.org/10.1111/j.1365-246X.1982.tb02782.x}

\leavevmode\vadjust pre{\hypertarget{ref-parasnis1989}{}}%
Parasnis, D. S. (1989). \emph{Temperatures in sweden, compilation}.

\leavevmode\vadjust pre{\hypertarget{ref-leg87_1983}{}}%
Party, L. 87. S. (1983). Leg 87 drills of {Honshu and SW Japan}. \emph{Geotimes}, \emph{28}, 15--18.

\leavevmode\vadjust pre{\hypertarget{ref-pasquale2012}{}}%
Pasquale, V., Chiozzi, P., Verdoya, M., \& Gola, G. (2012). Heat flow in the {Western Po Basin} and the surrounding orogenic belts. \emph{Geophys. J. Int.}, \emph{190}, 8--22. \url{https://doi.org/10.1111/j.1365-246X.2012.05486.x}

\leavevmode\vadjust pre{\hypertarget{ref-paterson1966}{}}%
Paterson, W. S. B., \& Law, L. K. (1966). Additional heat flow determinations in the area of {Mould Bay}, arctic {Canada}. \emph{Can. J. Earth Sci.}, \emph{3}, 237--246.

\leavevmode\vadjust pre{\hypertarget{ref-peng2015}{}}%
Peng, T., Wu, J.-W., Ren, Z.-Q., Xu, S.-P., \& Zhang, H.-C. (2015). Distribution of terrestrial heat flow and structural control in huainan-huaibei coalfield.chinese journal geophysics,. \emph{Chinese Journal Geophysics}, \emph{58}(7), 2391--2401. \url{https://doi.org/10.6038/cjg20150716}

\leavevmode\vadjust pre{\hypertarget{ref-perry2004}{}}%
Perry, H. K. C., Jaupart, C., Mareschal, J. C., Rolandone, F., \& Bienfait, G. (2004). Heat flow in the {Nipigon} arm of the {Keweenawan} rift, northwestern {Ontario, Canada}. \emph{Geophys. Res. Lett.}, \emph{31}, l15607, doi:10.1029/2004GL020159.

\leavevmode\vadjust pre{\hypertarget{ref-perry2006}{}}%
Perry, H. K. C., Jaupart, C., Mareschal, J.-C., \& Bienfait, G. (2006). Crustal heat production in the superior province. \emph{J. Geophys. Res.}, \emph{111}, b04401, doi:10.1029/2005JB003893.

\leavevmode\vadjust pre{\hypertarget{ref-perusini1982}{}}%
Perusini, P., Squarci, P., Taffi, L., Loddo, M., Mongelli, F., \& Tramacere, A. (1982). Misure di flusso di calore nella "dorsale medio toscana" tra monticiano e roccastrada. In \emph{Energia geotermica: Prospettive aperte dalle ricerche del CNR} (Vol. Cnr--pfe--speg--3, pp. 99--112).

\leavevmode\vadjust pre{\hypertarget{ref-pfister1998}{}}%
Pfister, M., Rybach, L., \& Simsek, S. (1998). Geothermal reconnaissance of the {Marmara Sea} region {(NW Turkey)}: Surface heat flow density in an area of active continental extension. \emph{Tectonophysics}, \emph{291}, 77--89.

\leavevmode\vadjust pre{\hypertarget{ref-phillips1969}{}}%
Phillips, J. D., Thompson, R. P., Von Herzen, R. P., \& Bowen, V. T. (1969). Mid-atlantic ridge near 43N latitude. \emph{J. Geophys. Res.}, \emph{74}, 3069.

\leavevmode\vadjust pre{\hypertarget{ref-pinet1991}{}}%
Pinet, C., Jaupart, C., Mareschal, J.-C., Gariépy, C., Bienfait, G., \& Lapointe, R. (1991). Heat flow and structure of the lithosphere in the eastern {Canadian Shield}. \emph{J. Geophys. Res.}, \emph{96}, 19941--19963.

\leavevmode\vadjust pre{\hypertarget{ref-plewa1988}{}}%
Plewa, M. (1988). Analiza gestości powierzchniowego strumienia cieplnego ziemi na obszarze polski. \emph{Zeszyty Naukowe AG, Krakow. Geofizyka Stosowana}, \emph{1}, 110--124.

\leavevmode\vadjust pre{\hypertarget{ref-plewa1989}{}}%
Plewa, M. (1989). \emph{Wyniki badań cieplnej przewodności wiaściwej i gestości powierzchniowego strumienia cieplnego ziemi w otworze kuźmina --- 1 na podstawie pomiarów laboratoryjnych probek saka{l}}.

\leavevmode\vadjust pre{\hypertarget{ref-plewa1991}{}}%
Plewa, M., Plewa, S., Poprawa, D., \& Tomaś, A. (1991). Catalogue of heat flow density data: poland. In \emph{Geothermal atlas of europe} (p. 122). Hermann Haack Verlagsgesellschaft mbH.

\leavevmode\vadjust pre{\hypertarget{ref-plewa1966}{}}%
Plewa, S. (1966). \emph{Regionalny obraz parameterow geotermicznych obszaru {Polski} (in polish)} (pp. pp. 88). Prace Geof. i Geol.

\leavevmode\vadjust pre{\hypertarget{ref-pollett2019}{}}%
Pollett, A., Hasterok, D., Raimondo, T., Halpin, J. A., Hand, M., Bendall, B., \& McLaren, S. (2019a). Heat flow in southern australia and connections with east antarctica. \emph{Geochemistry, Geophysics, Geosystems}, \emph{20}(11), 5352--5370. https://doi.org/\url{https://doi.org/10.1029/2019GC008418}

\leavevmode\vadjust pre{\hypertarget{ref-pollett2019b}{}}%
Pollett, A., Thiel, S., Bendall, B., Raimondo, T., \& Hand, M. (2019b). Mapping the gawler craton--musgrave province interface using integrated heat flow and magnetotellurics. \emph{Tectonophysics}, \emph{756}, 43--56. https://doi.org/\url{https://doi.org/10.1016/j.tecto.2019.02.017}

\leavevmode\vadjust pre{\hypertarget{ref-polyak1966}{}}%
Polyak, B. G. (1966). Geotermicheskie osobennosti oblasti sovremennogo vulkanizma (na primere kamchatki). - moskva: nauka.

\leavevmode\vadjust pre{\hypertarget{ref-polyak1996}{}}%
Polyak, B. G., Fernandez, M., Khutorskoy, M. D., Soto, J. I., Basov, I. A., Comas, M. C., et al. (1996). Heat flow in the alboran sea, western mediterranean. \emph{Tectonophysics}, \emph{263}(1-4), 191--218. \url{https://doi.org/10.1016/0040-1951(95)00178-6}

\leavevmode\vadjust pre{\hypertarget{ref-poort2004}{}}%
Poort, J., \& Klerkx, J. (2004). Absence of a regional surface thermal high in the baikal rift; new insights from detailed contouring of heat flow anomalies. \emph{Tectonophysics}, \emph{383}(3-4), 217--241. \url{https://doi.org/10.1016/j.tecto.2004.03.011}

\leavevmode\vadjust pre{\hypertarget{ref-poort2010}{}}%
Poort, J., Rimi, A., Lucazeau, F. A. M., \& Bouquerel, H. (2010). Low heat flow in the atlas mountains and the implications for the origin of the uplift. Retrieved from \href{http://meetingorganizer.copernicus.org/EGU2010/\%20EGU2010-10801-1.pdf}{http://meetingorganizer.copernicus.org/EGU2010/ EGU2010-10801-1.pdf}

\leavevmode\vadjust pre{\hypertarget{ref-poort2020}{}}%
Poort, J., Lucazeau, F., Le Gal, V., Dal Cin, M., Leroux, E., Bouzid, A., et al. (2020). Heat flow in the western mediterranean: Thermal anomalies on the margins, the seafloor and the transfer zones. \emph{Marine Geology}, \emph{419}, 106064. https://doi.org/\url{https://doi.org/10.1016/j.margeo.2019.106064}

\leavevmode\vadjust pre{\hypertarget{ref-popov1974}{}}%
Popov, A. K. (1974). Rezultaty izmereniy teplovogo potoka na akvatoriyakh. \emph{Geotermiya (Russian)}, \emph{1-2}, 81--86.

\leavevmode\vadjust pre{\hypertarget{ref-popov2003}{}}%
Popov, Y., Pohl, J., Romushkevich, R., Tertychnyi, V., \& Soffel, H. (2003). Geothermal characteristics of the {Ries} impact structure. \emph{Geophys. J. Int.}, \emph{154}, 355--378.

\leavevmode\vadjust pre{\hypertarget{ref-popov1998}{}}%
Popov, Y. A., Pimenov, V. P., Pevzner, L. A., Romushkevich, R. A., \& Popov, E. Y. (1998). Geothermal characteristics of the vorotilovo deep borehole drilled into the puchezh-katunk impact structure. \emph{Tectonophysics}, \emph{291}(1-4), 205--223. \url{https://doi.org/10.1016/s0040-1951(98)00041-9}

\leavevmode\vadjust pre{\hypertarget{ref-popov1999}{}}%
Popov, Y. A., Pevzner, S. L., Pimenov, V. P., \& Romushkevich, R. A. (1999). New geothermal data from the {Kola} superdeep well {SG-3}. \emph{Tectonophysics}, \emph{306}, 345--366.

\leavevmode\vadjust pre{\hypertarget{ref-popova1974}{}}%
Popova, A. K. (1974). Rezultaty izmereniya teplovogo potoka na akvatoriyakh (russ.). \emph{Geotermiya. Otchety Po Geotermicheskim Issledovaniyam V Sssr. Vyp. 1-2. Otchety Za 1971-1972 Gg. Moskva}, 81--86.

\leavevmode\vadjust pre{\hypertarget{ref-powell1997}{}}%
Powell, W. G. (1997). \emph{Thermal state of the lithosphere in the {Colorado Plateau--Basin and Range} transition zone, {Utah}} (PhD thesis).

\leavevmode\vadjust pre{\hypertarget{ref-powell1990}{}}%
Powell, W. G., \& Chapman, D. S. (1990). A detailed study of heat flow at the {Fifth Water Site, Utah}, in the {Basin and Range--Colorado Plateaus} transition. \emph{Tectonophysics}, \emph{176}, 291--314.

\leavevmode\vadjust pre{\hypertarget{ref-pribnow2000a}{}}%
Pribnow, D. F. C., Kinoshita, M., \& Stein, C. A. (2000). \emph{Thermal data collection and heat flow recalculations for ODP legs 101-180} (Vol. 120432, p. --). Retrieved from \url{http://www-odp.tamu.edu/publications/heatflow/}

\leavevmode\vadjust pre{\hypertarget{ref-prol1989}{}}%
Prol-Ledesma, R. M., Sugrobov, V. M., Flores, E. L., Juarez, G., Smirnov, Ya. B., Gorshkov, A. P., et al. (1989). Heat flow variations along the {Middle America} trench. \emph{Marine Geophysical Researches}, \emph{11}, 69--76.

\leavevmode\vadjust pre{\hypertarget{ref-prol2018}{}}%
Prol-Ledesma, R. M., Carrillo de la Cruz, J. L., Torres-Vera, M. A., Membrillo-Abad, A. S., \& Espinoza-Ojeda, O. M. (2018). Heat flow map and geothermal resources in mexico. \emph{Terra Digitalis}, \emph{2}(2), 1--15. \url{https://doi.org/10.22201/igg.25940694.2018.2.51.105}

\leavevmode\vadjust pre{\hypertarget{ref-pugh1977}{}}%
Pugh, D. T. (1977). Geothermal gradients in {British} lake sediments. \emph{Limnology and Oceanography}, \emph{22}, 581--596.

\leavevmode\vadjust pre{\hypertarget{ref-puranen1968}{}}%
Puranen, M., Jarvimaki, P., Hamalainen, U., \& Lehtinen, S. (1968). Terrestrial heat flow in {Finland}. \emph{Geoexploration}, \emph{6}, 151--162.

\leavevmode\vadjust pre{\hypertarget{ref-purss2001}{}}%
Purss, M. B. J., \& Cull, J. (2001). Heat-flow data in {Western Victoria}. \emph{Australian Journal of Earth Sciences}, \emph{48}(1), 1--4. \url{https://doi.org/10.1046/j.1440-0952.2001.00840.x}

\leavevmode\vadjust pre{\hypertarget{ref-pye1972}{}}%
Pye, G. D., \& Hyndman, R. D. (1972). Heat-flow measurements in baffin bay and the labrador sea. \emph{Journal of Geophysical Research}, \emph{77}, 934--944. \url{https://doi.org/10.1029/JB077i005p00938}

\leavevmode\vadjust pre{\hypertarget{ref-qiu2003}{}}%
Qiu, N. (2003). Geothermal regime in the QaIdam basin, northeast qinghai-tibet plateau. \emph{Geological Magazine}, \emph{140}(6), 707--719. \url{https://doi.org/10.1017/s0016756803008136}

\leavevmode\vadjust pre{\hypertarget{ref-rabinowitz1980}{}}%
Rabinowitz, P. D., \& Ludwig, W. J. (1980). Geophysical measurements at candidate drill sites along an east-west flow line in the central {Atlantic Ocean}. \emph{Marine Geology}, \emph{35}, 243--275.

\leavevmode\vadjust pre{\hypertarget{ref-raksaskulwong1995}{}}%
Raksaskulwong, M., \& Thienprasert, A. (1995). Heat flow studies and geothermal energy development in {Thailand}. In \emph{Terrestrial heat flow and geothermal energy in asia} (pp. 129--144). Science Publ.

\leavevmode\vadjust pre{\hypertarget{ref-ramaekers1991}{}}%
Ramaekers, J. J. F. (1991). Catalogue of heat flow density data: The netherlands. In \emph{Geothermal atlas of europe} (pp. 126--128). Hermann Haack Verlagsgesellschaft mbH.

\leavevmode\vadjust pre{\hypertarget{ref-rankin1974}{}}%
Rankin, D. S. (1974). \emph{Heat flow--heat production studies in nova scotia} (PhD thesis).

\leavevmode\vadjust pre{\hypertarget{ref-rankin1971}{}}%
Rankin, D. S., \& Hyndman, R. D. (1971). Shallow water heat flow measurements in bras d'or lake, nova scotia. \emph{Revue Canadienne Des Sciences de La Terre}, \emph{8}(1), 96--101. \url{https://doi.org/10.1139/e71-006}

\leavevmode\vadjust pre{\hypertarget{ref-rao1980}{}}%
Rao, G. V., \& Rao, R. U. M. (1980). A geothermal study of the {Jharia Gondwana Basin (India)}: Heat flow results from several holes and heat production of basement rocks. \emph{Earth Planet. Sci. Lett.}, \emph{48}, 397--05.

\leavevmode\vadjust pre{\hypertarget{ref-rao1983}{}}%
Rao, G. V., \& Rao, R. U. M. (1983). Heat flow in indian gondwana basins and heat production in basement rocks. \emph{Tectonophysics}, \emph{91}, 105--117.

\leavevmode\vadjust pre{\hypertarget{ref-rao1974}{}}%
Rao, R. U. M., \& Rao, G. V. (1974). Results of some geothermal studies in {Singhbhum Thrust Belt, India}. \emph{Geothermics}, \emph{3}, 153--161.

\leavevmode\vadjust pre{\hypertarget{ref-rao1970a}{}}%
Rao, R. U. M., Verma, R. K., Rao, G. V., \& Gupta, M. L. (1970a). Heat flow at damua and mohapani, satpura gondwana basin, india. \emph{Earth and Planetary Science Letters}, \emph{7}, 406--412. \url{https://doi.org/10.1016/0012-821x(70)90082-8}

\leavevmode\vadjust pre{\hypertarget{ref-rao1970b}{}}%
Rao, R. U. M., Verma, R. K., Rao, G. V., Hamza, V. M., Panda, P. K., \& Gupta, M. L. (1970b). Heat flow studies in the godavari valley (india). \emph{Tectonophysics}, \emph{10}, 165--181. \url{https://doi.org/10.1016/0040-1951(70)90105-8}

\leavevmode\vadjust pre{\hypertarget{ref-rao1976}{}}%
Rao, R. U. M., Rao, G. V., \& Narain, H. (1976). Radioactive heat generation and heat flow in the indian shield. \emph{Earth and Planetary Science Letters}, \emph{30}, 57--64.

\leavevmode\vadjust pre{\hypertarget{ref-rao2013}{}}%
Rao, S., Hu, S.-B., Zhu, C.-Q., Tang, X.-Y., Li, W.-W., \& Wang, J.-Y. (2013). Characteristics of heat flow and lithospheric thermal structure in the junggar basin, northwestern china. \emph{Chinese Journal of Geophysics}, \emph{56}(5), 661--673. \url{https://doi.org/10.1002/cjg2.20061}

\leavevmode\vadjust pre{\hypertarget{ref-rao2016}{}}%
Rao, S., Jiang, G.-Z., Gao, Y.-J., Hu, S.-B., \& Wang, J.-Y. (2016). The thermal structure of the lithosphere and heat source mechanism of geothermal field in weihe basin. \emph{Chinese Journal Geophysics}, \emph{59}, 2176--2190. \url{https://doi.org/10.6038/cjg20160622}

\leavevmode\vadjust pre{\hypertarget{ref-ravnik1991}{}}%
Ravnik, D. (1991). Catalogue of heat flow density data: yugoslavia. In \emph{Geothermal atlas of europe} (pp. 152--153). Hermann Haack Verlagsgesellschaft mbH.

\leavevmode\vadjust pre{\hypertarget{ref-ray2003}{}}%
Ray, L., Kumar, P. S., Reddy, G. K., Roy, S., Rao, G. V., Srinivasan, R., \& Rao, R. U. M. (2003). High mantle heat flow in a {Precambrian} granulite province: Evidence from southern {India}. \emph{J. Geophys. Res.}, \emph{108}, doi:10.1029/2001JB000688.

\leavevmode\vadjust pre{\hypertarget{ref-rehault1981}{}}%
Rehault, J. P. (1981). \emph{Evolution tectonique et sédimnetaire du bassin ligure, mediterranée occidentale} (PhD thesis).

\leavevmode\vadjust pre{\hypertarget{ref-reiter1976}{}}%
Reiter, M., Weidman, C., Edwards, C. L., \& Hartman, H. (1976a). \emph{Subsurface temperature data in {Jemez Mountains, New Mexico}} (No. 151).

\leavevmode\vadjust pre{\hypertarget{ref-reiter1977}{}}%
Reiter, M. A., \& Smith, R. B. (1977). Subsurface temperature data in the socorro peak KGRA, new mexico. \emph{Geothermal Energy Magazine}, \emph{5}, 37--41.

\leavevmode\vadjust pre{\hypertarget{ref-reiter1975}{}}%
Reiter, M. A., Edwards, C. L., Hartmann, H., \& Weidman, C. (1975). Terrestrial heat flow along the rio grande rift, new mexico and southern colorado. \emph{Geological Society of America Bulletin}, \emph{86}, 811--818. \url{https://doi.org/10.1130/0016-7606(1975)86\%3C811:thfatr\%3E2.0.co;2}

\leavevmode\vadjust pre{\hypertarget{ref-reiter1976b}{}}%
Reiter, M. A., Simmons, G., Chessman, M. D., England, T., Hartmann, H., \& Weidman, C. (1976b). \emph{Terrestrial heat flow near datil, new mexico} (No. 33-37) (p. --).

\leavevmode\vadjust pre{\hypertarget{ref-reitzel1961}{}}%
Reitzel, J. S. (1961). Some heat-flow measurements in the {North Atlantic}. \emph{Journal of Geophysical Research}, \emph{66}, 2267--2268. \url{https://doi.org/10.1029/JZ066i007p02267}

\leavevmode\vadjust pre{\hypertarget{ref-reitzel1963}{}}%
Reitzel, J. S. (1963). A region of uniform heat flow in the {North Atlantic}. \emph{Journal of Geophysical Research}, \emph{68}, 5191--5196. \url{https://doi.org/10.1029/JZ068i018p05191}

\leavevmode\vadjust pre{\hypertarget{ref-ren2015}{}}%
Ren, Z.-Q., Peng, T., Shen, S.-H., Zhang, H.-C., Xu, S.-P., \& Wu, J.-W. (2015). The distribution characteristics of current geothermal field in huainan coalfield. \emph{Acta Metallurgica Sinica}, \emph{21}(1), 147--154. \url{https://doi.org/10.16108/j.issn1006-7493.20141}

\leavevmode\vadjust pre{\hypertarget{ref-revelle1952}{}}%
Revelle, R., \& Maxwell, A. E. (1952). Heat flow through the floor of the eastern north {Pacific Ocean}. \emph{Nature}, \emph{170}, 199--200.

\leavevmode\vadjust pre{\hypertarget{ref-rhea1964}{}}%
Rhea, K., Northrop, J., \& Von Herzen, R. P. (1964). Heat-flow measurements between {North America and the Hawaiian Islands}. \emph{Marine Geol.}, \emph{1}, 220--224.

\leavevmode\vadjust pre{\hypertarget{ref-richardson1981}{}}%
Richardson, S. W., \& Jones, M. Q. W. (1981). Measurements of thermal conductivity of drill cuttings in the {Marchwood} geothermal borehole --- a preliminary assessment of the resource. In \emph{Investigations of the geothermal potential of the UK} (pp. 60--62). Institute of Geological Sciences.

\leavevmode\vadjust pre{\hypertarget{ref-richardson1978}{}}%
Richardson, S. W., \& Oxburgh, E. R. (1978). Heat flow, radiogenic heat production and crustal temperatures in england and wales. \emph{Journal of the Geological Society London}, \emph{135}(3), 323--337. \url{https://doi.org/10.1144/gsjgs.135.3.0323}

\leavevmode\vadjust pre{\hypertarget{ref-riedel2006}{}}%
Riedel, M., Novosel, I., Spence, G. D., Hyndman, R. D., Chapman, R. N., Solem, R. C., \& Lewis, T. (2006). Geophysical and geochemical signatures associated with gas {hydrate--related} venting in the northern {Cascadia} margin. \emph{Geol. Soc. Am. Bull.}, \emph{118}, 23--38. \href{https://doi.org/doi:\%2010.1130/B25720.1}{https://doi.org/doi: 10.1130/B25720.1}

\leavevmode\vadjust pre{\hypertarget{ref-rimi1990}{}}%
Rimi, A. (1990). Geothermal gradients and heat flow trends in morocco. \emph{Geothermics}, \emph{19}, 443--454. \url{https://doi.org/10.1016/0375-6505(90)90057-i}

\leavevmode\vadjust pre{\hypertarget{ref-rimi1987}{}}%
Rimi, A., \& Lucazeau, F. (1987). Heat flow density measurements in northern morocco. \emph{Journal of African Earth Sciences}, \emph{6}(6), 835--843. \url{https://doi.org/10.1016/0899-5362(87)90041-8}

\leavevmode\vadjust pre{\hypertarget{ref-rimi1998}{}}%
Rimi, A., Chalouan, A., \& Bahi, L. (1998). Heat flow in the westernmost part of the alpine mediterranean system (the rif, morocco). \emph{Tectonophysics}, \emph{285}, 135--146. \url{https://doi.org/10.1016/s0040-1951(97)00185-6}

\leavevmode\vadjust pre{\hypertarget{ref-ritter2004}{}}%
Ritter, U., Zielinski, G. W., Weiss, H. M., Zielinski, R. L., \& Saettern, J. (2004). Heat flow in the {V{o}ring Basin, mid-Norwegian shelf}. \emph{Petroleum Geoscience}, \emph{10}, 353--365.

\leavevmode\vadjust pre{\hypertarget{ref-roberts1984}{}}%
Roberts, D. G. (1984). \emph{Initial Reports DSDP}, \emph{81}, 898.

\leavevmode\vadjust pre{\hypertarget{ref-rolandone2002}{}}%
Rolandone, F., Jaupart, C., Mareschal, J. C., Gariépy, C., Bienfait, G., Carbonne, C., \& Lapointe, R. (2002). Surface heat flow, crustal temperatures and mantle heat flow in the {Proterozoic Trans-Hudson Orogen, Canadian Shield}. \emph{J. Geophys. Res.}, \emph{107}, doi:10.1029/2001JB000698.

\leavevmode\vadjust pre{\hypertarget{ref-rolandone2013}{}}%
Rolandone, F., Lucazeau, F. S. L., Mareschal, J.-C., Jorand, R., Goutorbe, B., \& Bouquerel, H. (2013). New heat flow measurements in oman and the thermal state of the arabian shield and platform. \emph{Tectonophysics}, \emph{589}, 77--89. \url{https://doi.org/10.1016/j.tecto.2012.12.034,}

\leavevmode\vadjust pre{\hypertarget{ref-rollin1991}{}}%
Rollin, K. (1991). Catalogue of heat flow density data: United kingdom. In \emph{Geothermal atlas of europe} (pp. 129--131). Hermann Haack Verlagsgesellschaft mbH.

\leavevmode\vadjust pre{\hypertarget{ref-rona1996}{}}%
Rona, P. A., Petersen, S., Becker, K., Herzen, R. P. V., Hannington, M. D., Herzig, P., et al. (1996). Heat flow and mineralogy of TAG relict high-temperature hydrothermal zones: {Mid-Atlantic Ridge} 26\(\circ\)n, 45\(\circ\)w. \emph{Geophys. Res. Lett.}, \emph{23}, 3507--3510.

\leavevmode\vadjust pre{\hypertarget{ref-roy1999}{}}%
Roy, S., \& Rao, R. U. M. (1999). Geothermal investigations in the 1993 {Latur} earthquake area, {Deccan} volcanic province, {India}. \emph{Tectonophysics}, \emph{306}, 237--252.

\leavevmode\vadjust pre{\hypertarget{ref-roy2000}{}}%
Roy, S., \& Rao, R. U. M. (2000). Heat flow in the {Indian} shield. \emph{J. Geophys. Res.}, \emph{105}, 25587--25604.

\leavevmode\vadjust pre{\hypertarget{ref-roy2008}{}}%
Roy, S., Ray, L., Bhattacharya, A., \& Sirnivasan, R. (2008). Heat flow and crustal thermal structure in the {Late Archaean Clospet granite batholith, south India}. \emph{Int. J. Earth Sci.}, \emph{97}, 245--256.

\leavevmode\vadjust pre{\hypertarget{ref-ruppel1995}{}}%
Ruppel, C., Von Herzen, R. P., \& Bonneville, A. (1995). Heat flux through an old (\(\sim\)175 ma) passive margin: Offshore southeastern {United States}. \emph{J. Geophys. Res.}, \emph{100}, 20037--20057.

\leavevmode\vadjust pre{\hypertarget{ref-keldysh1998}{}}%
\emph{R/v akademik mstislav keldysh 40th crusie report, 1998}. (1998).

\leavevmode\vadjust pre{\hypertarget{ref-rybach1991}{}}%
Rybach, L. (1991). Catalogue of heat flow density data: switzerland. In \emph{Geothermal atlas of europe} (pp. 111--112). Hermann Haack Verlagsgesellschaft mbH.

\leavevmode\vadjust pre{\hypertarget{ref-rybach1977}{}}%
Rybach, L., Werner, D., Mueller, S., \& Berset, G. (1977). Heat flow, heat production and crustal dynamics in the central {Alps, Switzerland}. \emph{Tectonophysics}, \emph{41}, 113--126.

\leavevmode\vadjust pre{\hypertarget{ref-saettem1988}{}}%
Saettem, J. (1988). Varmestrømsmaelinger i barentshavet (pp. 406--408).

\leavevmode\vadjust pre{\hypertarget{ref-safanda1995}{}}%
Safanda, J., Kresl, M., Cermak, V., Hasanean, A. R. G., Deebes, H. A., Abd-Alla, M. A., \& Moustafa, S. M. (1995). Subsurface temperature measurements and terrestrial heat flow estimates in the {Aswan region, Egypt}. \emph{Studia Geoph. Et Geod.}, \emph{39}, 162--176.

\leavevmode\vadjust pre{\hypertarget{ref-saki1985}{}}%
Saki, T., Kaneda, Y., \& Aoyagi, K. (1985). Measurement of heat flow in the continental shelf of the {Japan Sea}.

\leavevmode\vadjust pre{\hypertarget{ref-salat1967}{}}%
Salat, P. (1967). \emph{Terrestrial heat flow in the mecsek mts. (In hungarian)} (PhD thesis).

\leavevmode\vadjust pre{\hypertarget{ref-salat1968}{}}%
Salat, P. (1968). \emph{The measurements of terrestrial heat flow at budapest and recsk}.

\leavevmode\vadjust pre{\hypertarget{ref-salmi2014}{}}%
Salmi, M. S., Johnson, H. P., Tivey, M. A., \& Hutnak, M. (2014). Quantitative estimate of heat flow from a mid-ocean ridge axial valley, {Raven field, Juan de Fuca Ridge:} Observations and inferences. \emph{Journal of Geophysical Research}. \url{https://doi.org/10.1002/2014jb011086}

\leavevmode\vadjust pre{\hypertarget{ref-salnikov1976}{}}%
Salnikov, V. E. (1976a). Geotermicheskie gradienty i teplovoi potok v magni- togorskom megasinklinorii (russ.). \emph{Geotermiya. / Geotermicheskie Is- Sledovaniya V SSSR /. Chast 1 Moskva}, 36--44.

\leavevmode\vadjust pre{\hypertarget{ref-salnikov1976a}{}}%
Salnikov, V. E. (1976b). Teplovye potoki na yuzhnom urale (russ.). \emph{Geotermiya. / Geotermicheskie Issledovaniya V Sssr /. Chast 1. Moskva.}, 45--52.

\leavevmode\vadjust pre{\hypertarget{ref-salnikov1982}{}}%
Salnikov, V. E. (1982). Novye dannye o raspredelenii teplovogo potoka na yuzhnom urale (russ.). \emph{Doklady An SSSR}, \emph{265}(4), 944--947.

\leavevmode\vadjust pre{\hypertarget{ref-salnikov1977}{}}%
Salnikov, V. E., \& Ogarinov, I. S. (1977). Zona anomalno nizkikh teplovykh potokov na yuzhnom urale. \emph{Doklady an SSSR}, \emph{237}(1456-1459), 1456--1459.

\leavevmode\vadjust pre{\hypertarget{ref-saltus1991}{}}%
Saltus, R. W., \& Lachenbruch, A. H. (1991). Thermal evolution of the {Sierra Nevada}: Tectonic implications of new heat flow data. \emph{Tectonics}, \emph{10}, 325--344.

\leavevmode\vadjust pre{\hypertarget{ref-sams1986}{}}%
Sams, M., \& Thomas-Betts, A. (1986). \emph{Heat flow and temperature in the vicinity of the carmenellis plutons} (No. 23--25).

\leavevmode\vadjust pre{\hypertarget{ref-sarkar2005}{}}%
Sarkar, R. K., \& Singh, O. P. (2005). A note on the heat flow studies at sohagpur and raniganj coalfield areas, india. \emph{Acta Geophysica Polonica}, \emph{53}, 197--204.

\leavevmode\vadjust pre{\hypertarget{ref-sass1964a}{}}%
Sass, J. H. (1964a). Heat flow values from eastern australia. \emph{J. Geophys. Res.}, \emph{69}, 3889--3893.

\leavevmode\vadjust pre{\hypertarget{ref-sass1964b}{}}%
Sass, J. H. (1964b). Heat flow values from the {Precambrian shield of Western Australia}. \emph{J. Geophys. Res.}, \emph{69}, 299--308.

\leavevmode\vadjust pre{\hypertarget{ref-sass1980}{}}%
Sass, J. H., \& Behrendt, J. C. (1980). Heat flow from the liberian precambrian shield. \emph{Journal of Geophysical Research}, \emph{85}(b6), 3159--3162. \url{https://doi.org/10.1029/JB085iB06p03159}

\leavevmode\vadjust pre{\hypertarget{ref-sass1963}{}}%
Sass, J. H., \& Le Marne, A. E. (1963). Heat flow at {Broken Hill, New South Wales}. \emph{Geophys. J. Royal Astr. Soc.}, \emph{7}, 477--489.

\leavevmode\vadjust pre{\hypertarget{ref-sass1988}{}}%
Sass, J. H., \& Morgan, P. (1988). Conductive heat flux in VC-1 and the thermal regime of {Valles Caldera, Jemez Mountains, New Mexico}. \emph{J. Geophys. Res.}, \emph{93}, 6027--6039.

\leavevmode\vadjust pre{\hypertarget{ref-sass1970}{}}%
Sass, J. H., \& Munroe, R. J. (1970). Heat flow from deep boreholes on two island arcs. \emph{Journal of Geophysical Research}, \emph{75}, 4387--4395. \url{https://doi.org/10.1029/JB075i023p04387}

\leavevmode\vadjust pre{\hypertarget{ref-sass1967}{}}%
Sass, J. H., Clark, S. P., \& Jaeger, J. C. (1967). Heat flow in the {Snowy Mountains of Australia}. \emph{J. Geophys. Res.}, \emph{72}, 2635--2647.

\leavevmode\vadjust pre{\hypertarget{ref-sass1968}{}}%
Sass, J. H., Killeen, P. G., \& Mustonen, E. D. (1968). Heat flow and surface radioactivity in the quirke lake syncline near elliot lake, ontario, canada. \emph{Canadian Journal of Earth Sciences}, \emph{5}, 1417--1428. \url{https://doi.org/10.1139/e68-141}

\leavevmode\vadjust pre{\hypertarget{ref-sass1971b}{}}%
Sass, J. H., Lachenbruch, A. H., \& Munroe, R. J. (1971a). Thermal conductivity of rocks from measurements on fragments and its application to heat flow determinations. \emph{Journal of Geophysical Research}, \emph{76}, 3391--3401. \url{https://doi.org/10.1029/JB076i014p03391}

\leavevmode\vadjust pre{\hypertarget{ref-sass1971}{}}%
Sass, J. H., Lachenbruch, A. H., \& Jessop, A. M. (1971b). Uniform heat flow in a deep hole in the canadian shield and its paleoclimatic implications. \emph{Journal of Geophysical Research}, \emph{76}, 8586--8596. \url{https://doi.org/10.1029/JB076i035p08586}

\leavevmode\vadjust pre{\hypertarget{ref-sass1972}{}}%
Sass, J. H., Nielsen, B. L., Wollenberg, H. A., \& Munroe, R. J. (1972). Heat flow and surface radioactivity at two sites in south greenland. \emph{Journal of Geophysical Research}, \emph{77}, 6435--6444. \url{https://doi.org/10.1029/JB077i032p06435}

\leavevmode\vadjust pre{\hypertarget{ref-sass1974}{}}%
Sass, J. H., Nielson, B. L., Wollenberg, H. A., \& Munroe, R. J. (1974). Heat flow from eastern {Panama} and northwestern {Columbia}. \emph{Earth Planet. Sci. Lett.}, \emph{21}, 134--142.

\leavevmode\vadjust pre{\hypertarget{ref-sass1976}{}}%
Sass, J. H., Jaeger, J. C., \& Munroe, J. R. (1976). \emph{Heat flow and near surface radioactivity in australian continental crust} (No. 76-250).

\leavevmode\vadjust pre{\hypertarget{ref-sass1981b}{}}%
Sass, J. H., Munroe, R. J., \& Stone, C. (1981). \emph{Heat flow from five uranium test wells in west-central arizona} (No. 81-1089) (p. --).

\leavevmode\vadjust pre{\hypertarget{ref-sass1982}{}}%
Sass, J. H., Stone, C., \& Bills, D. J. (1982). \emph{Shallow subsurface temperatures and some estimates of heat flow from the colorado plateau of northeastern arizona} (No. 82-994) (p. --).

\leavevmode\vadjust pre{\hypertarget{ref-sass1983}{}}%
Sass, J. H., Lachenbruch, A. H., \& Smith, E. P. (1983). \emph{Temperature profiles from salt valley, utah, thermal conductivity of 10 samples from drill hole DOE-3, and preliminary estimates of heat flow} (No. 83-455) (p. --).

\leavevmode\vadjust pre{\hypertarget{ref-sass1985}{}}%
Sass, J. H., Lawver, L. A., \& Munroe, R. J. (1985). A heat-flow reconnaissance southeastern {Alaska}. \emph{Can. J. Earth Sci.}, \emph{22}, 416--421.

\leavevmode\vadjust pre{\hypertarget{ref-sass1994}{}}%
Sass, J. H., Lachenbruch, A. H., Galanis, S. P., Morgan, P., Priest, S. S., Moses, T. H., \& Munroe, R. J. (1994). Thermal regime of the southern basin and range province: 1. Heat flow data from arizona and the mojave desert of california and nevada. \emph{Journal of Geophysical Research}, \emph{99}(b11), 22093--22119. \url{https://doi.org/10.1029/94jb01891}

\leavevmode\vadjust pre{\hypertarget{ref-sass1997}{}}%
Sass, J. H., Williams, C. F., Lachenbruch, A. H., Galanis, S. P., \& Grubb, F. V. (1997). Thermal regime of the san andreas fault near parkfield, california. \emph{Journal of Geophysical Research}, \emph{102}(b12), 27575--27585. \url{https://doi.org/10.1029/JB102iB12p27575}

\leavevmode\vadjust pre{\hypertarget{ref-sato1984}{}}%
Sato, S., Asakura, N., Saki, T., Oikawa, N., \& Kaneda, Y. (1984). Preliminary results of geological and geophysical surveys in the {Ross Sea} and the {Dumont D'Urville Sea} off {Antarctica}. In \emph{Memoirs of the national institute of polar research} (pp. 62--92). National Institute of Polar Research.

\leavevmode\vadjust pre{\hypertarget{ref-saull1962}{}}%
Saull, V. A., Clark, T. H., Doig, R. P., \& Butler, R. B. (1962). Terrestrial heat flow in the {St. Lawrence} lowland of {Quebec}. \emph{Can. Min. Met. Bull.}, \emph{65}, 63--66.

\leavevmode\vadjust pre{\hypertarget{ref-savostin1979}{}}%
Savostin, L. A. (1979). Geotermicheskie issledovaniya. - v kn.: Geologo-geofizi- cheskie i podvodnye issledovaniya ozera baikal. Moskva: Institut okea- nologii an sssr, (russ.), 119--125.

\leavevmode\vadjust pre{\hypertarget{ref-schellschmidt2003}{}}%
Schellschmidt, R., Popov, Y. A., Kukkonen, I. T., Nover, G., Milanovsky, S. Y., Borevsky, L., et al. (2003). New heat flow data from the immediate vicinity of the kola superdeep borehole (p. --). Geophysical Research Abstracts.

\leavevmode\vadjust pre{\hypertarget{ref-schintgen2015}{}}%
Schintgen, T., Förster, A., Förster, H.-J., \& Norden, B. (2015). Surface heat flow and lithosphere thermal structure of the rhenohercynian zone in the greater luxembourg region. \emph{Geothermics}, \emph{56}, 93--109. https://doi.org/\url{http://dx.doi.org/10.1016/j.geothermics.2015.03.007}

\leavevmode\vadjust pre{\hypertarget{ref-schmidt2005}{}}%
Schmidt, M., Hensen, C., Mörz, T., Grevemeyer, C. M. I., Wallmann, K., Mau, S., \& N. Kaul, N. (2005). Methane hydrate accumulation in {{``Mound 11''}} mud volcano, {Costa Rica forearc}. \emph{Marine Geol.}, \emph{216}, 83--100.

\leavevmode\vadjust pre{\hypertarget{ref-schmidt2012}{}}%
Schmidt-Schierhorn, F., Kaul, N., Stephan, S., \& Villinger, H. (2012). Geophysical site survey results from north pond (mid-atlantic ridge). In \emph{Proceedings IODP} (Vol. 336, pp. 62 pp.). Integrated Ocean Drilling Program Management International, Inc. \url{https://doi.org/10.2204/iodp.proc.336.107.2012}

\leavevmode\vadjust pre{\hypertarget{ref-schoessler1959}{}}%
Schössler, K., \& J. Schwarzlose, 120. PP. (1959). \emph{Geophysikalische wärmeflussmessungen} (Vol. c75, pp. pp. 120). Freiberg. Forsuchungsh.

\leavevmode\vadjust pre{\hypertarget{ref-schroeder2011}{}}%
Schröder, H., Paulsen, T., \& Wonik, T. (2011). Thermal properties of the AND-2A borehole in the southern victoria land basin, McMurdo sound, antarctica. \emph{Geosphere}, \emph{7}(6), 1324--1330. \url{https://doi.org/10.1130/ges00690.1}

\leavevmode\vadjust pre{\hypertarget{ref-schubert1974}{}}%
Schubert, C. E., \& Peter, G. (1974). Heat flow northeast of guadeloupe island, lesser antilles. \emph{Journal of Geophysical Research}, \emph{79}, 2139--2140. \url{https://doi.org/10.1029/JB079i014p02139}

\leavevmode\vadjust pre{\hypertarget{ref-schuech1973}{}}%
Schuech, J. (1973). Measurements of heat flow in the red sea between 19 degrees and 26 degrees northern latitude (region of the brine deeps). \emph{Zeitschrift Für Geophysik}, \emph{39}, 859--862.

\leavevmode\vadjust pre{\hypertarget{ref-schultz1991}{}}%
Schultz, R., Haenel, R., \& Kockel, F. (1991). Catalogue of heat flow density data: Federal republic of germany (western federal states). In \emph{Geothermal atlas of europe} (p. 115). Hermann Haack Verlagsgesellschaft mbH.

\leavevmode\vadjust pre{\hypertarget{ref-schuetz2012b}{}}%
Schütz, F., Förster, H.-J., \& Förster, A. (2012a). Surface heat flow and pre-cenozoic lithosphere thermal structure of the northern sinai microplate in israel. \emph{Journal of Geophysical Research}, \emph{submitted}.

\leavevmode\vadjust pre{\hypertarget{ref-schuetz2012a}{}}%
Schütz, F., Norden, B., \& Förster, A., DESIRE Group. (2012b). Thermal properties of sediments in southern israel: A comprehensive data set for heat flow and geothermal energy studies. \emph{Basin Research}, \emph{24}(3), 357--376. \url{https://doi.org/10.1111/j.1365-2117.2011.00529.x}

\leavevmode\vadjust pre{\hypertarget{ref-schuetz2018}{}}%
Schütz, F., Winterleitner, G., \& Huenges, E. (2018). Geothermal exploration in a sedimentary basin: New continuous temperature data and physical rock properties from northern oman. \emph{Geothermal Energy}, \emph{6}(1), 5. \url{https://doi.org/10.1186/s40517-018-0091-6}

\leavevmode\vadjust pre{\hypertarget{ref-sclater1966}{}}%
Sclater, J. G. (1966). Heat flow in the northwest {Indian Ocean and Red Sea}. \emph{Philosophical Transaction of the Royal Astronomy Society, Ser. A}, \emph{259}, 271--278. \url{https://doi.org/10.1098/rsta.1966.0012}

\leavevmode\vadjust pre{\hypertarget{ref-sclater1967}{}}%
Sclater, J. G., \& Corry, C. E. (1967). Heat flow, hawaiian area. \emph{Journal of Geophysical Research}, \emph{72}, 3711--3715. \url{https://doi.org/10.1029/JZ072i014p03711}

\leavevmode\vadjust pre{\hypertarget{ref-sclater1979}{}}%
Sclater, J. G., \& Crowe, J. (1979). A heat flow survey at anomaly 13 on the reykjanes ridge: A critical test of the relation between heat flow and age. \emph{Journal of Geophysical Research}, \emph{84}, 1593--1602. \url{https://doi.org/10.1029/JB084iB04p01593}

\leavevmode\vadjust pre{\hypertarget{ref-sclater2001}{}}%
Sclater, J. G., \& Decesari, R. (2001). \emph{A complilation of marine heat flow data within the {Gulf of California and the California borderland}} (No. 01-4).

\leavevmode\vadjust pre{\hypertarget{ref-sclater1974}{}}%
Sclater, J. G., \& Erickson, A. J. (1974). Geothermal measurements on {Leg} 22 of the {D.V. Glomar Challenger}. \emph{Initial Reports DSDP}, \emph{22}, 387--396.

\leavevmode\vadjust pre{\hypertarget{ref-sclater1973}{}}%
Sclater, J. G., \& Klitgord, K. D. (1973). A detailed heat flow, topographic and magnetic survey across the {Galapagos} spreading centre at 86\(^\circ\)w. \emph{J. Geophys. Res.}, \emph{78}, 6591--6975.

\leavevmode\vadjust pre{\hypertarget{ref-sclater1970}{}}%
Sclater, J. G., Mudie, J. D., \& Harrison, C. G. A. (1970a). Detailed geophysical studies on the {Hawaiian Arch} near 24\(^\circ\)25'n 157\(^\circ\)40'w -- a closely spaced suite of heat-flow stations. \emph{J. Geophys. Res.}, \emph{75}, 333--348.

\leavevmode\vadjust pre{\hypertarget{ref-sclater1970a}{}}%
Sclater, J. G., Jones, F. J. W., \& Miller, S. P. (1970b). The relationship of heat flow, bottom topography and basement relief in peake and freen deeps, northeast atlantic. \emph{Tectonophysics}, \emph{10}, 283--300. \url{https://doi.org/10.1016/0040-1951(70)90111-3}

\leavevmode\vadjust pre{\hypertarget{ref-sclater1971}{}}%
Sclater, J. G., Anderson, P. N., \& Bell, M. L. (1971). Elevation of ridges and evolution of the central eastern pacific. \emph{Journal of Geophysical Research}, \emph{76}, 7888--7915. \url{https://doi.org/10.1029/JB076i032p07888}

\leavevmode\vadjust pre{\hypertarget{ref-sclater1972}{}}%
Sclater, J. G., Ritter, U. G., \& Dixon, F. S. (1972). Heat flow in the southwestern {Pacific}. \emph{J. Geophys. Res.}, \emph{77}, 5697--5704. \url{https://doi.org/10.1029/JB077i029p05697}

\leavevmode\vadjust pre{\hypertarget{ref-sclater1976}{}}%
Sclater, J. G., Karig, D., Lawver, L. A., \& Louden, K. E. (1976). Heat flow, depth, and crustal thickness of the marginal basins of the south philippine sea. \emph{Journal of Geophysical Research}, \emph{81}, 309--318. \url{https://doi.org/10.1029/JB081i002p00309}

\leavevmode\vadjust pre{\hypertarget{ref-sebagenzi1993}{}}%
Sebagenzi, M. N., Vasseur, G., \& Louis, P. (1993). First heat flow density determinations from southeastern zaire (central africa). \emph{Journal of African Earth Sciences}, \emph{16}(4), 413--423. \url{https://doi.org/10.1016/0899-5362(93)90100-5}

\leavevmode\vadjust pre{\hypertarget{ref-seck1984}{}}%
Seck, L. (1984). \emph{Mesures du flux de chaleur au sénégal} (Master's thesis).

\leavevmode\vadjust pre{\hypertarget{ref-secretariat1986}{}}%
Secretariat, A. (1986). Terrestrial heat flow map of southeast asia: ASCOPE t/p.

\leavevmode\vadjust pre{\hypertarget{ref-sekiguchi1986}{}}%
Sekiguchi, K. (1986). A method for determining terrestrial heat flow by using bore-hole data in the oil/gas basinal areas. In \emph{Contributions to petroleum geoscience dedicated to professor kazuo taguchi on the occasion of his retirement (in japanese with english abstract)} (pp. 199--208). Faculty of Science, Tohoku University.

\leavevmode\vadjust pre{\hypertarget{ref-sestini1970}{}}%
Sestini, S. (1970). Heat flow measurements in nonhomogeneous terrains with application to geothermal areas. \emph{Geothermics}, \emph{Spec. Issue 2, 2}, 424--436.

\leavevmode\vadjust pre{\hypertarget{ref-shalev2012}{}}%
Shalev, E., Lyakhovsky, V., Weinstein, Y., \& Ben-Avraham, Z. (2013). The thermal structure of israel and the dead sea fault. \emph{Tectonophysics}, \emph{602}(0), 69--77. \url{https://doi.org/10.1016/j.tecto.2012.09.011}

\leavevmode\vadjust pre{\hypertarget{ref-shankar2013}{}}%
Shankar, U., \& Riedel, M. (2013). Heat flow and gas hydrate saturation estimates from andaman sea, india. \emph{Marine and Petroleum Geology}, \emph{43}(0), 434--449. \url{https://doi.org/10.1016/j.marpetgeo.2012.12.004}

\leavevmode\vadjust pre{\hypertarget{ref-shastkevich1975}{}}%
Shastkevich, Yu. G., \& Zabolotnik, S. I. (1975). Potok vnutrizemnogo. \emph{Studia Geophysica Et Geodaetica}, \emph{2}, 197--200.

\leavevmode\vadjust pre{\hypertarget{ref-shelyagin1973}{}}%
Shelyagin, V. A., Buachidze, I. M., Buachidze, G. U., \& Sharshidze, M. P. (1973). Teplovoy potok s pribrezhnoy polosi chernogo morya i prilegayyschey chasti territorii gruzii. In \emph{Teplovye potoki iz kori i verkhney mantiyi ze iz kori i verkhney mantiyi zemli (in russian)} (pp. 39--46). Verkhnaya Mantiya Izd.Nauka.

\leavevmode\vadjust pre{\hypertarget{ref-shen1989a}{}}%
Shen, X.-J., Zhang, W.-R., \& Guan, H. (1989). Heat flow profile from yadong to qaidam running through the tibetan plateau(in chinese). \emph{Chinese Science Bulletin}, \emph{35}, 314--316.

\leavevmode\vadjust pre{\hypertarget{ref-sheridan1983}{}}%
Sheridan, R. E. (1983). \emph{Initial Reports DSDP}, \emph{76}.

\leavevmode\vadjust pre{\hypertarget{ref-shevaldin1988}{}}%
Shevaldin, Y. V., \& Balabashin, V. I. (1988). Some results of new geothermal technique test. \emph{Geothermal Investigation}, 107--109.

\leavevmode\vadjust pre{\hypertarget{ref-shevaldin1987}{}}%
Shevaldin, Y. V., Balabashin, V. I., \& Zimin, P. (1987). New data on geothermics of the tatar strait. \emph{Geological of the Pacific Ocean}, \emph{3}, 61--64.

\leavevmode\vadjust pre{\hypertarget{ref-shkola1979}{}}%
Shkola, I. (1979). Temperature gradients in hole nagursk-1 drilled in the alexander island, franz josef land archipelago. In \emph{Processing results from parametric drill hole nagursk-1 on alexandra land island, franz josef land archipelago (report 5280, leningrad)} (p. doi:10.1594/PANGAEA.628522). All-Russian Research Institute for Geology; Mineral Resources of the World Ocean. \url{https://doi.org/10.1594/pangaea.628522}

\leavevmode\vadjust pre{\hypertarget{ref-shyu2001}{}}%
Shyu, C. T., \& Liu, C. S. (2001). Heat flow of the southwestern end of the okinawa trough. \emph{Terrestrial Atmospheric and Oceanic Sciences}, \emph{12}(Suppl. SI5), 305--317.

\leavevmode\vadjust pre{\hypertarget{ref-shyu1998}{}}%
Shyu, C.-T., Hsu, S.-K., \& Liu, C.-S. (1998). Heat flows off southwest taiwan:measurements over mud diapirs and estimated from bottom simulating reflectors. \emph{Terrestrial Atmospheric and Oceanic Sciences}, \emph{9}(4), 795--812.

\leavevmode\vadjust pre{\hypertarget{ref-shyu2006}{}}%
Shyu, C.-T., Chen, Y.-J., Chaing, S.-T., \& Liu, C.-S. (2006). Heat flow measurements over bottom simulating reflectors offshore {Southwestern Taiwan}. \emph{Terr. Atmos. Ocean. Sci.}, \emph{17}, 845--869.

\leavevmode\vadjust pre{\hypertarget{ref-simbolon1985}{}}%
Simbolon, B. (1985). Heat flow in the salawati and bintuni basins.

\leavevmode\vadjust pre{\hypertarget{ref-simmons1968}{}}%
Simmons, G., \& Horai, K. (1968). Heat flow data, 2. \emph{Journal of Geophysical Research}, \emph{73}, 6608--6629.

\leavevmode\vadjust pre{\hypertarget{ref-simpson1987}{}}%
Simpson, B. (1987). Heat flow measurements on the {Bay of Plenty} coast, {New Zealand}. \emph{J. Volcan. Geotherm. Res.}, \emph{34}, 25--33.

\leavevmode\vadjust pre{\hypertarget{ref-skinner1985}{}}%
Skinner, N. J. (1985). Heat flow in {Figi}. \emph{New Zealand J. Geol. Geophys.}, \emph{28}, 1--4.

\leavevmode\vadjust pre{\hypertarget{ref-slagstad2009}{}}%
Slagstad, T., Balling, N., Elvebakk, H., Mittomme, K., Olesen, O., Olsen, L., \& Pascal, C. (2009). Heat-flow measurements in {Late Paleoproterozic to Permian} geological provinces in south and central {Norway} and a new heat-flow map of {Fennoscandia and the Norwegian--Greenland Sea}. \emph{Tectonophysics}, \emph{in review}.

\leavevmode\vadjust pre{\hypertarget{ref-smirnov1991}{}}%
Smirnov, Y. A., Sugrobov, V. M., \& Yanovsky, F. A. (1991). Terrestrial heat flow in kamchatkatka. \emph{J. Volcanol. Seismol.}, \emph{2}(in Russian), 41--65.

\leavevmode\vadjust pre{\hypertarget{ref-smirnov1980}{}}%
Smirnov, Ya. B., \& Sugrobov, V. M. (1980). Zemnoy teplovoy potok v kurilo-kamchatskoy i aleutskoy provintsiyakh. Ii. Karta izmerennogo i fonovogo teplovogo po- toka. - vulkanologiya i seismologiya.

\leavevmode\vadjust pre{\hypertarget{ref-smirnov1976}{}}%
Smirnov, Ya. B., Zelenov, K. K., Paduchikh, V. I., Turkov, V. P., \& Khutor- Skoi, M. D. (1976). Issledovanie teplovogo potoka na poligone 44 gr. 00' - 44 gr. 40' ssh. I 34 gr. 00' - 34 gr. 40' v.d. V chernom more. \emph{Geotermiya. / Geotermicheskie Is- Sledovaniya V SSSR /. Chast 1 Moskva}, \emph{1}, 97--99.

\leavevmode\vadjust pre{\hypertarget{ref-smirnov1983}{}}%
Smirnov, Ya. B., Ashirov, T. A., Merkushov, V. N., Sopiev, V. A., \& Dubrovskaya, E. B. (1983). Kaspiiskoe more. - v kn.: Metodicheskie i eksperimen- talnye osnovy geotermii. Moskva, nauka.(russ.)., 129--134.

\leavevmode\vadjust pre{\hypertarget{ref-smith1974}{}}%
Smith, D. L. (1974). Heat flow, radioactive heat generation, and theoretical tectonics for northern {Mexico}. \emph{Earth Planet. Sci. Lett.}, \emph{23}, 43--52.

\leavevmode\vadjust pre{\hypertarget{ref-smith1979}{}}%
Smith, D. L., Mukerls III, C. E., Jones, R. L., \& Cook, G. A. (1979). Distribution of heat flow and radioactive heat generation in northern {Mexico}. \emph{J. Geophys. Res.}, \emph{84}, 2371--2379.

\leavevmode\vadjust pre{\hypertarget{ref-smith1980}{}}%
Smith, R. N. (1980). Heat flow of the western {Snake River Plain}. \emph{Geothermal Res. Council Trans.}, \emph{4}, 89--92.

\leavevmode\vadjust pre{\hypertarget{ref-soinov1993}{}}%
Soinov, V. V. (1993). The geothermal survey results. In \emph{An oceanographic study of the east sea (the sea of japan) - korea and russia cooperative research} (pp. 228--234). Korea Ocean Research; Development Institute.

\leavevmode\vadjust pre{\hypertarget{ref-soinov1997}{}}%
Soinov, V. V. (1997). Heat flow of the northwest pacific. \emph{Geophysical Fields and Simulation of Tectonosphere}, \emph{Iii}, 14--21.

\leavevmode\vadjust pre{\hypertarget{ref-soinov1975}{}}%
Soinov, V. V., \& Veselov, O. V. (1975a). Heat flow data on the okhotsk sea. \emph{Trans Sakhalin Complex Sci. Res. Inst.}, \emph{37}, 243--246.

\leavevmode\vadjust pre{\hypertarget{ref-soinov1975b}{}}%
Soinov, V. V., \& Veselov, O. V. (1975b). Novye dannye o teplovom potoke v okhotskom more (russ.). \emph{Yuzhno-Sakhalinsk: DVNTS an SSSR}, 243--246.

\leavevmode\vadjust pre{\hypertarget{ref-soinov1972}{}}%
Soinov, V. V., Tikhomirov, V. M., Veselov, O. V., \& Yermin, G. D. (1972). Heat flow measurements during the {Philippine} expedition of {Sakhalin Complex Scientific Research Institute} in 1969. \emph{Trans. Sakhalin Complex Sci. Res. Inst.}, \emph{26}, 212--215.

\leavevmode\vadjust pre{\hypertarget{ref-soinov1984}{}}%
Soinov, V. V., Soloviev, V. I., Vlasenko, V. I., \& Salman, A. G. (1984). Teplovye potoki cherez dno vpadiny deryugina okhotskogo morya. - v kn.: Teoreticheskie i experimentalnye issledovaniya po geotermike morey i okeanov. Moskva: Nauka (russ.), 63--66.

\leavevmode\vadjust pre{\hypertarget{ref-sokolova1982}{}}%
Sokolova, L. S., \& Duchkov, A. D. (1982). Novye opredeleniya teplovogo potoka v sibiri (russ.). \emph{Geologiya I Geofizika}, \emph{7}, 121--124.

\leavevmode\vadjust pre{\hypertarget{ref-sokolova2008}{}}%
Sokolova, L. S., \& Duchkov, A. D. (2008). Heat flow in the {Altai--Sayan} area: New data. \emph{Russian Geology and Geophysics}, \emph{49}, 940--950.

\leavevmode\vadjust pre{\hypertarget{ref-solovyeva1976}{}}%
Solovyeva, L. N. (1976). Morfologiya kriolitozony sayano-baikalskoi oblasti (russ.). \emph{Novosibirsk Nauka}, 124p.

\leavevmode\vadjust pre{\hypertarget{ref-springer1998}{}}%
Springer, M., \& Förster, A. (1998). Heat-flow density across the {Central Andean} subduction zone. \emph{Tectonophysics}, \emph{291}, 123--139.

\leavevmode\vadjust pre{\hypertarget{ref-sroka1991}{}}%
Sroka, K. (1991). The new results of a surface heat flow investigations of earth crust prerformed in polish carpathians. \emph{Zeszyty Naukowe AG, Krakow. Geofizyka Stosowana}, \emph{8}.

\leavevmode\vadjust pre{\hypertarget{ref-stein1991b}{}}%
Stein, C. A., \& Abbott, D. H. (1991). Heat flow constraints on the south {Pacific Superswell}. \emph{J. Geophys. Res.}, \emph{96}, 16083--16099. \url{https://doi.org/10.1029/91jb00774}

\leavevmode\vadjust pre{\hypertarget{ref-stein2000}{}}%
Stein, J. S. (2000). \emph{Multiple scales of hydrothermal circulation in the oceanic crust: Studies from the {Juan de Fuca} ridge crest and flank} (PhD thesis).

\leavevmode\vadjust pre{\hypertarget{ref-stenz1954}{}}%
Stenz, E. (1954). Temperatury wglebne i stopien geotermiczny w ciechocinku (in polish). \emph{Acta Geophys. Polon.}, \emph{2}, 159--167.

\leavevmode\vadjust pre{\hypertarget{ref-studt1969}{}}%
Studt, F. E., \& Thompson, G. E. K. (1969). Geothermal heat flow in the north island of new zealand. \emph{New Zealand Journal of Geology and Geophysics}, \emph{12}, 673--683. \url{https://doi.org/10.1080/00288306.1969.10431105}

\leavevmode\vadjust pre{\hypertarget{ref-subono1983}{}}%
Subono, S. (1983). \emph{Flux de caleur terrestre dans la region su est de la {France}} (PhD thesis).

\leavevmode\vadjust pre{\hypertarget{ref-sukharev1969}{}}%
Sukharev, G. M., Taranukha, Yu. K., \& Vlasova, S. P. (1969). Teplovoi potok iz nedr azerbaidzhana (russ.). \emph{Sovetskaya Geologiya}, \emph{8}, 146--153.

\leavevmode\vadjust pre{\hypertarget{ref-sultan2004}{}}%
Sultan, N., Foucher, J. P., Cochonat, P., Tonnerre, T., Bourillet, J. F., Ondreas, H., et al. (2004). Dynamics of gas hydrate: Case of the congo continental slope. \emph{Marine Geology}, \emph{206}(1-4), 1--18. \url{https://doi.org/10.1016/j.margeo.2004.03.005}

\leavevmode\vadjust pre{\hypertarget{ref-sun2005}{}}%
Sun, Z., Zhang, W., Hu, B., Li, W., \& Pan, T. (2005). Geothermal field and its relation with coalbed methane distribution of the qinshui basin. \emph{Chinese Sci. Bull.}, \emph{50}, 111--117.

\leavevmode\vadjust pre{\hypertarget{ref-sundar1990}{}}%
Sundar, A., Gupta, M. L., \& Sharma, S. R. (1990). Heat flow in the trans-aravalli igneous suite, tusham, india. \emph{Journal of Geodynamics}, \emph{12}, 89--100. \url{https://doi.org/10.1016/0264-3707(90)90025-p}

\leavevmode\vadjust pre{\hypertarget{ref-sundvor1986}{}}%
Sundvor, E. (1986). \emph{Heat flow measurements on the western svalbard margin}.

\leavevmode\vadjust pre{\hypertarget{ref-sundvor1991}{}}%
Sundvor, E., \& Eldholm, O. (1991). Norway: Off-shore and north-east {Atlantic}. In \emph{Geothermal atlas of europe} (pp. 63--65). Hermann Haack Verlagsgesellschaft mbH.

\leavevmode\vadjust pre{\hypertarget{ref-sundvor1987}{}}%
Sundvor, E., \& Myhre, A. M. (1987). \emph{Heatflow measurements: Jan mayen ridge and norway basin} (No. Seismo-Series. 9).

\leavevmode\vadjust pre{\hypertarget{ref-sundvor1989}{}}%
Sundvor, E., Myhre, A. M., \& Eldholm, O. (1989). \emph{Heat flow measurements on the norwegian continental margin during the FLUNORGE project} (No. 27) (Vol. 27, p. --). University Bergen.

\leavevmode\vadjust pre{\hypertarget{ref-sundvor2000}{}}%
Sundvor, E., Eldholm, O., Gladczenko, T. P., \& Planke, Sverre. (2000). Norwegian-greenland sea thermal field. In \emph{Dynamics of the norwegian margin} (Vol. 167, pp. 397--410). Geological Society. \url{https://doi.org/10.1144/gsl.sp.2000.167.01.15}

\leavevmode\vadjust pre{\hypertarget{ref-swanberg1974}{}}%
Swanberg, C. A., Chessman, M. S., Simmons, G., Smithson, S. B., Gronlie, G., \& Heier, K. S. (1974). Heat flow---heat generation studies in {Norway}. \emph{Tectonophysics}, \emph{23}, 31--48.

\leavevmode\vadjust pre{\hypertarget{ref-takherist1989}{}}%
Takherist, D., \& Lesquer, A. (1989). Mise en evidence d'importantes variations regionales du flux de chaleur en algerie. \emph{Can. J Earth Sci.}, \emph{26}, 615--626.

\leavevmode\vadjust pre{\hypertarget{ref-taktikos1985}{}}%
Taktikos, S. (1985). \emph{Heat flow and subsurface temperature measurements for {Greece}}.

\leavevmode\vadjust pre{\hypertarget{ref-taktikos1991}{}}%
Taktikos, S. (1991). Catalogue of heat flow density data: greece. In \emph{Geothermal atlas of europe} (p. 118). Hermann Haack Verlagsgesellschaft mbH.

\leavevmode\vadjust pre{\hypertarget{ref-talwani1971}{}}%
Talwani, M., Windish, C. C., \& Langesth, M. G. (1971). Reykjanes ridge crest---a detailed geophysical study. \emph{J. Geophys. Res.}, \emph{76}, 473--517.

\leavevmode\vadjust pre{\hypertarget{ref-talwani1976}{}}%
Talwani, U., M. (1976). \emph{Initial reports DSDP} (Vol. 38, pp. 151--160). U.S. Gov't. Printing Office.

\leavevmode\vadjust pre{\hypertarget{ref-tammemagi1974}{}}%
Tammemagi, H. Y., \& Wheildon, J. (1974). Terrestrial heat flow and heat generation in south-west england. \emph{Geophysical Journal of the Royal Astronomical Society}, \emph{38}, 83--94. \url{https://doi.org/10.1111/j.1365-246X.1974.tb04110.x}

\leavevmode\vadjust pre{\hypertarget{ref-tammemagi1977}{}}%
Tammemagi, H. Y., \& Wheildon, J. (1977). Further data on the south-west england heat flow anomaly. \emph{Geophysical Journal of the Royal Astronomical Society}, \emph{49}, 531--539. \url{https://doi.org/10.1111/j.1365-246X.1977.tb03721.x}

\leavevmode\vadjust pre{\hypertarget{ref-tan2010}{}}%
Tan, J.-Q., Ju, Y.-W., Zhang, W.-Y., Hou, Q.-L., \& Tan, Y.-J. (2010). Heat flow and its coalbed gas effects in the central-south area of the huaibei coalfield, eastern {China}. \emph{Science China Earth Sciences}, \emph{53}(5), 672--682. \url{https://doi.org/10.1007/s11430-010-0050-y}

\leavevmode\vadjust pre{\hypertarget{ref-tanaka2002}{}}%
Tanaka, A., \& Ito, H. (2002). Temperature at the base of the seismogenic zone and its relationship to the focal depth of the western nagao prefecture area, zisin. \emph{J. Seismol. Soc. Japan}, \emph{55}, 1--10.

\leavevmode\vadjust pre{\hypertarget{ref-tanaka2004}{}}%
Tanaka, A., Yamano, M., Yano, Y., \& Sasada, M. (2004). Geothermal gradient and heat flow data in and around japan. In \emph{Digital geoscience map DGM p-5}. Geological Survey of Japan.

\leavevmode\vadjust pre{\hypertarget{ref-taranukha1971}{}}%
Taranukha, Yu. K., \& Kamalova, O. (1971). Vteplovye potoki i neftegazonosnost na primere dono-medveditskoi sistemy dislokatsii (russ.). \emph{Izvestiya Vuzov. Ser. Neft I Gaz.}, \emph{10}, 12--14.

\leavevmode\vadjust pre{\hypertarget{ref-taranukha1973}{}}%
Taranukha, Yu. K., \& Kamalova, O. V. K. (1973). Kharakteristike geotermicheskikh uslovii vala karpinskogo i prilegayushchei chasti prikaspiiskoi vpadiny (russ.). \emph{Izvestiya Vuzov, Ser. Neft I Gaz.}, \emph{2}, 3--6.

\leavevmode\vadjust pre{\hypertarget{ref-taylor1986}{}}%
Taylor, A., Judge, A. S., \& Allen, V. (1986). Terrestrial heat flow from project cesar, alpha ridge, arctic ocean. \emph{Journal of Geodynamics}, \emph{6}, 137--176. \url{https://doi.org/10.1016/0264-3707(86)90037-2}

\leavevmode\vadjust pre{\hypertarget{ref-taylor1979}{}}%
Taylor, A. E., \& Judge, A. S. (1979). Permafrost studies in northern quebec. \emph{Géographie Physique Et Quaternaire}, \emph{33}(3-4), 245--251. \url{https://doi.org/10.7202/1000361ar}

\leavevmode\vadjust pre{\hypertarget{ref-taylor1983}{}}%
Taylor, B., \& Hayes, D. E. (1983). Origin and history of the {South China Sea Basin}. In \emph{The tectonic and geologic evolution of southeast asian seas and islands: Part 2} (Vol. 27, pp. 23--56). Am. Geophys. Union.

\leavevmode\vadjust pre{\hypertarget{ref-tezcan1991}{}}%
Tezcan, A. K., \& Turgay, M. I. (1991). Catalogue of heat flow density data:turkey. In \emph{Geothermal atlas of europe} (pp. 84--85). Hermann Haack Verlagsgesellschaft mbH.

\leavevmode\vadjust pre{\hypertarget{ref-thamrin1986}{}}%
Thamrin, M. (1986). Terrestrial heat flow map of indonesian basins. \emph{Indonesian Petroleum Association}, 33--70.

\leavevmode\vadjust pre{\hypertarget{ref-thiede1988}{}}%
Thiede, J. (1988). \emph{Scientific cruise report of arctic expedition ARK IV/3}.

\leavevmode\vadjust pre{\hypertarget{ref-thienprasert1984}{}}%
Thienprasert, A., \& Raksaskulwong, M. (1984). Heat flow in northern thailand. \emph{Tectonophysics}, \emph{103}, 217--233. \url{https://doi.org/10.1016/0040-1951(84)90085-4}

\leavevmode\vadjust pre{\hypertarget{ref-thienprasert1978}{}}%
Thienprasert, A., Galoung, W., Matsubayashi, O., Uyeda, S., \& Watanabe, T. (1978). Geothermal gradients and heat flow in northern {Thailand}. \emph{United Nations ESCAP, CCOP Tech. Bull.}, \emph{12}, 17--31.

\leavevmode\vadjust pre{\hypertarget{ref-thompson1977}{}}%
Thompson, G. E. K. (1977). Temperature gradients within and adjacent to the {North Island Volcanic Belt}. \emph{NZ J. Geol. Geophys.}, \emph{20}, 85--97.

\leavevmode\vadjust pre{\hypertarget{ref-timareva1986}{}}%
Timareva, S. V. (1986). - in: Dokl. An sssr.

\leavevmode\vadjust pre{\hypertarget{ref-tomara1984}{}}%
Tomara, G. A., Kalinin, A. V., Krystev, T. I., \& Fadeev, V. E. (1984). Plotnost teplovogo potoka. - v kn.: Neftegazogeneticheskie issle- dovaniya bolgarskogo sektora chernogo morya. Sofiya, izdatelstvo bolgarskoi akademii nauk. (russ.)., 204--208.

\leavevmode\vadjust pre{\hypertarget{ref-townend1997}{}}%
Townend, J. (1997). Estimates of conductive heat flow through bottom-simulating reflectors on the hikurangi and southwest fiordland continental margins, new zealand. \emph{Marine Geology}, \emph{141}(1-4), 209--220. \url{https://doi.org/10.1016/s0025-3227(97)00073-x}

\leavevmode\vadjust pre{\hypertarget{ref-townend1999}{}}%
Townend, J. (1999). Heat flow through the west coast, {South Island, New Zealand}. \emph{NZ J. Geol. Geophys.}, \emph{42}, 21--31.

\leavevmode\vadjust pre{\hypertarget{ref-tsukahara1976}{}}%
Tsukahara, J. (1976). Terrestrial heat flow of the iwatsuki deep well observatory and crustal temperature profiles beneath the kanto district, japan. \emph{Res. Nots Ef National Res. Center for Disaster Prev.}, \emph{21}, 1--9.

\leavevmode\vadjust pre{\hypertarget{ref-tsumuraya1985}{}}%
Tsumuraya, Y., Tanahashi, M., Saki, T., Machihara, T., \& Asakura, N. (1985). Preliminary report of the marine geophysical and geological surveys off {Wilkes Land, Antarctica} in 1983--1984. \emph{Mem. Natl. Inst. Of Polar Res. Special Issue}, \emph{37}, 48--62.

\leavevmode\vadjust pre{\hypertarget{ref-tsybulya1988}{}}%
Tsybulya, L. A. (1988). And urban g.i. \emph{Nauka i Tekhnika. 1988}.

\leavevmode\vadjust pre{\hypertarget{ref-tsybulya1984}{}}%
Tsybulya, L. A., \& Urban, G. I. (1984). Teplovoi potok v volynsko-orshanskom pro- gibe. - doklady an bssr. 1984. \emph{T. 28}.

\leavevmode\vadjust pre{\hypertarget{ref-tsybulya1981}{}}%
Tsybulya, L. A., \& Zhuk, M. S. (1981). Geotermicheskaya kharakteristika osadoch- nykh otlozhenii i teplovoi potok v raione g. Minska. - doklady an bssr. 1981. \emph{T. 25}.

\leavevmode\vadjust pre{\hypertarget{ref-tsybulya1985}{}}%
Tsybulya, L. A., \& Zhuk, M. S. (1985). Teplovoi potok belorusskoi anteklizy. - doklady an bssr. 1985. \emph{T. 29}.

\leavevmode\vadjust pre{\hypertarget{ref-tucholke2001}{}}%
Tucholke, B. E., Fujioka, K., Ishihara, T., Hirth, G., \& Kinoshita, M. (2001). Submersible study of an oceanic megamullion in the central {North Atlantic}. \emph{J. Geophys. Res.}, \emph{106}, 16145--16161.

\leavevmode\vadjust pre{\hypertarget{ref-udintsev1973}{}}%
Udintsev, G. B., \& Lubimova, E. A. (1973). \emph{Izv. Akad. Nauk SSSR}, \emph{Ser. Fizz. Zemli}(1).

\leavevmode\vadjust pre{\hypertarget{ref-udintsov1971}{}}%
Udintsov, G. B., Smirnov, Y. B., Popova, A. K., Shekatov, B. V., \& Suvilov, F. V. (1971). New data on heat flow through the floors of the {Indian and Pacific Oceans}. \emph{Dokl. Acak. Nauk. SSSR}, \emph{200}, 453-456 (AGI English Transl. 242-244).

\leavevmode\vadjust pre{\hypertarget{ref-metodika1981}{}}%
Unknown. (1981).

\leavevmode\vadjust pre{\hypertarget{ref-urban1988}{}}%
Urban, G. I., \& Tsybulya, L. A. (1988). Teplovoe role rizhskogo plutona. - eesti nsv teaduste akadeemia toimetised. Geologia. 1988. \emph{T. 37}.

\leavevmode\vadjust pre{\hypertarget{ref-urlaub2009}{}}%
Urlaub, M., Schmidt-Aursch, MechitaC., Jokat, W., \& Kaul, N. (2009). Gravity crustal models and heat flow measurements for the eurasia basin, arctic ocean. \emph{Marine Geophysical Researches}, \emph{30}(4), 277--292. \url{https://doi.org/10.1007/s11001-010-9093-x}

\leavevmode\vadjust pre{\hypertarget{ref-uyeda1964}{}}%
Uyeda, S., \& Horai, K. (1964). Terrestrial heat flow in {Japan}. \emph{J. Geophys. Res.}, \emph{69}, 2121--2141.

\leavevmode\vadjust pre{\hypertarget{ref-uyeda1982}{}}%
Uyeda, S., \& Horai, K. (1982). Heat flow measurements, DSDP {Leg} 60. \emph{Initial Reports DSDP}, \emph{60}, 789--800. \url{https://doi.org/10.1016/0040-1951(82)90007-5}

\leavevmode\vadjust pre{\hypertarget{ref-uyeda1962}{}}%
Uyeda, S., Horai, K., Yasui, M., \& Akamatsu, H. (1962). Heat-flow measurements over the japan trench. \emph{Journal of Geophysical Research}, \emph{67}, 1186--1188. \url{https://doi.org/10.1029/JZ067i003p01186}

\leavevmode\vadjust pre{\hypertarget{ref-uyeda1973}{}}%
Uyeda, S., Watanabe, T., Mizushima, N., Yasui, M., \& Horie, S. (1973). Terrestrial heat flow in lake biwa, central japan. \emph{Proc. Japan Acad.}, \emph{49}, 341--346.

\leavevmode\vadjust pre{\hypertarget{ref-uyeda1982a}{}}%
Uyeda, S., Eguchi, T., Lum, H. K., Lee, A. K., \& Singh, J. (1982a). A heat flow measurement in peninsular {Malaysia}. \emph{United Nations ESCAP, CCOP Tech. Bull.}, \emph{15}, 45--50.

\leavevmode\vadjust pre{\hypertarget{ref-uyeda1982b}{}}%
Uyeda, S., Eguchi, T., Kamal, S., \& Modjo, W. S. (1982b). Preliminary study on geothermal gradient and heat flow in {Java}. \emph{United Nations ESCAP, CCOP Tech. Bull.}, \emph{15}, 15--27.

\leavevmode\vadjust pre{\hypertarget{ref-vacquier1984}{}}%
Vacquier, V. (1984). Oil fields - a source of heat flow data. \emph{Tectonophysics}, \emph{103}, 81--98. \url{https://doi.org/10.1016/0040-1951(84)90076-3}

\leavevmode\vadjust pre{\hypertarget{ref-vacquier1985}{}}%
Vacquier, V. (1985). Calculation of terrestrial heat flow solely from oil well logging records.

\leavevmode\vadjust pre{\hypertarget{ref-vacquier1964}{}}%
Vacquier, V., \& Von Herzen, R. P. (1964). Evidence for connection between heat flow and the mid-atlantic ridge magnetic anomaly. \emph{Journal of Geophysical Research}, \emph{69}, 1093--1101. \url{https://doi.org/10.1029/JZ069i006p01093}

\leavevmode\vadjust pre{\hypertarget{ref-vacquier1966}{}}%
Vacquier, V., Uyeda, S., Yasui, M., Sclater, J. G., Corry, C., \& Watanabe, T. (1966). Studies of the thermal state of the {Earth, the 19th paper---Heat flow measurements in the northwestern Pacific}. \emph{Bull. Earthquake Res. Inst. Tokyo}, \emph{44}, 1519--1535. \url{https://doi.org/10.4095/100771}

\leavevmode\vadjust pre{\hypertarget{ref-vacquier1967}{}}%
Vacquier, V., Sclater, J. G., \& Corry, C. E. (1967). Studies of the thermal state of the {Earth, The 21st paper---Heat flow, eastern Pacific}. \emph{Bull. Earthquake Res. Inst. Tokyo}, \emph{45}, 375--393.

\leavevmode\vadjust pre{\hypertarget{ref-vangool1987}{}}%
Van Gool, M., Huson, W. J., Prawirasasra, R., \& Owen, T. R. (1987). Heat flow and seismic obervations in the northwestern {Banda Arc}. \emph{J. Geophys. Res.}, \emph{92}, 2581--2586.

\leavevmode\vadjust pre{\hypertarget{ref-vanhinte1987}{}}%
Van Hinte, J. E. (1987). \emph{Initial Reports DSDP}, \emph{83}, 80--81.

\leavevmode\vadjust pre{\hypertarget{ref-vanneste2003}{}}%
Vanneste, M., Poort, J., De Batist, M., \& Klerkx, J. (2003). Atypical heat-flow near gas hydrate irregularities and cold seeps in the {Baikal Rift Zone}. \emph{Marine Petrol. Geol.}, \emph{19}, 1257--1274.

\leavevmode\vadjust pre{\hypertarget{ref-vartanyan1984}{}}%
Vartanyan, K. S., \& Gordienko, V. V. (1984). Novye znacheniya teplovogo potoka na territorii armyanskoi ssr. - izvestiya an arm. ssr. \emph{Ser. Nauki o Zemle. 1984}.

\leavevmode\vadjust pre{\hypertarget{ref-vasseur1980}{}}%
Vasseur, G. (1980). Some aspects of heat flow in france. In \emph{Advances in european geothermal research} (pp. 170--175). European Science Fundation.

\leavevmode\vadjust pre{\hypertarget{ref-vasseur1982}{}}%
Vasseur, G. (1982). Synthèse des résultats du flux géothermique en france. \emph{Annales Géophysiques}, \emph{38}, 189--201.

\leavevmode\vadjust pre{\hypertarget{ref-vasseur1983}{}}%
Vasseur, G., Bernard, P., Van de Meulebrouch, J., Kast, Y., \& Jolivet, J. (1983). Holocene paleotemperatures deduced from geothermal measurements. \emph{Paleogeogr. Paleoclim. Paleoecology}, \emph{43}, 237--259.

\leavevmode\vadjust pre{\hypertarget{ref-veliciu1979}{}}%
Veliciu, S., \& Demetrescu, C. (1979). Heat flow in {Romania} and some relations to geological and geophysical features. In \emph{Terrestrial heat flow in europe} (pp. 253--260). Springer Verlag.

\leavevmode\vadjust pre{\hypertarget{ref-veliciu1984}{}}%
Veliciu, S., \& Visarion, M. (1984). Geothermal models for the east carpathians. \emph{Tectonophysics}, \emph{103}(1-4), 157--165. \url{https://doi.org/10.1016/0040-1951(84)90080-5}

\leavevmode\vadjust pre{\hypertarget{ref-veliciu1977}{}}%
Veliciu, S., Cristian, M., Paraschiv, D., \& Visarion, M. (1977). Preliminary data of heat flow distribution in romania. \emph{Geothermics}, \emph{6}(1-2), 95--98. \url{https://doi.org/10.1016/0375-6505(77)90044-x}

\leavevmode\vadjust pre{\hypertarget{ref-velinov1983}{}}%
Velinov, T., \& Bojadgieva, K. (1983). \emph{Heat flow in bulgaria (manuscript)}.

\leavevmode\vadjust pre{\hypertarget{ref-verheijen1979}{}}%
Verheijen, P. J. T., \& Ajakaiye, D. E. (1979). Heat flow measurements in the {Ririwai Ring Complex, Nigeria}. \emph{Tectonophysics}, \emph{54}, 27--32.

\leavevmode\vadjust pre{\hypertarget{ref-verma1965}{}}%
Verma, R. K., \& Rao, R. U. M. (1965). Terrestrial heat flow in {Kolar Gold Field}, {India}. \emph{J. Geophys. Res.}, \emph{70}, 1353--1356.

\leavevmode\vadjust pre{\hypertarget{ref-verma1966}{}}%
Verma, R. K., Rao, R. U. M., \& Gupta, M. L. (1966). Terrestrial heat flow in mosabani mine, singhbhum district, bihar, india. \emph{Journal of Geophysical Research}, \emph{71}, 4943--4948. \url{https://doi.org/10.1029/JZ071i020p04943}

\leavevmode\vadjust pre{\hypertarget{ref-verma1968}{}}%
Verma, R. K., Gupta, M. L., Hamza, V. M., Rao, G. V., \& Rao, R. U. M. (1968). Heat flow and crustal structure near cambay, gujarat, india. \emph{Bull. Natn. Geophys. Res. Inst.}, \emph{6}, 153--166.

\leavevmode\vadjust pre{\hypertarget{ref-vermeesch2004}{}}%
Vermeesch, P., Poort, J., Duchkov, A. D., Klerkx, J., \& De Batist, M. (2004). {Lake Issyk Kul (Tien Shan)}: Unually low heat-flow in an active intermontane basin. \emph{Russian Geology and Geophysics}, \emph{45}, 574--584.

\leavevmode\vadjust pre{\hypertarget{ref-verzhbitskii2001}{}}%
Verzhbitskii, E. V. (2001). Geothermal studies in the pechora sea (in russian). \emph{Okeanologiya}, \emph{41}, 456--461.

\leavevmode\vadjust pre{\hypertarget{ref-verzhbitskii2005}{}}%
Verzhbitskii, E. V., Lobkovskii, L. I., Pokryskin, A. A., \& Soltanovskii, I. I. (2005). Anomalous geothermal regime, seismic and gravitational landslide activity in the northeastern part of the {Black Sea} continental slope. \emph{Oceanology}, \emph{45}, 580--587.

\leavevmode\vadjust pre{\hypertarget{ref-verzhbitskii1980}{}}%
Verzhbitskii, V. G., \& Zolotarev, V. G. (1980). Studies of the heat flow in the rift zone of the {Red Sea (in Russian)}. \emph{Okeanologiya (Oceanology)}, \emph{20}, 882--886.

\leavevmode\vadjust pre{\hypertarget{ref-verzhbitsky1989}{}}%
Verzhbitsky, E. V., \& Zolotarev, V. G. (1989). Heat flow and the eurasian-african plate boundary in the eastern part of the azores-gibraltar fracture zone. \emph{Journal of Geodynamics}, \emph{11}, 267--273. \url{https://doi.org/10.1016/0264-3707(89)90009-4}

\leavevmode\vadjust pre{\hypertarget{ref-veselov2000}{}}%
Veselov, O. B. (2000). Heat flow structure of the okhotsk sea region. In \emph{Structure of the earth crust and oil-and-gas presence prospects in regions of north-west pacific margin - vol.1} (Vol. 1, pp. 107--129). Inst. Marine Geol. Geophys., Far East Branch, Russian Academy of Sciences, Yuzhno-Sakhalinsk.

\leavevmode\vadjust pre{\hypertarget{ref-veselov1982}{}}%
Veselov, O. B., \& Lipina, E. H. (1982). \emph{Catalogue data: Heat flow of eastern asia, australia and western {Pacific}}.

\leavevmode\vadjust pre{\hypertarget{ref-veselov2004}{}}%
Veselov, O. V. (2004). Personal communication, 2003. In \emph{CD rom: Geothermal gradient and heat flow data in and around japan} (p. --). Geological Survey of Japan, AIST, 2004.

\leavevmode\vadjust pre{\hypertarget{ref-veselov1979}{}}%
Veselov, O. V., \& Soinov, V. V. (1979). (Otvetstwennye ispolniteli) vyyasnit rol teplovogo polya zemli v geodinamike v predelakh okrainnykh morei tikhogo okeana: Teplovoy potok okhotomorskogo regiona (metodika, apparatura, re- zultaty). \emph{Moskva: Vntits}, b8597.

\leavevmode\vadjust pre{\hypertarget{ref-veselov1974}{}}%
Veselov, O. V., Volkova, N. A., Yeremin, G. D., Kozlov, N. A., \& Soinov, V. V. (1974a). Heat flow measurements in the zone transitional from the asiatic continent to the pacific ocean. \emph{Doklady Akad. Nauk SSSR}, \emph{217}, 897--900.

\leavevmode\vadjust pre{\hypertarget{ref-veselov1974b}{}}%
Veselov, O. V., Volkova, N. A., Eremin, G. D., Kozlov, N. A., \& Soinov, V. V. (1974b). Is-sledovanie teplovogo potoka v severo-zapadnoy chasti tikhogo okeana. - v kn.: Geotermiya. Otchety po geotermicheskim issledovaniyam v sssr. Vyp. 1-2. Moskva: Gin an sssr, (russ.), 87--90.

\leavevmode\vadjust pre{\hypertarget{ref-veselov1975a}{}}%
Veselov, O. V., Volkova, N. A., \& Soinov, V. V. (1975a). Geothermal researches in the deep part of the east china sea. In \emph{Geophysical researches of the crust and upper mantle structure in the transition zone from asian continent to the pacific ocean} (Vol. 30, pp. 300--302). Akad. Nauk SSSR.

\leavevmode\vadjust pre{\hypertarget{ref-veselov1975b}{}}%
Veselov, O. V., Yeremin, G. D., \& Soinov, V. V. (1975b). Heat flow determination during the second complex oceanic expedition of the sakhalin complex scientific research institute. In \emph{Geophysical researches of the crust and upper mantle structure in the transition zone from asian continent to the pacific ocean} (Vol. 30, pp. 298--300). Akad. Nauk SSSR.

\leavevmode\vadjust pre{\hypertarget{ref-vidal1984}{}}%
Vidal, O., Vasseur, G., \& Lucazeau, F. (1984). Mesures geothermiques dans la region du cezallier. Geothermal measurements in the cezallier region. In \emph{Geothermalisme actuel (cezallier). Present-day geothermal activity, cezallier} (Vol. 81--10, pp. 153--162). Documents - B.R.G.M.

\leavevmode\vadjust pre{\hypertarget{ref-vigneresse1987}{}}%
Vigneresse, J. L., Jolivet, J., Cuney, M., \& Bienfait, G. (1987). Heat flow, heat production and granite depth in western france. \emph{Geophysical Research Letters}, \emph{14}, 275--278. \url{https://doi.org/10.1029/GL014i003p00275}

\leavevmode\vadjust pre{\hypertarget{ref-villinger1984}{}}%
Villinger, H. (1984). New heat flow values off the west coast of {Morrocco}. \emph{Initial Reports DSDP}, \emph{79}, 377--381.

\leavevmode\vadjust pre{\hypertarget{ref-villinger2000}{}}%
Villinger, H., \& Cruise Participants. (2000). \emph{Report and preliminary results of SONNE-cruise SO145/leg 1, balboa - talcahuano, 21.12.1999 - 28.1.2000} (No. Nr. 154).

\leavevmode\vadjust pre{\hypertarget{ref-vitorello1980}{}}%
Vitorello, I., Hamza, V. M., \& Pollack, H. N. (1980). Heat flow and radiogenic heat production in {Brazil}. \emph{J. Geophys. Res.}, \emph{85}, 3778--3788.

\leavevmode\vadjust pre{\hypertarget{ref-vlasenko1984}{}}%
Vlasenko, V. I., Salman, A. G., Tomara, G. A., \& Baranov, B. A. (1984). Dannye izmereniy teplovogo potoka v vostochnoy chasti arkticheskogo basseyna. - v kn.: Teoreticheskie i experimentalnye issledovaniya po geotermike morey i oke- anov. Moskva: Nauka, (russ.), 47--51.

\leavevmode\vadjust pre{\hypertarget{ref-vogt1999}{}}%
Vogt, P. R., Gardner, J., Crane, K., Sundvor, E., Bowles, F., \& Cherkashev, G. (1999). Ground-truthing 11- to 12-kHz side-scan sonar imagery in the {Norwegian-Greenland Sea}: Part i: Pockmarks on the {Vestnesa Ridge and Storegga} slide margin. \emph{Geo-Marine Letters}, \emph{19}, 97--110.

\leavevmode\vadjust pre{\hypertarget{ref-vonherzen2001}{}}%
Von Herzen, R., Ruppel, C., Molnar, P., Nettles, M., Nagihara, S., \& Ekström, G. (2001). A constraint on the shear stress at the {Pacific-Australian} plate boundary from heat flow and seismicity at the {Kermadec} forearc. \emph{J. Geophys. Res.}, \emph{106}, 6817--6833.

\leavevmode\vadjust pre{\hypertarget{ref-vonherzen1959}{}}%
Von Herzen, R. P. (1959). Heat-flow values from the {South-Eastern Pacific}. \emph{Nature}, \emph{183}, 882--883. \url{https://doi.org/10.1038/183882a0}

\leavevmode\vadjust pre{\hypertarget{ref-vonherzen1964}{}}%
Von Herzen, R. P. (1964). Ocean-floor heat-flow measurements west of the {United States and Baja California}. \emph{Marine Geol.}, \emph{1}, 225--239.

\leavevmode\vadjust pre{\hypertarget{ref-vonherzen1973}{}}%
Von Herzen, R. P. (1973). Geothermal measurement, leg 21. \emph{Initial Reports DSDP}, \emph{21}, 443--457.

\leavevmode\vadjust pre{\hypertarget{ref-vonherzen1965}{}}%
Von Herzen, R. P., \& Langseth, M. G. (1965). Present status of oceanic heat flow measurements. \emph{Phys. Chem. Earth}, \emph{6}, 365--407.

\leavevmode\vadjust pre{\hypertarget{ref-vonherzen1972}{}}%
Von Herzen, R. P., \& Simmons, G. (1972). Two heat flow profiles across the {Atlantic Ocean}. \emph{Earth Planet. Sci. Lett.}, \emph{15}, 19--27.

\leavevmode\vadjust pre{\hypertarget{ref-vonherzen1963}{}}%
Von Herzen, R. P., \& Uyeda, S. (1963). Heat flow through the eastern {Pacific Ocean} floor. \emph{J. Geophys. Res.}, \emph{68}, 4219--4250. \url{https://doi.org/10.1126/science.140.3572.1207}

\leavevmode\vadjust pre{\hypertarget{ref-vonherzen1967}{}}%
Von Herzen, R. P., \& Vacquier, V. (1967). Terrestrial heat flow in {Lake Malawi, Africa}. \emph{J. Geophys. Res.}, \emph{72}, 4221--4226.

\leavevmode\vadjust pre{\hypertarget{ref-vonherzen1970}{}}%
Von Herzen, R. P., Simmons, G., \& Folinsbee, A. (1970). Heat flow between the {Caribbean Sea and the Mid-Atlantic Ridge}. \emph{J. Geophys. Res.}, \emph{75}, 1973--1984.

\leavevmode\vadjust pre{\hypertarget{ref-vonherzen1971}{}}%
Von Herzen, R. P., Fiske, R. J., \& Sutton, D. (1971). Geothermal measurements on {Leg. 8}. \emph{Initial Reports DSDP}, 837--849.

\leavevmode\vadjust pre{\hypertarget{ref-vonherzen1974}{}}%
Von Herzen, R. P., Finckh, P., \& Hsu, K. J. (1974). Heat flow measurements in {Swiss} lakes. \emph{J. Geophys.}, \emph{40}, 141--172.

\leavevmode\vadjust pre{\hypertarget{ref-vonherzen1982a}{}}%
Von Herzen, R. P., Hutchison, I., Jemsek, J., \& Sclater, J. G. (1982a). Geothermal flux in western mediterranean basins. \emph{EOS Trans. AGU}, --.

\leavevmode\vadjust pre{\hypertarget{ref-vonherzen1982b}{}}%
Von Herzen, R. P., Detrick, R. S., Crough, S. T., Epp, D., \& Fehn, U. (1982b). Thermal origin of the {Hawaiian Swell}: Heat flow evidence and thermal model. \emph{J. Geophys. Res.}, \emph{87}, 6711--6723.

\leavevmode\vadjust pre{\hypertarget{ref-vonherzen1989}{}}%
Von Herzen, R. P., Cordery, M. J., Detrick, R. S., \& Fang, C. (1989). Heat flow and the thermal origin of hot spot swells: The {Hawaiian} swell revisited. \emph{J. Geophys. Res.}, \emph{94}, 13783--13799.

\leavevmode\vadjust pre{\hypertarget{ref-vasseur1978}{}}%
V.Vasseur, \& Groupe FLUXCHAF. (1978). Nouvelles determinations du flux geothermique en {France}. \emph{C.R. Acad. Sci. Paris}, \emph{286 (D)}, 933--936.

\leavevmode\vadjust pre{\hypertarget{ref-wang1990b}{}}%
Wang, J. A., Xu, Q., \& Zhang, W. R. (1990). Geothermal characteristics and deep thermal structure of yunnan area, SW china (in chinese with english abstract). \emph{Seismol. Geol.}, \emph{12}, 367-\/-379.

\leavevmode\vadjust pre{\hypertarget{ref-wang2002a}{}}%
Wang, L.-S., Liu, S., \& Xiao, Y. (2002). Distribution characteristics of geothermal heat flow in bohai basin. \emph{Bulletin of the Chinese Academy of Sciences}, \emph{47}, 151--155.

\leavevmode\vadjust pre{\hypertarget{ref-wang2001}{}}%
Wang, S., Lijuan, H., \& Wang, J. (2001). Thermal regime and petroleum systems in {Junggar} basin, northwest {China}. \emph{Phys. Earth Planet. Int.}, \emph{126}, 237--248.

\leavevmode\vadjust pre{\hypertarget{ref-wang2013}{}}%
Wang, W., \& Liu, J.-G. (2013). Underground temperature calculation of mined bed in pyrite mine of mawei mountain according to temperature characteristics of surrounding rock. \emph{Science Technology and Engineering}, \emph{13}(17), 4893--4897.

\leavevmode\vadjust pre{\hypertarget{ref-wang1987}{}}%
Wang, Y. (1987). \emph{Geothermics and oil-gas generation in north jiangsu basin} (Master's thesis).

\leavevmode\vadjust pre{\hypertarget{ref-wang2003a}{}}%
Wang, Y.-X., Wang, J.-W., \& Hu, S.-B. (2003). \emph{Thermal history and structure of eastern depression in the liaohe basin thermal evolution} (Master's thesis). \emph{Geological Science}.

\leavevmode\vadjust pre{\hypertarget{ref-watanabe1972}{}}%
Watanabe, T. (1972). On heat flow in the sagami bay and heat flow distribution around the izu peninsula. In \emph{Izu peninsula} (pp. 277--286). Tokai Univ. Press.

\leavevmode\vadjust pre{\hypertarget{ref-watanabe1970}{}}%
Watanabe, T., Epp, D., Uyeda, S., Langseth, M. G., \& Yasui, M. (1970). Heat flow in the philippine sea. \emph{Tectonophysics}, \emph{10}, 205--224.

\leavevmode\vadjust pre{\hypertarget{ref-watanabe1975}{}}%
Watanabe, T., Von Herzen, R. P., \& Erickson, A. (1975). Geothermal studies {Leg} 31. \emph{Initial Reports DSDP}, \emph{31}, 573--576.

\leavevmode\vadjust pre{\hypertarget{ref-watermez1980}{}}%
Watermez, P. (1980). Flux de chaleur sur le {Massif Amrmoricain} et sur la marge continental. In \emph{Essai de modelisation de l'evolution thermique de la marge continentale. These 3eme cycle}. Centre Oceanologique de Bretagne.

\leavevmode\vadjust pre{\hypertarget{ref-wesierska1973b}{}}%
Wesierska, M. (1973a). \emph{A study of heat flux density in {Poland}} (No. 60).

\leavevmode\vadjust pre{\hypertarget{ref-wesierska1973a}{}}%
Wesierska, M. (1973b). A study of terrestrial heat flux density in {Poland}. \emph{Mat. I. Prace}, \emph{60}, 135--144.

\leavevmode\vadjust pre{\hypertarget{ref-wheat2004}{}}%
Wheat, C. G., Mottl, M. J., Fisher, A. T., D.Kadko, Davis, E. E., \& Baker, E. (2004). Heat flow through a basaltic outcrop on a sedimented young ridge flank. \emph{Geochem. Geophys. Geosys.}, \emph{5}, q12006, doi:10.1029/2004GC000700.

\leavevmode\vadjust pre{\hypertarget{ref-whieldon1978}{}}%
Wheildon, J. (1978). Heat flow measurement in the {Port More} borehole. In \emph{Geology of the causway coast} (Vol. 2, pp. 155--156). Geol. Surv. N. Ireland.

\leavevmode\vadjust pre{\hypertarget{ref-wheildon1977}{}}%
Wheildon, J., Francis, M. F., \& Thomas-Betts, A. (1977). Investigation of the {S.W.} England thermal anomaly zone. \emph{Semin. Geotherm. Energy (Commission of the European Communities)}, \emph{1}, 175--188.

\leavevmode\vadjust pre{\hypertarget{ref-wheildon1980}{}}%
Wheildon, J., Francis, M. F., Ellis, J. R. L., \& Thomas-Betts, A. (1980). Exploration and interpretation of the {South West England} geothermal anomaly. In \emph{Advances in european geothermal research: Proceedings of the second international seminar on the results of EC geothermal research strasbourg} (pp. 456--465). D. Reidel Publishing.

\leavevmode\vadjust pre{\hypertarget{ref-wheildon1984a}{}}%
Wheildon, J., King, G., Crook, C. N., \& Thomas-Betts, A. (1984a). The eastern highleands granites: Heat flow, heat production and model studies. In \emph{Investigations of the geothermal potential of the UK}. British Geol. Surv.

\leavevmode\vadjust pre{\hypertarget{ref-wheildon1984b}{}}%
Wheildon, J., King, G., Crook, C. N., \& Thomas-Betts, A. (1984b). The lake district granites: Heat flow, heat production and model studies. In \emph{Investigations of the geothermal potential of the UK}. British Geol. Surv.

\leavevmode\vadjust pre{\hypertarget{ref-whieldon1985}{}}%
Wheildon, J., Gebski, J. S., \& Thomas-Betts, A. (1985). Further investigations of the UK heat flow field 1981-1987. In \emph{Investigations of the geothermal potential of the UK}. Britishis Geological Survey.

\leavevmode\vadjust pre{\hypertarget{ref-wheildon1994}{}}%
Wheildon, J., Morgan, P., Williamson, K. H., Evans, T. R., \& Swanberg, C. A. (1994). Heat flow in the kenya rift zone. \emph{Tectonophysics}, \emph{236}(1-4), 131--149. \url{https://doi.org/10.1016/0040-1951(94)90173-2}

\leavevmode\vadjust pre{\hypertarget{ref-white1989}{}}%
White, P. (1989). Downhole logging. In \emph{Antarctic cenozoic history from the CIROS-1 drillhole, McMurdo sound} (Vol. 245, pp. 7--14). Department of Scientific; Industrial Research Bulletin.

\leavevmode\vadjust pre{\hypertarget{ref-whiteford1990}{}}%
Whiteford, P. C. (1990). \emph{Heat flow measurements in the bay of plenty, new zealand} (No. 221).

\leavevmode\vadjust pre{\hypertarget{ref-wiggins2002}{}}%
Wiggins, S. M., Hildebrand, J. A., \& Gieskes, J. M. (2002). Geothermal state and fluid flow within ODP hole 843B: Results from wireline logging. \emph{Earth and Planetary Science Letters}, \emph{195}(3-4), 239--248. \url{https://doi.org/10.1016/s0012-821x(01)00590-8}

\leavevmode\vadjust pre{\hypertarget{ref-wilhelm2004}{}}%
Wilhelm, H., Heidinger, P., Safanda, J., Cermak, V., Burkhardt, H., \& Popov, Y. (2004). High resolution temperature measurements in the borehole {Yaxcopoil-1, Mexico}. \emph{Meteoritics Planet. Sci.}, \emph{39}, 813--819.

\leavevmode\vadjust pre{\hypertarget{ref-williams1996}{}}%
Williams, C. F. (1996). Temperature and the seismic/aseismic transition: Observations from the 1992 landers earthquake. \emph{Geophysical Research Letters}, \emph{23}(16), 2029--2032. \url{https://doi.org/10.1029/96gl02066}

\leavevmode\vadjust pre{\hypertarget{ref-williams1994}{}}%
Williams, C. F., \& Galanis, S. P. (1994). \emph{Heat-flow measurements in the vicinity of the hayward fault, california} (No. 94-692).

\leavevmode\vadjust pre{\hypertarget{ref-williams1995}{}}%
Williams, C. F., Galanis, S. P., Grubb, F., \& Moses, T. H. (1995). \emph{The thermal regime of santa maria province, california}.

\leavevmode\vadjust pre{\hypertarget{ref-williams2004}{}}%
Williams, C. F., Grubb, F., \& Galanis, S. P. (2004). Heat flow in the {SAFOD} pilot hole and implications for the strength of the {San Andreas Fault}. \emph{Geophys. Res. Lett.}, \emph{31}, l15S14, doi:10.1029/2003GL019352.

\leavevmode\vadjust pre{\hypertarget{ref-williams1983}{}}%
Williams, D. L., \& Herzen, R. P. V. (1983). On the terrestrial heat flow and physical limnology of {Crater Lake, Oregon}. \emph{J. Geophys. Res.}, \emph{88}, 1094--1104.

\leavevmode\vadjust pre{\hypertarget{ref-williams1974}{}}%
Williams, D. L., Von Herzen, R. P., Sclater, J. G., \& Anderson, R. N. (1974). The {Galapagos} spreading centre, lithospheric cooling and hydrothermal circulation. \emph{Geophys. J. Roy. Astr. Soc.}, \emph{38}, 587--608.

\leavevmode\vadjust pre{\hypertarget{ref-williams1977}{}}%
WIlliams, D. L., Lee, T. C., Green, K. E., \& Hobart, M. A. (1977). A geothermal study of the {Mid-Atlantic} ridge near 37\(^\circ\)n. \emph{Bull. Geol. Soc. Am.}, \emph{88}, 531--540.

\leavevmode\vadjust pre{\hypertarget{ref-williams1979}{}}%
Williams, D. L., Becker, K., Lawver, L. A., \& Von Herzen, R. P. (1979a). Heat flow at the spreading centers of the guaymas basin, gulf of california. \emph{Journal of Geophysical Research}, \emph{84}, 6757--6769. \url{https://doi.org/10.1029/JB084iB12p06757}

\leavevmode\vadjust pre{\hypertarget{ref-williams1979a}{}}%
Williams, D. L., Green, K., van Andel, T. H., Von Herzen, R. P., Dymond, J. R., \& Crane, K. (1979b). The hydrothermal mounds of the {Galapagos} rift: Observations with {DSRV Alvin} and detailed heat flow studies. \emph{J. Geophys. Res.}, \emph{84}, 7467--7484.

\leavevmode\vadjust pre{\hypertarget{ref-wimbush1971}{}}%
Wimbush, M., \& Sclater, J. G. (1971). Geothermal heat flux evaluated from turbulent fluctuations above the sea floor. \emph{Journal of Geophysical Research}, \emph{76}, 529--536. \url{https://doi.org/10.1029/JB076i002p00529}

\leavevmode\vadjust pre{\hypertarget{ref-wright1980}{}}%
Wright, J. A., Jessop, A. M., Judge, A. S., \& Lewis, T. J. (1980). Geothermal measurements in newfoundland. \emph{Canadian Journal of Earth Sciences}, \emph{17}, 1370--1376. \url{https://doi.org/10.1139/e80-144}

\leavevmode\vadjust pre{\hypertarget{ref-wronski1977}{}}%
Wronski, E. B. (1977). Two heat flow values for tasmania. \emph{Geophysical Journal of the Royal Astronomy Society}, \emph{48}, 131--133. \url{https://doi.org/10.1111/j.1365-246X.1977.tb01291.x}

\leavevmode\vadjust pre{\hypertarget{ref-wu1990}{}}%
Wu, G.-F., Zu, J.-H., \& Xie, Y.-Z. (1990). Heat flow along the no. 5 china's geoscience section. \emph{Chinese Sci. Bull.}, \emph{35}(2), 126--129.

\leavevmode\vadjust pre{\hypertarget{ref-wu2012}{}}%
Wu, L., Zhao, L., \& Luo, X.-G. (2012). Characteristics of geothermal field and estimation of heat flow in wudang district of guiyang. \emph{Site Investigation Science and Technology}, \emph{3}, 41--43.

\leavevmode\vadjust pre{\hypertarget{ref-wu1993}{}}%
Wu, Q.-F. (1993). Geothermal characteristics and seismological activity (in chinese). \emph{Earthquake Science of Huabei}, \emph{11}, 42--47.

\leavevmode\vadjust pre{\hypertarget{ref-wu2005}{}}%
Wu, S., Ou, Y.-C., \& Lu, J.-L. (2005). Exploration and assessment of geothermal resources at in hepu basin in guangxi. \emph{Journal of Guilin University of Technology}, \emph{25}, 155--160.

\leavevmode\vadjust pre{\hypertarget{ref-xianjie1984}{}}%
Xianjie, C. S. (1984). \emph{Heat Flow Measurement on Xizhang (Tibetan) Plateau}.

\leavevmode\vadjust pre{\hypertarget{ref-xiao1982}{}}%
Xiao, D., \& Ji'an, W. (1982). Terrestrial heat flow in anhui province (in chinese with english abstract). In \emph{Research on geology (i)} (pp. 82--89). Culture relics publishing house.

\leavevmode\vadjust pre{\hypertarget{ref-xiao2004}{}}%
Xiao, W., Liu, Z., \& Du, J.-H. (2004). Characteristics of temperature and pressure system in erlian basin. \emph{Xinjiang Petroleum Geology}, \emph{25}, 610--613.

\leavevmode\vadjust pre{\hypertarget{ref-xiao2013}{}}%
Xiao, W., Zhang, T., Zheng, Y., \& Gao, J. (2013). Heat flow measurements on the lomonosov ridge, arctic ocean. \emph{Acta Oceanologica Sinica}, \emph{32}(12), 25--30. \url{https://doi.org/10.1007/s13131-013-0384-3}

\leavevmode\vadjust pre{\hypertarget{ref-xu1995b}{}}%
Xu, J., Ehara, S., \& Ping, X. H. (1995). Preliminary report of heat flow in the {GGT} profile from {Mznzhouli to Suifenhe, northeast China}. \emph{CCOP Tech. Bull.}, \emph{25}, 79--87.

\leavevmode\vadjust pre{\hypertarget{ref-xu2010}{}}%
Xu, M., Zhao, P., Zhu, C.-Q., \& Hu, S.-B. (2010). Borehole temperature logging and terrestrial heat flow distribution in jianghan basin. \emph{Scientia Geologica Sinica}, \emph{45}(1), 317--323.

\leavevmode\vadjust pre{\hypertarget{ref-xu2011}{}}%
Xu, M., Zhu, C.-Q., Tian, Y.-T., Rao, S., \& Hu, S.-B. (2011). Borehole temperature logging and characteristics of subsurface temperature in sichuan basin. \emph{Chinese Journal Geophysics}, \emph{54}(4), 1052--1060. \url{https://doi.org/10.3969/j.issn.0001-5733.2011.04.020}

\leavevmode\vadjust pre{\hypertarget{ref-xu2006}{}}%
Xu, X., Shi, X., Luo, X., Liu, F., Guo, X., Sha, Z., \& Yang, X. (2006). Marine heat flow measurements in the {Xisha Troufh, South China Sea}. \emph{Marine Geol. Quaternary Geol.}, \emph{26}, 51--57.

\leavevmode\vadjust pre{\hypertarget{ref-yamano1981}{}}%
Yamano, M. (1985a). Heat flow measurements. In \emph{Preliminary report of the hakuho maru cruise KH84-1} (pp. 265--271). Ocean Res. Inst., Univ. Tokyo.

\leavevmode\vadjust pre{\hypertarget{ref-yamano1985b}{}}%
Yamano, M. (1985b). \emph{Heat flow studies of the circum-pacific subduction zones} (PhD thesis).

\leavevmode\vadjust pre{\hypertarget{ref-yamano2004}{}}%
Yamano, M. (2004). Unpublished data. In \emph{CD rom: Geothermal gradient and heat flow data in and around japan} (p. --). Geological Survey of Japan, AIST, 2004.

\leavevmode\vadjust pre{\hypertarget{ref-yamano1999}{}}%
Yamano, M., \& Goto, S. (1999). High heat flow anomalies on the seaward slope of the japan trench (abstract). \emph{EOS Trans. AGU}, \emph{80}, f929.

\leavevmode\vadjust pre{\hypertarget{ref-yamano1998}{}}%
Yamano, M., \& Kinoshita, M. (1998). Thermal structure of the shikoku basin and southwest japan subduction zone. \emph{Bulletin of the Earthquake Research Institute, University of Tokyo}, \emph{73}, 105--123.

\leavevmode\vadjust pre{\hypertarget{ref-yamano1990}{}}%
Yamano, M., \& Uyeda, S. (1990). Heat-flow studies in the {Peru Trench} subduction zone. \emph{Proc. Ocean Drilling Program, Sci. Results}, \emph{112}, 653--661.

\leavevmode\vadjust pre{\hypertarget{ref-yamano1983}{}}%
Yamano, M., Fujii, M., \& Fujisawa, H. (1983). Heat flow measurement. In \emph{Preliminary report of the hakuho maru cruise KH82-4} (pp. 218--225). Ocean Res. Inst., Univ. Tokyo.

\leavevmode\vadjust pre{\hypertarget{ref-yamano1984}{}}%
Yamano, M., Honda, S., \& Uyeda, S. (1984). Nankai trough: A hot trench? \emph{Marine Geophysical Research}, \emph{6}, 187--203.

\leavevmode\vadjust pre{\hypertarget{ref-yamano1986}{}}%
Yamano, M., Uyeda, S., Furukawa, Y., \& Dehghani, G. D. (1986). Heat flow measurements in the northern and middle {Ryukyu Arc} area on {R/V Sonne} in 1984. \emph{Bull. Earthquake Res. Inst.}, in press.

\leavevmode\vadjust pre{\hypertarget{ref-yamano1987}{}}%
Yamano, M., Uyeda, S., Uyeshima, M., Kinoshita, M., Nagihara, S., Boh, R., \& Fujisawa, H. (1987). Report on DELP 1985 cruises in the japan sea, part v: Heat flow measurements. \emph{Bull. Earthq. Res. Inst.}, \emph{62}, 417--432.

\leavevmode\vadjust pre{\hypertarget{ref-yamano1989a}{}}%
Yamano, M., Uyeda, S., Foucher, J. P., \& Sibuet, J. C. (1989). Heat flow anomaly in the middle okinawa trough. \emph{Tectonophysics}, \emph{159}, 307--318.

\leavevmode\vadjust pre{\hypertarget{ref-yamano1992}{}}%
Yamano, M., Foucher, J. P., Kinoshita, M., Fisher, A. T., \& Hyndman, R. D. (1992). Heat flow and fluid flow regime in the western nankai accretionary prism. \emph{Earth and Planetary Science Letters}, \emph{109}, 451--462. \url{https://doi.org/10.1016/0012-821x(92)90105-5}

\leavevmode\vadjust pre{\hypertarget{ref-yamano2003}{}}%
Yamano, M., Kinoshita, M., Goto, S., \& Matsubayashi, O. (2003). Extremely high heat flow anomaly in the middle part of the nankai trough. \emph{Physics and Chemistry of the Earth}, \emph{28}(9-11), 487--497. \url{https://doi.org/10.1016/s1474-7065(03)00068-8}

\leavevmode\vadjust pre{\hypertarget{ref-yamano2008}{}}%
Yamano, M., Kinoshita, M., \& Goto, S. (2008). High heat flow anomalies on an old oceanic plate observed seaward of the {Japan Trench}. \emph{Int. J. Earth Sci. (Geol. Rundish.)}, \emph{97}, 345--352.

\leavevmode\vadjust pre{\hypertarget{ref-yamano2014}{}}%
Yamano, M., Hamamoto, H., Kawada, Y., \& Goto, S. (2014). Heat flow anomaly on the seaward side of the japan trench associated with deformation of the incoming pacific plate. \emph{Earth and Planetary Science Letters}, \emph{407}(0), 196--204. \url{https://doi.org/10.1016/j.epsl.2014.09.039}

\leavevmode\vadjust pre{\hypertarget{ref-yamazaki1986}{}}%
Yamazaki, T. (1986). Heat flow measurements in the central pacific basin (GH81-4 area). \emph{Geol. Surv. Japan Cruise Rep.}, \emph{21}, 49--55.

\leavevmode\vadjust pre{\hypertarget{ref-yamazaki1992}{}}%
Yamazaki, T. (1992a). Heat flow in the izu-ogasawara (bonin)-mariana arc. \emph{Bull. Geol. Surv. Japan}, \emph{43}, 207--235.

\leavevmode\vadjust pre{\hypertarget{ref-yamazaki1992b}{}}%
Yamazaki, T. (1992b). Heat flow in the south of the nova-canton trough, central equitorial pacific (GH82-4 area). \emph{Geol. Surv. Japan Cruise Rep.}, \emph{22}, 71--83.

\leavevmode\vadjust pre{\hypertarget{ref-yamazaki1994}{}}%
Yamazaki, T. (1994). Heat flow in the penrhyn basin, south pacific (GH83-3 area). \emph{Geol. Surv. Japan Cruise Rep.}, \emph{23}, 201--207.

\leavevmode\vadjust pre{\hypertarget{ref-yang2004}{}}%
Yang, S., Hu, S., Cai, D., Feng, X., Chen, L., \& Gao, L. (2004). Present-day heat flow, thermal history and tectonic subsidence of the {East China Sea Basin}. \emph{Marine Petrol. Geol.}, \emph{21}, 1095--1105.

\leavevmode\vadjust pre{\hypertarget{ref-yang2006}{}}%
Yang, Y.-S., Ma, Y.-S., \& Hu, S.-B. (2006). Present-day geothermal characteristics of south china. \emph{Acta Physica Sinica}, \emph{49}, 1118--1126.

\leavevmode\vadjust pre{\hypertarget{ref-yasui2004}{}}%
Yasui, M. (2004). Unpublished data. In \emph{CD rom: Geothermal gradient and heat flow data in and around japan} (p. --). Geological Survey of Japan, AIST, 2004.

\leavevmode\vadjust pre{\hypertarget{ref-yasu1965}{}}%
Yasui, M., \& Watanabe, T. (1965). Studies of the thermal state of the {Earth}. {The 16th paper---Terrestrial heat flow in the Japan Sea}. \emph{Bull. Earthq. Res. Inst.}, \emph{43}, 549--563.

\leavevmode\vadjust pre{\hypertarget{ref-yasui1966}{}}%
Yasui, M., Kishii, T., Watanabe, T., \& Uyeda, S. (1966). Studies of the thermal state of the {Earth}. {The 18th paper---Terrestrial heat flow of the Japan Sea (2)}. \emph{Bull. Earthquake Res. Inst.}, \emph{44}, 1501--1518.

\leavevmode\vadjust pre{\hypertarget{ref-yasui1967}{}}%
Yasui, M., Kishii, T., \& Sudo, K. (1967). Terrestrial heat flow in the okhotsk sea (1). \emph{Oceanogrl. Mag.}, \emph{19}, 147--156.

\leavevmode\vadjust pre{\hypertarget{ref-yasui1968a}{}}%
Yasui, M., Kishii, T., Watanabe, T., \& Uyeda, S. (1968a). Heat flow in the sea of japan. In \emph{Crust and upper mantle of the {Pacific} area} (Vol. 12, pp. 3--16). American Geophysical Union Monograph.

\leavevmode\vadjust pre{\hypertarget{ref-yasui1968b}{}}%
Yasui, M., Nagasaka, K., Kishii, T., \& Halunen, A. J. (1968b). Terrestrial heat flow in the {Okhotsk Sea (2)}. \emph{Oceanogrl. Mag.}, \emph{20}, 73--86.

\leavevmode\vadjust pre{\hypertarget{ref-yasui1970}{}}%
Yasui, M., Epp, D., Nagasaka, K., \& Kishii, I. (1970). Terrestrial heat flow in the seas round the nansei shoto (ryukyu islands). \emph{Tectonophysics}, \emph{10}, 225--234. \url{https://doi.org/10.1016/0040-1951(70)90108-3}

\leavevmode\vadjust pre{\hypertarget{ref-yorath1983}{}}%
Yorath, C. J., \& Hyndman, R. D. (1983). Subsidence and thermal history of {Queen Charlotte Basin}. \emph{Can. J. Earth Sci.}, \emph{20}, 135--159.

\leavevmode\vadjust pre{\hypertarget{ref-zakowicz1975}{}}%
Zakowicz, K. (1975). \emph{Analiza w{l}asnosci fizycznych karbonu okolicy tyszowiec na tle ich rozwoju litologicznego}.

\leavevmode\vadjust pre{\hypertarget{ref-zhang2018}{}}%
Zhang, C., Jiang, G., Shi, Y., Wang, Z., Wang, Y., Li, S., et al. (2018). Terrestrial heat flow and crustal thermal structure of the gonghe-guide area, northeastern qinghai-tibetan plateau. \emph{Geothermics}, \emph{72}, 182--192. \url{https://doi.org/10.1016/j.geothermics.2017.11.011}

\leavevmode\vadjust pre{\hypertarget{ref-zheng2016}{}}%
Zheng, Y., Li, H., \& Gong, Z. (2016). Geothermal study at the wenchuan earthquake fault scientific drilling project-hole 1 ({WFSD}-1): Borehole temperature, thermal conductivity, and well log data. \emph{Journal of Asian Earth Sciences}, \emph{117}, 23--32. \url{https://doi.org/10.1016/j.jseaes.2015.11.025}

\leavevmode\vadjust pre{\hypertarget{ref-zhevago1972}{}}%
Zhevago, V. S. (1972). Geotermiya i termalnye vody kazakhstana (russ.). \emph{Alma-Ata Nauka}, \emph{254}.

\leavevmode\vadjust pre{\hypertarget{ref-ziagos1985}{}}%
Ziagos, J. P., Blackwell, D. D., \& Mooser, F. (1985). Heat flow in southern mexico and the thermal effects of subduction. \emph{Journal of Geophysical Research}, \emph{90}(b7), 5410--5420. \url{https://doi.org/10.1029/JB090iB07p05410}

\leavevmode\vadjust pre{\hypertarget{ref-zielinski1986}{}}%
Zielinski, G. W., Gunleiksrud, T., Saettem, J., Zuidberg, H. M., \& Geise, J. M. (1986). Deep heatflow measurements in quaternary sediments on the norwegian continental shelf. In \emph{Proceedings-offshore technology conference 18} (pp. 277--282).

\leavevmode\vadjust pre{\hypertarget{ref-zlotnicki1980}{}}%
Zlotnicki, V., Sclater, J. G., Norton, I. O., \& Von Herzen, R. P. (1980). Heat flow through the floor of the scotia, far south atlantic and weddell seas. \emph{Geophysical Research Letters}, \emph{7}, 421--424. \url{https://doi.org/10.1029/GL007i006p00421}

\leavevmode\vadjust pre{\hypertarget{ref-zolotarev1986}{}}%
Zolotarev, V. G. (1986). Geotermicheskaya model adenskogo rifta. \emph{Okeanologiya}, \emph{26}(6), 947--952.

\leavevmode\vadjust pre{\hypertarget{ref-zolotarev1980}{}}%
Zolotarev, V. G., \& Sochelnikov, V. V. (1980). Geotermicheskiye ussloviya afrikansko sicilianskogo podnatiya. (In russian). \emph{Izv. Akad. Nauk Sssr, Ser. Fizika Zemli}, \emph{16}(3), 202--206.

\leavevmode\vadjust pre{\hypertarget{ref-zolotarev1988}{}}%
Zolotarev, V. G., \& Sochelnikov, V. V. (1988). Teplovoe pole krasnomorskogo rifta. - v kn.: Geotermicheskie issledovaniya na dne akvatorii. Moskva, na- uka. (russ.)., 41--48.

\leavevmode\vadjust pre{\hypertarget{ref-zolotarev1979}{}}%
Zolotarev, V. G., Sochelnikov, V. V., \& Malovitskii, Ya. P. (1979). Rezultaty izmereniya teplovogo potoka v basseinakh chernogo i sredizemnogo morei. \emph{Okeanologiya}, \emph{T.19 Vyp. 6}, 1059--1065.

\leavevmode\vadjust pre{\hypertarget{ref-zolotarev1989a}{}}%
Zolotarev, V. G., Kondurin, A. V., \& Sochelnikov, V. V. (1989). \emph{Internal report}.

\leavevmode\vadjust pre{\hypertarget{ref-zuev1968}{}}%
Zuev, Yu. N., \& Firsov, F. V. (1968). Dokl. An ussr. \emph{N 11}.

\leavevmode\vadjust pre{\hypertarget{ref-zuev1982}{}}%
Zuev, Yu. N., \& Polikarpov, A. A. (1982). (Russ.). \emph{Dokl. An USSR}, \emph{10}.

\leavevmode\vadjust pre{\hypertarget{ref-zuev1984}{}}%
Zuev, Yu. N., \& Polikarpov, A. A. (1984). In: Zemnaya kora \& verkhnyaya mantiya gima- laev. \emph{Pamira}.

\leavevmode\vadjust pre{\hypertarget{ref-zuev1977}{}}%
Zuev, Yu. N., \& TalVirsky, B. B. (1977). Zemnaya kora \& verkhnyaya mantiya sred- ney azii (russ.). \emph{Moskva Nauka}.

\leavevmode\vadjust pre{\hypertarget{ref-zui1985}{}}%
Zui, V. I., Urban, G. I., Veselko, A. V., \& Zhuk, M. S. (1985). Geotermicheskie isledovaniya v kaliningradskoi oblasti i litovskoi ssr. In \emph{Seismologicheskie i geotermicheskie issledovaniya v belorussi} (pp. 88--94). Nauka i Tekhnika.

\leavevmode\vadjust pre{\hypertarget{ref-zuo2013}{}}%
Zuo, Y.-H., Qiu, N.-S., Deng, Y.-X., Rao, S., Xu, S.-M., \& Li, J.-G. (2013). Terrestrial heat flow in the qagan sag, inner {Mongolia}. \emph{Chinese Journal of Geophysics}, \emph{56}(5), 559--571. \url{https://doi.org/10.1002/cjg2.20053}

\end{CSLReferences}

\cleardoublepage

\hypertarget{appendix-appendix}{%
\appendix}


\markboth{Appendix: Chapter 2}{Appendix: Chapter 2}

\hypertarget{effects-of-thermo-kinematic-boundary-conditions-on-plate-coupling-in-subduction-zones}{%
\chapter{Effects of Thermo-kinematic Boundary Conditions on Plate Coupling in Subduction Zones}\label{effects-of-thermo-kinematic-boundary-conditions-on-plate-coupling-in-subduction-zones}}

\clearpage

\hypertarget{antDepth}{%
\section{Serpentine Stability Depth Through Time}\label{antDepth}}

Stability of serpentine progressively increases with depth along the plate interface as the subducting oceanic plate continuously cools and hydrates the shallow upper-plate mantle. However, this phenomenon ceases after approximately 5 Ma and dynamics change. From approximately 5 Ma to tens of Ma afterwards, the lower limit of serpentine dehydration stabilizes (Figure \ref{fig:antDepth}). In theory, serpentine dehydration should continue to increase as long as water continues to flux from the oceanic plate and the shallow upper-plate remains stagnant and cooling. Stability of serpentine dehydration through tens of Ma is direct result of the correspondence between mechanical coupling and absence of serpentine along the plate interface. Notably, using Lagrangian frameworks to implement metamorphic reactions is an advantageous numerical feature allowing for such behavior.

\begin{figure}[htbp]

{\centering \includegraphics[width=1\linewidth,]{assets/figs/chpt2/figA1} 

}

\caption[Serpentine stability depth vs. time]{Serpentine stability depth at the plate interface vs. time for models cda, cdf, and cdp with $Z_{UP}$ = 46, 62, 78, and 94 km. Serpentine stabilization deepens for approximately 5 Ma of subduction and then remains roughly constant for $\leq$ 10 Ma. The exceptions are models with very thin $Z_{UP}$, which exhibit transient behavior for at least 15 Ma. Overall serpentine stability depth after approximately 5 Ma depends on upper-plate thickness.}\label{fig:antDepth}
\end{figure}

Numerical experiments in this study suggest a negative dynamic feedback regulating coupling and serpentine dehydration can help explain how similar configurations, in terms of depths to subducting plates beneath arcs (\protect\hyperlink{ref-england2004}{England et al., 2004}) and thin upper-plates (\protect\hyperlink{ref-currie2006}{Currie \& Hyndman, 2006}), may occur in subduction zones with different thermo-kinematic boundary conditions and subduction durations. The results indicate that subduction zones quickly (\textless{} 5 Ma) develop and stabilize quasi-permanent, generalized configurations with coupling depth dependent on upper-plate thickness.

Notable exceptions occur in models with the thinnest upper-plates (\(Z_{UP}\) = 46 km). Rapid extension due to thin upper-plates form spreading centers in the upper-plate within 5 Ma. Passive asthenospheric upwelling near spreading centers diverts heat from deep within the upper-plate mantle. Enough heat is apparently diverted to disrupt thermal feedbacks regulating coupling and serpentine stability near the plate interface. In principle, diversion of heat from the plate interface could lead to cooler conditions, deeper serpentine stability, and thus deeper coupling. Further testing to confirm this behavior may artificially increase upper-plate strength in thin upper-plate thickness experiments to prevent high rates of spreading.

\begin{figure}[htbp]

{\centering \includegraphics[width=1\linewidth,]{assets/figs/chpt2/figA2} 

}

\caption[Results for model cdf with $Z_{UP}$ = 78 km at 1.64 Ma]{Visualizing model cdf with $Z_{UP}$ = 78 km at 1.64 Ma. (a) Rock type. (b) Temperature. (c) Viscosity. (d) Streamlines. Early subduction is facilitated by the prescribed initial weak layer cutting the lithosphere. Strain is localized in the weak serpentine layer along the plate interface. The shallow upper-plate mantle is stagnant and loses heat to the subducting plate, promoting serpentine stabiliization to greater depths. Rock type colors are the same as Figure \ref{fig:init}.}\label{fig:cdfStep1}
\end{figure}

\begin{figure}[htbp]

{\centering \includegraphics[width=1\linewidth,]{assets/figs/chpt2/figA3} 

}

\caption[Results for model cdf with $Z_{UP}$ = 78 km at 5.05 Ma]{Visualizing model cdf with $Z_{UP}$ = 78 km at 5.05 Ma. (a) Rock type. (b) Temperature. (c) Viscosity. (d) Streamlines. By 5 Ma balance is achieved between cooling and heating in the shallow and deep upper-plate mantle, respectively. A feedback regulating heat transfer, serpentine destabilization, and mechanical coupling is already stabilizing coupling depth. Rock type colors are the same as Figure \ref{fig:init}.}\label{fig:cdfStep2}
\end{figure}

\begin{figure}[htbp]

{\centering \includegraphics[width=1\linewidth,]{assets/figs/chpt2/figA4} 

}

\caption[Results for model cdf with $Z_{UP}$ = 78 km at 9.93 Ma]{Visualizing standard model cdf with $Z_{UP}$ = 78 km at 9.93 Ma. (a) Rock type. (b) Temperature. (c) Viscosity. (d) Streamlines. Geodynamics remain approximately constant from 5 Ma (cf. Figure \ref{fig:cdfStep2}). The system remains in steady state for as long water fluxes to the upper-plate mantle and serpentine is stable. Rock type colors are the same as Figure \ref{fig:init}.}\label{fig:cdfStep3}
\end{figure}

\begin{figure}[htbp]

{\centering \includegraphics[width=1\linewidth,]{assets/figs/chpt2/figA7} 

}

\caption[Surface heat flow for all numerical experiments]{Surface heat flow calculated at approximately 10 Ma for all numerical experiments. Normalized distance is the distance from the left boundary to the trench, divided by the distance between the trench and arc. Grayscale corresponds to $\Phi$. High amplitude fluctuations near the arc region (normalized distance = 1.0) correspond to vertical migration of fluids and melts. In the backarc region (normalized distance $\geq$ 1.0), these fluctuations correspond to lithospheric extension. Backarc extension is most apparent for high-$\Phi$ experiments (lighter gray lines). Experiments with no extension show a tight distribution of surface heat flow in the backarc region (darker gray lines).}\label{fig:hf}
\end{figure}

\clearpage

\hypertarget{regSummary}{%
\section{Regression Summaries}\label{regSummary}}

The form of the preferred quadratic regression model in Section \ref{cdEstimators} (Figure \ref{fig:results} \& Table \ref{tab:zcResults}) implies a lower limit to coupling depth of approximately 60 km, even for thin upper-plate thickness and, presumably, under warm conditions during nascent subduction. In principle, thin upper-plates could allow effective heat transfer in a flowing shallow asthenospheric mantle---hindering deep stabilization of serpentine. Olivine and pyroxene would be the stable mantle minerals, and strong, shallow coupling between plates would be expected Gerya et al. (\protect\hyperlink{ref-gerya2008}{2008}). However, even the warmest numerical experiments (low-\(\Phi\) \& thin upper-plate thickness) in this study eventually stabilize serpentine in the shallow upper-plate mantle. This is evident by increasing depth of mechanical coupling with time for the first 5 Ma of subduction (Figure \ref{fig:antDepth}).

\begin{figure}[htbp]

{\centering \includegraphics[width=1\linewidth,]{assets/figs/chpt2/figA5} 

}

\caption[Coupling depths determined from numerical experiments]{Coupling depths ($Z_{cpl}$, grayscale) determined from numerical experiments. Model names are listed along the top axis and correspond to the range of thermal parameter $\Phi$ values along the bottom axis. Note that the ($\Phi$) axis is not linear. $Z_{cpl}$ increases systematically with increasing $Z_{UP}$ (change in grayscale down columns) for all models. Trends in $Z_{cpl}$ with respect to $\Phi$ (change in grayscale across rows) are less apparent.}\label{fig:results}
\end{figure}

\begin{table}

\caption{\label{tab:anova}Summary of ANOVA test}
\centering
\begin{threeparttable}
\begin{tabular}[t]{lrrrl}
\toprule
\multicolumn{1}{c}{$Z_{UP}$ Groups} & \multicolumn{1}{c}{$Z_{cpl}$ Estimate} & \multicolumn{1}{c}{Upper Bound} & \multicolumn{1}{c}{Lower Bound} & \multicolumn{1}{c}{p value} \\
\multicolumn{1}{c}{} & \multicolumn{1}{c}{$[km]$} & \multicolumn{1}{c}{$[km]$} & \multicolumn{1}{c}{$[km]$} & \multicolumn{1}{c}{}\\
\midrule
\cellcolor{gray!6}{62-46} & \cellcolor{gray!6}{8.3} & \cellcolor{gray!6}{2.5} & \cellcolor{gray!6}{14.0} & \cellcolor{gray!6}{1.84e-03}\\
78-46 & 18.0 & 12.3 & 23.7 & 1.08e-10\\
\cellcolor{gray!6}{94-46} & \cellcolor{gray!6}{33.6} & \cellcolor{gray!6}{27.8} & \cellcolor{gray!6}{39.3} & \cellcolor{gray!6}{1.99e-11}\\
78-62 & 9.8 & 4.0 & 15.5 & 1.83e-04\\
\cellcolor{gray!6}{94-62} & \cellcolor{gray!6}{25.3} & \cellcolor{gray!6}{19.6} & \cellcolor{gray!6}{31.0} & \cellcolor{gray!6}{1.99e-11}\\
\addlinespace
94-78 & 15.6 & 9.8 & 21.3 & 7.31e-09\\
\bottomrule
\end{tabular}
\begin{tablenotes}
\item Pair-wise Tukey's test comparing means between groups. Estimates are differences between means. Null hypothesis is that means are not different
\end{tablenotes}
\end{threeparttable}
\end{table}

\begin{table}

\caption{\label{tab:regSummary}Summary of regression models}
\centering
\begin{threeparttable}
\begin{tabular}[t]{llrrl}
\toprule
Model & Term & Estimate & Std. Error & p value\\
\midrule
\cellcolor{gray!6}{1} & \cellcolor{gray!6}{Intercept} & \cellcolor{gray!6}{89.4} & \cellcolor{gray!6}{3.7} & \cellcolor{gray!6}{2.24e-33}\\
1 & $\phi$ & -0.1 & 0.1 & 1.55e-01\\
\cellcolor{gray!6}{2} & \cellcolor{gray!6}{Intercept} & \cellcolor{gray!6}{36.4} & \cellcolor{gray!6}{3.2} & \cellcolor{gray!6}{7.15e-17}\\
2 & $Z_{UP}$ & 0.7 & 0.0 & 3.73e-23\\
\cellcolor{gray!6}{3} & \cellcolor{gray!6}{Intercept} & \cellcolor{gray!6}{58.9} & \cellcolor{gray!6}{1.7} & \cellcolor{gray!6}{1.43e-41}\\
\addlinespace
3 & $Z_{UP}^2$ & 0.0 & 0.0 & 2.98e-24\\
\cellcolor{gray!6}{4} & \cellcolor{gray!6}{Intercept} & \cellcolor{gray!6}{69.2} & \cellcolor{gray!6}{14.0} & \cellcolor{gray!6}{6.25e-06}\\
4 & $Z_{UP}$ & -0.3 & 0.4 & 4.63e-01\\
\cellcolor{gray!6}{4} & \cellcolor{gray!6}{$Z_{UP}^2$} & \cellcolor{gray!6}{0.0} & \cellcolor{gray!6}{0.0} & \cellcolor{gray!6}{1.95e-02}\\
5 & Intercept & 41.1 & 3.3 & 1.14e-18\\
\addlinespace
\cellcolor{gray!6}{5} & \cellcolor{gray!6}{$Z_{UP}$} & \cellcolor{gray!6}{0.7} & \cellcolor{gray!6}{0.0} & \cellcolor{gray!6}{1.12e-24}\\
5 & $\phi$ & -0.1 & 0.0 & 1.18e-03\\
\cellcolor{gray!6}{6} & \cellcolor{gray!6}{Intercept} & \cellcolor{gray!6}{63.6} & \cellcolor{gray!6}{2.1} & \cellcolor{gray!6}{8.29e-39}\\
6 & $Z_{UP}^2$ & 0.0 & 0.0 & 5.68e-26\\
\cellcolor{gray!6}{6} & \cellcolor{gray!6}{$\phi$} & \cellcolor{gray!6}{-0.1} & \cellcolor{gray!6}{0.0} & \cellcolor{gray!6}{6.98e-04}\\
\addlinespace
7 & Intercept & 73.8 & 12.9 & 3.39e-07\\
\cellcolor{gray!6}{7} & \cellcolor{gray!6}{$Z_{UP}$} & \cellcolor{gray!6}{-0.3} & \cellcolor{gray!6}{0.4} & \cellcolor{gray!6}{4.23e-01}\\
7 & $Z_{UP}^2$ & 0.0 & 0.0 & 1.12e-02\\
\cellcolor{gray!6}{7} & \cellcolor{gray!6}{$\phi$} & \cellcolor{gray!6}{-0.1} & \cellcolor{gray!6}{0.0} & \cellcolor{gray!6}{7.28e-04}\\
\bottomrule
\end{tabular}
\begin{tablenotes}
\item \uline{\textit{models}}: 1: $[z_c=\phi]$, 2: $[z_c=Z_{UP}]$, 3: $[z_c=Z_{UP}^2]$, 4: $[z_c=Z_{UP}+Z_{UP}^2]$, 5: $[z_c=Z_{UP}+\phi]$, 6: [$z_c=Z_{UP}^2+\phi]$, 7: $[z_c=Z_{UP}+Z_{UP}^2+\phi]$
\end{tablenotes}
\end{threeparttable}
\end{table}

\begingroup
\renewcommand{\arraystretch}{0.5}

\begin{longtable}[t]{lrrr}
\caption{\label{tab:zcResults}Coupling depth results}\\
\toprule
Model & $Z_{UP}$ $[km]$ & $\Phi$ $[km/100]$ & $Z_{cpl}$ $[km]$\\
\midrule
\endfirsthead
\caption[]{\label{tab:zcResults}Coupling depth results \textit{(continued)}}\\
\toprule
Model & $Z_{UP}$ $[km]$ & $\Phi$ $[km/100]$ & $Z_{cpl}$ $[km]$\\
\midrule
\endhead

\endfoot
\bottomrule
\endlastfoot
\cellcolor{gray!6}{cda} & \cellcolor{gray!6}{46} & \cellcolor{gray!6}{13.0} & \cellcolor{gray!6}{66}\\
cdb & 46 & 21.5 & 74\\
\cellcolor{gray!6}{cdc} & \cellcolor{gray!6}{46} & \cellcolor{gray!6}{26.1} & \cellcolor{gray!6}{69}\\
cdd & 46 & 32.6 & 67\\
\cellcolor{gray!6}{cde} & \cellcolor{gray!6}{46} & \cellcolor{gray!6}{22.0} & \cellcolor{gray!6}{72}\\
cdf & 46 & 36.3 & 78\\
\cellcolor{gray!6}{cdg} & \cellcolor{gray!6}{46} & \cellcolor{gray!6}{44.0} & \cellcolor{gray!6}{78}\\
cdh & 46 & 55.0 & 59\\
\cellcolor{gray!6}{cdi} & \cellcolor{gray!6}{46} & \cellcolor{gray!6}{34.0} & \cellcolor{gray!6}{80}\\
cdj & 46 & 56.1 & 70\\
\cellcolor{gray!6}{cdk} & \cellcolor{gray!6}{46} & \cellcolor{gray!6}{68.0} & \cellcolor{gray!6}{58}\\
cdl & 46 & 85.0 & 65\\
\cellcolor{gray!6}{cdm} & \cellcolor{gray!6}{46} & \cellcolor{gray!6}{44.0} & \cellcolor{gray!6}{79}\\
cdn & 46 & 72.6 & 70\\
\cellcolor{gray!6}{cdo} & \cellcolor{gray!6}{46} & \cellcolor{gray!6}{88.0} & \cellcolor{gray!6}{68}\\
cdp & 46 & 110.0 & 64\\
\cellcolor{gray!6}{cda} & \cellcolor{gray!6}{62} & \cellcolor{gray!6}{13.0} & \cellcolor{gray!6}{80}\\
cdb & 62 & 21.5 & 79\\
\cellcolor{gray!6}{cdc} & \cellcolor{gray!6}{62} & \cellcolor{gray!6}{26.1} & \cellcolor{gray!6}{78}\\
cdd & 62 & 32.6 & 77\\
\cellcolor{gray!6}{cde} & \cellcolor{gray!6}{62} & \cellcolor{gray!6}{22.0} & \cellcolor{gray!6}{87}\\
cdf & 62 & 36.3 & 82\\
\cellcolor{gray!6}{cdg} & \cellcolor{gray!6}{62} & \cellcolor{gray!6}{44.0} & \cellcolor{gray!6}{75}\\
cdh & 62 & 55.0 & 70\\
\cellcolor{gray!6}{cdi} & \cellcolor{gray!6}{62} & \cellcolor{gray!6}{34.0} & \cellcolor{gray!6}{91}\\
cdj & 62 & 56.1 & 77\\
\cellcolor{gray!6}{cdk} & \cellcolor{gray!6}{62} & \cellcolor{gray!6}{68.0} & \cellcolor{gray!6}{72}\\
cdl & 62 & 85.0 & 67\\
\cellcolor{gray!6}{cdm} & \cellcolor{gray!6}{62} & \cellcolor{gray!6}{44.0} & \cellcolor{gray!6}{88}\\
cdn & 62 & 72.6 & 77\\
\cellcolor{gray!6}{cdo} & \cellcolor{gray!6}{62} & \cellcolor{gray!6}{88.0} & \cellcolor{gray!6}{74}\\
cdp & 62 & 110.0 & 75\\
\cellcolor{gray!6}{cda} & \cellcolor{gray!6}{78} & \cellcolor{gray!6}{13.0} & \cellcolor{gray!6}{87}\\
cdb & 78 & 21.5 & 94\\
\cellcolor{gray!6}{cdc} & \cellcolor{gray!6}{78} & \cellcolor{gray!6}{26.1} & \cellcolor{gray!6}{97}\\
cdd & 78 & 32.6 & 97\\
\cellcolor{gray!6}{cde} & \cellcolor{gray!6}{78} & \cellcolor{gray!6}{22.0} & \cellcolor{gray!6}{90}\\
cdf & 78 & 36.3 & 90\\
\cellcolor{gray!6}{cdg} & \cellcolor{gray!6}{78} & \cellcolor{gray!6}{44.0} & \cellcolor{gray!6}{88}\\
cdh & 78 & 55.0 & 85\\
\cellcolor{gray!6}{cdi} & \cellcolor{gray!6}{78} & \cellcolor{gray!6}{34.0} & \cellcolor{gray!6}{97}\\
cdj & 78 & 56.1 & 91\\
\cellcolor{gray!6}{cdk} & \cellcolor{gray!6}{78} & \cellcolor{gray!6}{68.0} & \cellcolor{gray!6}{84}\\
cdl & 78 & 85.0 & 77\\
\cellcolor{gray!6}{cdm} & \cellcolor{gray!6}{78} & \cellcolor{gray!6}{44.0} & \cellcolor{gray!6}{78}\\
cdn & 78 & 72.6 & 87\\
\cellcolor{gray!6}{cdo} & \cellcolor{gray!6}{78} & \cellcolor{gray!6}{88.0} & \cellcolor{gray!6}{85}\\
cdp & 78 & 110.0 & 78\\
\cellcolor{gray!6}{cda} & \cellcolor{gray!6}{94} & \cellcolor{gray!6}{13.0} & \cellcolor{gray!6}{95}\\
cdb & 94 & 21.5 & 101\\
\cellcolor{gray!6}{cdc} & \cellcolor{gray!6}{94} & \cellcolor{gray!6}{26.1} & \cellcolor{gray!6}{108}\\
cdd & 94 & 32.6 & 113\\
\cellcolor{gray!6}{cde} & \cellcolor{gray!6}{94} & \cellcolor{gray!6}{22.0} & \cellcolor{gray!6}{100}\\
cdf & 94 & 36.3 & 104\\
\cellcolor{gray!6}{cdg} & \cellcolor{gray!6}{94} & \cellcolor{gray!6}{44.0} & \cellcolor{gray!6}{104}\\
cdh & 94 & 55.0 & 104\\
\cellcolor{gray!6}{cdi} & \cellcolor{gray!6}{94} & \cellcolor{gray!6}{34.0} & \cellcolor{gray!6}{101}\\
cdj & 94 & 56.1 & 102\\
\cellcolor{gray!6}{cdk} & \cellcolor{gray!6}{94} & \cellcolor{gray!6}{68.0} & \cellcolor{gray!6}{101}\\
cdl & 94 & 85.0 & 107\\
\cellcolor{gray!6}{cdm} & \cellcolor{gray!6}{94} & \cellcolor{gray!6}{44.0} & \cellcolor{gray!6}{106}\\
cdn & 94 & 72.6 & 102\\
\cellcolor{gray!6}{cdo} & \cellcolor{gray!6}{94} & \cellcolor{gray!6}{88.0} & \cellcolor{gray!6}{98}\\
cdp & 94 & 110.0 & 108\\*
\end{longtable}

\endgroup

\begin{figure}[htbp]

{\centering \includegraphics[width=1\linewidth,]{assets/figs/chpt2/figA6} 

}

\caption[Bivariate regressions of coupling results]{Bivariate regressions. (a) Coupling depth ($Z_{cpl}$) vs. upper-plate thickness ($Z_{UP}$) shows $Z_{cpl}$ increasing approximately quadratically with increasing $Z_{UP}$. The correlation is highly significant (see Tables \ref{tab:anova} and \ref{tab:regSummary}) and explains more than 80\% of the variance in $Z_{cpl}$. $Z_{UP}$ alone estimates $Z_{cpl}$ well. (b) $Z_{cpl}$ vs. thermal parameter ($\Phi$) shows no significant correlation (no line fits with a slope significantly different from zero). $\Phi$ has little effect on $Z_{cpl}$ and cannot be used as a standalone estimator.}\label{fig:biv}
\end{figure}

\clearpage

\hypertarget{deHydration}{%
\section{(De)hydration Model}\label{deHydration}}

The material properties used in the numerical experiments are listed in Table \ref{tab:materials} and Table \ref{tab:melts}. For details about the sedimentation and erosion, melting and extraction, and rheological models, refer to Sizova et al. (\protect\hyperlink{ref-sizova2010}{2010}). Here we discuss only the hydrodynamic model, because it is the most relevant aspect of the numerical experiments.

Hydrodynamics in the numerical models control the timing and magnitude of mantle wedge hydration. The main sources of water delivered to the mantle are altered basaltic crust and seafloor sediments, which we assumed to contain up to 5 \(wt.\% H_{2}O\). We assumed a gradual expulsion of water from pore space and through quasi-continuous dehydration reactions occurring within the slab. Water content is computed using the following equation:
\begin{equation}
  \chi_{H_{2}O} = \chi_{H_{2}O_{init}}\times\left(1-\frac{\Delta z}{150\times 10^{3}}\right)
\end{equation}
where \(\chi_{H_{2}O_{init}}\) = 5 \(wt.\%\) and \(\Delta z\) is a marker's depth below the topographical surface.

If a rock marker dehydrates, an independent water particle is instantaneously generated at the same location with the respective \(H_{2}O\) content. The new water particle is moved in accordance to the local velocity field, described by the following equation:
\begin{equation}
  \begin{aligned}
    \vec{v}_{\text{water}} & = (\vec{v}_x,\ \vec{v}_z) \\
    \vec{v}_z & = \vec{v}_z - \vec{v}_{z(\text{percolation})} \\
  \end{aligned}
\end{equation}
where \(\vec{v}_{water}\) is the velocity vector of the water particle, \(\vec{v}_{x}\) and \(\vec{v}_{z}\) are the local velocity vectors of the solid state mantle or crust, and \(\vec{v}_{z(percolation)}\) is a imposed constant upward percolation velocity (10 \(cm/year\)). We implicitly neglect kinetics of reactions, as material properties of markers change instantaneously at equilibrium reactions.

\begin{table}

\caption{\label{tab:melts}Melting curves used in numerical experiments}
\centering
\resizebox{\linewidth}{!}{
\begin{threeparttable}
\begin{tabular}[t]{lrrlrlrllrr}
\toprule
Material & a & b & c & d & e & f & g & h & i & j\\
\midrule
\cellcolor{gray!6}{sediments} & \cellcolor{gray!6}{1200} & \cellcolor{gray!6}{889} & \cellcolor{gray!6}{1.79e+04} & \cellcolor{gray!6}{54} & \cellcolor{gray!6}{2.02e+04} & \cellcolor{gray!6}{831} & \cellcolor{gray!6}{6.00e-02} & \cellcolor{gray!6}{} & \cellcolor{gray!6}{1262} & \cellcolor{gray!6}{0.009}\\
felsic crust & 1200 & 889 & 1.79e+04 & 54 & 2.02e+04 & 831 & 6.00e-02 &  & 1262 & 0.009\\
\cellcolor{gray!6}{basalt} & \cellcolor{gray!6}{1600} & \cellcolor{gray!6}{973} & \cellcolor{gray!6}{7.04e+05} & \cellcolor{gray!6}{354} & \cellcolor{gray!6}{7.78e+07} & \cellcolor{gray!6}{935} & \cellcolor{gray!6}{3.50e-03} & \cellcolor{gray!6}{6.2e-05} & \cellcolor{gray!6}{1423} & \cellcolor{gray!6}{0.105}\\
gabbro & 1600 & 973 & 7.04e+05 & 354 & 7.78e+07 & 935 & 3.50e-03 & 6.2e-05 & 1423 & 0.105\\
\cellcolor{gray!6}{mantle dry} & \cellcolor{gray!6}{} & \cellcolor{gray!6}{} & \cellcolor{gray!6}{} & \cellcolor{gray!6}{} & \cellcolor{gray!6}{} & \cellcolor{gray!6}{1394} & \cellcolor{gray!6}{1.33e-01} & \cellcolor{gray!6}{-5.1e-05} & \cellcolor{gray!6}{2073} & \cellcolor{gray!6}{0.114}\\
mantle hydrated & 2400 & 1240 & 4.98e+04 & 323 &  &  & 1.27e+05 & 3.5e-05 & 2073 & 0.114\\
\cellcolor{gray!6}{serpentine} & \cellcolor{gray!6}{2400} & \cellcolor{gray!6}{1240} & \cellcolor{gray!6}{4.98e+04} & \cellcolor{gray!6}{323} & \cellcolor{gray!6}{} & \cellcolor{gray!6}{} & \cellcolor{gray!6}{1.27e+05} & \cellcolor{gray!6}{3.5e-05} & \cellcolor{gray!6}{2073} & \cellcolor{gray!6}{0.114}\\
\bottomrule
\end{tabular}
\begin{tablenotes}
\item \uline{\textit{solidus curve}}: $T(P)=[b+\frac{c}{(P+d)}+\frac{e}{(P+d)^2}]$ at $P<a$ and $[f+gP+hP^2]$ at $P\geq a$
\item \uline{\textit{liquidus curve}}: $T(P) = i+jP$ with $T$ in $[K]$ and $P$ in $[MPa]$
\item \uline{\textit{reference}}: Schmidt \& Poli (1998)
\end{tablenotes}
\end{threeparttable}}
\end{table}

\hypertarget{rheologicSensitivity}{%
\section{Rheologic Sensitivity Tests on Plate Coupling}\label{rheologicSensitivity}}

Numerical modelling practitioners simulating subduction zones approach mechanical coupling between plates differently. A simple, but highly effective approach, is prescribing a layer of arbitrary strength extending from the surface to an arbitrary depth or temperature along the plate interface. This approach effectively inhibits transfer of shear stress between plates and is analogous to controlling a no-slip condition at the interface (plates move with the same velocity vector beyond the coupling point). Numerous models use this method (e.g. \protect\hyperlink{ref-peacock1996}{Peacock, 1996}; \protect\hyperlink{ref-peacock1999b}{Peacock \& Wang, 1999}; \protect\hyperlink{ref-syracuse2010}{Syracuse et al., 2010}; \protect\hyperlink{ref-wada2009}{Wada \& Wang, 2009}) in part because it allows fine-tuning to specific subduction zone configurations. Serpentine-or talc-rich horizons are typically invoked to justify implementing such a condition at shallow interface depths.

The experiments outlined in Section \ref{numMethods} do not explicitly define coupling, but rather use a rheologic model that explicitly follows experimentally determined flow laws and mineral stability fields. This approach conceptually follows and extends petrologic explanations for a weak interface (\protect\hyperlink{ref-hyndman2003}{Hyndman \& Peacock, 2003}; \protect\hyperlink{ref-peacock1999a}{Peacock \& Hyndman, 1999}). As a corollary, dehydration of serpentine, or possibly talc, at higher temperatures must strengthen the interface (\protect\hyperlink{ref-agard2016}{Agard et al., 2016}). Noting that talc is unstable at P \textgreater{} 2.0 GPa in an ultramafic rock (\protect\hyperlink{ref-schmidt1998}{Schmidt \& Poli, 1998}), a serpentine rheology is arguably the most relevant candidate responsible for a strength increase, and thus coupling, at \gls{pt} conditions inferred for coupling in active subduction zones (\protect\hyperlink{ref-syracuse2010}{Syracuse et al., 2010}; \protect\hyperlink{ref-wada2009}{Wada \& Wang, 2009}).

Sensitivity tests of the rheologic model presented in Section \ref{rheologicModel} were run using diverse experiments adjusting the rheology of serpentine (compared to Table \ref{tab:materials}), the shape and position of the antigorite-out reaction (compared to \eqref{eq:antstab}), and certain hydrodynamic parameters. For brevity, these results are not presented here. The experiments included:

\begin{enumerate}
\def\labelenumi{\arabic{enumi}.}
\tightlist
\item
  antigorite \(\leftarrow\) wet olivine flow law
\item
  antigorite and wet olivine \(\leftarrow\) dry olivine flow law
\item
  isothermal antigorite reaction at 690 \(^{\circ}\)C
\item
  antigorite reaction isothermal Clapeyron slope at 715 \(^{\circ}\)C
\item
  antigorite reaction with positive linear Clapeyron slope
\item
  linear release of \(H_{2}O\) with depth
\item
  no fluid-induced weakening
\end{enumerate}

Only experiments 5 and 7 listed above were inconsistent with the results presented Section \ref{chpt2Results}. Experiment 5 results in transient coupling depths and discontinuous antigorite stability in the upper-plate mantle, whereas experiment 7 results in two-sided subduction (e.g. \protect\hyperlink{ref-gerya2008}{Gerya et al., 2008}). These sensitivity experiments imply numerical coupling mechanisms are mostly contingent on fluid flux to the upper-plate mantle and the implementation of serpentine stability. The experiments also show coupling is relatively insensitive to the exact flow law parameters.

\cleardoublepage

\markboth{Appendix: Chapter 3}{Appendix: Chapter 3}

\hypertarget{a-comparison-of-surface-heat-flow-interpolations-near-subduction-zones}{%
\chapter{A Comparison of Surface Heat Flow Interpolations Near Subduction Zones}\label{a-comparison-of-surface-heat-flow-interpolations-near-subduction-zones}}

\clearpage

\hypertarget{krigeOpt}{%
\section{Kriging System and Optimization}\label{krigeOpt}}

\hypertarget{ordinary-kriging}{%
\subsection{Ordinary Kriging}\label{ordinary-kriging}}

This study applies local isotropic ordinary Kriging methods under the following general assumptions:

\begin{itemize}
\tightlist
\item
  \(\hat{\gamma}(h)\) is directionally invariant (isotropic)
\item
  \(\hat{\gamma}(h)\) is evaluated in two-dimensions and neglects elevation
\item
  The first and second moments of \(Z(u)\) are assumed to follow the conditions:
\end{itemize}

\begin{equation}
  \begin{aligned}
    &E[Z(u)] = \hat{Z}(u) = constant \\
    &E[(Z(u + h) - \hat{Z}(u))(Z(u) - \hat{Z}(u))] = C(h)
  \end{aligned}
  \label{eq:krigeAssumptions}
\end{equation}
where \(h\) is the lag distance, \(C(h)\) is the covariance function, \(E[Z(u)]\) is the expected value of the random variable \(Z(u)\), and \(\hat{Z}(u)\) is the arithmetic mean of \(Z(u)\).

Equation \eqref{eq:krigeAssumptions} is known as ``weak second-order stationarity''. It assumes the underlying probability distribution of the observations \(Z(u)\) does not change in space and the covariance \(C(h)\) only depends on the distance \(h\) between two observations. These assumptions are expected to be valid in cases where the underlying natural process is stochastic, spatially continuous, and has the property of additivity such that \(\frac{1}{n}\sum_{i=1}^n Z(u_i)\) has the same meaning as \(Z(u)\) (\protect\hyperlink{ref-bardossy1997}{Bárdossy, 1997}).

The following are two illustrative cases where Equation \eqref{eq:krigeAssumptions} is likely valid:

\begin{quote}
The thickness of a sedimentary unit with a homogeneous concentration of radioactive elements can be approximated by \(q_s = q_b + \int A \,dz\), where \(q_b\) is a constant heat flux entering the bottom of the layer and \(A\) is the heat production within the layer with thickness \(z\) (\protect\hyperlink{ref-furlong2013}{Furlong \& Chapman, 2013}). If one has two samples, \(Z(u_1)\) = 31 mW/m\(^2\) and \(Z(u_2)\) = 30.5 mW/m\(^2\), their corresponding thicknesses would be \(Z'(u_1)\) = 1000 m and \(Z'(u_2)\) = 500 m for \(A\) = 0.001 mW/m\(^3\) and \(q_b\) = 30 mW/m\(^2\). The variable, \(Z(u)\), in this case is additive because the arithmetic mean of the samples is a good approximation of the average sedimentary layer thickness, \((Z(u_1) + Z(u_2)) /\) 2 = 750 m.
\end{quote}

\begin{quote}
The age of young oceanic lithosphere can be approximated by \(q_s(t) = kT_b(\pi\kappa t)^{-1/2}\), where \(q_s(t)\) is surface heat flow of a plate with age, \(t\), \(T_b\) is the temperature at the base of the plate, \(k\) is thermal conductivity, and \(\kappa = k/\rho C_p\) is thermal diffusivity (\protect\hyperlink{ref-stein1992}{Stein \& Stein, 1992}). Using reasonable values for \(k\) = 3.138 W/mK, \(\rho\) = 3330 kg/m\(^3\), \(C_p\) = 1171 J/kgK, \(T_b\) = 1350 \(^{\circ}\)C, two samples, \(Z(u_1)\) = 180 mW/m\(^2\) and \(Z(u_2)\) = 190 mW/m\(^2\), would correspond to plates with ages of \(Z'(u_1)\) = 10 Ma, and \(Z'(u_2)\) = 9 Ma, respectively. Since \(Z(u_1) + Z(u_2) /\) 2 = 185 mW/m\(^2\) and \(Z'(185~mW/m^2)\) = 9.5 Ma = \(Z'(u_1) + Z'(u_2) /\) 2, the variable \(Z(u)\) in this case is also additive.
\end{quote}

Equation \eqref{eq:krigeAssumptions} is likely invalid in regions that transition among two or more tectonic regimes, however. For example, the expected (mean) heat flow \(E[Z(u)]\) will change when moving from a spreading center to a subduction zone and thus \(E[Z(u)] \neq constant\) over the region of interest. In other words, stationarity is violated and Kriging estimates may become spurious. Careful selection of Kriging parameters (outlined below; e.g.~maximum point-pairs to use for local Kriging) can reduce or eliminate violations of stationarity assumptions embodied in \eqref{eq:krigeAssumptions}.

The second step is fitting a variogram model \(\gamma(h)\) to the experimental variogram. This study fits six popular variogram models with sills (or theoretical sills) to the experimental variogram. The models are defined as (\protect\hyperlink{ref-pebesma2004}{Pebesma, 2004}):
\begin{equation}
  \begin{aligned}
    Bes &\leftarrow \gamma(h) = 1 - \frac{h}{a}\ K_1\left(\frac{h}{a}\right) \quad \text{for } \  h \geq 0 \\
    Cir &\leftarrow \gamma(h) =
    \begin{cases}
      \frac{2}{\pi}\frac{h}{a}\ \sqrt{1-\left(\frac{h}{a}\right)^2} + \frac{2}{\pi}\ arcsin\left(\frac{h}{a}\right) \quad \text{for } \  0 \leq h \leq a \\
      nug + sill \quad \text{for } \  h > a
    \end{cases} \\
    Exp &\leftarrow \gamma(h) = 1 - exp\left(\frac{-h}{a}\right) \quad \text{for } \  h \geq 0 \\
    Gau &\leftarrow \gamma(h) = 1 - exp\left(\left[\frac{-h}{a}\right]^2\right) \quad \text{for } \  h \geq 0 \\
    Lin &\leftarrow \gamma(h) =
    \begin{cases}
      \frac{h}{a} \quad \text{for } \  0 \leq h \leq a \\
      nug + sill \quad \text{for } \  h > a
    \end{cases} \\
    Sph &\leftarrow \gamma(h) =
    \begin{cases}
      \frac{3}{2}\frac{h}{a} - \frac{1}{2}\left(\frac{h}{a}\right)^3 \quad \text{for } \  0 \leq h \leq a \\
      nug + sill \quad \text{for } \  h > a
    \end{cases} \\
  \end{aligned}
  \label{eq:varMods}
\end{equation}
where \(h\) is the lag distance, \(nug\) is the nugget, \(sill\) is the sill, \(a\) is the effective range, \(K_1\) is a modified Bessel function. The models are Bessel, Circular, Exponential, Gaussian, Linear, and Spherical. For models without explicit sills (Bes, Exp, Gau), the effective range \(a\) is the distance where the variogram reaches 95\% of its maximum defined as 4\(a\), 3\(a\), and \(\sqrt{3}a\) for Bes, Exp, and Gau, respectively (\protect\hyperlink{ref-graler2016}{Gräler et al., 2016}; \protect\hyperlink{ref-pebesma2004}{Pebesma, 2004}). The function \texttt{fit.variogram} in \texttt{gstat} is used to try all variogram models. The best model is selected by the minimum weighted least squares (\protect\hyperlink{ref-pebesma2004}{Pebesma, 2004}) error with weights proportional to the number of points in each lag divided by the squared lag distance \(wt = N(h)_k/h_k^2\). Gaussian models produce spurious results in every case and are not included in the final analysis. Moreover, Circular models produce indistinguishable results from Spherical models, and so too were omitted from the final analysis.

Ordinary Kriging is used for interpolation, which estimates unknown observations \(\hat{Z}(u)\) as a linear combination of all known observations (\protect\hyperlink{ref-bardossy1997}{Bárdossy, 1997}):
\begin{equation}
  \hat{Z}(u) = \sum_{i=1}^n \lambda_i Z(u_i)
  \label{eq:linEstimate}
\end{equation}

The conditions in Equation \eqref{eq:krigeAssumptions} set up a constrained minimization problem that can be solved with a system of linear equations. The expected value of \(Z(u)\) is assumed to be the mean according to \eqref{eq:krigeAssumptions}, so the weights must be:
\begin{equation}
  \begin{aligned}
    E[\hat{Z}(u)] &= \sum_{i=1}^n \lambda_i E[Z(u_i)] \\
    \sum_{i=1}^n \lambda_i &= 1
  \end{aligned}
  \label{eq:unbiased}
\end{equation}

This constraint is known as the unbiased condition, which states that the sum of the weights must equal one. However, there is an infinite set of real numbers one could use for the weights, \(\lambda_i\). The goal is to find the set of weights in Equation \eqref{eq:linEstimate} that minimizes the estimation variance. This can be solved by minimizing the covariance function, \(C(h)\) from Equation \eqref{eq:krigeAssumptions}:
\begin{equation}
  \begin{aligned}
    & \sigma^2(u) = Var[Z(u) - \hat{Z}(u)] = \\
    & E\left[(Z(u) - \sum_{i=1}^n \lambda_i Z(u_i))^2\right] = \\
    & E\left[Z(u)^2 + \sum_{j=1}^n \sum_{i=1}^n \lambda_j \lambda_i Z(u_j)Z(u_i) - 2 \sum_{i=1}^n \lambda_i Z(u_i)Z(u)\right] = \\
    & C(0) + \sum_{j=1}^n \sum_{i=1}^n \lambda_j \lambda_i C(u_i - u_j) - 2 \sum_{i=1}^n \lambda_i C(u_i - u)
  \end{aligned}
  \label{eq:minVar}
\end{equation}

Minimizing Equation \eqref{eq:minVar} with respect to the unbiased condition (Equation \eqref{eq:unbiased}), yields the best linear unbiased estimator (BLUE, \protect\hyperlink{ref-bardossy1997}{Bárdossy, 1997}) for Equation \eqref{eq:linEstimate} and together comprise the Kriging system of equations. The functions \texttt{krige} and \texttt{krige.cv} in \texttt{gstat} are used for surface heat flow interpolation and error estimation by k-fold cross-validation (\protect\hyperlink{ref-pebesma2004}{Pebesma, 2004}).

\hypertarget{nloptr}{%
\subsection{\texorpdfstring{Optimization with \texttt{nloptr}}{Optimization with nloptr}}\label{nloptr}}

Achieving accurate Kriging results depends on one's choice of many Kriging parameters, \(\Theta\). In this study, we investigate a set of parameters:
\begin{equation} 
  \Theta = \{model,\ n_{lag},\ cut,\ n_{max},\ shift\}
  \label{eq:params}
\end{equation}
where \(model\) is one of the variogram models defined in Equation \eqref{eq:varMods}, \(n_{lag}\) is the number of lags, \(cut\) is a lag cutoff proportionality constant, \(n_{max}\) is the maximum point-pairs for local Kriging, and \(shift\) is a horizontal lag shift constant. The lag cutoff constant \(cut\) controls the maximum separation distance between pairs of points used to calculate the experimental variogram (i.e.~the x-axis range or ``width'' of the experimental variogram). The horizontal lag shift constant \(shift\) removes the first few lags from being evaluated by effectively shifting all lags to the left proportionally by \(shift\). This is necessary to avoid negative ranges when fitting experimental variograms with anomalously high variances at small lag distances.

The goal is to find \(\Theta\) such that the Kriging function \(f(x_i;\ \Theta)\) gives the minimum error defined by a cost function \(C(\Theta)\), which represents the overall goodness of fit of the interpolation. This study defines a cost function that simultaneously considers errors between the experimental variogram \(\hat{\gamma}(h)\) and modelled variogram \(\gamma(h)\), and between surface heat flow observations \(Z(u_i)\) and Kriging estimates \(\hat{Z}(u)\) (after \protect\hyperlink{ref-li2018}{Li et al., 2018}):
\begin{equation}
  \begin{aligned}
    C(\Theta) &= w_{vgrm}\ C_{vgrm}(\Theta) + w_{interp}\ C_{interp}(\Theta) \\
    &w_{vgrm} + w_{interp} = 1
  \end{aligned}
  \label{eq:cost}
\end{equation}
where \(C_{vgrm}(\Theta)\) is the normalized \gls{rmse} evaluated during variogram fitting and \(C_{interp}(\Theta)\) is the normalized \gls{rmse} evaluated during Kriging. Weighted ordinary least squares is used to evaluate \(C_{vgrm}(\Theta)\), whereas k-fold cross-validation is used to evaluate \(C_{interp}(\Theta)\). K-fold splits the dataset \(|Z(u_i)|\) into \(k\) equal intervals, removes observations from an interval, and then estimates the removed observations by fitting a variogram model to data in the remaining \(k-1\) intervals. This process is repeated over all \(k\) intervals so that the whole dataset has been cross-validated. The final expression to minimize becomes:
\begin{equation}
  \begin{aligned}
    &C(\Theta) = \\
    &\frac{w_{vgrm}}{\sigma_{vgrm}}\ \left(\frac{1}{N(h)}\ \sum_{k=1}^{N}\ w(h_k)\ [\hat{\gamma}(h_k)-\gamma(h_k;\ \Theta)]^2\right)^{1/4} + \\
    &\frac{w_{interp}}{\sigma_{interp}}\ \left(\frac{1}{M}\ \sum_{i=1}^{M}\ [Z(u_i)-\hat{Z}(u_i;\ \Theta)]^2\right)^{1/2}
  \end{aligned}
  \label{eq:costExp}
\end{equation}
where \(N(h)\) is the number of point-pairs used to evaluate the experimental variogram and \(w(h_k) = N(h)_k/h_k^2\) are weights defining the importance of the \(kth\) lag on the variogram model fit. \(Z(u_i)\) and \(\hat{Z}(u_i;\ \Theta)\) are the observed and estimated values, respectively, and m is the number of measurements in \(|Z(u_i)|\). The \glspl{rmse} are normalized by dividing by \(\sigma_{vgrm}\) and \(\sigma_{interp}\), which represent the standard deviation of the experimental variogram \(\hat{\gamma}(h)\) and surface heat flow observations \(Z(u_i)\), respectively. The weights \(w_{vgrm}\) and \(w_{interp}\) were varied between 0 and 1 to test the effects on \(C(\Theta)\). Preferred weights of \(w_{vgrm}\) = \(w_{interp}\) = 0.5 are selected to balance the effects of \(C_{vgrm}(\Theta)\) and \(C_{interp}(\Theta)\) on the cost function.



\begin{figure}[htbp]

{\centering \includegraphics[width=1\linewidth,]{assets/figs/chpt3/vgrmSummary} 

}

\caption[Summary of optimized variogram models]{Summary of optimized Kriging parameters. Cost does not correlate strongly with most Kriging parameters (solid black line with ivory 95\% confidence intervals), indicating the optimization procedure is successfully generalizable across subduction zone segments. The exception is a correlation between cost and the logarithm of the experimental variogram sill. Note that parameter values adjust from an initial value (solid white line) during the optimization procedure.}\label{fig:vgrmSummaryPlot}
\end{figure}

Minimization of \(C(\Theta)\) is achieved by non-linear constrained optimization using algorithms defined in the R package \texttt{nloptr} (\protect\hyperlink{ref-ypma2014}{Ypma, 2014}). Global search methods had limited success compared to local search methods. See \href{https://nlopt.readthedocs.io/en/latest/NLopt_Introduction/}{the official documentation} for more information on \texttt{nloptr} and available optimization algorithms. The run used to produce the visualizations in this study apply the \texttt{NLOPT\_LN\_COBYLA} method (constrained optimization by linear approximation, \protect\hyperlink{ref-powell1994}{Powell, 1994}) with 50 max iterations, leave-one-out cross-validation (k-fold \(=\) the number of observations) in the evaluated segment, and cost function weights of \(w_{vgrm}\) = \(w_{interp}\) = 0.5 (Figure \ref{fig:optTrace}). All data, code, and instructions to reproduce results in this study can be found at \url{https://github.com/buchanankerswell/kerswell_kohn_backarc}.



\begin{figure}[htbp]

{\centering \includegraphics[width=1\linewidth,]{assets/figs/chpt3/optTrace} 

}

\caption[Cost function minimization for Kriging interpolations]{Cost function minimization for Kriging interpolations. Most variogram models (panels) converge on a local optimum for most Kriging domains (lines) after 15-20 iterations. Each line represents one of thirteen subduction zone segments. See text for bound constraints and other options passed to the optimization procedure.}\label{fig:optTrace}
\end{figure}

\clearpage

\hypertarget{vgrmModelsAppendix}{%
\section{Variogram Models}\label{vgrmModelsAppendix}}

\begin{figure}
\centering
\includegraphics{assets/figs/chpt3/AlaskaAleutiansVgrms.png}
\caption[Fitted variograms for Alaska Aleutians]{Fitted variograms for Alaska Aleutians}
\end{figure}

\begin{figure}
\centering
\includegraphics{assets/figs/chpt3/AndesVgrms.png}
\caption[Fitted variograms for Andes]{Fitted variograms for Andes}
\end{figure}

\begin{figure}
\centering
\includegraphics{assets/figs/chpt3/CentralAmericaVgrms.png}
\caption[Fitted variograms for Central America]{Fitted variograms for Central America}
\end{figure}

\begin{figure}
\centering
\includegraphics{assets/figs/chpt3/KamchatkaMarianasVgrms.png}
\caption[Fitted variograms for Kamchatka Marianas]{Fitted variograms for Kamchatka Marianas}
\end{figure}

\begin{figure}
\centering
\includegraphics{assets/figs/chpt3/KyushuRyukyuVgrms.png}
\caption[Fitted variograms for Kyushu Ryukyu]{Fitted variograms for Kyushu Ryukyu}
\end{figure}

\begin{figure}
\centering
\includegraphics{assets/figs/chpt3/LesserAntillesVgrms.png}
\caption[Fitted variograms for Lesser Antilles]{Fitted variograms for Lesser Antilles}
\end{figure}

\begin{figure}
\centering
\includegraphics{assets/figs/chpt3/NPhilippinesVgrms.png}
\caption[Fitted variograms for N Philippines]{Fitted variograms for N Philippines}
\end{figure}

\begin{figure}
\centering
\includegraphics{assets/figs/chpt3/NewBritainSolomonVgrms.png}
\caption[Fitted variograms for New Britain Solomon]{Fitted variograms for New Britain Solomon}
\end{figure}

\begin{figure}
\centering
\includegraphics{assets/figs/chpt3/SPhilippinesVgrms.png}
\caption[Fitted variograms for S Philippines]{Fitted variograms for S Philippines}
\end{figure}

\begin{figure}
\centering
\includegraphics{assets/figs/chpt3/ScotiaVgrms.png}
\caption[Fitted variograms for Scotia]{Fitted variograms for Scotia}
\end{figure}

\begin{figure}
\centering
\includegraphics{assets/figs/chpt3/SumatraBandaSeaVgrms.png}
\caption[Fitted variograms for Sumatra Banda Sea]{Fitted variograms for Sumatra Banda Sea}
\end{figure}

\begin{figure}
\centering
\includegraphics{assets/figs/chpt3/TongaNewZealandVgrms.png}
\caption[Fitted variograms for Tonga New Zealand]{Fitted variograms for Tonga New Zealand}
\end{figure}

\begin{figure}
\centering
\includegraphics{assets/figs/chpt3/VanuatuVgrms.png}
\caption[Fitted variograms for Vanuatu]{Fitted variograms for Vanuatu}
\end{figure}

\clearpage

\begingroup
\renewcommand{\arraystretch}{0.5}

\begingroup\fontsize{9}{11}\selectfont

\begin{ThreePartTable}
\begin{TableNotes}
\item \uline{\textit{key}}: $n_{max}$: max point-pairs, $RMSE_K$: Kriging accuracy
\end{TableNotes}
\begin{longtable}[t]{llrrrrrrrr}
\caption{\label{tab:vgrmSummaryTableLong}Optimum variogram models and Kriging accuracy}\\
\toprule
\multicolumn{1}{c}{Segment} & \multicolumn{1}{c}{Model} & \multicolumn{1}{c}{Cutoff} & \multicolumn{1}{c}{Lags} & \multicolumn{1}{c}{Shift} & \multicolumn{1}{c}{$n_{max}$} & \multicolumn{1}{c}{Sill} & \multicolumn{1}{c}{Range} & \multicolumn{1}{c}{Cost} & \multicolumn{1}{c}{$RMSE_K$} \\
\cmidrule(l{0pt}r{0pt}){1-1} \cmidrule(l{0pt}r{0pt}){2-2} \cmidrule(l{0pt}r{0pt}){3-3} \cmidrule(l{0pt}r{0pt}){4-4} \cmidrule(l{0pt}r{0pt}){5-5} \cmidrule(l{0pt}r{0pt}){6-6} \cmidrule(l{0pt}r{0pt}){7-7} \cmidrule(l{0pt}r{0pt}){8-8} \cmidrule(l{0pt}r{0pt}){9-9} \cmidrule(l{0pt}r{0pt}){10-10}
 &  &  &  &  &  & $(mWm^{-2})^2$ & km & mW/m$^2$ & mW/m$^2$\\
\midrule
\endfirsthead
\caption[]{\label{tab:vgrmSummaryTableLong}Optimum variogram models and Kriging accuracy \textit{(continued)}}\\
\toprule
\multicolumn{1}{c}{Segment} & \multicolumn{1}{c}{Model} & \multicolumn{1}{c}{Cutoff} & \multicolumn{1}{c}{Lags} & \multicolumn{1}{c}{Shift} & \multicolumn{1}{c}{$n_{max}$} & \multicolumn{1}{c}{Sill} & \multicolumn{1}{c}{Range} & \multicolumn{1}{c}{Cost} & \multicolumn{1}{c}{$RMSE_K$} \\
\cmidrule(l{0pt}r{0pt}){1-1} \cmidrule(l{0pt}r{0pt}){2-2} \cmidrule(l{0pt}r{0pt}){3-3} \cmidrule(l{0pt}r{0pt}){4-4} \cmidrule(l{0pt}r{0pt}){5-5} \cmidrule(l{0pt}r{0pt}){6-6} \cmidrule(l{0pt}r{0pt}){7-7} \cmidrule(l{0pt}r{0pt}){8-8} \cmidrule(l{0pt}r{0pt}){9-9} \cmidrule(l{0pt}r{0pt}){10-10}
 &  &  &  &  &  & $(mWm^{-2})^2$ & km & mW/m$^2$ & mW/m$^2$\\
\midrule
\endhead

\endfoot
\bottomrule
\insertTableNotes
\endlastfoot
\cellcolor{gray!6}{Alaska Aleutians} & \cellcolor{gray!6}{Bes} & \cellcolor{gray!6}{1.0} & \cellcolor{gray!6}{16.3} & \cellcolor{gray!6}{1.0} & \cellcolor{gray!6}{8.4} & \cellcolor{gray!6}{841} & \cellcolor{gray!6}{77} & \cellcolor{gray!6}{0.498} & \cellcolor{gray!6}{74.6}\\
Alaska Aleutians & Exp & 1.0 & 15.0 & 1.9 & 5.0 & 837 & 111 & 0.665 & 14.2\\
\cellcolor{gray!6}{Alaska Aleutians} & \cellcolor{gray!6}{Lin} & \cellcolor{gray!6}{3.0} & \cellcolor{gray!6}{20.6} & \cellcolor{gray!6}{3.4} & \cellcolor{gray!6}{8.0} & \cellcolor{gray!6}{790} & \cellcolor{gray!6}{243} & \cellcolor{gray!6}{0.621} & \cellcolor{gray!6}{15.1}\\
Alaska Aleutians & Sph & 2.4 & 19.1 & 5.3 & 6.4 & 818 & 734 & 0.629 & 14.5\\
\cellcolor{gray!6}{Andes} & \cellcolor{gray!6}{Bes} & \cellcolor{gray!6}{8.7} & \cellcolor{gray!6}{26.5} & \cellcolor{gray!6}{2.1} & \cellcolor{gray!6}{6.2} & \cellcolor{gray!6}{2566} & \cellcolor{gray!6}{5} & \cellcolor{gray!6}{0.312} & \cellcolor{gray!6}{38.0}\\
Andes & Exp & 1.6 & 20.8 & 8.5 & 12.4 & 4631 & 165 & 0.294 & 34.9\\
\cellcolor{gray!6}{Andes} & \cellcolor{gray!6}{Lin} & \cellcolor{gray!6}{3.6} & \cellcolor{gray!6}{24.8} & \cellcolor{gray!6}{5.0} & \cellcolor{gray!6}{11.8} & \cellcolor{gray!6}{6084} & \cellcolor{gray!6}{933} & \cellcolor{gray!6}{0.297} & \cellcolor{gray!6}{38.7}\\
Andes & Sph & 2.8 & 18.2 & 5.3 & 11.6 & 5457 & 558 & 0.296 & 35.0\\
\cellcolor{gray!6}{Central America} & \cellcolor{gray!6}{Bes} & \cellcolor{gray!6}{5.3} & \cellcolor{gray!6}{30.4} & \cellcolor{gray!6}{1.0} & \cellcolor{gray!6}{11.9} & \cellcolor{gray!6}{2085} & \cellcolor{gray!6}{4} & \cellcolor{gray!6}{0.267} & \cellcolor{gray!6}{40.4}\\
Central America & Exp & 4.9 & 21.2 & 3.9 & 12.4 & 4683 & 265 & 0.248 & 33.4\\
\cellcolor{gray!6}{Central America} & \cellcolor{gray!6}{Lin} & \cellcolor{gray!6}{5.1} & \cellcolor{gray!6}{27.1} & \cellcolor{gray!6}{1.0} & \cellcolor{gray!6}{7.7} & \cellcolor{gray!6}{2218} & \cellcolor{gray!6}{14} & \cellcolor{gray!6}{0.253} & \cellcolor{gray!6}{35.4}\\
Central America & Sph & 6.2 & 27.2 & 4.0 & 13.1 & 2926 & 271 & 0.251 & 33.0\\
\cellcolor{gray!6}{Kamchatka Marianas} & \cellcolor{gray!6}{Bes} & \cellcolor{gray!6}{3.9} & \cellcolor{gray!6}{25.1} & \cellcolor{gray!6}{1.0} & \cellcolor{gray!6}{11.0} & \cellcolor{gray!6}{1713} & \cellcolor{gray!6}{10} & \cellcolor{gray!6}{0.449} & \cellcolor{gray!6}{36.4}\\
Kamchatka Marianas & Exp & 1.0 & 18.9 & 1.0 & 8.4 & 1783 & 64 & 0.428 & 30.5\\
\cellcolor{gray!6}{Kamchatka Marianas} & \cellcolor{gray!6}{Lin} & \cellcolor{gray!6}{1.0} & \cellcolor{gray!6}{22.2} & \cellcolor{gray!6}{6.0} & \cellcolor{gray!6}{6.4} & \cellcolor{gray!6}{1797} & \cellcolor{gray!6}{1528} & \cellcolor{gray!6}{0.424} & \cellcolor{gray!6}{31.2}\\
Kamchatka Marianas & Sph & 1.7 & 18.5 & 7.5 & 6.9 & 1787 & 1355 & 0.424 & 31.2\\
\cellcolor{gray!6}{Kyushu Ryukyu} & \cellcolor{gray!6}{Bes} & \cellcolor{gray!6}{2.4} & \cellcolor{gray!6}{20.4} & \cellcolor{gray!6}{1.8} & \cellcolor{gray!6}{5.9} & \cellcolor{gray!6}{1843} & \cellcolor{gray!6}{3} & \cellcolor{gray!6}{0.491} & \cellcolor{gray!6}{33.4}\\
Kyushu Ryukyu & Exp & 2.4 & 21.4 & 5.8 & 7.8 & 1898 & 34 & 0.487 & 33.3\\
\cellcolor{gray!6}{Kyushu Ryukyu} & \cellcolor{gray!6}{Lin} & \cellcolor{gray!6}{3.2} & \cellcolor{gray!6}{19.8} & \cellcolor{gray!6}{3.3} & \cellcolor{gray!6}{8.3} & \cellcolor{gray!6}{1898} & \cellcolor{gray!6}{183} & \cellcolor{gray!6}{0.487} & \cellcolor{gray!6}{37.8}\\
Kyushu Ryukyu & Sph & 3.0 & 20.0 & 3.3 & 8.1 & 1903 & 216 & 0.488 & 34.2\\
\cellcolor{gray!6}{Lesser Antilles} & \cellcolor{gray!6}{Bes} & \cellcolor{gray!6}{2.0} & \cellcolor{gray!6}{18.3} & \cellcolor{gray!6}{2.5} & \cellcolor{gray!6}{7.2} & \cellcolor{gray!6}{554} & \cellcolor{gray!6}{13} & \cellcolor{gray!6}{0.329} & \cellcolor{gray!6}{20.9}\\
Lesser Antilles & Exp & 1.5 & 25.1 & 1.4 & 10.5 & 657 & 68 & 0.309 & 12.4\\
\cellcolor{gray!6}{Lesser Antilles} & \cellcolor{gray!6}{Lin} & \cellcolor{gray!6}{1.5} & \cellcolor{gray!6}{24.2} & \cellcolor{gray!6}{1.1} & \cellcolor{gray!6}{11.0} & \cellcolor{gray!6}{653} & \cellcolor{gray!6}{77} & \cellcolor{gray!6}{0.297} & \cellcolor{gray!6}{13.3}\\
Lesser Antilles & Sph & 2.7 & 24.2 & 3.3 & 10.2 & 582 & 122 & 0.306 & 12.6\\
\cellcolor{gray!6}{N Philippines} & \cellcolor{gray!6}{Bes} & \cellcolor{gray!6}{1.4} & \cellcolor{gray!6}{18.3} & \cellcolor{gray!6}{1.0} & \cellcolor{gray!6}{7.9} & \cellcolor{gray!6}{1258} & \cellcolor{gray!6}{19} & \cellcolor{gray!6}{0.548} & \cellcolor{gray!6}{32.0}\\
N Philippines & Exp & 2.1 & 15.0 & 1.3 & 5.9 & 1266 & 25 & 0.567 & 26.7\\
\cellcolor{gray!6}{N Philippines} & \cellcolor{gray!6}{Lin} & \cellcolor{gray!6}{3.0} & \cellcolor{gray!6}{20.0} & \cellcolor{gray!6}{1.0} & \cellcolor{gray!6}{8.6} & \cellcolor{gray!6}{1310} & \cellcolor{gray!6}{40} & \cellcolor{gray!6}{0.552} & \cellcolor{gray!6}{27.3}\\
N Philippines & Sph & 1.0 & 17.8 & 4.2 & 8.6 & 946 & 516 & 0.550 & 27.3\\
\cellcolor{gray!6}{New Britain Solomon} & \cellcolor{gray!6}{Bes} & \cellcolor{gray!6}{3.9} & \cellcolor{gray!6}{20.6} & \cellcolor{gray!6}{3.5} & \cellcolor{gray!6}{10.2} & \cellcolor{gray!6}{744} & \cellcolor{gray!6}{61} & \cellcolor{gray!6}{0.694} & \cellcolor{gray!6}{6.8}\\
New Britain Solomon & Exp & 1.6 & 16.1 & 1.0 & 7.4 & 723 & 68 & 0.732 & 8.0\\
\cellcolor{gray!6}{New Britain Solomon} & \cellcolor{gray!6}{Lin} & \cellcolor{gray!6}{2.0} & \cellcolor{gray!6}{20.2} & \cellcolor{gray!6}{5.1} & \cellcolor{gray!6}{10.2} & \cellcolor{gray!6}{693} & \cellcolor{gray!6}{228} & \cellcolor{gray!6}{0.609} & \cellcolor{gray!6}{28.2}\\
New Britain Solomon & Sph & 1.2 & 18.7 & 3.6 & 10.1 & 694 & 320 & 0.657 & 7.0\\
\cellcolor{gray!6}{S Philippines} & \cellcolor{gray!6}{Bes} & \cellcolor{gray!6}{4.1} & \cellcolor{gray!6}{16.5} & \cellcolor{gray!6}{1.1} & \cellcolor{gray!6}{5.3} & \cellcolor{gray!6}{1086} & \cellcolor{gray!6}{20} & \cellcolor{gray!6}{0.465} & \cellcolor{gray!6}{33.9}\\
S Philippines & Exp & 1.3 & 19.0 & 2.0 & 5.7 & 1227 & 271 & 0.466 & 21.9\\
\cellcolor{gray!6}{S Philippines} & \cellcolor{gray!6}{Lin} & \cellcolor{gray!6}{3.2} & \cellcolor{gray!6}{29.0} & \cellcolor{gray!6}{1.0} & \cellcolor{gray!6}{5.0} & \cellcolor{gray!6}{1014} & \cellcolor{gray!6}{40} & \cellcolor{gray!6}{0.464} & \cellcolor{gray!6}{22.9}\\
S Philippines & Sph & 1.3 & 28.2 & 8.0 & 5.3 & 1056 & 578 & 0.466 & 21.9\\
\cellcolor{gray!6}{Scotia} & \cellcolor{gray!6}{Bes} & \cellcolor{gray!6}{3.1} & \cellcolor{gray!6}{20.7} & \cellcolor{gray!6}{3.2} & \cellcolor{gray!6}{10.0} & \cellcolor{gray!6}{2120} & \cellcolor{gray!6}{195} & \cellcolor{gray!6}{0.247} & \cellcolor{gray!6}{}\\
Scotia & Exp & 2.6 & 15.6 & 4.4 & 7.9 & 4503 & 1148 & 0.230 & 10.9\\
\cellcolor{gray!6}{Scotia} & \cellcolor{gray!6}{Lin} & \cellcolor{gray!6}{3.0} & \cellcolor{gray!6}{23.8} & \cellcolor{gray!6}{3.2} & \cellcolor{gray!6}{8.0} & \cellcolor{gray!6}{1876} & \cellcolor{gray!6}{563} & \cellcolor{gray!6}{0.243} & \cellcolor{gray!6}{10.9}\\
Scotia & Sph & 2.7 & 20.8 & 4.8 & 7.9 & 3655 & 1766 & 0.228 & 10.9\\
\cellcolor{gray!6}{Sumatra Banda Sea} & \cellcolor{gray!6}{Bes} & \cellcolor{gray!6}{3.2} & \cellcolor{gray!6}{20.1} & \cellcolor{gray!6}{1.2} & \cellcolor{gray!6}{10.3} & \cellcolor{gray!6}{1604} & \cellcolor{gray!6}{63} & \cellcolor{gray!6}{0.307} & \cellcolor{gray!6}{}\\
Sumatra Banda Sea & Exp & 3.1 & 24.1 & 1.0 & 10.6 & 2128 & 245 & 0.267 & 19.4\\
\cellcolor{gray!6}{Sumatra Banda Sea} & \cellcolor{gray!6}{Lin} & \cellcolor{gray!6}{6.6} & \cellcolor{gray!6}{23.0} & \cellcolor{gray!6}{5.8} & \cellcolor{gray!6}{12.1} & \cellcolor{gray!6}{4199} & \cellcolor{gray!6}{1547} & \cellcolor{gray!6}{0.266} & \cellcolor{gray!6}{20.4}\\
Sumatra Banda Sea & Sph & 6.6 & 21.0 & 5.1 & 12.8 & 10598 & 5850 & 0.266 & 20.4\\
\cellcolor{gray!6}{Tonga New Zealand} & \cellcolor{gray!6}{Bes} & \cellcolor{gray!6}{5.6} & \cellcolor{gray!6}{17.2} & \cellcolor{gray!6}{7.0} & \cellcolor{gray!6}{7.5} & \cellcolor{gray!6}{1566} & \cellcolor{gray!6}{186} & \cellcolor{gray!6}{0.531} & \cellcolor{gray!6}{40.7}\\
Tonga New Zealand & Exp & 1.0 & 18.9 & 8.6 & 6.3 & 2072 & 1657 & 0.533 & 20.8\\
\cellcolor{gray!6}{Tonga New Zealand} & \cellcolor{gray!6}{Lin} & \cellcolor{gray!6}{3.7} & \cellcolor{gray!6}{24.9} & \cellcolor{gray!6}{3.6} & \cellcolor{gray!6}{10.1} & \cellcolor{gray!6}{1293} & \cellcolor{gray!6}{321} & \cellcolor{gray!6}{0.521} & \cellcolor{gray!6}{23.8}\\
Tonga New Zealand & Sph & 4.9 & 23.7 & 5.5 & 7.4 & 1307 & 436 & 0.534 & 19.9\\
\cellcolor{gray!6}{Vanuatu} & \cellcolor{gray!6}{Bes} & \cellcolor{gray!6}{3.0} & \cellcolor{gray!6}{20.5} & \cellcolor{gray!6}{3.6} & \cellcolor{gray!6}{10.3} & \cellcolor{gray!6}{3101} & \cellcolor{gray!6}{113} & \cellcolor{gray!6}{0.518} & \cellcolor{gray!6}{59.5}\\
Vanuatu & Exp & 3.0 & 19.8 & 3.4 & 8.3 & 3188 & 197 & 0.549 & 17.8\\
\cellcolor{gray!6}{Vanuatu} & \cellcolor{gray!6}{Lin} & \cellcolor{gray!6}{1.2} & \cellcolor{gray!6}{20.4} & \cellcolor{gray!6}{2.6} & \cellcolor{gray!6}{10.8} & \cellcolor{gray!6}{2918} & \cellcolor{gray!6}{286} & \cellcolor{gray!6}{0.517} & \cellcolor{gray!6}{54.6}\\
Vanuatu & Sph & 1.4 & 17.9 & 3.3 & 7.3 & 2970 & 468 & 0.537 & 17.5\\*
\end{longtable}
\end{ThreePartTable}
\endgroup{}

\endgroup

\clearpage

\hypertarget{thermoglobe-summary}{%
\section{ThermoGlobe Summary}\label{thermoglobe-summary}}



\begin{figure}[htbp]

{\centering \includegraphics[width=1\linewidth,]{assets/figs/chpt3/hfSummary} 

}

\caption[ThermoGlobe observations near subduction zones]{Distribution of ThermoGlobe observations from Lucazeau (\protect\hyperlink{ref-lucazeau2019}{2019}) cropped within 1000 km-radius buffers around 13 active subduction zone segments. Heat flow distributions are centered between 41 and 108 mW/m\(^2\), generally right-skewed, and irregularly distributed. Skewness reflects near-surface perturbations from geothermal systems and tectonic regions with high thermal activity while irregularity reflects complex heat exchange acting across multiple spatial scales from 10\(^-1\) to 10\(^3\) km.}\label{fig:hfSummaryPlot}
\end{figure}

\begin{table}

\caption{\label{tab:hfSummaryTable}ThermoGlobe heat flow summary}
\centering
\begin{threeparttable}
\begin{tabular}[t]{lrrrrrrr}
\toprule
Segment & n & Min & Max & Median & IQR & Mean & $\sigma$\\
\midrule
\cellcolor{gray!6}{Alaska Aleutians} & \cellcolor{gray!6}{290} & \cellcolor{gray!6}{6} & \cellcolor{gray!6}{196} & \cellcolor{gray!6}{66} & \cellcolor{gray!6}{27} & \cellcolor{gray!6}{71} & \cellcolor{gray!6}{28}\\
Andes & 1398 & 7 & 250 & 108 & 120 & 119 & 66\\
\cellcolor{gray!6}{Central America} & \cellcolor{gray!6}{1441} & \cellcolor{gray!6}{8} & \cellcolor{gray!6}{250} & \cellcolor{gray!6}{89} & \cellcolor{gray!6}{123} & \cellcolor{gray!6}{110} & \cellcolor{gray!6}{67}\\
Kamchatka Marianas & 2268 & 1 & 248 & 78 & 51 & 83 & 42\\
\cellcolor{gray!6}{Kyushu Ryukyu} & \cellcolor{gray!6}{1895} & \cellcolor{gray!6}{3} & \cellcolor{gray!6}{250} & \cellcolor{gray!6}{76} & \cellcolor{gray!6}{42} & \cellcolor{gray!6}{84} & \cellcolor{gray!6}{42}\\
Lesser Antilles & 3008 & 13 & 242 & 41 & 8 & 46 & 18\\
\cellcolor{gray!6}{N Philippines} & \cellcolor{gray!6}{569} & \cellcolor{gray!6}{3} & \cellcolor{gray!6}{231} & \cellcolor{gray!6}{71} & \cellcolor{gray!6}{26} & \cellcolor{gray!6}{75} & \cellcolor{gray!6}{33}\\
New Britain Solomon & 100 & 3 & 143 & 58 & 34 & 61 & 26\\
\cellcolor{gray!6}{S Philippines} & \cellcolor{gray!6}{458} & \cellcolor{gray!6}{1} & \cellcolor{gray!6}{224} & \cellcolor{gray!6}{71} & \cellcolor{gray!6}{32} & \cellcolor{gray!6}{74} & \cellcolor{gray!6}{33}\\
Scotia & 25 & 13 & 145 & 81 & 62 & 79 & 43\\
\cellcolor{gray!6}{Sumatra Banda Sea} & \cellcolor{gray!6}{1415} & \cellcolor{gray!6}{1} & \cellcolor{gray!6}{247} & \cellcolor{gray!6}{59} & \cellcolor{gray!6}{63} & \cellcolor{gray!6}{67} & \cellcolor{gray!6}{42}\\
Tonga New Zealand & 356 & 5 & 218 & 49 & 41 & 60 & 37\\
\cellcolor{gray!6}{Vanuatu} & \cellcolor{gray!6}{137} & \cellcolor{gray!6}{1} & \cellcolor{gray!6}{223} & \cellcolor{gray!6}{61} & \cellcolor{gray!6}{62} & \cellcolor{gray!6}{80} & \cellcolor{gray!6}{52}\\
\bottomrule
\end{tabular}
\begin{tablenotes}
\item \uline{\textit{key}}: n: [\# of observations], all other units are in mW/m$^2$
\item \uline{\textit{note}}: ThermoGlobe data are filtered for quality, restricted to [0, 250) mW/m$^2$, and cropped within 1000 km-radius buffers of segment boundaries
\end{tablenotes}
\end{threeparttable}
\end{table}

\clearpage

\hypertarget{interpDiffAppendix}{%
\section{Comparing Similarity and Kriging Interpolations}\label{interpDiffAppendix}}



\begin{figure}[htbp]

{\centering \includegraphics[width=1\linewidth,]{assets/figs/chpt3/interpDiffSummary} 

}

\caption[Differences between Similarity and Kriging interpolations]{Differences between Similarity and Kriging interpolations by segment, computed as Similarity-Kriging. Differences are centered near zero with medians ranging from -1 to 14 mW/m\(^2\), but broadly distributed with \DIFdelbeginFL \DIFdelFL{interquartile ranges }\DIFdelendFL \DIFaddbeginFL \glspl{iqr} \DIFaddendFL from 15 to 50 mW/m\(^2\) and some long tails extending from -1000 to 205 mW/m\(^2\). Positive medians and right skew indicate a general tendency towards higher surface heat flow predictions by Similarity compared to Kriging. The broadest distributions (Andes and Central America) reflect less subtle differences between methods. Distributions are colored by quartiles (25\%, 50\%, 75\%). Similarity interpolation from Lucazeau (\protect\hyperlink{ref-lucazeau2019}{2019}).}\label{fig:diffSummaryPlot}
\end{figure}



\begin{figure}[htbp]

{\centering \includegraphics[width=1\linewidth,]{assets/figs/chpt3/interpSigmaDiffSummary} 

}

\caption[Differences between Similarity and Kriging uncertainties]{Summary of differences between Similarity and Kriging uncertainties computed as Similarity-Kriging. Differences are centered at slightly negative values with median differences ranging from -23 to -3 mW/m\(^2\), and relatively narrowly distributed with \DIFdelbeginFL \DIFdelFL{interquartile ranges }\DIFdelendFL \DIFaddbeginFL \glspl{iqr} \DIFaddendFL from 4 to 13 mW/m\(^2\) and some long tails extending from -50 to 70 mW/m\(^2\). Negative medians indicate greater uncertainties by Kriging compared to Similarity. Distributions are colored by quantiles (25\%, 50\%, 75\%). Similarity data from Lucazeau (\protect\hyperlink{ref-lucazeau2019}{2019}). Refer to Figure \ref{fig:diffSummaryPlot} for estimate differences.}\label{fig:sigmaDiffSummaryPlot}
\end{figure}

\begin{table}

\caption{\label{tab:diffSummaryTable}Summary of Similarity-Kriging prediction differences}
\centering
\begin{threeparttable}
\begin{tabular}[t]{lrrrrrr}
\toprule
Segment & Min & Max & Median & IQR & Mean & $\sigma$\\
\midrule
\cellcolor{gray!6}{Alaska Aleutians} & \cellcolor{gray!6}{-1000} & \cellcolor{gray!6}{126} & \cellcolor{gray!6}{2} & \cellcolor{gray!6}{22} & \cellcolor{gray!6}{-1} & \cellcolor{gray!6}{43}\\
Andes & -124 & 169 & 0 & 41 & 0 & 33\\
\cellcolor{gray!6}{Central America} & \cellcolor{gray!6}{-128} & \cellcolor{gray!6}{205} & \cellcolor{gray!6}{12} & \cellcolor{gray!6}{50} & \cellcolor{gray!6}{20} & \cellcolor{gray!6}{42}\\
Kamchatka Marianas & -144 & 178 & 4 & 18 & 6 & 23\\
\cellcolor{gray!6}{Kyushu Ryukyu} & \cellcolor{gray!6}{-123} & \cellcolor{gray!6}{167} & \cellcolor{gray!6}{4} & \cellcolor{gray!6}{21} & \cellcolor{gray!6}{6} & \cellcolor{gray!6}{23}\\
Lesser Antilles & -129 & 106 & 4 & 15 & 2 & 21\\
\cellcolor{gray!6}{N Philippines} & \cellcolor{gray!6}{-144} & \cellcolor{gray!6}{141} & \cellcolor{gray!6}{8} & \cellcolor{gray!6}{25} & \cellcolor{gray!6}{11} & \cellcolor{gray!6}{22}\\
New Britain Solomon & -70 & 169 & 7 & 21 & 10 & 22\\
\cellcolor{gray!6}{S Philippines} & \cellcolor{gray!6}{-79} & \cellcolor{gray!6}{189} & \cellcolor{gray!6}{6} & \cellcolor{gray!6}{25} & \cellcolor{gray!6}{9} & \cellcolor{gray!6}{23}\\
Scotia & -126 & 199 & 3 & 40 & 4 & 34\\
\cellcolor{gray!6}{Sumatra Banda Sea} & \cellcolor{gray!6}{-153} & \cellcolor{gray!6}{144} & \cellcolor{gray!6}{3} & \cellcolor{gray!6}{21} & \cellcolor{gray!6}{2} & \cellcolor{gray!6}{22}\\
Tonga New Zealand & -142 & 188 & -1 & 24 & 0 & 27\\
\cellcolor{gray!6}{Vanuatu} & \cellcolor{gray!6}{-147} & \cellcolor{gray!6}{204} & \cellcolor{gray!6}{14} & \cellcolor{gray!6}{31} & \cellcolor{gray!6}{13} & \cellcolor{gray!6}{34}\\
\bottomrule
\end{tabular}
\begin{tablenotes}
\item \uline{\textit{note}}: All units are mW/m$^2$
\end{tablenotes}
\end{threeparttable}
\end{table}

\begin{table}

\caption{\label{tab:sigmaDiffSummaryTable}Summary of Similarity-Kriging uncertainty differences}
\centering
\resizebox{\linewidth}{!}{
\begin{threeparttable}
\begin{tabular}[t]{llrrrrrr}
\toprule
Segment & Model & Min & Max & Median & IQR & Mean & $\sigma$\\
\midrule
\cellcolor{gray!6}{Alaska Aleutians} & \cellcolor{gray!6}{Bes} & \cellcolor{gray!6}{-24} & \cellcolor{gray!6}{45} & \cellcolor{gray!6}{-3} & \cellcolor{gray!6}{7} & \cellcolor{gray!6}{-2} & \cellcolor{gray!6}{8}\\
Andes & Exp & -46 & 46 & -23 & 12 & -22 & 11\\
\cellcolor{gray!6}{Central America} & \cellcolor{gray!6}{Exp} & \cellcolor{gray!6}{-50} & \cellcolor{gray!6}{57} & \cellcolor{gray!6}{-20} & \cellcolor{gray!6}{12} & \cellcolor{gray!6}{-20} & \cellcolor{gray!6}{13}\\
Kamchatka Marianas & Sph & -21 & 70 & -3 & 6 & -1 & 8\\
\cellcolor{gray!6}{Kyushu Ryukyu} & \cellcolor{gray!6}{Lin} & \cellcolor{gray!6}{-43} & \cellcolor{gray!6}{33} & \cellcolor{gray!6}{-11} & \cellcolor{gray!6}{7} & \cellcolor{gray!6}{-10} & \cellcolor{gray!6}{7}\\
Lesser Antilles & Lin & -27 & 18 & -12 & 8 & -12 & 6\\
\cellcolor{gray!6}{N Philippines} & \cellcolor{gray!6}{Bes} & \cellcolor{gray!6}{-38} & \cellcolor{gray!6}{29} & \cellcolor{gray!6}{-21} & \cellcolor{gray!6}{13} & \cellcolor{gray!6}{-21} & \cellcolor{gray!6}{10}\\
New Britain Solomon & Lin & -12 & 19 & -7 & 5 & -4 & 7\\
\cellcolor{gray!6}{S Philippines} & \cellcolor{gray!6}{Lin} & \cellcolor{gray!6}{-38} & \cellcolor{gray!6}{0} & \cellcolor{gray!6}{-23} & \cellcolor{gray!6}{11} & \cellcolor{gray!6}{-23} & \cellcolor{gray!6}{7}\\
Scotia & Sph & -11 & 3 & -7 & 4 & -6 & 4\\
\cellcolor{gray!6}{Sumatra Banda Sea} & \cellcolor{gray!6}{Sph} & \cellcolor{gray!6}{-36} & \cellcolor{gray!6}{40} & \cellcolor{gray!6}{-4} & \cellcolor{gray!6}{6} & \cellcolor{gray!6}{-2} & \cellcolor{gray!6}{8}\\
Tonga New Zealand & Lin & -15 & 59 & -5 & 7 & -1 & 12\\
\cellcolor{gray!6}{Vanuatu} & \cellcolor{gray!6}{Lin} & \cellcolor{gray!6}{-24} & \cellcolor{gray!6}{36} & \cellcolor{gray!6}{-11} & \cellcolor{gray!6}{10} & \cellcolor{gray!6}{-7} & \cellcolor{gray!6}{13}\\
\bottomrule
\end{tabular}
\begin{tablenotes}
\item \uline{\textit{note}}: Showing optimal Kriging models only, difference is calculated as Similarity-Kriging
\item \uline{\textit{key}}: Cost: [mW/m$^2$], n: number of target locations (grid size), all other units are mW/m$^2$
\end{tablenotes}
\end{threeparttable}}
\end{table}

\clearpage

\begin{figure}[htbp]

{\centering \includegraphics[width=1\linewidth,]{assets/figs/chpt3/AlaskaAleutiansDiffComp} 

}

\caption[Similarity and Kriging interpolations for Alaska Aleutians]{Similarity (a) and Kriging (b) interpolations for Alaska Aleutians. Refer to the main text for explanation of panels and colors.}\label{fig:alaskaAleutiansDiff}
\end{figure}

\begin{figure}[htbp]

{\centering \includegraphics[width=1\linewidth,]{assets/figs/chpt3/AndesDiffComp} 

}

\caption[Similarity and Kriging interpolations for Andes]{Similarity (a) and Kriging (b) interpolations for Andes. Refer to the main text for explanation of panels and colors.}\label{fig:andesDiff}
\end{figure}

\begin{figure}[htbp]

{\centering \includegraphics[width=1\linewidth,]{assets/figs/chpt3/KamchatkaMarianasDiffComp} 

}

\caption[Similarity and Kriging interpolations for Kamchatka Marianas]{Similarity (a) and Kriging (b) interpolations for Kamchatka Marianas. Refer to the main text for explanation of panels and colors.}\label{fig:kamchatkaMarianasDiff}
\end{figure}

\begin{figure}[htbp]

{\centering \includegraphics[width=1\linewidth,]{assets/figs/chpt3/LesserAntillesDiffComp} 

}

\caption[Similarity and Kriging interpolations for Lesser Antilles]{Similarity (a) and Kriging (b) interpolations for Lesser Antilles. Refer to the main text for explanation of panels and colors.}\label{fig:lesserAntillesDiff}
\end{figure}

\begin{figure}[htbp]

{\centering \includegraphics[width=1\linewidth,]{assets/figs/chpt3/NPhilippinesDiffComp} 

}

\caption[Similarity and Kriging interpolations for N Philippines]{Similarity (a) and Kriging (b) interpolations for N Philippines. Refer to the main text for explanation of panels and colors.}\label{fig:nPhilippinesDiff}
\end{figure}

\begin{figure}[htbp]

{\centering \includegraphics[width=1\linewidth,]{assets/figs/chpt3/NewBritainSolomonDiffComp} 

}

\caption[Similarity and Kriging interpolations for New Britain Solomon]{Similarity (a) and Kriging (b) interpolations for New Britain Solomon. Refer to the main text for explanation of panels and colors.}\label{fig:newBritainSolomonDiff}
\end{figure}

\begin{figure}[htbp]

{\centering \includegraphics[width=1\linewidth,]{assets/figs/chpt3/SPhilippinesDiffComp} 

}

\caption[Similarity and Kriging interpolations for S Philippines]{Similarity (a) and Kriging (b) interpolations for S Philippines. Refer to the main text for explanation of panels and colors.}\label{fig:sPhilippinesDiff}
\end{figure}

\begin{figure}[htbp]

{\centering \includegraphics[width=1\linewidth,]{assets/figs/chpt3/SumatraBandaSeaDiffComp} 

}

\caption[Similarity and Kriging interpolations for Sumatra Banda Sea]{Similarity (a) and Kriging (b) interpolations for Sumatra Banda Sea. Refer to the main text for explanation of panels and colors.}\label{fig:sumatraBandaSeaDiff}
\end{figure}

\begin{figure}[htbp]

{\centering \includegraphics[width=1\linewidth,]{assets/figs/chpt3/TongaNewZealandDiffComp} 

}

\caption[Similarity and Kriging interpolations for Tonga New Zealand]{Similarity (a) and Kriging (b) interpolations for Tonga New Zealand. Refer to the main text for explanation of panels and colors.}\label{fig:tongaNewZealandDiff}
\end{figure}

\clearpage

\hypertarget{lateralDiffAppendix}{%
\section{Upper-plate Surface Heat Flow}\label{lateralDiffAppendix}}

\begin{figure}[htbp]

{\centering \includegraphics[width=1\linewidth,]{assets/figs/chpt3/AlaskaAleutiansUpperPlate} 

}

\caption[Alaska Aleutians upper-plate sectors]{Surface heat flow profiles for Alaska Aleutians upper-plate sectors. Refer to the main text for explanation of panels and colors.}\label{fig:alaskaAleutiansUpper}
\end{figure}

\begin{figure}[htbp]

{\centering \includegraphics[width=1\linewidth,]{assets/figs/chpt3/AndesUpperPlate} 

}

\caption[Andes upper-plate sectors]{Surface heat flow profiles for Andes upper-plate sectors. Refer to the main text for explanation of panels and colors.}\label{fig:andesUpper}
\end{figure}

\begin{figure}[htbp]

{\centering \includegraphics[width=1\linewidth,]{assets/figs/chpt3/CentralAmericaUpperPlate} 

}

\caption[Central America upper-plate sectors]{Surface heat flow profiles for Central America upper-plate sectors. Refer to the main text for explanation of panels and colors.}\label{fig:centralAmericaUpper}
\end{figure}

\begin{figure}[htbp]

{\centering \includegraphics[width=1\linewidth,]{assets/figs/chpt3/kamchatkaMarianasUpperPlate} 

}

\caption[Kamchatka Marianas upper-plate sectors]{Surface heat flow profiles for Kamchatka Marianas upper-plate sectors. Refer to the main text for explanation of panels and colors.}\label{fig:kamchatkaMarianasUpper}
\end{figure}

\begin{figure}[htbp]

{\centering \includegraphics[width=1\linewidth,]{assets/figs/chpt3/LesserAntillesUpperPlate} 

}

\caption[Lesser Antilles upper-plate sectors]{Surface heat flow profiles for Lesser Antilles upper-plate sectors. Refer to the main text for explanation of panels and colors.}\label{fig:lesserAntillesUpper}
\end{figure}

\begin{figure}[htbp]

{\centering \includegraphics[width=1\linewidth,]{assets/figs/chpt3/NPhilippinesUpperPlate} 

}

\caption[N Philippines upper-plate sectors]{Surface heat flow profiles for N Philippines upper-plate sectors. Refer to the main text for explanation of panels and colors.}\label{fig:nPhilippinesUpper}
\end{figure}

\begin{figure}[htbp]

{\centering \includegraphics[width=1\linewidth,]{assets/figs/chpt3/SPhilippinesUpperPlate} 

}

\caption[S Philippines upper-plate sectors]{Surface heat flow profiles for S Philippines upper-plate sectors. Refer to the main text for explanation of panels and colors.}\label{fig:sPhilippinesUpper}
\end{figure}

\begin{figure}[htbp]

{\centering \includegraphics[width=1\linewidth,]{assets/figs/chpt3/ScotiaUpperPlate} 

}

\caption[Scotia upper-plate sectors]{Surface heat flow profiles for Scotia upper-plate sectors. Refer to the main text for explanation of panels and colors.}\label{fig:scotiaUpper}
\end{figure}

\begin{figure}[htbp]

{\centering \includegraphics[width=1\linewidth,]{assets/figs/chpt3/TongaNewZealandUpperPlate} 

}

\caption[Tonga New Zealand upper-plate sectors]{Surface heat flow profiles for Tonga New Zealand upper-plate sectors. Refer to the main text for explanation of panels and colors.}\label{fig:tongaNewZealandUpper}
\end{figure}

\begin{figure}[htbp]

{\centering \includegraphics[width=1\linewidth,]{assets/figs/chpt3/VanuatuUpperPlate} 

}

\caption[Vanuatu upper-plate sectors]{Surface heat flow profiles for Vanuatu upper-plate sectors. Refer to the main text for explanation of panels and colors.}\label{fig:vanuatuUpper}
\end{figure}

\clearpage

\begingroup
\renewcommand{\arraystretch}{0.5}

\begingroup\fontsize{10}{12}\selectfont

\begin{ThreePartTable}
\begin{TableNotes}
\item \uline{\textit{note}}: Similarity and Kriging prediction counts are the same. Surface heat flow units are mW/m$^2$.
\end{TableNotes}
\begin{longtable}[t]{lrrrrrrrrr}
\caption{\label{tab:sectorSummaryTable}Summary of upper-plate surface heat flow}\\
\toprule
\multicolumn{2}{c}{ } & \multicolumn{3}{c}{ThermoGlobe} & \multicolumn{3}{c}{Similarity} & \multicolumn{2}{c}{Kriging} \\
\cmidrule(l{3pt}r{3pt}){3-5} \cmidrule(l{3pt}r{3pt}){6-8} \cmidrule(l{3pt}r{3pt}){9-10}
Segment & Sector & n & Median & IQR & n & Median & IQR & Median & IQR\\
\midrule
\endfirsthead
\caption[]{\label{tab:sectorSummaryTable}Summary of upper-plate surface heat flow \textit{(continued)}}\\
\toprule
\multicolumn{2}{c}{ } & \multicolumn{3}{c}{ThermoGlobe} & \multicolumn{3}{c}{Similarity} & \multicolumn{2}{c}{Kriging} \\
\cmidrule(l{3pt}r{3pt}){3-5} \cmidrule(l{3pt}r{3pt}){6-8} \cmidrule(l{3pt}r{3pt}){9-10}
Segment & Sector & n & Median & IQR & n & Median & IQR & Median & IQR\\
\midrule
\endhead

\endfoot
\bottomrule
\insertTableNotes
\endlastfoot
\cellcolor{gray!6}{Alaska Aleutians} & \cellcolor{gray!6}{1} & \cellcolor{gray!6}{5} & \cellcolor{gray!6}{96.1} & \cellcolor{gray!6}{42.6} & \cellcolor{gray!6}{80} & \cellcolor{gray!6}{82.7} & \cellcolor{gray!6}{33.0} & \cellcolor{gray!6}{103.0} & \cellcolor{gray!6}{34.0}\\
Alaska Aleutians & 2 & 1 & 62.0 & 0.0 & 69 & 75.2 & 16.8 & 74.6 & 18.4\\
\cellcolor{gray!6}{Alaska Aleutians} & \cellcolor{gray!6}{5} & \cellcolor{gray!6}{1} & \cellcolor{gray!6}{62.0} & \cellcolor{gray!6}{0.0} & \cellcolor{gray!6}{68} & \cellcolor{gray!6}{75.0} & \cellcolor{gray!6}{16.7} & \cellcolor{gray!6}{68.7} & \cellcolor{gray!6}{7.7}\\
Alaska Aleutians & 6 & 13 & 50.0 & 22.2 & 115 & 74.0 & 17.5 & 64.6 & 13.7\\
\cellcolor{gray!6}{Alaska Aleutians} & \cellcolor{gray!6}{7} & \cellcolor{gray!6}{2} & \cellcolor{gray!6}{55.0} & \cellcolor{gray!6}{11.1} & \cellcolor{gray!6}{35} & \cellcolor{gray!6}{76.6} & \cellcolor{gray!6}{13.2} & \cellcolor{gray!6}{53.0} & \cellcolor{gray!6}{15.9}\\
Alaska Aleutians & 8 & 4 & 45.6 & 15.1 & 79 & 79.9 & 6.8 & 56.1 & 19.3\\
\cellcolor{gray!6}{Alaska Aleutians} & \cellcolor{gray!6}{9} & \cellcolor{gray!6}{2} & \cellcolor{gray!6}{134.6} & \cellcolor{gray!6}{60.5} & \cellcolor{gray!6}{74} & \cellcolor{gray!6}{80.7} & \cellcolor{gray!6}{14.4} & \cellcolor{gray!6}{83.5} & \cellcolor{gray!6}{32.7}\\
Alaska Aleutians & 11 & 2 & 41.9 & 25.1 & 84 & 75.3 & 13.7 & 53.0 & 16.3\\
\cellcolor{gray!6}{Alaska Aleutians} & \cellcolor{gray!6}{12} & \cellcolor{gray!6}{8} & \cellcolor{gray!6}{74.5} & \cellcolor{gray!6}{15.2} & \cellcolor{gray!6}{86} & \cellcolor{gray!6}{76.3} & \cellcolor{gray!6}{17.3} & \cellcolor{gray!6}{78.4} & \cellcolor{gray!6}{36.1}\\
Alaska Aleutians & 13 & 6 & 84.0 & 15.8 & 72 & 77.8 & 16.7 & 83.3 & 15.2\\
\cellcolor{gray!6}{Alaska Aleutians} & \cellcolor{gray!6}{14} & \cellcolor{gray!6}{4} & \cellcolor{gray!6}{63.5} & \cellcolor{gray!6}{20.0} & \cellcolor{gray!6}{86} & \cellcolor{gray!6}{74.2} & \cellcolor{gray!6}{11.5} & \cellcolor{gray!6}{62.2} & \cellcolor{gray!6}{13.2}\\
Andes & 4 & 14 & 74.5 & 89.5 & 127 & 75.3 & 13.8 & 104.0 & 31.5\\
\cellcolor{gray!6}{Andes} & \cellcolor{gray!6}{5} & \cellcolor{gray!6}{68} & \cellcolor{gray!6}{69.0} & \cellcolor{gray!6}{59.8} & \cellcolor{gray!6}{114} & \cellcolor{gray!6}{78.7} & \cellcolor{gray!6}{16.1} & \cellcolor{gray!6}{114.3} & \cellcolor{gray!6}{41.2}\\
Andes & 6 & 39 & 61.0 & 40.5 & 122 & 73.6 & 23.4 & 99.7 & 31.2\\
\cellcolor{gray!6}{Andes} & \cellcolor{gray!6}{7} & \cellcolor{gray!6}{23} & \cellcolor{gray!6}{81.0} & \cellcolor{gray!6}{112.0} & \cellcolor{gray!6}{120} & \cellcolor{gray!6}{77.3} & \cellcolor{gray!6}{40.3} & \cellcolor{gray!6}{94.0} & \cellcolor{gray!6}{67.7}\\
Andes & 8 & 30 & 94.0 & 69.2 & 141 & 101.4 & 89.6 & 101.4 & 46.0\\
\cellcolor{gray!6}{Andes} & \cellcolor{gray!6}{9} & \cellcolor{gray!6}{45} & \cellcolor{gray!6}{61.0} & \cellcolor{gray!6}{57.0} & \cellcolor{gray!6}{130} & \cellcolor{gray!6}{74.6} & \cellcolor{gray!6}{56.3} & \cellcolor{gray!6}{89.3} & \cellcolor{gray!6}{58.2}\\
Andes & 10 & 11 & 45.0 & 19.5 & 94 & 68.2 & 24.7 & 86.9 & 48.1\\
\cellcolor{gray!6}{Andes} & \cellcolor{gray!6}{11} & \cellcolor{gray!6}{4} & \cellcolor{gray!6}{41.9} & \cellcolor{gray!6}{8.3} & \cellcolor{gray!6}{88} & \cellcolor{gray!6}{69.8} & \cellcolor{gray!6}{19.2} & \cellcolor{gray!6}{39.6} & \cellcolor{gray!6}{6.5}\\
Andes & 12 & 4 & 36.0 & 8.2 & 91 & 67.4 & 19.8 & 51.8 & 11.1\\
\cellcolor{gray!6}{Andes} & \cellcolor{gray!6}{13} & \cellcolor{gray!6}{36} & \cellcolor{gray!6}{71.0} & \cellcolor{gray!6}{7.0} & \cellcolor{gray!6}{88} & \cellcolor{gray!6}{74.0} & \cellcolor{gray!6}{21.2} & \cellcolor{gray!6}{101.4} & \cellcolor{gray!6}{59.9}\\
Central America & 1 & 72 & 42.0 & 13.0 & 64 & 56.2 & 24.9 & 40.8 & 15.5\\
\cellcolor{gray!6}{Central America} & \cellcolor{gray!6}{2} & \cellcolor{gray!6}{2} & \cellcolor{gray!6}{50.2} & \cellcolor{gray!6}{4.1} & \cellcolor{gray!6}{41} & \cellcolor{gray!6}{76.7} & \cellcolor{gray!6}{22.6} & \cellcolor{gray!6}{35.7} & \cellcolor{gray!6}{21.9}\\
Central America & 4 & 1 & 37.7 & 0.0 & 59 & 77.7 & 24.2 & 45.6 & 18.1\\
\cellcolor{gray!6}{Central America} & \cellcolor{gray!6}{5} & \cellcolor{gray!6}{41} & \cellcolor{gray!6}{34.7} & \cellcolor{gray!6}{6.6} & \cellcolor{gray!6}{39} & \cellcolor{gray!6}{82.6} & \cellcolor{gray!6}{25.7} & \cellcolor{gray!6}{40.9} & \cellcolor{gray!6}{12.3}\\
Central America & 6 & 94 & 50.9 & 20.1 & 39 & 81.7 & 11.6 & 59.3 & 21.3\\
\cellcolor{gray!6}{Central America} & \cellcolor{gray!6}{7} & \cellcolor{gray!6}{2} & \cellcolor{gray!6}{76.4} & \cellcolor{gray!6}{11.5} & \cellcolor{gray!6}{48} & \cellcolor{gray!6}{81.5} & \cellcolor{gray!6}{11.1} & \cellcolor{gray!6}{66.3} & \cellcolor{gray!6}{4.3}\\
Central America & 8 & 10 & 63.0 & 15.1 & 44 & 71.8 & 26.4 & 63.6 & 16.3\\
\cellcolor{gray!6}{Kamchatka Marianas} & \cellcolor{gray!6}{3} & \cellcolor{gray!6}{25} & \cellcolor{gray!6}{186.0} & \cellcolor{gray!6}{112.0} & \cellcolor{gray!6}{81} & \cellcolor{gray!6}{70.4} & \cellcolor{gray!6}{44.1} & \cellcolor{gray!6}{59.1} & \cellcolor{gray!6}{48.6}\\
Kamchatka Marianas & 4 & 43 & 64.5 & 150.8 & 78 & 74.2 & 49.2 & 60.6 & 37.7\\
\cellcolor{gray!6}{Kamchatka Marianas} & \cellcolor{gray!6}{5} & \cellcolor{gray!6}{79} & \cellcolor{gray!6}{54.0} & \cellcolor{gray!6}{63.5} & \cellcolor{gray!6}{123} & \cellcolor{gray!6}{95.4} & \cellcolor{gray!6}{46.2} & \cellcolor{gray!6}{87.8} & \cellcolor{gray!6}{50.2}\\
Kamchatka Marianas & 6 & 116 & 69.6 & 64.5 & 86 & 75.3 & 52.5 & 67.4 & 51.4\\
\cellcolor{gray!6}{Kamchatka Marianas} & \cellcolor{gray!6}{7} & \cellcolor{gray!6}{301} & \cellcolor{gray!6}{75.0} & \cellcolor{gray!6}{50.0} & \cellcolor{gray!6}{113} & \cellcolor{gray!6}{78.6} & \cellcolor{gray!6}{43.3} & \cellcolor{gray!6}{76.3} & \cellcolor{gray!6}{42.1}\\
Kamchatka Marianas & 8 & 126 & 81.8 & 55.0 & 118 & 73.6 & 38.4 & 69.8 & 55.0\\
\cellcolor{gray!6}{Kamchatka Marianas} & \cellcolor{gray!6}{9} & \cellcolor{gray!6}{172} & \cellcolor{gray!6}{89.0} & \cellcolor{gray!6}{82.8} & \cellcolor{gray!6}{153} & \cellcolor{gray!6}{76.0} & \cellcolor{gray!6}{61.0} & \cellcolor{gray!6}{72.4} & \cellcolor{gray!6}{57.2}\\
Kamchatka Marianas & 10 & 59 & 83.7 & 30.8 & 98 & 91.7 & 36.7 & 80.7 & 33.7\\
\cellcolor{gray!6}{Kamchatka Marianas} & \cellcolor{gray!6}{11} & \cellcolor{gray!6}{27} & \cellcolor{gray!6}{80.0} & \cellcolor{gray!6}{39.8} & \cellcolor{gray!6}{94} & \cellcolor{gray!6}{83.7} & \cellcolor{gray!6}{30.6} & \cellcolor{gray!6}{68.7} & \cellcolor{gray!6}{32.4}\\
Kamchatka Marianas & 12 & 48 & 78.2 & 41.2 & 117 & 75.8 & 24.8 & 69.7 & 25.7\\
\cellcolor{gray!6}{Kamchatka Marianas} & \cellcolor{gray!6}{13} & \cellcolor{gray!6}{55} & \cellcolor{gray!6}{68.0} & \cellcolor{gray!6}{36.5} & \cellcolor{gray!6}{108} & \cellcolor{gray!6}{75.5} & \cellcolor{gray!6}{27.1} & \cellcolor{gray!6}{61.6} & \cellcolor{gray!6}{31.1}\\
Kyushu Ryukyu & 1 & 74 & 69.5 & 41.8 & 52 & 75.8 & 40.3 & 78.9 & 29.8\\
\cellcolor{gray!6}{Kyushu Ryukyu} & \cellcolor{gray!6}{2} & \cellcolor{gray!6}{25} & \cellcolor{gray!6}{80.0} & \cellcolor{gray!6}{40.0} & \cellcolor{gray!6}{43} & \cellcolor{gray!6}{77.6} & \cellcolor{gray!6}{13.1} & \cellcolor{gray!6}{76.6} & \cellcolor{gray!6}{16.0}\\
Kyushu Ryukyu & 3 & 6 & 67.5 & 18.2 & 61 & 86.2 & 17.8 & 74.9 & 37.3\\
\cellcolor{gray!6}{Kyushu Ryukyu} & \cellcolor{gray!6}{4} & \cellcolor{gray!6}{28} & \cellcolor{gray!6}{77.5} & \cellcolor{gray!6}{26.2} & \cellcolor{gray!6}{43} & \cellcolor{gray!6}{84.9} & \cellcolor{gray!6}{24.6} & \cellcolor{gray!6}{93.0} & \cellcolor{gray!6}{49.0}\\
Kyushu Ryukyu & 5 & 103 & 88.0 & 77.0 & 48 & 72.4 & 27.2 & 74.3 & 39.3\\
\cellcolor{gray!6}{Kyushu Ryukyu} & \cellcolor{gray!6}{6} & \cellcolor{gray!6}{25} & \cellcolor{gray!6}{126.0} & \cellcolor{gray!6}{94.0} & \cellcolor{gray!6}{39} & \cellcolor{gray!6}{80.4} & \cellcolor{gray!6}{19.0} & \cellcolor{gray!6}{81.7} & \cellcolor{gray!6}{70.0}\\
Kyushu Ryukyu & 7 & 42 & 60.0 & 70.2 & 33 & 76.3 & 16.7 & 64.2 & 53.1\\
\cellcolor{gray!6}{Kyushu Ryukyu} & \cellcolor{gray!6}{8} & \cellcolor{gray!6}{36} & \cellcolor{gray!6}{43.4} & \cellcolor{gray!6}{30.8} & \cellcolor{gray!6}{23} & \cellcolor{gray!6}{62.1} & \cellcolor{gray!6}{37.6} & \cellcolor{gray!6}{49.5} & \cellcolor{gray!6}{35.5}\\
Lesser Antilles & 1 & 3 & 54.4 & 0.4 & 23 & 54.0 & 7.5 & 49.9 & 4.8\\
\cellcolor{gray!6}{Lesser Antilles} & \cellcolor{gray!6}{3} & \cellcolor{gray!6}{10} & \cellcolor{gray!6}{38.1} & \cellcolor{gray!6}{31.9} & \cellcolor{gray!6}{20} & \cellcolor{gray!6}{57.7} & \cellcolor{gray!6}{24.0} & \cellcolor{gray!6}{59.4} & \cellcolor{gray!6}{26.7}\\
Lesser Antilles & 5 & 15 & 55.0 & 36.2 & 29 & 73.0 & 32.7 & 71.4 & 36.2\\
\cellcolor{gray!6}{Lesser Antilles} & \cellcolor{gray!6}{6} & \cellcolor{gray!6}{24} & \cellcolor{gray!6}{74.4} & \cellcolor{gray!6}{89.3} & \cellcolor{gray!6}{17} & \cellcolor{gray!6}{68.6} & \cellcolor{gray!6}{83.1} & \cellcolor{gray!6}{81.1} & \cellcolor{gray!6}{103.9}\\
Lesser Antilles & 7 & 6 & 78.2 & 26.8 & 29 & 68.4 & 36.0 & 84.0 & 34.3\\
\cellcolor{gray!6}{Lesser Antilles} & \cellcolor{gray!6}{8} & \cellcolor{gray!6}{14} & \cellcolor{gray!6}{54.5} & \cellcolor{gray!6}{32.0} & \cellcolor{gray!6}{46} & \cellcolor{gray!6}{64.9} & \cellcolor{gray!6}{20.3} & \cellcolor{gray!6}{59.5} & \cellcolor{gray!6}{21.7}\\
N Philippines & 1 & 2 & 46.3 & 2.3 & 30 & 65.3 & 18.0 & 45.4 & 11.6\\
\cellcolor{gray!6}{N Philippines} & \cellcolor{gray!6}{2} & \cellcolor{gray!6}{3} & \cellcolor{gray!6}{44.0} & \cellcolor{gray!6}{3.3} & \cellcolor{gray!6}{20} & \cellcolor{gray!6}{71.6} & \cellcolor{gray!6}{26.2} & \cellcolor{gray!6}{45.9} & \cellcolor{gray!6}{11.7}\\
N Philippines & 3 & 2 & 75.4 & 33.5 & 17 & 66.2 & 33.9 & 46.8 & 13.2\\
\cellcolor{gray!6}{N Philippines} & \cellcolor{gray!6}{4} & \cellcolor{gray!6}{5} & \cellcolor{gray!6}{23.0} & \cellcolor{gray!6}{7.0} & \cellcolor{gray!6}{33} & \cellcolor{gray!6}{75.7} & \cellcolor{gray!6}{34.2} & \cellcolor{gray!6}{44.5} & \cellcolor{gray!6}{14.1}\\
N Philippines & 6 & 1 & 51.0 & 0.0 & 30 & 81.1 & 14.3 & 45.3 & 15.2\\
\cellcolor{gray!6}{New Britain Solomon} & \cellcolor{gray!6}{3} & \cellcolor{gray!6}{1} & \cellcolor{gray!6}{37.7} & \cellcolor{gray!6}{0.0} & \cellcolor{gray!6}{26} & \cellcolor{gray!6}{83.2} & \cellcolor{gray!6}{24.9} & \cellcolor{gray!6}{38.9} & \cellcolor{gray!6}{13.3}\\
New Britain Solomon & 4 & 1 & 2.9 & 0.0 & 16 & 95.2 & 46.1 & 24.3 & 13.4\\
\cellcolor{gray!6}{New Britain Solomon} & \cellcolor{gray!6}{5} & \cellcolor{gray!6}{3} & \cellcolor{gray!6}{36.8} & \cellcolor{gray!6}{12.1} & \cellcolor{gray!6}{64} & \cellcolor{gray!6}{57.9} & \cellcolor{gray!6}{29.5} & \cellcolor{gray!6}{43.1} & \cellcolor{gray!6}{9.6}\\
New Britain Solomon & 6 & 3 & 35.2 & 10.6 & 38 & 52.5 & 10.5 & 31.4 & 9.1\\
\cellcolor{gray!6}{New Britain Solomon} & \cellcolor{gray!6}{8} & \cellcolor{gray!6}{1} & \cellcolor{gray!6}{58.2} & \cellcolor{gray!6}{0.0} & \cellcolor{gray!6}{19} & \cellcolor{gray!6}{56.6} & \cellcolor{gray!6}{27.8} & \cellcolor{gray!6}{45.7} & \cellcolor{gray!6}{5.4}\\
S Philippines & 2 & 6 & 127.5 & 37.6 & 83 & 88.2 & 45.0 & 102.8 & 19.2\\
\cellcolor{gray!6}{S Philippines} & \cellcolor{gray!6}{4} & \cellcolor{gray!6}{4} & \cellcolor{gray!6}{97.0} & \cellcolor{gray!6}{105.8} & \cellcolor{gray!6}{62} & \cellcolor{gray!6}{73.0} & \cellcolor{gray!6}{24.7} & \cellcolor{gray!6}{49.0} & \cellcolor{gray!6}{11.9}\\
S Philippines & 5 & 5 & 62.8 & 4.6 & 68 & 69.6 & 18.0 & 56.1 & 12.6\\
\cellcolor{gray!6}{S Philippines} & \cellcolor{gray!6}{6} & \cellcolor{gray!6}{3} & \cellcolor{gray!6}{62.8} & \cellcolor{gray!6}{19.0} & \cellcolor{gray!6}{72} & \cellcolor{gray!6}{76.8} & \cellcolor{gray!6}{27.4} & \cellcolor{gray!6}{48.6} & \cellcolor{gray!6}{24.8}\\
S Philippines & 7 & 5 & 46.0 & 5.0 & 46 & 76.9 & 14.6 & 46.7 & 5.5\\
\cellcolor{gray!6}{S Philippines} & \cellcolor{gray!6}{8} & \cellcolor{gray!6}{4} & \cellcolor{gray!6}{45.5} & \cellcolor{gray!6}{7.2} & \cellcolor{gray!6}{65} & \cellcolor{gray!6}{81.4} & \cellcolor{gray!6}{18.0} & \cellcolor{gray!6}{44.4} & \cellcolor{gray!6}{4.8}\\
Scotia & 2 & 3 & 143.0 & 5.5 & 28 & 120.0 & 51.2 & 127.2 & 11.1\\
\cellcolor{gray!6}{Scotia} & \cellcolor{gray!6}{3} & \cellcolor{gray!6}{9} & \cellcolor{gray!6}{134.0} & \cellcolor{gray!6}{37.0} & \cellcolor{gray!6}{54} & \cellcolor{gray!6}{90.2} & \cellcolor{gray!6}{38.9} & \cellcolor{gray!6}{127.1} & \cellcolor{gray!6}{8.9}\\
Sumatra Banda Sea & 1 & 339 & 21.0 & 10.8 & 69 & 74.4 & 15.2 & 86.4 & 123.3\\
\cellcolor{gray!6}{Sumatra Banda Sea} & \cellcolor{gray!6}{3} & \cellcolor{gray!6}{23} & \cellcolor{gray!6}{80.0} & \cellcolor{gray!6}{24.2} & \cellcolor{gray!6}{59} & \cellcolor{gray!6}{75.4} & \cellcolor{gray!6}{22.6} & \cellcolor{gray!6}{70.2} & \cellcolor{gray!6}{23.7}\\
Sumatra Banda Sea & 4 & 208 & 113.0 & 46.2 & 112 & 85.2 & 32.1 & 90.5 & 42.6\\
\cellcolor{gray!6}{Sumatra Banda Sea} & \cellcolor{gray!6}{5} & \cellcolor{gray!6}{192} & \cellcolor{gray!6}{123.0} & \cellcolor{gray!6}{32.5} & \cellcolor{gray!6}{95} & \cellcolor{gray!6}{85.4} & \cellcolor{gray!6}{36.9} & \cellcolor{gray!6}{100.4} & \cellcolor{gray!6}{58.9}\\
Sumatra Banda Sea & 6 & 40 & 103.0 & 13.0 & 73 & 72.9 & 50.0 & 67.2 & 62.1\\
\cellcolor{gray!6}{Sumatra Banda Sea} & \cellcolor{gray!6}{7} & \cellcolor{gray!6}{86} & \cellcolor{gray!6}{70.5} & \cellcolor{gray!6}{31.5} & \cellcolor{gray!6}{72} & \cellcolor{gray!6}{71.7} & \cellcolor{gray!6}{24.7} & \cellcolor{gray!6}{71.7} & \cellcolor{gray!6}{30.0}\\
Sumatra Banda Sea & 8 & 40 & 78.0 & 18.5 & 64 & 66.7 & 18.0 & 56.2 & 26.7\\
\cellcolor{gray!6}{Sumatra Banda Sea} & \cellcolor{gray!6}{9} & \cellcolor{gray!6}{30} & \cellcolor{gray!6}{77.5} & \cellcolor{gray!6}{25.2} & \cellcolor{gray!6}{83} & \cellcolor{gray!6}{68.8} & \cellcolor{gray!6}{28.8} & \cellcolor{gray!6}{42.5} & \cellcolor{gray!6}{39.1}\\
Sumatra Banda Sea & 10 & 5 & 75.0 & 51.2 & 91 & 70.7 & 24.7 & 53.8 & 18.4\\
\cellcolor{gray!6}{Sumatra Banda Sea} & \cellcolor{gray!6}{11} & \cellcolor{gray!6}{1} & \cellcolor{gray!6}{71.2} & \cellcolor{gray!6}{0.0} & \cellcolor{gray!6}{67} & \cellcolor{gray!6}{72.3} & \cellcolor{gray!6}{12.4} & \cellcolor{gray!6}{62.4} & \cellcolor{gray!6}{7.5}\\
Sumatra Banda Sea & 12 & 0 &  &  & 85 & 80.0 & 19.0 & 69.3 & 20.5\\
\cellcolor{gray!6}{Tonga New Zealand} & \cellcolor{gray!6}{1} & \cellcolor{gray!6}{75} & \cellcolor{gray!6}{47.0} & \cellcolor{gray!6}{39.0} & \cellcolor{gray!6}{44} & \cellcolor{gray!6}{56.9} & \cellcolor{gray!6}{24.3} & \cellcolor{gray!6}{47.9} & \cellcolor{gray!6}{21.1}\\
Tonga New Zealand & 2 & 44 & 39.5 & 20.8 & 34 & 49.7 & 29.0 & 42.6 & 18.5\\
\cellcolor{gray!6}{Tonga New Zealand} & \cellcolor{gray!6}{3} & \cellcolor{gray!6}{30} & \cellcolor{gray!6}{64.0} & \cellcolor{gray!6}{36.0} & \cellcolor{gray!6}{64} & \cellcolor{gray!6}{73.6} & \cellcolor{gray!6}{38.2} & \cellcolor{gray!6}{95.3} & \cellcolor{gray!6}{90.0}\\
Tonga New Zealand & 4 & 1 & 24.3 & 0.0 & 48 & 76.0 & 28.2 & 49.8 & 60.8\\
\cellcolor{gray!6}{Tonga New Zealand} & \cellcolor{gray!6}{5} & \cellcolor{gray!6}{1} & \cellcolor{gray!6}{15.1} & \cellcolor{gray!6}{0.0} & \cellcolor{gray!6}{68} & \cellcolor{gray!6}{80.7} & \cellcolor{gray!6}{37.4} & \cellcolor{gray!6}{40.5} & \cellcolor{gray!6}{51.2}\\
Tonga New Zealand & 6 & 29 & 31.2 & 15.0 & 48 & 79.7 & 35.6 & 127.9 & 120.0\\
\cellcolor{gray!6}{Tonga New Zealand} & \cellcolor{gray!6}{7} & \cellcolor{gray!6}{35} & \cellcolor{gray!6}{28.5} & \cellcolor{gray!6}{7.1} & \cellcolor{gray!6}{53} & \cellcolor{gray!6}{71.9} & \cellcolor{gray!6}{24.0} & \cellcolor{gray!6}{47.4} & \cellcolor{gray!6}{32.0}\\
Tonga New Zealand & 8 & 7 & 49.0 & 49.2 & 64 & 81.0 & 43.8 & 58.0 & 24.1\\
\cellcolor{gray!6}{Tonga New Zealand} & \cellcolor{gray!6}{9} & \cellcolor{gray!6}{4} & \cellcolor{gray!6}{31.1} & \cellcolor{gray!6}{23.2} & \cellcolor{gray!6}{58} & \cellcolor{gray!6}{73.8} & \cellcolor{gray!6}{34.8} & \cellcolor{gray!6}{53.1} & \cellcolor{gray!6}{46.2}\\
Tonga New Zealand & 10 & 4 & 59.7 & 47.0 & 48 & 74.3 & 29.3 & 71.0 & 48.9\\
\cellcolor{gray!6}{Tonga New Zealand} & \cellcolor{gray!6}{11} & \cellcolor{gray!6}{5} & \cellcolor{gray!6}{31.8} & \cellcolor{gray!6}{19.7} & \cellcolor{gray!6}{52} & \cellcolor{gray!6}{79.3} & \cellcolor{gray!6}{33.4} & \cellcolor{gray!6}{53.8} & \cellcolor{gray!6}{41.1}\\
Vanuatu & 1 & 9 & 96.0 & 72.0 & 68 & 81.6 & 17.5 & 84.5 & 37.5\\
\cellcolor{gray!6}{Vanuatu} & \cellcolor{gray!6}{2} & \cellcolor{gray!6}{4} & \cellcolor{gray!6}{91.4} & \cellcolor{gray!6}{32.7} & \cellcolor{gray!6}{44} & \cellcolor{gray!6}{103.0} & \cellcolor{gray!6}{51.3} & \cellcolor{gray!6}{79.8} & \cellcolor{gray!6}{49.8}\\
Vanuatu & 3 & 6 & 54.5 & 116.8 & 27 & 101.7 & 60.0 & 96.9 & 85.0\\
\cellcolor{gray!6}{Vanuatu} & \cellcolor{gray!6}{4} & \cellcolor{gray!6}{3} & \cellcolor{gray!6}{125.0} & \cellcolor{gray!6}{9.5} & \cellcolor{gray!6}{34} & \cellcolor{gray!6}{110.8} & \cellcolor{gray!6}{67.7} & \cellcolor{gray!6}{119.4} & \cellcolor{gray!6}{87.2}\\
Vanuatu & 5 & 4 & 174.5 & 18.8 & 36 & 107.5 & 75.6 & 133.6 & 41.7\\
\cellcolor{gray!6}{Vanuatu} & \cellcolor{gray!6}{6} & \cellcolor{gray!6}{2} & \cellcolor{gray!6}{123.0} & \cellcolor{gray!6}{18.0} & \cellcolor{gray!6}{30} & \cellcolor{gray!6}{118.1} & \cellcolor{gray!6}{48.1} & \cellcolor{gray!6}{110.0} & \cellcolor{gray!6}{29.8}\\
Vanuatu & 7 & 2 & 57.0 & 2.9 & 20 & 109.8 & 18.0 & 71.9 & 25.3\\*
\end{longtable}
\end{ThreePartTable}
\endgroup{}

\endgroup

\cleardoublepage

\markboth{Appendix: Chapter 4}{Appendix: Chapter 4}

\hypertarget{computing-rates-and-distributions-of-rock-recovery-in-subduction-zones}{%
\chapter{Computing Rates and Distributions of Rock Recovery in Subduction Zones}\label{computing-rates-and-distributions-of-rock-recovery-in-subduction-zones}}

\clearpage

\hypertarget{gmm}{%
\section{Gaussian Mixture Models}\label{gmm}}

Let the traced markers represent a \(d\)-dimensional array of \(n\) random independent variables \(x_i \in \mathbb{R}^{n \times d}\). Assume markers \(x_i\) were drawn from \(k\) discrete probability distributions with parameters \(\Phi\). The probability distribution of markers \(x_i\) can be modelled with a mixture of \(k\) components:
\begin{equation}
  p(x_i | \Phi) = \sum_{j=1}^k \pi_j p(x_i | \Theta_j)
  \label{eq:gmix}
\end{equation}
where \(p(x_i | \Theta_j)\) is the probability of \(x_i\) under the \(j^{th}\) mixture component and \(\pi_j\) is the mixture proportion representing the probability that \(x_i\) belongs to the \(j^{th}\) component \((\pi_j \geq 0; \sum_{j=1}^k \pi_j = 1)\).

Assuming \(\Theta_j\) describes a Gaussian probability distributions with mean \(\mu_j\) and covariance \(\Sigma_j\), Equation \eqref{eq:gmix} becomes:
\begin{equation}
  p(x_i | \Phi) = \sum_{j=1}^k \pi_j \mathcal{N}(x_i | \mu_j, \Sigma_j)
  \label{eq:mix}
\end{equation}
where
\begin{equation}
  \mathcal{N}(x_i | \mu_j, \Sigma_j) = \frac{exp\{ -\frac{1}{2}(x_i - \mu_j)(x_i - \mu_j)^T \Sigma_j^{-1}\}}{\sqrt{det(2 \pi \Sigma_j)}}
  \label{eq:gauss}
\end{equation}

The parameters \(\mu_j\) and \(\Sigma_j\), representing the center and shape of each cluster, are estimated by maximizing the log of the likelihood function, \(L(x_i | \Phi) = \prod_{i=1}^n p(x_i | \Phi)\):
\begin{equation}
  log~L(x_i | \Phi) = log \prod_{i=1}^n p(x_i | \Phi) = \sum_{i=1}^n log \left[ \sum_{j=1}^k \pi_j p(x_i | \Theta_j) \right]
  \label{eq:loglik}
\end{equation}

Taking the derivative of Equation \eqref{eq:loglik} with respect to each parameter, \(\pi\), \(\mu\), \(\Sigma\), setting the equation to zero, and solving for each parameter gives the maximum likelihood estimators:
\begin{equation}
  \begin{aligned}
    N_j &= \sum_{i=1}^n \omega_{i} \\
    \pi_j &= \frac{N_j}{n} \\
    \mu_j &= \frac{1}{N_j} \sum_{i=1}^n \omega_{i} x_i \\
    \Sigma_j &= \frac{1}{N_j} \sum_{i=1}^n \omega_{i} (x_i - \mu_j)(x_i - \mu_j)^T
  \end{aligned}
  \label{eq:mle}
\end{equation}
where \(\omega_{i}\) (\(\omega_{i} \geq 0; \sum_{j=1}^k \omega_{i} = 1\)) are membership weights representing the probability of an observation \(x_i\) belonging to the \(j^{th}\) Gaussian and \(N_j\) represents the number of observations belonging to the \(j^{th}\) Gaussian. Please note that \(\omega_{i}\) is unknown for markers so maximum likelihood estimator cannot be computed with Equation \eqref{eq:mle}. The solution to this problem is the Expectation-Maximization algorithm, which is defined below.

General purpose functions in the R package \texttt{Mclust} (\protect\hyperlink{ref-scrucca2016}{Scrucca et al., 2016}) are used to fit Gaussian mixture models. ``Fitting'' refers to adjusting all \(k\) Gaussian parameters \(\mu_j\) and \(\Sigma_j\) until the data and Gaussian ellipsoids achieve maximum likelihood defined by Equation \eqref{eq:loglik}. After Banfield \& Raftery (\protect\hyperlink{ref-banfield1993}{1993}), covariance matrices \(\Sigma\) in \texttt{Mclust} are parameterized to be flexible in their shape, volume, and orientation (\protect\hyperlink{ref-scrucca2016}{Scrucca et al., 2016}):
\begin{equation}
  \Sigma_j = \lambda_j D_j A_j D_j^T
  \label{eq:eigen}
\end{equation}
where \(D_j\) is the orthogonal eigenvector matrix, \(A_j\) and \(\lambda_j\) are diagonal matrices of values proportional to the eigenvalues. This implementation allows fixing one, two, or three geometric elements of the covariance matrices. That is, the volume \(\lambda_j\), shape \(A_j\), and orientation \(D_j\) of Gaussian clusters can change or be fixed among all \(k\) clusters (e.g. \protect\hyperlink{ref-celeux1995}{Celeux \& Govaert, 1995}; \protect\hyperlink{ref-fraley2002}{Fraley \& Raftery, 2002}). Fourteen parameterizations of Equation \eqref{eq:eigen} are tried, representing different geometric combinations of the covariance matrices \(\Sigma\) (see \protect\hyperlink{ref-scrucca2016}{Scrucca et al., 2016}) and the Bayesian information criterion is computed (\protect\hyperlink{ref-schwarz1978}{Schwarz, 1978}). The parameterization for Equation \eqref{eq:eigen} is chosen by Bayesian information criterion.

\hypertarget{expectation-maximization}{%
\section{Expectation-Maximization}\label{expectation-maximization}}

The Expectation-Maximization algorithm estimates Gaussian mixture model parameters by initializing \(k\) Gaussians with parameters \((\pi_j, \mu_j, \Sigma_j)\), then iteratively computing membership weights with Equation \eqref{eq:posterior} and updating Gaussian parameters with Equation \eqref{eq:mle} until reaching a convergence threshold (\protect\hyperlink{ref-dempster1977}{Dempster et al., 1977}).

The \emph{expectation} (E-)step involves a ``latent'' multinomial variable \(z_{i} \in \{1, 2, \dots, k\}\) representing the unknown classifications of \(x_i\) with a joint distribution \(p(x_i,z_{i}) = p(x_i | z_{i})p(z_{j})\). Membership weights \(\omega_{i}\) are equivalent to the conditional probability \(p(z_{i} | x_i)\), which represents the probability of observation \(x_i\) belonging to the \(j^{th}\) Gaussian. Given initial guesses for Gaussian parameters \(\pi_j\), \(\mu_j\), \(\Sigma_j\), membership weights are computed using Bayes Theorem (E-step):
\begin{equation}
  p(z_{i} | x_i) = \frac{p(x_i | z_{i})p(z_{j})}{p(x_i)} = \frac{\pi_j \mathcal{N}(\mu_j, \Sigma_j)}{\sum_{j=1}^k \pi_j \mathcal{N}(\mu_j, \Sigma_j)} = \omega_{i}
  \label{eq:posterior}
\end{equation}
and Gaussian estimates are updated during the \emph{maximization} (M-)step by applying \(\omega_{i}\) to Equation \eqref{eq:mle}. This step gives markers \(x_i\) class labels \(z_i \in \{1, \dots, k\}\) representing assignment to one of \(k\) clusters (Figure \ref{fig:class}).

\cleardoublepage

\hypertarget{vis}{%
\section{Marker Classifications}\label{vis}}

The following pages contain visualizations of marker classifications results including marker \gls{pt} distributions and geodynamic snapshots for all 64 subduction zone simulations presented in the main text of this study. All data and visualizations are available online at \url{https://github.com/buchanankerswell/kerswell_et_al_marx}.

\clearpage

\blandscape

\begin{figure}
\centering
\includegraphics{assets/figs/chpt4/cda46_bigComp.png}
\caption[PT distribution of recovered markers from model cda46]{PT distribution of recovered markers from model cda46. Refer to the main text for explanation of panels and colors.}
\end{figure}

\elandscape

\blandscape

\begin{figure}
\centering
\includegraphics{assets/figs/chpt4/cda62_bigComp.png}
\caption[PT distribution of recovered markers from model cda62]{PT distribution of recovered markers from model cda62. Refer to the main text for explanation of panels and colors.}
\end{figure}

\elandscape

\blandscape

\begin{figure}
\centering
\includegraphics{assets/figs/chpt4/cda78_bigComp.png}
\caption[PT distribution of recovered markers from model cda78]{PT distribution of recovered markers from model cda78. Refer to the main text for explanation of panels and colors.}
\end{figure}

\elandscape

\blandscape

\begin{figure}
\centering
\includegraphics{assets/figs/chpt4/cda94_bigComp.png}
\caption[PT distribution of recovered markers from model cda94]{PT distribution of recovered markers from model cda94. Refer to the main text for explanation of panels and colors.}
\end{figure}

\elandscape

\blandscape

\begin{figure}
\centering
\includegraphics{assets/figs/chpt4/cdb46_bigComp.png}
\caption[PT distribution of recovered markers from model cdb46]{PT distribution of recovered markers from model cdb46. Refer to the main text for explanation of panels and colors.}
\end{figure}

\elandscape

\blandscape

\begin{figure}
\centering
\includegraphics{assets/figs/chpt4/cdb62_bigComp.png}
\caption[PT distribution of recovered markers from model cdb62]{PT distribution of recovered markers from model cdb62. Refer to the main text for explanation of panels and colors.}
\end{figure}

\elandscape

\blandscape

\begin{figure}
\centering
\includegraphics{assets/figs/chpt4/cdb78_bigComp.png}
\caption[PT distribution of recovered markers from model cdb78]{PT distribution of recovered markers from model cdb78. Refer to the main text for explanation of panels and colors.}
\end{figure}

\elandscape

\blandscape

\begin{figure}
\centering
\includegraphics{assets/figs/chpt4/cdb94_bigComp.png}
\caption[PT distribution of recovered markers from model cdb94]{PT distribution of recovered markers from model cdb94. Refer to the main text for explanation of panels and colors.}
\end{figure}

\elandscape

\blandscape

\begin{figure}
\centering
\includegraphics{assets/figs/chpt4/cdc46_bigComp.png}
\caption[PT distribution of recovered markers from model cdc46]{PT distribution of recovered markers from model cdc46. Refer to the main text for explanation of panels and colors.}
\end{figure}

\elandscape

\blandscape

\begin{figure}
\centering
\includegraphics{assets/figs/chpt4/cdc62_bigComp.png}
\caption[PT distribution of recovered markers from model cdc62]{PT distribution of recovered markers from model cdc62. Refer to the main text for explanation of panels and colors.}
\end{figure}

\elandscape

\blandscape

\begin{figure}
\centering
\includegraphics{assets/figs/chpt4/cdc78_bigComp.png}
\caption[PT distribution of recovered markers from model cdc78]{PT distribution of recovered markers from model cdc78. Refer to the main text for explanation of panels and colors.}
\end{figure}

\elandscape

\blandscape

\begin{figure}
\centering
\includegraphics{assets/figs/chpt4/cdc94_bigComp.png}
\caption[PT distribution of recovered markers from model cdc94]{PT distribution of recovered markers from model cdc94. Refer to the main text for explanation of panels and colors.}
\end{figure}

\elandscape

\blandscape

\begin{figure}
\centering
\includegraphics{assets/figs/chpt4/cdd46_bigComp.png}
\caption[PT distribution of recovered markers from model cdd46]{PT distribution of recovered markers from model cdd46. Refer to the main text for explanation of panels and colors.}
\end{figure}

\elandscape

\blandscape

\begin{figure}
\centering
\includegraphics{assets/figs/chpt4/cdd62_bigComp.png}
\caption[PT distribution of recovered markers from model cdd62]{PT distribution of recovered markers from model cdd62. Refer to the main text for explanation of panels and colors.}
\end{figure}

\elandscape

\blandscape

\begin{figure}
\centering
\includegraphics{assets/figs/chpt4/cdd78_bigComp.png}
\caption[PT distribution of recovered markers from model cdd78]{PT distribution of recovered markers from model cdd78. Refer to the main text for explanation of panels and colors.}
\end{figure}

\elandscape

\blandscape

\begin{figure}
\centering
\includegraphics{assets/figs/chpt4/cdd94_bigComp.png}
\caption[PT distribution of recovered markers from model cdd94]{PT distribution of recovered markers from model cdd94. Refer to the main text for explanation of panels and colors.}
\end{figure}

\elandscape

\blandscape

\begin{figure}
\centering
\includegraphics{assets/figs/chpt4/cde46_bigComp.png}
\caption[PT distribution of recovered markers from model cde46]{PT distribution of recovered markers from model cde46. Refer to the main text for explanation of panels and colors.}
\end{figure}

\elandscape

\blandscape

\begin{figure}
\centering
\includegraphics{assets/figs/chpt4/cde62_bigComp.png}
\caption[PT distribution of recovered markers from model cde62]{PT distribution of recovered markers from model cde62. Refer to the main text for explanation of panels and colors.}
\end{figure}

\elandscape

\blandscape

\begin{figure}
\centering
\includegraphics{assets/figs/chpt4/cde78_bigComp.png}
\caption[PT distribution of recovered markers from model cde78]{PT distribution of recovered markers from model cde78. Refer to the main text for explanation of panels and colors.}
\end{figure}

\elandscape

\blandscape

\begin{figure}
\centering
\includegraphics{assets/figs/chpt4/cde94_bigComp.png}
\caption[PT distribution of recovered markers from model cde94]{PT distribution of recovered markers from model cde94. Refer to the main text for explanation of panels and colors.}
\end{figure}

\elandscape

\blandscape

\begin{figure}
\centering
\includegraphics{assets/figs/chpt4/cdf46_bigComp.png}
\caption[PT distribution of recovered markers from model cdf46]{PT distribution of recovered markers from model cdf46. Refer to the main text for explanation of panels and colors.}
\end{figure}

\elandscape

\blandscape

\begin{figure}
\centering
\includegraphics{assets/figs/chpt4/cdf62_bigComp.png}
\caption[PT distribution of recovered markers from model cdf62]{PT distribution of recovered markers from model cdf62. Refer to the main text for explanation of panels and colors.}
\end{figure}

\elandscape

\blandscape

\begin{figure}
\centering
\includegraphics{assets/figs/chpt4/cdf78_bigComp.png}
\caption[PT distribution of recovered markers from model cdf78]{PT distribution of recovered markers from model cdf78. Refer to the main text for explanation of panels and colors.}
\end{figure}

\elandscape

\blandscape

\begin{figure}
\centering
\includegraphics{assets/figs/chpt4/cdf94_bigComp.png}
\caption[PT distribution of recovered markers from model cdf94]{PT distribution of recovered markers from model cdf94. Refer to the main text for explanation of panels and colors.}
\end{figure}

\elandscape

\blandscape

\begin{figure}
\centering
\includegraphics{assets/figs/chpt4/cdg46_bigComp.png}
\caption[PT distribution of recovered markers from model cdg46]{PT distribution of recovered markers from model cdg46. Refer to the main text for explanation of panels and colors.}
\end{figure}

\elandscape

\blandscape

\begin{figure}
\centering
\includegraphics{assets/figs/chpt4/cdg62_bigComp.png}
\caption[PT distribution of recovered markers from model cdg62]{PT distribution of recovered markers from model cdg62. Refer to the main text for explanation of panels and colors.}
\end{figure}

\elandscape

\blandscape

\begin{figure}
\centering
\includegraphics{assets/figs/chpt4/cdg78_bigComp.png}
\caption[PT distribution of recovered markers from model cdg78]{PT distribution of recovered markers from model cdg78. Refer to the main text for explanation of panels and colors.}
\end{figure}

\elandscape

\blandscape

\begin{figure}
\centering
\includegraphics{assets/figs/chpt4/cdg94_bigComp.png}
\caption[PT distribution of recovered markers from model cdg94]{PT distribution of recovered markers from model cdg94. Refer to the main text for explanation of panels and colors.}
\end{figure}

\elandscape

\blandscape

\begin{figure}
\centering
\includegraphics{assets/figs/chpt4/cdh46_bigComp.png}
\caption[PT distribution of recovered markers from model cdh46]{PT distribution of recovered markers from model cdh46. Refer to the main text for explanation of panels and colors.}
\end{figure}

\elandscape

\blandscape

\begin{figure}
\centering
\includegraphics{assets/figs/chpt4/cdh62_bigComp.png}
\caption[PT distribution of recovered markers from model cdh62]{PT distribution of recovered markers from model cdh62. Refer to the main text for explanation of panels and colors.}
\end{figure}

\elandscape

\blandscape

\begin{figure}
\centering
\includegraphics{assets/figs/chpt4/cdh78_bigComp.png}
\caption[PT distribution of recovered markers from model cdh78]{PT distribution of recovered markers from model cdh78. Refer to the main text for explanation of panels and colors.}
\end{figure}

\elandscape

\blandscape

\begin{figure}
\centering
\includegraphics{assets/figs/chpt4/cdh94_bigComp.png}
\caption[PT distribution of recovered markers from model cdh94]{PT distribution of recovered markers from model cdh94. Refer to the main text for explanation of panels and colors.}
\end{figure}

\elandscape

\blandscape

\begin{figure}
\centering
\includegraphics{assets/figs/chpt4/cdi46_bigComp.png}
\caption[PT distribution of recovered markers from model cdi46]{PT distribution of recovered markers from model cdi46. Refer to the main text for explanation of panels and colors.}
\end{figure}

\elandscape

\blandscape

\begin{figure}
\centering
\includegraphics{assets/figs/chpt4/cdi62_bigComp.png}
\caption[PT distribution of recovered markers from model cdi62]{PT distribution of recovered markers from model cdi62. Refer to the main text for explanation of panels and colors.}
\end{figure}

\elandscape

\blandscape

\begin{figure}
\centering
\includegraphics{assets/figs/chpt4/cdi78_bigComp.png}
\caption[PT distribution of recovered markers from model cdi78]{PT distribution of recovered markers from model cdi78. Refer to the main text for explanation of panels and colors.}
\end{figure}

\elandscape

\blandscape

\begin{figure}
\centering
\includegraphics{assets/figs/chpt4/cdi94_bigComp.png}
\caption[PT distribution of recovered markers from model cdi94]{PT distribution of recovered markers from model cdi94. Refer to the main text for explanation of panels and colors.}
\end{figure}

\elandscape

\blandscape

\begin{figure}
\centering
\includegraphics{assets/figs/chpt4/cdj46_bigComp.png}
\caption[PT distribution of recovered markers from model cdj46]{PT distribution of recovered markers from model cdj46. Refer to the main text for explanation of panels and colors.}
\end{figure}

\elandscape

\blandscape

\begin{figure}
\centering
\includegraphics{assets/figs/chpt4/cdj62_bigComp.png}
\caption[PT distribution of recovered markers from model cdj62]{PT distribution of recovered markers from model cdj62. Refer to the main text for explanation of panels and colors.}
\end{figure}

\elandscape

\blandscape

\begin{figure}
\centering
\includegraphics{assets/figs/chpt4/cdj78_bigComp.png}
\caption[PT distribution of recovered markers from model cdj78]{PT distribution of recovered markers from model cdj78. Refer to the main text for explanation of panels and colors.}
\end{figure}

\elandscape

\blandscape

\begin{figure}
\centering
\includegraphics{assets/figs/chpt4/cdj94_bigComp.png}
\caption[PT distribution of recovered markers from model cdj94]{PT distribution of recovered markers from model cdj94. Refer to the main text for explanation of panels and colors.}
\end{figure}

\elandscape

\blandscape

\begin{figure}
\centering
\includegraphics{assets/figs/chpt4/cdk46_bigComp.png}
\caption[PT distribution of recovered markers from model cdk46]{PT distribution of recovered markers from model cdk46. Refer to the main text for explanation of panels and colors.}
\end{figure}

\elandscape

\blandscape

\begin{figure}
\centering
\includegraphics{assets/figs/chpt4/cdk62_bigComp.png}
\caption[PT distribution of recovered markers from model cdk62]{PT distribution of recovered markers from model cdk62. Refer to the main text for explanation of panels and colors.}
\end{figure}

\elandscape

\blandscape

\begin{figure}
\centering
\includegraphics{assets/figs/chpt4/cdk78_bigComp.png}
\caption[PT distribution of recovered markers from model cdk78]{PT distribution of recovered markers from model cdk78. Refer to the main text for explanation of panels and colors.}
\end{figure}

\elandscape

\blandscape

\begin{figure}
\centering
\includegraphics{assets/figs/chpt4/cdk94_bigComp.png}
\caption[PT distribution of recovered markers from model cdk94]{PT distribution of recovered markers from model cdk94. Refer to the main text for explanation of panels and colors.}
\end{figure}

\elandscape

\blandscape

\begin{figure}
\centering
\includegraphics{assets/figs/chpt4/cdl46_bigComp.png}
\caption[PT distribution of recovered markers from model cdl46]{PT distribution of recovered markers from model cdl46. Refer to the main text for explanation of panels and colors.}
\end{figure}

\elandscape

\blandscape

\begin{figure}
\centering
\includegraphics{assets/figs/chpt4/cdl62_bigComp.png}
\caption[PT distribution of recovered markers from model cdl62]{PT distribution of recovered markers from model cdl62. Refer to the main text for explanation of panels and colors.}
\end{figure}

\elandscape

\blandscape

\begin{figure}
\centering
\includegraphics{assets/figs/chpt4/cdl78_bigComp.png}
\caption[PT distribution of recovered markers from model cdl78]{PT distribution of recovered markers from model cdl78. Refer to the main text for explanation of panels and colors.}
\end{figure}

\elandscape

\blandscape

\begin{figure}
\centering
\includegraphics{assets/figs/chpt4/cdl94_bigComp.png}
\caption[PT distribution of recovered markers from model cdl94]{PT distribution of recovered markers from model cdl94. Refer to the main text for explanation of panels and colors.}
\end{figure}

\elandscape

\blandscape

\begin{figure}
\centering
\includegraphics{assets/figs/chpt4/cdm46_bigComp.png}
\caption[PT distribution of recovered markers from model cdm46]{PT distribution of recovered markers from model cdm46. Refer to the main text for explanation of panels and colors.}
\end{figure}

\elandscape

\blandscape

\begin{figure}
\centering
\includegraphics{assets/figs/chpt4/cdm62_bigComp.png}
\caption[PT distribution of recovered markers from model cdm62]{PT distribution of recovered markers from model cdm62. Refer to the main text for explanation of panels and colors.}
\end{figure}

\elandscape

\blandscape

\begin{figure}
\centering
\includegraphics{assets/figs/chpt4/cdm78_bigComp.png}
\caption[PT distribution of recovered markers from model cdm78]{PT distribution of recovered markers from model cdm78. Refer to the main text for explanation of panels and colors.}
\end{figure}

\elandscape

\blandscape

\begin{figure}
\centering
\includegraphics{assets/figs/chpt4/cdm94_bigComp.png}
\caption[PT distribution of recovered markers from model cdm94]{PT distribution of recovered markers from model cdm94. Refer to the main text for explanation of panels and colors.}
\end{figure}

\elandscape

\blandscape

\begin{figure}
\centering
\includegraphics{assets/figs/chpt4/cdn46_bigComp.png}
\caption[PT distribution of recovered markers from model cdn46]{PT distribution of recovered markers from model cdn46. Refer to the main text for explanation of panels and colors.}
\end{figure}

\elandscape

\blandscape

\begin{figure}
\centering
\includegraphics{assets/figs/chpt4/cdn62_bigComp.png}
\caption[PT distribution of recovered markers from model cdn62]{PT distribution of recovered markers from model cdn62. Refer to the main text for explanation of panels and colors.}
\end{figure}

\elandscape

\blandscape

\begin{figure}
\centering
\includegraphics{assets/figs/chpt4/cdn78_bigComp.png}
\caption[PT distribution of recovered markers from model cdn78]{PT distribution of recovered markers from model cdn78. Refer to the main text for explanation of panels and colors.}
\end{figure}

\elandscape

\blandscape

\begin{figure}
\centering
\includegraphics{assets/figs/chpt4/cdn94_bigComp.png}
\caption[PT distribution of recovered markers from model cdn94]{PT distribution of recovered markers from model cdn94. Refer to the main text for explanation of panels and colors.}
\end{figure}

\elandscape

\blandscape

\begin{figure}
\centering
\includegraphics{assets/figs/chpt4/cdo46_bigComp.png}
\caption[PT distribution of recovered markers from model cdo46]{PT distribution of recovered markers from model cdo46. Refer to the main text for explanation of panels and colors.}
\end{figure}

\elandscape

\blandscape

\begin{figure}
\centering
\includegraphics{assets/figs/chpt4/cdo62_bigComp.png}
\caption[PT distribution of recovered markers from model cdo62]{PT distribution of recovered markers from model cdo62. Refer to the main text for explanation of panels and colors.}
\end{figure}

\elandscape

\blandscape

\begin{figure}
\centering
\includegraphics{assets/figs/chpt4/cdo78_bigComp.png}
\caption[PT distribution of recovered markers from model cdo78]{PT distribution of recovered markers from model cdo78. Refer to the main text for explanation of panels and colors.}
\end{figure}

\elandscape

\blandscape

\begin{figure}
\centering
\includegraphics{assets/figs/chpt4/cdo94_bigComp.png}
\caption[PT distribution of recovered markers from model cdo94]{PT distribution of recovered markers from model cdo94. Refer to the main text for explanation of panels and colors.}
\end{figure}

\elandscape

\blandscape

\begin{figure}
\centering
\includegraphics{assets/figs/chpt4/cdp46_bigComp.png}
\caption[PT distribution of recovered markers from model cdp46]{PT distribution of recovered markers from model cdp46. Refer to the main text for explanation of panels and colors.}
\end{figure}

\elandscape

\blandscape

\begin{figure}
\centering
\includegraphics{assets/figs/chpt4/cdp62_bigComp.png}
\caption[PT distribution of recovered markers from model cdp62]{PT distribution of recovered markers from model cdp62. Refer to the main text for explanation of panels and colors.}
\end{figure}

\elandscape

\blandscape

\begin{figure}
\centering
\includegraphics{assets/figs/chpt4/cdp78_bigComp.png}
\caption[PT distribution of recovered markers from model cdp78]{PT distribution of recovered markers from model cdp78. Refer to the main text for explanation of panels and colors.}
\end{figure}

\elandscape

\blandscape

\begin{figure}
\centering
\includegraphics{assets/figs/chpt4/cdp94_bigComp.png}
\caption[PT distribution of recovered markers from model cdp94]{PT distribution of recovered markers from model cdp94. Refer to the main text for explanation of panels and colors.}
\end{figure}

\elandscape

\end{document}
